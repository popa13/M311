\documentclass[20pt,a4paper]{extarticle}
\usepackage[utf8]{inputenc}
\usepackage[english]{babel}

\usepackage{amsmath}
\usepackage{amsfonts}
\usepackage{amssymb}
\usepackage{mathtools}
\usepackage{systeme}
\sysdelim..

\usepackage{graphicx}
\usepackage{caption}
\usepackage{subcaption}
\usepackage{lmodern}
\usepackage{tikz}
\usetikzlibrary{calc}
\usepackage{titlesec}
\usepackage{environ}
\usepackage{xcolor}
\usepackage{fancyhdr}
\usepackage[colorlinks = true, linkcolor = black]{hyperref}
\usepackage{xparse}
\usepackage{enumitem}
\usepackage{comment}
\usepackage{wrapfig}
\usepackage{soul}
\usepackage[capitalise]{cleveref}
\usepackage{circledsteps}

\usepackage[left=1cm,right=1cm,top=1cm,bottom=3cm]{geometry}
\usepackage{multicol}
\usepackage[indent=0pt]{parskip}

\newcommand{\spaceP}{\vspace*{0.5cm}}
\newcommand{\range}{\mathrm{range}\,}
\newcommand{\ra}{\rightarrow}
\newcommand{\curl}{\mathrm{curl} \,}
\newcommand{\hint}[1]{\scalebox{2}{$\displaystyle\int_{\scalebox{0.35}{$#1$}}$}\,}
\newcommand{\hiint}[1]{\scalebox{2}{$\displaystyle\iint_{\scalebox{0.35}{$#1$}}$}\,}
\newcommand{\hiiint}[1]{\scalebox{2}{$\displaystyle\iiint_{\scalebox{0.35}{$#1$}}$}\,}
\renewcommand{\div}{\mathrm{div}\,}

\DeclareMathOperator{\Span}{span}

\makeatletter
\renewcommand*\env@matrix[1][*\c@MaxMatrixCols c]{%
  \hskip -\arraycolsep
  \let\@ifnextchar\new@ifnextchar
  \array{#1}}
\makeatother

%% Redefining sections
\newcommand{\sectionformat}[1]{%
    \begin{tikzpicture}[baseline=(title.base)]
        \node[rectangle, draw] (title) {#1};
    \end{tikzpicture}
    
    \noindent\hrulefill
}

\newif\ifhNotes 

\hNotesfalse

\ifhNotes
	\newcommand{\hideNotes}[1]{%
	\phantom{#1}
	}
	\newcommand{\hideNotesU}[1]{%
	\underline{\hspace{1mm}\phantom{#1}\hspace{1mm}}
	}
\else
	\newcommand{\hideNotes}[1]{#1}
	\newcommand{\hideNotesU}[1]{\textcolor{blue}{#1}}
\fi

% default values copied from titlesec documentation page 23
% parameters of \titleformat command are explained on page 4
\titleformat%
    {\section}% <command> is the sectioning command to be redefined, i. e., \part, \chapter, \section, \subsection, \subsubsection, \paragraph or \subparagraph.
    {\normalfont\large\scshape}% <format>
    {}% <label> the number
    {0em}% <sep> length. horizontal separation between label and title body
    {\centering\sectionformat}% code preceding the title body  (title body is taken as argument)

%% Set counters for sections to none
\setcounter{secnumdepth}{0}

%% Set the footer/headers
\pagestyle{fancy}
\fancyhf{}
\renewcommand{\headrulewidth}{0pt}
\renewcommand{\footrulewidth}{2pt}
\lfoot{P.-O. Paris{\'e}}
\cfoot{MATH 311}
\rfoot{Page \thepage}

%% Defining example environment
\newcounter{example}
\NewEnviron{example}%
	{%
	\noindent\refstepcounter{example}\fcolorbox{gray!40}{gray!40}{\textsc{\textcolor{red}{Example~\theexample.}}}%
	%\fcolorbox{black}{white}%
		{  %\parbox{0.95\textwidth}%
			{
			\BODY
			}%
		}%
	}

\newcounter{theorem}
\NewEnviron{theorem}%
	{%
	\noindent\refstepcounter{theorem}\fcolorbox{gray!40}{gray!40}{\textsc{\textcolor{black}{Theorem~\thetheorem.}}}%
	%\fcolorbox{black}{white}%
		{  %\parbox{0.95\textwidth}%
			{
			\BODY
			}%
		}%
	}

\newcounter{definition}
\NewEnviron{definition}%
	{%
	\noindent\refstepcounter{definition}\fcolorbox{gray!40}{gray!40}{\textsc{\textcolor{black}{Definition~\thedefinition.}}}%
	%\fcolorbox{black}{white}%
		{  %\parbox{0.95\textwidth}%
			{
			\BODY
			}%
		}%
	}

\newcounter{algo}
\NewEnviron{algorithm}
	{%
	\noindent\refstepcounter{algo}\fcolorbox{gray!40}{gray!40}{\textsc{\textcolor{black}{Algorithm~\thealgo.}}}%
	%\fcolorbox{black}{white}%
		{  %\parbox{0.95\textwidth}%
			{
			\BODY
			}%
		}%
	}

\NewEnviron{goal}
	{%
	\noindent\fcolorbox{gray!40}{gray!40}{\textsc{\textcolor{black}{Goal:}}}%
	%\fcolorbox{black}{white}%
		{  %\parbox{0.95\textwidth}%
			{
			\BODY
			}%
		}%
	}

\NewEnviron{solution}%
	{%
	\noindent \fcolorbox{gray!40}{gray!40}{\textsc{\textcolor{blue}{Solution.}}}%
	%\fcolorbox{black}{white}%
		{  %\parbox{0.95\textwidth}%
			{
			%\textcolor{blue}
			}%
		}%
	}

\NewEnviron{proof}%
	{%
	\noindent \fcolorbox{gray!40}{gray!40}{\textsc{\textcolor{blue}{Proof.}}}%
	%\fcolorbox{black}{white}%
		{  %\parbox{0.95\textwidth}%
			{
			\textcolor{blue}{%
			\BODY
			}
			}%
		}%
	}
%%% Ignorer les notes
%\excludecomment{notes}

%%%%
\begin{document}
\thispagestyle{empty}

\begin{center}
\vspace*{0.75cm}

{\Huge \textsc{Math 311}}

\vspace*{1.5cm}

{\LARGE \textsc{Chapter 9}} 

\vspace*{0.75cm}

\noindent\textsc{Section 9.2: Operators and Similarity}

\vspace*{0.25cm}

\tableofcontents

\vfill

\noindent \textsc{Created by: Pierre-Olivier Paris{\'e}} \\
\textsc{Spring 2024}
\end{center}

\newpage

\section{Operators}

\begin{definition}
A linear transformation $T : V \ra W$ is called an \textbf{linear operator} if $V = W$. We will therefore write $T : V \ra V$, where $V$ is a vector space.
\end{definition}

\subsubsection{$B$-matrix}

Recall that if $T : \mathbb{R}^n \ra \mathbb{R}^n$ is a linear operator and $E = \{ \mathbf{e_1} , \mathbf{e_2} , \ldots , \mathbf{e_n} \}$ is the standard basis, then the matrix representing $T$ on the basis $E$ is
	\[
		A = \begin{bmatrix} T (\mathbf{e_1}) & T (\mathbf{e_2}) & \cdots & T (\mathbf{e_n}) \end{bmatrix} .
	\]

\begin{definition}
Let 
	\begin{itemize}
		\item $V$ be a vector space;
		\item $T : V \ra V$ be a linear operator;
		\item $B = \{ \mathbf{b_1} , \mathbf{b_2} , \ldots , \mathbf{b_n} \}$ be a basis.
	\end{itemize}
The \textbf{$\mathbf{B}$-matrix} of $T$ is the matrix representing $T$ on the basis $B$:
	\[
		M_B (T) := \begin{bmatrix} C_B (T (\mathbf{b_1})) & C_B (T (\mathbf{b_2})) & \cdots & C_B (T (\mathbf{b_n})) \end{bmatrix} .
	\]
\end{definition}

\underline{Properties:}
	\begin{enumerate}[label=\Circled{\arabic*}]
		\item $C_B (T (\mathbf{v})) = M_B (T) C_B (\mathbf{v})$ for all $\mathbf{v} \in V$. 
		\item $T$ is an isomorphism if and only if $M_B (T)$ is invertible. More over, $M_B (T^{-1}) = (M_B (T))^{-1}$.
	\end{enumerate}

\newpage 

\section{Change of Basis}

\begin{example}
Consider the square with vertices $(0, 0)$, $(1, 1)$, $(0, 2)$, $(-1, 1)$. Find a basis $D$ on which the coordinates of the vertices become $(0, 0)$, $(1, 0)$, $(1, 1)$, $(0, 1)$.
\end{example}

\begin{solution}

\end{solution}

\vfill 

\underline{Goal:} Given two basis 
	\begin{itemize}
		\begin{multicols}{2}
		\item $B = \{ \mathbf{b_1}, \mathbf{b_2} , \ldots , \mathbf{b_n} \}$;
		\item $D = \{ \mathbf{d_1} , \mathbf{d_2} , \ldots , \mathbf{d_n} \}$;
		\end{multicols}
	\end{itemize}
\vspace*{-10pt}
how do we get $C_D (\mathbf{v})$ from $C_B (\mathbf{v})$?

\newpage 

\setcounter{example}{0}

\begin{example}[Continued]

\end{example}

\vfill 

\begin{definition}
We define the \textbf{change matrix} from $B$ to $D$ as
	\[
		P_{D \leftarrow B} = \begin{bmatrix} C_D (\mathbf{b_1}) & C_D (\mathbf{b_2}) & \cdots & C_D (\mathbf{b_n}) \end{bmatrix} .
	\]
\end{definition}

\underline{Properties:}
	\begin{enumerate}[label=\Circled{\arabic*}]
		\item For any vector $\mathbf{v} \in V$, we have $C_D (\mathbf{v}) = P_{D \leftarrow B} C_B (\mathbf{v})$.
		\item $P_{B \leftarrow B} = I_n$.
		\item $P_{D \leftarrow B}$ is invertible and $(P_{D \leftarrow B})^{-1} = P_{B \leftarrow D}$.
	\end{enumerate}

\newpage 


\begin{example}
Let $V = \mathbb{R}^2$ and $B = \{ (1, 2), (0, 1) \}$, $D = \{ (1, 1) , (-1, 1) \}$. 
	\begin{enumerate}[label=\alph*)]
	\item Find $P_{D \leftarrow B}$.
	\item Verify that $C_D (\mathbf{x}) = P_{D \leftarrow B} C_B (\mathbf{x})$. 
	\item Find $P_{B \leftarrow D}$, verify that $C_B (\mathbf{x}) = P_{B \leftarrow D} C_D (\mathbf{x})$.
	\end{enumerate}
\end{example}

\begin{solution}

\end{solution}

\newpage 

\phantom{2}

\newpage 

\section{Diagonalisation and Change of Basis}

\begin{example}
Let $A = \begin{bmatrix} 11 & -6 \\ 12 & -6 \end{bmatrix}$, $P = \begin{bmatrix} 2 & 3 \\ 3 & 4 \end{bmatrix}$, and $D = \begin{bmatrix} 2 & 0 \\ 0 & 3 \end{bmatrix}$.
	\begin{enumerate}[label=\alph*)]
		\item Verify that $P^{-1} A P = D$.
		\item Find a basis $B$ such that $M_B (T_A) = D$.
	\end{enumerate}
\end{example}

\newpage 

\phantom{2}

\vfill 

\begin{theorem}
	\begin{enumerate}[label=\Circled{\arabic*}]
		\item Let $A$ be an $n \times n$ matrix and $E$ be standard basis of $\mathbb{R}^n$. 
		\item Let $B$ be a basis of $\mathbb{R}^n$.
		\item Let $P$ be the invertible matrix whose columns are the vectors in $B$ in order.
	\end{enumerate}
Then
	\[
		M_B (T_A) = P^{-1} A P .
	\]
\end{theorem}

\end{document}