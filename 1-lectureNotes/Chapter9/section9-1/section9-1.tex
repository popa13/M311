\documentclass[20pt,a4paper]{extarticle}
\usepackage[utf8]{inputenc}
\usepackage[english]{babel}

\usepackage{amsmath}
\usepackage{amsfonts}
\usepackage{amssymb}
\usepackage{mathtools}
\usepackage{systeme}
\sysdelim..

\usepackage{graphicx}
\usepackage{caption}
\usepackage{subcaption}
\usepackage{lmodern}
\usepackage{tikz}
\usetikzlibrary{calc}
\usepackage{titlesec}
\usepackage{environ}
\usepackage{xcolor}
\usepackage{fancyhdr}
\usepackage[colorlinks = true, linkcolor = black]{hyperref}
\usepackage{xparse}
\usepackage{enumitem}
\usepackage{comment}
\usepackage{wrapfig}
\usepackage{soul}
\usepackage[capitalise]{cleveref}
\usepackage{circledsteps}

\usepackage[left=1cm,right=1cm,top=1cm,bottom=3cm]{geometry}
\usepackage{multicol}
\usepackage[indent=0pt]{parskip}

\newcommand{\spaceP}{\vspace*{0.5cm}}
\newcommand{\range}{\mathrm{range}\,}
\newcommand{\ra}{\rightarrow}
\newcommand{\curl}{\mathrm{curl} \,}
\newcommand{\hint}[1]{\scalebox{2}{$\displaystyle\int_{\scalebox{0.35}{$#1$}}$}\,}
\newcommand{\hiint}[1]{\scalebox{2}{$\displaystyle\iint_{\scalebox{0.35}{$#1$}}$}\,}
\newcommand{\hiiint}[1]{\scalebox{2}{$\displaystyle\iiint_{\scalebox{0.35}{$#1$}}$}\,}
\renewcommand{\div}{\mathrm{div}\,}

\DeclareMathOperator{\Span}{span}

\makeatletter
\renewcommand*\env@matrix[1][*\c@MaxMatrixCols c]{%
  \hskip -\arraycolsep
  \let\@ifnextchar\new@ifnextchar
  \array{#1}}
\makeatother

%% Redefining sections
\newcommand{\sectionformat}[1]{%
    \begin{tikzpicture}[baseline=(title.base)]
        \node[rectangle, draw] (title) {#1};
    \end{tikzpicture}
    
    \noindent\hrulefill
}

\newif\ifhNotes 

\hNotesfalse

\ifhNotes
	\newcommand{\hideNotes}[1]{%
	\phantom{#1}
	}
	\newcommand{\hideNotesU}[1]{%
	\underline{\hspace{1mm}\phantom{#1}\hspace{1mm}}
	}
\else
	\newcommand{\hideNotes}[1]{#1}
	\newcommand{\hideNotesU}[1]{\textcolor{blue}{#1}}
\fi

% default values copied from titlesec documentation page 23
% parameters of \titleformat command are explained on page 4
\titleformat%
    {\section}% <command> is the sectioning command to be redefined, i. e., \part, \chapter, \section, \subsection, \subsubsection, \paragraph or \subparagraph.
    {\normalfont\large\scshape}% <format>
    {}% <label> the number
    {0em}% <sep> length. horizontal separation between label and title body
    {\centering\sectionformat}% code preceding the title body  (title body is taken as argument)

%% Set counters for sections to none
\setcounter{secnumdepth}{0}

%% Set the footer/headers
\pagestyle{fancy}
\fancyhf{}
\renewcommand{\headrulewidth}{0pt}
\renewcommand{\footrulewidth}{2pt}
\lfoot{P.-O. Paris{\'e}}
\cfoot{MATH 311}
\rfoot{Page \thepage}

%% Defining example environment
\newcounter{example}
\NewEnviron{example}%
	{%
	\noindent\refstepcounter{example}\fcolorbox{gray!40}{gray!40}{\textsc{\textcolor{red}{Example~\theexample.}}}%
	%\fcolorbox{black}{white}%
		{  %\parbox{0.95\textwidth}%
			{
			\BODY
			}%
		}%
	}

\newcounter{theorem}
\NewEnviron{theorem}%
	{%
	\noindent\refstepcounter{theorem}\fcolorbox{gray!40}{gray!40}{\textsc{\textcolor{black}{Theorem~\thetheorem.}}}%
	%\fcolorbox{black}{white}%
		{  %\parbox{0.95\textwidth}%
			{
			\BODY
			}%
		}%
	}

\newcounter{definition}
\NewEnviron{definition}%
	{%
	\noindent\refstepcounter{definition}\fcolorbox{gray!40}{gray!40}{\textsc{\textcolor{black}{Definition~\thedefinition.}}}%
	%\fcolorbox{black}{white}%
		{  %\parbox{0.95\textwidth}%
			{
			\BODY
			}%
		}%
	}

\newcounter{algo}
\NewEnviron{algorithm}
	{%
	\noindent\refstepcounter{algo}\fcolorbox{gray!40}{gray!40}{\textsc{\textcolor{black}{Algorithm~\thealgo.}}}%
	%\fcolorbox{black}{white}%
		{  %\parbox{0.95\textwidth}%
			{
			\BODY
			}%
		}%
	}

\NewEnviron{goal}
	{%
	\noindent\fcolorbox{gray!40}{gray!40}{\textsc{\textcolor{black}{Goal:}}}%
	%\fcolorbox{black}{white}%
		{  %\parbox{0.95\textwidth}%
			{
			\BODY
			}%
		}%
	}

\NewEnviron{solution}%
	{%
	\noindent \fcolorbox{gray!40}{gray!40}{\textsc{\textcolor{blue}{Solution.}}}%
	%\fcolorbox{black}{white}%
		{  %\parbox{0.95\textwidth}%
			{
			%\textcolor{blue}
			}%
		}%
	}

\NewEnviron{proof}%
	{%
	\noindent \fcolorbox{gray!40}{gray!40}{\textsc{\textcolor{blue}{Proof.}}}%
	%\fcolorbox{black}{white}%
		{  %\parbox{0.95\textwidth}%
			{
			\textcolor{blue}{%
			\BODY
			}
			}%
		}%
	}
%%% Ignorer les notes
%\excludecomment{notes}

%%%%
\begin{document}
\thispagestyle{empty}

\begin{center}
\vspace*{0.75cm}

{\Huge \textsc{Math 311}}

\vspace*{1.5cm}

{\LARGE \textsc{Chapter 9}} 

\vspace*{0.75cm}

\noindent\textsc{Section 9.1: The Matrix of a Linear Transformation}

\vspace*{0.25cm}

\tableofcontents

\vfill

\noindent \textsc{Created by: Pierre-Olivier Paris{\'e}} \\
\textsc{Spring 2024}
\end{center}

\newpage

\section{Coordinate Vector}

	Let $V$ be a vector space with $\dim V = n$ and $\mathbf{v} \in V$. 

	Given a basis $B = \{ \mathbf{b_1} , \mathbf{b_2} , \ldots , \mathbf{v_n} \}$ of $V$, recall that $C_B : V \ra \mathbb{R}^n$ is given by
		\[
			C_B (\mathbf{v}) = \begin{bmatrix} v_1 \\ v_2 \\ \vdots \\ v_n \end{bmatrix} .
		\]

	\begin{example}
	Let $\mathbf{x} = (2, 1, 3)$ and 
		$$
			B = \{ (1, 0, 1) , (1, 1, 0) , (0, 1, 1) \} 
		$$
	be a basis of $\mathbb{R}^3$. Find $C_B (\mathbf{x})$.
	\end{example}

	\begin{solution}

	\end{solution}

	\newpage 

\section{Matrix of a Linear transformation}

	Suppose we have the transformation
		\[
			T \begin{bmatrix} x \\ y \\ z \end{bmatrix} =  \begin{bmatrix} x + z \\ 2z \\ y - z\\  x + 2y \end{bmatrix}.
		\]
	Notice that, if we apply $T$ to the standard basis of $\mathbb{R}^3$, we get
		\[
			T \begin{bmatrix} 1 \\ 0 \\ 0 \end{bmatrix} = \begin{bmatrix} 1 \\ 0 \\ 0 \\ 1 \end{bmatrix} = \mathbf{a_1} , \quad T \begin{bmatrix} 0 \\ 1 \\ 0 \end{bmatrix} = \begin{bmatrix} 0 \\ 0 \\ 1 \\ 2 \end{bmatrix} = \mathbf{a_2} , \quad T \begin{bmatrix} 0 \\ 0 \\ 1 \end{bmatrix} = \begin{bmatrix} 1 \\ 2 \\ -1 \\ 0 \end{bmatrix} = \mathbf{a_3} .
		\]
	Then, setting
		\[
			 A = \begin{bmatrix} \mathbf{a_1} & \mathbf{a_2} & \mathbf{a_3} \end{bmatrix} = \begin{bmatrix} 1 & 0 & 1 \\ 0 & 0 & 2 \\ 0 & 1 & -1 \\ 1 & 2 & 0 \end{bmatrix} \quad \Rightarrow \quad T (\mathbf{x}) = \begin{bmatrix} 1 & 0 & 1 \\ 0 & 0 & 2 \\ 0 & 1 & -1 \\ 1 & 2 & 0 \end{bmatrix} \begin{bmatrix} x \\ y \\ z \end{bmatrix}
		\]

	The matrix $A$ is called the \textbf{matrix representation of the linear transformation} in term of the standard basis of $\mathbb{R}^3$ and $\mathbb{R}^4$. 

	What if we change basis?

		
	\newpage 

	\begin{example}
	Let $T : \mathbb{R}^3 \ra \mathbb{R}^4$ be the linear transformation defined by
		\[
			T \begin{bmatrix} x \\ y \\ z \end{bmatrix} = \begin{bmatrix} x + z \\ 2z \\ y - z\\  x + 2y \end{bmatrix}.
		\]
	We assume we have two basis:
		\begin{itemize}[label=\textbullet]
			\item a basis $B = \{ \begin{bmatrix} 1 & 0 & 1 \end{bmatrix}^\top , \begin{bmatrix} 1 & 1 & 0 \end{bmatrix}^\top , \begin{bmatrix} 0 & 1 & 1 \end{bmatrix}^\top \}$ of $\mathbb{R}^3$.
			\item a basis $D = {\small \{ \begin{bmatrix} 1 & 0 & 1 & 0 \end{bmatrix}^\top , \begin{bmatrix} 0 & 1 & 0 & 1 \end{bmatrix}^\top , \begin{bmatrix} 1 & 1 & 0 & 0 \end{bmatrix}^\top, \begin{bmatrix} 1 & 0 & 0& 1 \end{bmatrix}^\top }\}$ of $\mathbb{R}^4$.
		\end{itemize}
	Find a matrix representing $T$ on these basis. 
	\end{example}

	\begin{solution}

	\end{solution}

	\newpage 

	\phantom{2}

	\newpage 

	\phantom{2}

	\newpage 

	\subsection{General Procedure}

	To find the \textbf{matrix representation} of $T : \mathbb{R}^n \ra \mathbb{R}^m$ on a basis $B$ of $\mathbb{R}^n$ and on a basis $D$ of $\mathbb{R}^m$, we follow these steps:
		\begin{enumerate}[label=\Circled{\arabic*}]
			\item Evaluate $\mathbf{t_1} = T(\mathbf{b_1})$, $\mathbf{t_2} = T (\mathbf{b_2})$, $\ldots$, $\mathbf{t_n} = T (\mathbf{b_n})$.
			\item Find $C_D (\mathbf{t_1})$, $C_D (\mathbf{t_2})$, $\ldots$, $C_D (\mathbf{t_n})$.
			\item Set the $m \times n$ matrix
				\[
					A = \begin{bmatrix} C_D (\mathbf{t_1}) & C_D (\mathbf{t_2}) & \cdots & C_D (\mathbf{t_n}) \end{bmatrix} .
				\]
			\item Then we have, for any $\mathbf{x} \in \mathbb{R}^n$,
				\[
					C_D T (\mathbf{x}) = T_A C_B (\mathbf{x}) = A C_B (\mathbf{x}) .
				\]
		\end{enumerate}






\end{document}