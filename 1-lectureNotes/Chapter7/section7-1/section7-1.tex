\documentclass[20pt,a4paper]{extarticle}
\usepackage[utf8]{inputenc}
\usepackage[english]{babel}

\usepackage{amsmath}
\usepackage{amsfonts}
\usepackage{amssymb}
\usepackage{mathtools}
\usepackage{systeme}
\sysdelim..

\usepackage{graphicx}
\usepackage{caption}
\usepackage{subcaption}
\usepackage{lmodern}
\usepackage{tikz}
\usetikzlibrary{calc}
\usepackage{titlesec}
\usepackage{environ}
\usepackage{xcolor}
\usepackage{fancyhdr}
\usepackage[colorlinks = true, linkcolor = black]{hyperref}
\usepackage{xparse}
\usepackage{enumitem}
\usepackage{comment}
\usepackage{wrapfig}
\usepackage{soul}
\usepackage[capitalise]{cleveref}
\usepackage{circledsteps}

\usepackage[left=1cm,right=1cm,top=1cm,bottom=3cm]{geometry}
\usepackage{multicol}
\usepackage[indent=0pt]{parskip}

\newcommand{\spaceP}{\vspace*{0.5cm}}
\newcommand{\range}{\mathrm{range}\,}
\newcommand{\ra}{\rightarrow}
\newcommand{\curl}{\mathrm{curl} \,}
\newcommand{\hint}[1]{\scalebox{2}{$\displaystyle\int_{\scalebox{0.35}{$#1$}}$}\,}
\newcommand{\hiint}[1]{\scalebox{2}{$\displaystyle\iint_{\scalebox{0.35}{$#1$}}$}\,}
\newcommand{\hiiint}[1]{\scalebox{2}{$\displaystyle\iiint_{\scalebox{0.35}{$#1$}}$}\,}
\renewcommand{\div}{\mathrm{div}\,}

\DeclareMathOperator{\Span}{span}

\makeatletter
\renewcommand*\env@matrix[1][*\c@MaxMatrixCols c]{%
  \hskip -\arraycolsep
  \let\@ifnextchar\new@ifnextchar
  \array{#1}}
\makeatother

%% Redefining sections
\newcommand{\sectionformat}[1]{%
    \begin{tikzpicture}[baseline=(title.base)]
        \node[rectangle, draw] (title) {#1};
    \end{tikzpicture}
    
    \noindent\hrulefill
}

\newif\ifhNotes 

\hNotesfalse

\ifhNotes
	\newcommand{\hideNotes}[1]{%
	\phantom{#1}
	}
	\newcommand{\hideNotesU}[1]{%
	\underline{\hspace{1mm}\phantom{#1}\hspace{1mm}}
	}
\else
	\newcommand{\hideNotes}[1]{#1}
	\newcommand{\hideNotesU}[1]{\textcolor{blue}{#1}}
\fi

% default values copied from titlesec documentation page 23
% parameters of \titleformat command are explained on page 4
\titleformat%
    {\section}% <command> is the sectioning command to be redefined, i. e., \part, \chapter, \section, \subsection, \subsubsection, \paragraph or \subparagraph.
    {\normalfont\large\scshape}% <format>
    {}% <label> the number
    {0em}% <sep> length. horizontal separation between label and title body
    {\centering\sectionformat}% code preceding the title body  (title body is taken as argument)

%% Set counters for sections to none
\setcounter{secnumdepth}{0}

%% Set the footer/headers
\pagestyle{fancy}
\fancyhf{}
\renewcommand{\headrulewidth}{0pt}
\renewcommand{\footrulewidth}{2pt}
\lfoot{P.-O. Paris{\'e}}
\cfoot{MATH 311}
\rfoot{Page \thepage}

%% Defining example environment
\newcounter{example}
\NewEnviron{example}%
	{%
	\noindent\refstepcounter{example}\fcolorbox{gray!40}{gray!40}{\textsc{\textcolor{red}{Example~\theexample.}}}%
	%\fcolorbox{black}{white}%
		{  %\parbox{0.95\textwidth}%
			{
			\BODY
			}%
		}%
	}

\newcounter{theorem}
\NewEnviron{theorem}%
	{%
	\noindent\refstepcounter{theorem}\fcolorbox{gray!40}{gray!40}{\textsc{\textcolor{black}{Theorem~\thetheorem.}}}%
	%\fcolorbox{black}{white}%
		{  %\parbox{0.95\textwidth}%
			{
			\BODY
			}%
		}%
	}

\newcounter{definition}
\NewEnviron{definition}%
	{%
	\noindent\refstepcounter{definition}\fcolorbox{gray!40}{gray!40}{\textsc{\textcolor{black}{Definition~\thedefinition.}}}%
	%\fcolorbox{black}{white}%
		{  %\parbox{0.95\textwidth}%
			{
			\BODY
			}%
		}%
	}

\newcounter{algo}
\NewEnviron{algorithm}
	{%
	\noindent\refstepcounter{algo}\fcolorbox{gray!40}{gray!40}{\textsc{\textcolor{black}{Algorithm~\thealgo.}}}%
	%\fcolorbox{black}{white}%
		{  %\parbox{0.95\textwidth}%
			{
			\BODY
			}%
		}%
	}

\NewEnviron{goal}
	{%
	\noindent\fcolorbox{gray!40}{gray!40}{\textsc{\textcolor{black}{Goal:}}}%
	%\fcolorbox{black}{white}%
		{  %\parbox{0.95\textwidth}%
			{
			\BODY
			}%
		}%
	}

\NewEnviron{solution}%
	{%
	\noindent \fcolorbox{gray!40}{gray!40}{\textsc{\textcolor{blue}{Solution.}}}%
	%\fcolorbox{black}{white}%
		{  %\parbox{0.95\textwidth}%
			{
			%\textcolor{blue}
			}%
		}%
	}

\NewEnviron{proof}%
	{%
	\noindent \fcolorbox{gray!40}{gray!40}{\textsc{\textcolor{blue}{Proof.}}}%
	%\fcolorbox{black}{white}%
		{  %\parbox{0.95\textwidth}%
			{
			\textcolor{blue}{%
			\BODY
			}
			}%
		}%
	}
%%% Ignorer les notes
%\excludecomment{notes}

%%%%
\begin{document}
\thispagestyle{empty}

\begin{center}
\vspace*{0.75cm}

{\Huge \textsc{Math 311}}

\vspace*{1.5cm}

{\LARGE \textsc{Chapter 7}} 

\vspace*{0.75cm}

\noindent\textsc{Section 7.1: Linear Transformations}

\vspace*{0.25cm}

\tableofcontents

\vfill

\noindent \textsc{Created by: Pierre-Olivier Paris{\'e}} \\
\textsc{Spring 2024}
\end{center}

\newpage

\section{Linear Transformations}

Given an $m \times n$ matrix $A$, we introduced the transformation $T_A : \mathbb{R}^n \ra \mathbb{R}^m$ defined by
	\[
		T_A (\mathbf{x}) = A \mathbf{x} \quad (\mathbf{x} \in \mathbb{R}^n ) .
	\]
From the properties of matrix multiplication, we have
	\begin{enumerate}
		\item[(T1)] $T_A (\mathbf{x} + \mathbf{y}) = A (\mathbf{x} + \mathbf{y}) = A \mathbf{x} + A\mathbf{y} = T_A (\mathbf{x}) + T_A (\mathbf{y})$.
		\item[(T2)] $T_A (a \mathbf{x}) = A (a \mathbf{x}) = a (A \mathbf{x}) = a T_A (\mathbf{x})$. 
	\end{enumerate}
The transformations satisfying (T1) and (T2) are very special and play an important role in linear algebra.


\begin{definition}
Let $V$ and $W$ be two vector spaces. A transformation $T : V \ra W$ is called a \textbf{linear transformation} if it satisfies the following two conditions for any vectors $\mathbf{v_1}$ and $\mathbf{v_2}$ in $V$ and any scalars $a$:
	\begin{enumerate}
		\item[(T1)] $T (\mathbf{v_1} + \mathbf{v_2}) = T (\mathbf{v_1}) + T (\mathbf{v_2})$.
		\item[(T2)] $T (a \mathbf{v_1}) = a T (\mathbf{v_1})$. 
	\end{enumerate}
\end{definition}

\textbf{Notations:}
	\begin{enumerate}[label=\Circled{\arabic*}]
		\item The \textbf{identity transformation} is the transformation $1_V : V \ra V$ given by $1_V (\mathbf{v}) = \mathbf{v}$, for any $\mathbf{v} \in V$.
		\item The \textbf{zero transformation} is the transformation $0 : V \ra W$ given by $0 (\mathbf{v}) = \mathbf{0}$, for any $\mathbf{v} \in V$.
	\end{enumerate}

\newpage 


\begin{example}
Show that the following transformation is a linear transformation.
	\[
		D : \mathbf{P}_n \ra \mathbf{P_{n-1}}, \qquad D (p (x)) = p' (x) .
	\]
\end{example}

\begin{solution}

\end{solution}

\newpage 

\section{Properties}

\begin{example}
Let $T : \mathbb{R}^2 \ra \mathbb{R}^2$ be a linear transformation such that
	\[
		T  \begin{bmatrix} 1 \\ 1 \end{bmatrix}  = \begin{bmatrix} 2 \\ -3 \end{bmatrix} \quad \text{ and } \quad T  \begin{bmatrix} 1 \\ -2 \end{bmatrix}  = \begin{bmatrix} 5 \\ 1 \end{bmatrix} .
	\]
Find $T \begin{bmatrix} 4 \\ 3 \end{bmatrix}$.
\end{example}

\begin{solution}

\end{solution}

\newpage 

\phantom{2}

\vfill

\begin{theorem}
If $T : V \ra W$ is a linear transformation and $\mathbf{v_1}, \mathbf{v_2}, \ldots , \mathbf{v_k} \in V$ and $v_1, v_2, \ldots , v_k \in \mathbb{R}$, then
	\[
		T (v_1 \mathbf{v_1} + v_2 \mathbf{v_2} + \cdots + v_k \mathbf{v_k} ) = v_1 T (\mathbf{v_1}) + v_2 T (\mathbf{v_2}) + \cdots + v_k T (\mathbf{v_k}) .
	\]
\end{theorem}

\newpage 

\begin{example}
Find the expression of the linear transformation $T : \mathbb{R}^2 \ra \mathbb{R}^3$ such that
	\[
		T \begin{bmatrix} 1 \\ 2 \end{bmatrix} = \begin{bmatrix} 1 \\ 0 \\ 1 \end{bmatrix}, \quad \text{ and } \quad T \begin{bmatrix} -1 \\ 0 \end{bmatrix} = \begin{bmatrix} 0 \\ 1 \\ 1 \end{bmatrix} .
	\]
\end{example}

\begin{solution}

\end{solution}

\newpage 

\phantom{2}

\vfill

\begin{theorem}
Let 
\begin{enumerate}[label=\Circled{\arabic*}]
\item $V$ and $W$ be vector spaces
\item $\{ \mathbf{e_1} , \mathbf{e_2} , \ldots , \mathbf{e_n} \}$ be a basis for $V$. 
\item $\mathbf{w_1}$, $\mathbf{w_2}$, $\ldots$, $\mathbf{w_n}$ be vectors in $W$
\end{enumerate}
Then there exists a unique linear transformation $T : V \ra W$ satisfying $T (\mathbf{e_i}) = \mathbf{w_i}$, for any $i = 1, 2, \ldots , n$. In particular, the action of $T$ on a given $\mathbf{v} = v_1 \mathbf{e_1} + v_2 \mathbf{e_2} + \cdots + v_n \mathbf{v_n}$ is
	\[
		T (\mathbf{v}) = v_1 \mathbf{w_1} + v_2 \mathbf{w_2} + \cdots + v_n \mathbf{w_n} .
	\]
\end{theorem}


\end{document}