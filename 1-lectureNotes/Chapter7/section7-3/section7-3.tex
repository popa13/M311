\documentclass[20pt,a4paper]{extarticle}
\usepackage[utf8]{inputenc}
\usepackage[english]{babel}

\usepackage{amsmath}
\usepackage{amsfonts}
\usepackage{amssymb}
\usepackage{mathtools}
\usepackage{systeme}
\sysdelim..

\usepackage{graphicx}
\usepackage{caption}
\usepackage{subcaption}
\usepackage{lmodern}
\usepackage{tikz}
\usetikzlibrary{calc}
\usepackage{titlesec}
\usepackage{environ}
\usepackage{xcolor}
\usepackage{fancyhdr}
\usepackage[colorlinks = true, linkcolor = black]{hyperref}
\usepackage{xparse}
\usepackage{enumitem}
\usepackage{comment}
\usepackage{wrapfig}
\usepackage{soul}
\usepackage[capitalise]{cleveref}
\usepackage{circledsteps}

\usepackage[left=1cm,right=1cm,top=1cm,bottom=3cm]{geometry}
\usepackage{multicol}
\usepackage[indent=0pt]{parskip}

\newcommand{\spaceP}{\vspace*{0.5cm}}
\newcommand{\range}{\mathrm{range}\,}
\newcommand{\ra}{\rightarrow}
\newcommand{\curl}{\mathrm{curl} \,}
\newcommand{\hint}[1]{\scalebox{2}{$\displaystyle\int_{\scalebox{0.35}{$#1$}}$}\,}
\newcommand{\hiint}[1]{\scalebox{2}{$\displaystyle\iint_{\scalebox{0.35}{$#1$}}$}\,}
\newcommand{\hiiint}[1]{\scalebox{2}{$\displaystyle\iiint_{\scalebox{0.35}{$#1$}}$}\,}
\renewcommand{\div}{\mathrm{div}\,}

\DeclareMathOperator{\Span}{span}

\makeatletter
\renewcommand*\env@matrix[1][*\c@MaxMatrixCols c]{%
  \hskip -\arraycolsep
  \let\@ifnextchar\new@ifnextchar
  \array{#1}}
\makeatother

%% Redefining sections
\newcommand{\sectionformat}[1]{%
    \begin{tikzpicture}[baseline=(title.base)]
        \node[rectangle, draw] (title) {#1};
    \end{tikzpicture}
    
    \noindent\hrulefill
}

\newif\ifhNotes 

\hNotesfalse

\ifhNotes
	\newcommand{\hideNotes}[1]{%
	\phantom{#1}
	}
	\newcommand{\hideNotesU}[1]{%
	\underline{\hspace{1mm}\phantom{#1}\hspace{1mm}}
	}
\else
	\newcommand{\hideNotes}[1]{#1}
	\newcommand{\hideNotesU}[1]{\textcolor{blue}{#1}}
\fi

% default values copied from titlesec documentation page 23
% parameters of \titleformat command are explained on page 4
\titleformat%
    {\section}% <command> is the sectioning command to be redefined, i. e., \part, \chapter, \section, \subsection, \subsubsection, \paragraph or \subparagraph.
    {\normalfont\large\scshape}% <format>
    {}% <label> the number
    {0em}% <sep> length. horizontal separation between label and title body
    {\centering\sectionformat}% code preceding the title body  (title body is taken as argument)

%% Set counters for sections to none
\setcounter{secnumdepth}{0}

%% Set the footer/headers
\pagestyle{fancy}
\fancyhf{}
\renewcommand{\headrulewidth}{0pt}
\renewcommand{\footrulewidth}{2pt}
\lfoot{P.-O. Paris{\'e}}
\cfoot{MATH 311}
\rfoot{Page \thepage}

%% Defining example environment
\newcounter{example}
\NewEnviron{example}%
	{%
	\noindent\refstepcounter{example}\fcolorbox{gray!40}{gray!40}{\textsc{\textcolor{red}{Example~\theexample.}}}%
	%\fcolorbox{black}{white}%
		{  %\parbox{0.95\textwidth}%
			{
			\BODY
			}%
		}%
	}

\newcounter{theorem}
\NewEnviron{theorem}%
	{%
	\noindent\refstepcounter{theorem}\fcolorbox{gray!40}{gray!40}{\textsc{\textcolor{black}{Theorem~\thetheorem.}}}%
	%\fcolorbox{black}{white}%
		{  %\parbox{0.95\textwidth}%
			{
			\BODY
			}%
		}%
	}

\newcounter{definition}
\NewEnviron{definition}%
	{%
	\noindent\refstepcounter{definition}\fcolorbox{gray!40}{gray!40}{\textsc{\textcolor{black}{Definition~\thedefinition.}}}%
	%\fcolorbox{black}{white}%
		{  %\parbox{0.95\textwidth}%
			{
			\BODY
			}%
		}%
	}

\newcounter{algo}
\NewEnviron{algorithm}
	{%
	\noindent\refstepcounter{algo}\fcolorbox{gray!40}{gray!40}{\textsc{\textcolor{black}{Algorithm~\thealgo.}}}%
	%\fcolorbox{black}{white}%
		{  %\parbox{0.95\textwidth}%
			{
			\BODY
			}%
		}%
	}

\NewEnviron{goal}
	{%
	\noindent\fcolorbox{gray!40}{gray!40}{\textsc{\textcolor{black}{Goal:}}}%
	%\fcolorbox{black}{white}%
		{  %\parbox{0.95\textwidth}%
			{
			\BODY
			}%
		}%
	}

\NewEnviron{solution}%
	{%
	\noindent \fcolorbox{gray!40}{gray!40}{\textsc{\textcolor{blue}{Solution.}}}%
	%\fcolorbox{black}{white}%
		{  %\parbox{0.95\textwidth}%
			{
			%\textcolor{blue}
			}%
		}%
	}

\NewEnviron{proof}%
	{%
	\noindent \fcolorbox{gray!40}{gray!40}{\textsc{\textcolor{blue}{Proof.}}}%
	%\fcolorbox{black}{white}%
		{  %\parbox{0.95\textwidth}%
			{
			\textcolor{blue}{%
			\BODY
			}
			}%
		}%
	}
%%% Ignorer les notes
%\excludecomment{notes}

%%%%
\begin{document}
\thispagestyle{empty}

\begin{center}
\vspace*{0.75cm}

{\Huge \textsc{Math 311}}

\vspace*{1.5cm}

{\LARGE \textsc{Chapter 7}} 

\vspace*{0.75cm}

\noindent\textsc{Section 7.3: Coordinates Isomorphism and Composition}

\vspace*{0.25cm}

\tableofcontents

\vfill

\noindent \textsc{Created by: Pierre-Olivier Paris{\'e}} \\
\textsc{Spring 2024}
\end{center}

\newpage

\section{Coordinates of a Vector}

Assume that $V$ is a vector space with $n = \dim V < \infty$.

Let 
	\begin{enumerate}[label=\Circled{\arabic*}]
	\item $B = \{ \mathbf{b_1} , \mathbf{b_2} , \ldots , \mathbf{b_n} \}$ be a basis of $V$
	\item $E = \{ \mathbf{e_1} , \mathbf{e_2} , \ldots , \mathbf{e_n} \}$ be the standard basis for $\mathbb{R}^n$. 
	\end{enumerate}

Given any $\mathbf{v} \in V$ with $\mathbf{v} = v_1 \mathbf{v_1} + v_2 \mathbf{v_2} + \cdots + v_n \mathbf{v_n}$, define the linear transformation $C_B : V \ra \mathbb{R}^n$ as
	\[
		C_B (\mathbf{v}) = v_1 \mathbf{e_1} + v_2 \mathbf{e_2} + \cdots + v_n \mathbf{e_n} = \begin{bmatrix} v_1 \\ v_2 \\ \vdots \\ v_n \end{bmatrix} .
	\]

\begin{definition}
The linear transformation $C_B$ is called the \textbf{coordinates isomorphism}.
\end{definition}

\textbf{Notes:}
	\begin{enumerate}[label=\Circled{\arabic*}]
		\item The coordinates isomorphism $C_B$ gives a way to regard any vector space $V$ of dimension $n$ as $\mathbb{R}^n$.
		\item In mathematical jargon, we say that $V$ ($\dim V = n$) and $\mathbb{R}^n$ are \textbf{isomorphic}.
		\item More generally, two vector spaces $V$ and $W$ are isomorphic if there is a linear transformation $T : V \ra W$ which is onto and one-to-one.
	\end{enumerate}

\section{Composition}

\begin{definition}
Let $V$, $W$ and $U$ be vector spaces. Let $T : V \ra W$ and $S : W \ra U$ be linear transformations. The \textbf{composite transformation} $ST : V \ra U$ of $T$ and $S$ is defined by
	\[
		ST (\mathbf{v}) = S (T (\mathbf{v})) \quad \mathbf{v} \in V .
	\]
\end{definition}

\textbf{Notes:}
	\begin{enumerate}[label=\Circled{\arabic*}]
		\item $ST$ is a linear transformation.
		\item $TS$ might not be defined unless $U = V$. 
		\item We say that $T : V \ra W$ is an \textbf{isomorphism} if there exists a linear transformation $S : W \ra V$ such that:
		\begin{multicols}{2}
			\begin{itemize}
				\item $ST = 1_V$.
				\item $TS = 1_W$. 
			\end{itemize}
		\end{multicols}
		In this case, $S = T^{-1}$ is called the inverse of $T$.
	\end{enumerate}

\begin{example}
Let $B = \{ 1 , x, x^2 \}$ be a basis for $\mathbf{P_2}$. 
	\begin{enumerate}[label=\alph*)]
		\item Find the coordinate transformation $C_B$.
		\item Find $C_B^{-1}$.
	\end{enumerate}
\end{example}

\begin{solution}

\end{solution}

\newpage 

\phantom{2}

\end{document}