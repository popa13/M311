\documentclass[20pt,a4paper]{extarticle}
\usepackage[utf8]{inputenc}
\usepackage[english]{babel}

\usepackage{amsmath}
\usepackage{amsfonts}
\usepackage{amssymb}
\usepackage{mathtools}
\usepackage{systeme}
\sysdelim..

\usepackage{graphicx}
\usepackage{caption}
\usepackage{subcaption}
\usepackage{lmodern}
\usepackage{tikz}
\usetikzlibrary{calc}
\usepackage{titlesec}
\usepackage{environ}
\usepackage{xcolor}
\usepackage{fancyhdr}
\usepackage[colorlinks = true, linkcolor = black]{hyperref}
\usepackage{xparse}
\usepackage{enumitem}
\usepackage{comment}
\usepackage{wrapfig}
\usepackage{soul}
\usepackage[capitalise]{cleveref}
\usepackage{circledsteps}

\usepackage[left=1cm,right=1cm,top=1cm,bottom=3cm]{geometry}
\usepackage{multicol}
\usepackage[indent=0pt]{parskip}

\newcommand{\spaceP}{\vspace*{0.5cm}}
\newcommand{\range}{\mathrm{range}\,}
\newcommand{\ra}{\rightarrow}
\newcommand{\curl}{\mathrm{curl} \,}
\newcommand{\hint}[1]{\scalebox{2}{$\displaystyle\int_{\scalebox{0.35}{$#1$}}$}\,}
\newcommand{\hiint}[1]{\scalebox{2}{$\displaystyle\iint_{\scalebox{0.35}{$#1$}}$}\,}
\newcommand{\hiiint}[1]{\scalebox{2}{$\displaystyle\iiint_{\scalebox{0.35}{$#1$}}$}\,}
\renewcommand{\div}{\mathrm{div}\,}

\DeclareMathOperator{\Span}{span}

\makeatletter
\renewcommand*\env@matrix[1][*\c@MaxMatrixCols c]{%
  \hskip -\arraycolsep
  \let\@ifnextchar\new@ifnextchar
  \array{#1}}
\makeatother

%% Redefining sections
\newcommand{\sectionformat}[1]{%
    \begin{tikzpicture}[baseline=(title.base)]
        \node[rectangle, draw] (title) {#1};
    \end{tikzpicture}
    
    \noindent\hrulefill
}

\newif\ifhNotes 

\hNotesfalse

\ifhNotes
	\newcommand{\hideNotes}[1]{%
	\phantom{#1}
	}
	\newcommand{\hideNotesU}[1]{%
	\underline{\hspace{1mm}\phantom{#1}\hspace{1mm}}
	}
\else
	\newcommand{\hideNotes}[1]{#1}
	\newcommand{\hideNotesU}[1]{\textcolor{blue}{#1}}
\fi

% default values copied from titlesec documentation page 23
% parameters of \titleformat command are explained on page 4
\titleformat%
    {\section}% <command> is the sectioning command to be redefined, i. e., \part, \chapter, \section, \subsection, \subsubsection, \paragraph or \subparagraph.
    {\normalfont\large\scshape}% <format>
    {}% <label> the number
    {0em}% <sep> length. horizontal separation between label and title body
    {\centering\sectionformat}% code preceding the title body  (title body is taken as argument)

%% Set counters for sections to none
\setcounter{secnumdepth}{0}

%% Set the footer/headers
\pagestyle{fancy}
\fancyhf{}
\renewcommand{\headrulewidth}{0pt}
\renewcommand{\footrulewidth}{2pt}
\lfoot{P.-O. Paris{\'e}}
\cfoot{MATH 311}
\rfoot{Page \thepage}

%% Defining example environment
\newcounter{example}
\NewEnviron{example}%
	{%
	\noindent\refstepcounter{example}\fcolorbox{gray!40}{gray!40}{\textsc{\textcolor{red}{Example~\theexample.}}}%
	%\fcolorbox{black}{white}%
		{  %\parbox{0.95\textwidth}%
			{
			\BODY
			}%
		}%
	}

\newcounter{theorem}
\NewEnviron{theorem}%
	{%
	\noindent\refstepcounter{theorem}\fcolorbox{gray!40}{gray!40}{\textsc{\textcolor{black}{Theorem~\thetheorem.}}}%
	%\fcolorbox{black}{white}%
		{  %\parbox{0.95\textwidth}%
			{
			\BODY
			}%
		}%
	}

\newcounter{definition}
\NewEnviron{definition}%
	{%
	\noindent\refstepcounter{definition}\fcolorbox{gray!40}{gray!40}{\textsc{\textcolor{black}{Definition~\thedefinition.}}}%
	%\fcolorbox{black}{white}%
		{  %\parbox{0.95\textwidth}%
			{
			\BODY
			}%
		}%
	}

\newcounter{algo}
\NewEnviron{algorithm}
	{%
	\noindent\refstepcounter{algo}\fcolorbox{gray!40}{gray!40}{\textsc{\textcolor{black}{Algorithm~\thealgo.}}}%
	%\fcolorbox{black}{white}%
		{  %\parbox{0.95\textwidth}%
			{
			\BODY
			}%
		}%
	}

\NewEnviron{goal}
	{%
	\noindent\fcolorbox{gray!40}{gray!40}{\textsc{\textcolor{black}{Goal:}}}%
	%\fcolorbox{black}{white}%
		{  %\parbox{0.95\textwidth}%
			{
			\BODY
			}%
		}%
	}

\NewEnviron{solution}%
	{%
	\noindent \fcolorbox{gray!40}{gray!40}{\textsc{\textcolor{blue}{Solution.}}}%
	%\fcolorbox{black}{white}%
		{  %\parbox{0.95\textwidth}%
			{
			%\textcolor{blue}
			}%
		}%
	}

\NewEnviron{proof}%
	{%
	\noindent \fcolorbox{gray!40}{gray!40}{\textsc{\textcolor{blue}{Proof.}}}%
	%\fcolorbox{black}{white}%
		{  %\parbox{0.95\textwidth}%
			{
			\textcolor{blue}{%
			\BODY
			}
			}%
		}%
	}
%%% Ignorer les notes
%\excludecomment{notes}

%%%%
\begin{document}
\thispagestyle{empty}

\begin{center}
\vspace*{0.75cm}

{\Huge \textsc{Math 311}}

\vspace*{1.5cm}

{\LARGE \textsc{Chapter 7}} 

\vspace*{0.75cm}

\noindent\textsc{Section 7.2: Kernel and Image of a Linear Transformation}

\vspace*{0.25cm}

\tableofcontents

\vfill

\noindent \textsc{Created by: Pierre-Olivier Paris{\'e}} \\
\textsc{Spring 2024}
\end{center}

\newpage

\section{Definitions}

For this section, $T : V \ra W$ is assumed to be a linear transformation.

\begin{definition}
\begin{enumerate}[label=\Circled{\arabic*}]
	\item The \textbf{kernel} of $T$ is the set
		\[
			\ker T := \{ \mathbf{v} \in V \, : \, T (\mathbf{v}) = \mathbf{0} \} .
		\]
	\item The \textbf{image} of $T$ is the set
		\[
			\mathrm{im}\, T := \{ T (\mathbf{v}) \, : \, \mathbf{v} \in V \} .
		\]
\end{enumerate}
\end{definition}

\begin{example}
Let $A$ be an $m \times n$ matrix and consider $T_A (\mathbf{x}) = A \mathbf{x}$, where $\mathbf{x} \in \mathbb{R}^n$. 
	\begin{enumerate}[label=\alph*)]
		\item We have $\ker T_A = \{ \mathbf{x} \in \mathbb{R}^n \, : \, A \mathbf{x} = \mathbf{0} \} = \mathrm{null} \, A$.
		\item We have $\mathrm{im} \, T_A = \{ A \mathbf{x} \, : \, \mathbf{x} \in \mathbb{R}^n \} = \mathrm{im} \, A$.
	\end{enumerate}
\end{example}

\vspace*{16pt}

\begin{theorem}
We have that $\ker T$ is a subspace of $V$ and $\mathrm{im} \, T$ is a subspace of $W$.
\end{theorem}

\textbf{Notes:}
	\begin{enumerate}[label=\Circled{\arabic*}]
		\item We set $\mathrm{nullity}\,T := \dim (\ker T)$.
		\item We set $\mathrm{rank}\, T := \dim (\mathrm{im} \, T )$. 
	\end{enumerate}

\newpage 

\begin{example}
Define $T : \mathbf{M_{nn}} \ra \mathbf{M_{nn}}$ by $T (A) := A - A^\top$. Find (a) $\ker T$ (b) $\mathrm{im} \, T$.
\end{example}

\begin{solution}

\end{solution}

\newpage 

\phantom{2}

\newpage 

\section{One-to-One and Onto Transformations}

\begin{definition}
Assume that $T: V \ra W$ is a linear transformation.
	\begin{enumerate}[label=\Circled{\arabic*}] 
		\item $T$ is said to be \textbf{onto} if $\mathrm{im} \, T = W$.
		\item $T$ is said to be \textbf{one-to-one} if $\ker T = \{ \mathbf{0} \}$. 
	\end{enumerate}
\end{definition}

\vspace*{16pt}

\begin{example}
Let $T :  \mathbb{R}^3 \ra \mathbb{R}^4$ defined by $T (x, y, z) = (x, x, y, y)$.
	\begin{enumerate}[label=\alph*)]
		\item Is $T$ onto?
		\item Is $T$ one-to-one? If not, find $\mathrm{nullity} \, T$.
	\end{enumerate}
\end{example}

\newpage 

\phantom{2}

\newpage 

\section{Dimension Theorem}

\begin{theorem}
Let $T : V \ra W$ be any linear transformation with $n = \dim V < \infty$. Then
	\[
		\dim V = \mathrm{nullity}\, T + \mathrm{rank} \, T .
	\]
\end{theorem}

\textbf{Idea of the Proof.} We let
	\begin{itemize}
		\item $r = \mathrm{rank} \, T = \dim (\mathrm{im} \, T)$;
		\item $k = \mathrm{nullity} \, T = \dim (\ker T)$.
	\end{itemize} 

Let $\{ \mathbf{w_1} , \mathbf{w_2} , \ldots , \mathbf{w_r} \}$ be a basis for $\mathrm{im} \, T$. Then there are vectors $\mathbf{e_1}$, $\mathbf{e_2}$, $\ldots$, $\mathbf{e_r}$ such that $T (\mathbf{e_i}) = \mathbf{w_i}$. 

Let $\{ \mathbf{f_1} , \mathbf{f_2} , \ldots , \mathbf{f_k} \}$ be a basis for $\ker T$. 

Then the idea is to show that $\{ \mathbf{e_1} , \mathbf{e_2} , \ldots , \mathbf{e_r} , \mathbf{f_1} , \ldots , \mathbf{f_k} \}$ is a basis for $V$, so that we get
	\[
		n = k + r
	\]
showing the claim. \hfill $\square$



\end{document}