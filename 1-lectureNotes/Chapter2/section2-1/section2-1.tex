\documentclass[20pt,a4paper]{extarticle}
\usepackage[utf8]{inputenc}
\usepackage[english]{babel}

\usepackage{amsmath}
\usepackage{amsfonts}
\usepackage{amssymb}
\usepackage{mathtools}
\usepackage{systeme}
\sysdelim..

\usepackage{graphicx}
\usepackage{caption}
\usepackage{subcaption}
\usepackage{lmodern}
\usepackage{tikz}
\usetikzlibrary{calc}
\usepackage{titlesec}
\usepackage{environ}
\usepackage{xcolor}
\usepackage{fancyhdr}
\usepackage[colorlinks = true, linkcolor = black]{hyperref}
\usepackage{xparse}
\usepackage{enumitem}
\usepackage{comment}
\usepackage{wrapfig}
\usepackage{soul}
\usepackage[capitalise]{cleveref}

\usepackage[left=2cm,right=2cm,top=2cm,bottom=2cm]{geometry}
\usepackage{multicol}
\usepackage[indent=0pt]{parskip}

\newcommand{\spaceP}{\vspace*{0.5cm}}
\newcommand{\Span}{\mathrm{Span}\,}
\newcommand{\range}{\mathrm{range}\,}
\newcommand{\ra}{\rightarrow}
\newcommand{\curl}{\mathrm{curl} \,}
\newcommand{\hint}[1]{\scalebox{2}{$\displaystyle\int_{\scalebox{0.35}{$#1$}}$}\,}
\newcommand{\hiint}[1]{\scalebox{2}{$\displaystyle\iint_{\scalebox{0.35}{$#1$}}$}\,}
\newcommand{\hiiint}[1]{\scalebox{2}{$\displaystyle\iiint_{\scalebox{0.35}{$#1$}}$}\,}
\renewcommand{\div}{\mathrm{div}\,}

\makeatletter
\renewcommand*\env@matrix[1][*\c@MaxMatrixCols c]{%
  \hskip -\arraycolsep
  \let\@ifnextchar\new@ifnextchar
  \array{#1}}
\makeatother

%% Redefining sections
\newcommand{\sectionformat}[1]{%
    \begin{tikzpicture}[baseline=(title.base)]
        \node[rectangle, draw] (title) {#1};
    \end{tikzpicture}
    
    \noindent\hrulefill
}

\newif\ifhNotes 

\hNotesfalse

\ifhNotes
	\newcommand{\hideNotes}[1]{%
	\phantom{#1}
	}
	\newcommand{\hideNotesU}[1]{%
	\underline{\hspace{1mm}\phantom{#1}\hspace{1mm}}
	}
\else
	\newcommand{\hideNotes}[1]{#1}
	\newcommand{\hideNotesU}[1]{\textcolor{blue}{#1}}
\fi

% default values copied from titlesec documentation page 23
% parameters of \titleformat command are explained on page 4
\titleformat%
    {\section}% <command> is the sectioning command to be redefined, i. e., \part, \chapter, \section, \subsection, \subsubsection, \paragraph or \subparagraph.
    {\normalfont\large\scshape}% <format>
    {}% <label> the number
    {0em}% <sep> length. horizontal separation between label and title body
    {\centering\sectionformat}% code preceding the title body  (title body is taken as argument)

%% Set counters for sections to none
\setcounter{secnumdepth}{0}

%% Set the footer/headers
\pagestyle{fancy}
\fancyhf{}
\renewcommand{\headrulewidth}{0pt}
\renewcommand{\footrulewidth}{2pt}
\lfoot{P.-O. Paris{\'e}}
\cfoot{MATH 311}
\rfoot{Page \thepage}

%% Defining example environment
\newcounter{example}[section]
\NewEnviron{example}%
	{%
	\noindent\refstepcounter{example}\fcolorbox{gray!40}{gray!40}{\textsc{\textcolor{red}{Example~\theexample.}}}%
	%\fcolorbox{black}{white}%
		{  %\parbox{0.95\textwidth}%
			{
			\BODY
			}%
		}%
	}

\newcounter{theorem}
\NewEnviron{theorem}%
	{%
	\noindent\refstepcounter{theorem}\fcolorbox{gray!40}{gray!40}{\textsc{\textcolor{black}{Theorem~\thetheorem.}}}%
	%\fcolorbox{black}{white}%
		{  %\parbox{0.95\textwidth}%
			{
			\BODY
			}%
		}%
	}

\newcounter{definition}[section]
\NewEnviron{definition}%
	{%
	\noindent\refstepcounter{definition}\fcolorbox{gray!40}{gray!40}{\textsc{\textcolor{black}{Definition~\thedefinition.}}}%
	%\fcolorbox{black}{white}%
		{  %\parbox{0.95\textwidth}%
			{
			\BODY
			}%
		}%
	}

\NewEnviron{algorithm}
	{%
	\noindent\refstepcounter{definition}\fcolorbox{gray!40}{gray!40}{\textsc{\textcolor{black}{Algorithm~\thedefinition.}}}%
	%\fcolorbox{black}{white}%
		{  %\parbox{0.95\textwidth}%
			{
			\BODY
			}%
		}%
	}

\NewEnviron{solution}%
	{%
	\noindent \fcolorbox{gray!40}{gray!40}{\textsc{\textcolor{blue}{Solution.}}}%
	%\fcolorbox{black}{white}%
		{  %\parbox{0.95\textwidth}%
			{
			%\textcolor{blue}
			}%
		}%
	}

\NewEnviron{proof}%
	{%
	\noindent \fcolorbox{gray!40}{gray!40}{\textsc{\textcolor{blue}{Proof.}}}%
	%\fcolorbox{black}{white}%
		{  %\parbox{0.95\textwidth}%
			{
			\textcolor{blue}{%
			\BODY
			}
			}%
		}%
	}
%%% Ignorer les notes
%\excludecomment{notes}

%%%%
\begin{document}
\thispagestyle{empty}

\begin{center}
\vspace*{2.5cm}

{\Huge \textsc{Math 311}}

\vspace*{1.5cm}

{\LARGE \textsc{Chapter 2}} 

\vspace*{0.75cm}

\noindent\textsc{Section 2.1: Matrix Addition, Scalar Multiplication, and Transposition}

\vspace*{0.75cm}

\tableofcontents

\vfill

\noindent \textsc{Created by: Pierre-Olivier Paris{\'e}} \\
\textsc{Spring 2024}
\end{center}

\newpage

\section{Matrices}

\begin{definition}
\begin{itemize}
	\item A \textbf{matrix} is an array of numbers and the numbers are called \textbf{entries}. 
	\item A matrix has $m$ \textbf{rows} and $n$ \textbf{columns} and is called an $\mathbf{m \times n}$ \textbf{matrix}. The numbers of rows and columns, $m \times n$, are called the \textbf{dimensions} of the matrix.
	\item A \textbf{column matrix}, or $n$-vector or column vector is an $n \times 1$ matrix.
	\item A \textbf{row matrix}, or row vector, is an $1 \times n$ matrix.
	\item The $\mathbf{(i, j)}$\textbf{-entry} of a matrix is the number lying in row $i$ and column $j$.
\end{itemize}
\end{definition}

\begin{example}
Below are matrices. 

(a) Identify the $2 \times 3$ matrix. (b) Identify the $3 \times 2$ matrix. (c) Identify the column vector and indicate its dimensions. (d) Identify the row vector and indicate its dimensions. \\ 
(e) What is the $(1, 2)$-entry of the matrix $B$.
\[
	A = \begin{bmatrix} 1 \\ 3 \\ 4 \end{bmatrix} , \, B = \begin{bmatrix} 1 & 2 & -1 \\ 0 & 5 & 6 \end{bmatrix},  \, C = \begin{bmatrix} 1 & 3 & 4 \end{bmatrix},  \, D = \begin{bmatrix} 1 & 4 \\ 4 & 10 \\ -3 & -1 \end{bmatrix} .
\]
\end{example}

\subsubsection*{Notations and Conventions}

The general notations for an $m \times n$ matrix $A$:
\begin{itemize}
	\item $A = \begin{bmatrix} a_{11} & a_{12} & a_{13} & \cdots & a_{1n} \\ 
	a_{21} & a_{22} & a_{23} & \cdots & a_{2n} \\ 
	a_{31} & a_{32} & a_{33} & \cdots & a_{3n} \\ 
	\vdots & \vdots & \vdots & \ddots & \vdots \\ 
	a_{m1} & a_{m2} & a_{m3} & \cdots & a_{mn} \end{bmatrix}$
	\item For example, a generic $3 \times 5$ matrix:
		\[
			A = \begin{bmatrix} 
			a_{11} & a_{12} & a_{13} & a_{14} & a_{15} \\ 
			a_{21} & a_{22} & a_{23} & a_{24} & a_{25} \\ 
			a_{31} & a_{32} & a_{33} & a_{34} & a_{35} 
			\end{bmatrix}
		\]
	\item A shorcut: $A = [a_{ij}]$.
	\item Another shortcut: $A = [ \mathbf{a}_1 \, \, \mathbf{a}_2 \, \, \cdots \, \, \mathbf{a_n}]$, where $\mathbf{a}_1$, $\mathbf{a}_2$, $\ldots$, $\mathbf{a}_n$ represents the columns of the matrix $A$. 
\end{itemize}

Here are conventions to keep in mind:
\begin{itemize}
	\item If a matrix has size $m \times n$, it has $m$ rows and $n$ columns.
	\item If we speak of the $(i, j)$-entry of a matrix, it lies in row $i$ and column $j$.
	\item If an entry is denoted by $a_{ij}$, the first subscript $i$ refers to the row and the second subscript $j$ to the column in which the number $a_{ij}$ lies.
\end{itemize}

\newpage 

\section{Matrix Equality}

\begin{definition}
Two matrices $A$ and $B$ are \textbf{equal} (denoted as $A = B$) if the following conditions are met:
	\begin{itemize}
		\item They have the same size.
		\item Corresponding entries are equal.
	\end{itemize}
\end{definition}

\begin{example}
Let $a$ be a real number and
	\[
		A = \begin{bmatrix} a^2 & (a - 1)^2 \\ 2 (a - 2) & a^2 - 3a + 2 \end{bmatrix} 
	\] 
and 
	\[
		B = \begin{bmatrix} a^2 & a^2 - 2a + 1 \\ 2a - 4 & (a - 2) (a - 1) \end{bmatrix}.
	\]
and
	\[
		C = \begin{bmatrix} a^2 & (a - 1)^2 \\ 2a - 5 & (a - 2) (a - 1) \end{bmatrix}
	\]
(a) Do we have $A = B$? (b) Do we have $A = C$?
\end{example}

\begin{solution}

\end{solution}

\newpage

\section{Matrix Addition}

\begin{definition}
If $A$ and $B$ are matrices of the same size, their \textbf{sum} $A + B$ is the matrix formed by adding corresponding entries.
\end{definition}

\begin{example}
If $A = \begin{bmatrix} -2 & 3 & 2 \\ 3 & 4 & -1 \end{bmatrix}$ and $B = \begin{bmatrix} 1 & 1 & -1 \\ 2 & 0 & 6 \end{bmatrix}$, then
	\[
		A + B = \begin{bmatrix} -2 + 1 & 3 + 1 & 2 + (-1) \\ 3 + 2 & 4 + 0 & (-1) + 6 \end{bmatrix} = \begin{bmatrix} -1 & 4 & 1 \\ 5 & 4 & 5 \end{bmatrix} .
	\]
\end{example}

\vspace*{12pt}

\begin{example}
Find the values of $a$, $b$, and $c$ if 
	\[
		\begin{bmatrix} a & b & c \end{bmatrix} + \begin{bmatrix} c & a & b \end{bmatrix} = \begin{bmatrix} 3 & 2 & -1 \end{bmatrix}.
	\]
\end{example}

\begin{solution}

\end{solution}

\newpage 

\begin{theorem}
Let $A$, $B$, and $C$ be arbitrary $m \times n$ matrices. Then the following holds:
	\begin{enumerate}
		\item $A + B = B + A$ (commutative law).
		\item $A + (B + C) = (A + B) + C$ (associative law).
	\end{enumerate}
\end{theorem}

\begin{proof}
We will prove property 1. Let $A = [a_{ij}]$ and $B = [b_{ij}]$. Then
	\[
		A + B = [a_{ij} + b_{ij}] = [b_{ij} + a_{ij}] = B + A . \tag*{$\square$}
	\]
\end{proof}

\vspace*{-24pt}
\begin{definition}
Let $A$ and $B$ be two $m \times n$ matrix.
\begin{itemize}
	\item \textbf{Zero matrix}: The $m \times n$ matrix $O$ in which every entry is zero.
	\item \textbf{Negative}: The $m \times n$ matrix $-A$ in which every entry is obtained by multiplying entries of $A$ by $-1$.
	\item \textbf{Difference:} It is defined by $A - B = A + (-B)$.
\end{itemize}
\end{definition}

\begin{example}
Let $A = \begin{bmatrix} 3 & -1 \\ 1 & 2 \end{bmatrix}$, $B = \begin{bmatrix} 1 & -1 \\ -2 & 0 \end{bmatrix}$, $C = \begin{bmatrix} 1 & 0 \\ 3 & 1 \end{bmatrix}$. Compute (a) $-A$; (b) $A + B - C$.
\end{example}

\begin{solution}

\end{solution}

\newpage 

\vspace*{4cm}

\begin{theorem}
For any $m \times n$ matrix $A$:
	\begin{enumerate}
		\item $O + A = A$.
		\item $A + (-A) = O$. 
	\end{enumerate}
\end{theorem}

\begin{example}
Find the entries of the matrix $X$ if it satisfies the following equation:
	$$ 
	\begin{bmatrix} 3 & 2 \\ -1 & 1 \end{bmatrix} + X = \begin{bmatrix} 1 & 0 \\ -1 & 2 \end{bmatrix}.
	$$
\end{example}

\begin{solution}

\end{solution}

\newpage 

\section{Scalar Multiplication}

\begin{definition}
If $k$ is a number and $A$ a matrix, then the \textbf{scalar multiple} $kA$ is the matrix obtained by multiplying each entry of $A$ by $k$.
\end{definition}

\underline{Note:} the number $k$ is called a \textbf{scalar}.

\begin{example}
If $A = \begin{bmatrix} 3 & -1 & 4 \\ 2 & 0 & 6 \end{bmatrix}$ and $B = \begin{bmatrix} 1 & 2 & -1 \\ 0 & 3 & 2 \end{bmatrix}$. Compute $3A - 2B$.
\end{example}

\begin{solution}

\end{solution}

\newpage

\begin{example}
Show that if $kA = O$, then $k = 0$ or $A = O$.
\end{example}

\begin{solution}

\end{solution}

\vfill 

\begin{theorem}
Let $A$ and $B$ be two $m \times n$ matrices and $k, l$ be two numbers. Then
	\begin{enumerate}
		\item $k (A + B) = kA + kB$ (distributive law I).
		\item $(k + l)A = kA + lA$ (distributive law II).
		\item $(kl)A = k (lA)$.
		\item $1A = A$ and $(-1)A = -A$.
	\end{enumerate}
\end{theorem}

\newpage

\section{Transposition}

\begin{definition}
If $A$ is an $m \times n$ matrix, the \textbf{transpose} of $A$, written $A^T$, is the $n \times m$ matrix whose rows are the columns of $A$ in the same order.
\end{definition}

\underline{Note:} Based on the definition, we can write
	\[
		A^\top = [a_{ij}]^{\top} = [a_{ji}] .
	\]

\begin{example}
Find the transpose of each of the following matrices.
	\[
		A = \begin{bmatrix} 1 & 2 \\ 3 & 4 \\ 5 & 6 \end{bmatrix} \quad \text{and} \quad B = \begin{bmatrix} 3 & 1 & -1 \\ 1 & 3 & 2 \\ -1 & 2 & 1 \end{bmatrix} .
	\]
\end{example}

\begin{solution}

\end{solution}

\newpage 

\begin{theorem}
Let $A$ and $B$ denote matrices of the same size, and let $k$ denote a scalar.
\begin{enumerate}
	\item If $A$ is an $m \times n$ matrix, then $A^{\top}$ is an $n \times m$ matrix.
	\item $(A^\top)^\top = A$.
	\item $(k A)^\top = k A^\top$.
	\item $(A + B)^\top = A^\top + B^\top$.
\end{enumerate}
\end{theorem}

\begin{proof}
We will prove property 4. Write $A = [a_{ij}]$ and $B = [b_{ij}]$ and $A + B = [c_{ij}]$ with $c_{ij} = a_{ij} + b_{ij}$. Therefore,
	\[
		(A + B)^\top = [c_{ji}] = [a_{ji} + b_{ji}] = [a_{ji}] + [b_{ji}] = A^\top + B^\top . \tag*{$\square$} 
	\]
\end{proof}

\vspace*{-24pt}

\begin{example}
Find the values of the entries of the matrix $A$ if
	\[
		\Big( A + 3 \begin{bmatrix} 1 & -1 & 0 \\ 1 & 2 & 4 \end{bmatrix} \Big)^{\top} = \begin{bmatrix} 2 & 1 \\ 0 & 5 \\ 3 & 8 \end{bmatrix} .
	\]
\end{example}

\newpage 

\phantom{2}

\vspace*{8cm}

\begin{definition}
A matrix $A$ is symmetric if $A = A^\top$. 
\end{definition}

\begin{example}
Show that if $A$ and $B$ are symmetric $n \times n$ matrices, then $A + B$ is symmetric.
\end{example}

\begin{solution}

\end{solution}

\end{document}