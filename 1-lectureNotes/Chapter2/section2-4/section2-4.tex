\documentclass[20pt,a4paper]{extarticle}
\usepackage[utf8]{inputenc}
\usepackage[english]{babel}

\usepackage{amsmath}
\usepackage{amsfonts}
\usepackage{amssymb}
\usepackage{mathtools}
\usepackage{systeme}
\sysdelim..

\usepackage{graphicx}
\usepackage{caption}
\usepackage{subcaption}
\usepackage{lmodern}
\usepackage{tikz}
\usetikzlibrary{calc}
\usepackage{titlesec}
\usepackage{environ}
\usepackage{xcolor}
\usepackage{fancyhdr}
\usepackage[colorlinks = true, linkcolor = black]{hyperref}
\usepackage{xparse}
\usepackage{enumitem}
\usepackage{comment}
\usepackage{wrapfig}
\usepackage{soul}
\usepackage[capitalise]{cleveref}

\usepackage[left=1cm,right=1cm,top=1cm,bottom=3cm]{geometry}
\usepackage{multicol}
\usepackage[indent=0pt]{parskip}

\newcommand{\spaceP}{\vspace*{0.5cm}}
\newcommand{\Span}{\mathrm{Span}\,}
\newcommand{\range}{\mathrm{range}\,}
\newcommand{\ra}{\rightarrow}
\newcommand{\curl}{\mathrm{curl} \,}
\newcommand{\hint}[1]{\scalebox{2}{$\displaystyle\int_{\scalebox{0.35}{$#1$}}$}\,}
\newcommand{\hiint}[1]{\scalebox{2}{$\displaystyle\iint_{\scalebox{0.35}{$#1$}}$}\,}
\newcommand{\hiiint}[1]{\scalebox{2}{$\displaystyle\iiint_{\scalebox{0.35}{$#1$}}$}\,}
\renewcommand{\div}{\mathrm{div}\,}

\makeatletter
\renewcommand*\env@matrix[1][*\c@MaxMatrixCols c]{%
  \hskip -\arraycolsep
  \let\@ifnextchar\new@ifnextchar
  \array{#1}}
\makeatother

%% Redefining sections
\newcommand{\sectionformat}[1]{%
    \begin{tikzpicture}[baseline=(title.base)]
        \node[rectangle, draw] (title) {#1};
    \end{tikzpicture}
    
    \noindent\hrulefill
}

\newif\ifhNotes 

\hNotesfalse

\ifhNotes
	\newcommand{\hideNotes}[1]{%
	\phantom{#1}
	}
	\newcommand{\hideNotesU}[1]{%
	\underline{\hspace{1mm}\phantom{#1}\hspace{1mm}}
	}
\else
	\newcommand{\hideNotes}[1]{#1}
	\newcommand{\hideNotesU}[1]{\textcolor{blue}{#1}}
\fi

% default values copied from titlesec documentation page 23
% parameters of \titleformat command are explained on page 4
\titleformat%
    {\section}% <command> is the sectioning command to be redefined, i. e., \part, \chapter, \section, \subsection, \subsubsection, \paragraph or \subparagraph.
    {\normalfont\large\scshape}% <format>
    {}% <label> the number
    {0em}% <sep> length. horizontal separation between label and title body
    {\centering\sectionformat}% code preceding the title body  (title body is taken as argument)

%% Set counters for sections to none
\setcounter{secnumdepth}{0}

%% Set the footer/headers
\pagestyle{fancy}
\fancyhf{}
\renewcommand{\headrulewidth}{0pt}
\renewcommand{\footrulewidth}{2pt}
\lfoot{P.-O. Paris{\'e}}
\cfoot{MATH 311}
\rfoot{Page \thepage}

%% Defining example environment
\newcounter{example}
\NewEnviron{example}%
	{%
	\noindent\refstepcounter{example}\fcolorbox{gray!40}{gray!40}{\textsc{\textcolor{red}{Example~\theexample.}}}%
	%\fcolorbox{black}{white}%
		{  %\parbox{0.95\textwidth}%
			{
			\BODY
			}%
		}%
	}

\newcounter{theorem}
\NewEnviron{theorem}%
	{%
	\noindent\refstepcounter{theorem}\fcolorbox{gray!40}{gray!40}{\textsc{\textcolor{black}{Theorem~\thetheorem.}}}%
	%\fcolorbox{black}{white}%
		{  %\parbox{0.95\textwidth}%
			{
			\BODY
			}%
		}%
	}

\newcounter{definition}
\NewEnviron{definition}%
	{%
	\noindent\refstepcounter{definition}\fcolorbox{gray!40}{gray!40}{\textsc{\textcolor{black}{Definition~\thedefinition.}}}%
	%\fcolorbox{black}{white}%
		{  %\parbox{0.95\textwidth}%
			{
			\BODY
			}%
		}%
	}

\newcounter{algo}
\NewEnviron{algorithm}
	{%
	\noindent\refstepcounter{algo}\fcolorbox{gray!40}{gray!40}{\textsc{\textcolor{black}{Algorithm~\thealgo.}}}%
	%\fcolorbox{black}{white}%
		{  %\parbox{0.95\textwidth}%
			{
			\BODY
			}%
		}%
	}

\NewEnviron{solution}%
	{%
	\noindent \fcolorbox{gray!40}{gray!40}{\textsc{\textcolor{blue}{Solution.}}}%
	%\fcolorbox{black}{white}%
		{  %\parbox{0.95\textwidth}%
			{
			%\textcolor{blue}
			}%
		}%
	}

\NewEnviron{proof}%
	{%
	\noindent \fcolorbox{gray!40}{gray!40}{\textsc{\textcolor{blue}{Proof.}}}%
	%\fcolorbox{black}{white}%
		{  %\parbox{0.95\textwidth}%
			{
			\textcolor{blue}{%
			\BODY
			}
			}%
		}%
	}
%%% Ignorer les notes
%\excludecomment{notes}

%%%%
\begin{document}
\thispagestyle{empty}

\begin{center}
\vspace*{2.5cm}

{\Huge \textsc{Math 311}}

\vspace*{1.5cm}

{\LARGE \textsc{Chapter 2}} 

\vspace*{0.75cm}

\noindent\textsc{Section 2.4: Matrix Inverses}

\vspace*{0.75cm}

\tableofcontents

\vfill

\noindent \textsc{Created by: Pierre-Olivier Paris{\'e}} \\
\textsc{Spring 2024}
\end{center}

\newpage

\section{Inverses of A Matrix}

\underline{Numbers:} To solve the equation $2x + 1 = 0$:
	\begin{align*}
	2x + 1 = 0 \iff 2x = -1 \iff \frac{2x}{2} = -\frac{1}{2} \iff x = - \frac{1}{2} .
	\end{align*}
The number $2^{-1} = \frac{1}{2}$ is called the \textbf{inverse} of $2$ because $2 (2^{-1}) = 1$.

\begin{definition}
If $A$ is a square matrix, a matrix $B$ is called an \textbf{inverse} of $A$ if and only if
	\[
		AB = I \quad \text{ and } \quad BA = I .
	\]
If $A$ has an inverse, then $A$ is called an \textbf{invertible matrix}.
\end{definition}

\begin{example}
Show that $B = \begin{bmatrix} -1 & 1 \\ 1 & 0 \end{bmatrix}$ is an inverse of $A = \begin{bmatrix} 0 & 1 \\ 1 & 1 \end{bmatrix}$.
\end{example}

\begin{solution}

\end{solution}

\newpage 

\begin{example}
Show that $A = \begin{bmatrix} 0 & 1 \\ 0 & 3 \end{bmatrix}$ has no inverse.
\end{example}

\begin{solution}

\end{solution}

\vspace*{8cm}

\underline{Note:} There are non-zero matrices that do not have an inverse!

\vspace*{0.5cm}

\begin{theorem}
If $B$ and $C$ are both inverses of a matrix $A$, then $B = C$.
\end{theorem}

\begin{proof}
Since $B$ and $C$ are both inverses of $A$, we have $AC = I = CA$ and $AB = I = BA$. Therefore,
	\[
		B = IB = (CA) B = C (AB) = C I = C . \tag*{$\square$}
	\]
\end{proof}
\underline{Note:}
\begin{itemize}
	\item The last result tells us that when $A$ has an inverse, it is unique (there is only one inverse).
	\item So, we denote the inverse of $A$ by $A^{-1}$.
	\item If $B$ satisfies $AB = I$ and $BA = I$, then $B = A^{-1}$ (Inverse Criterion).
\end{itemize}

\newpage 

\subsubsection{Inverses of $2 \times 2$ matrices}

\begin{example}
Find the inverse of $A = \begin{bmatrix} 5 & -3 \\ 7 & 4 \end{bmatrix}$.
\end{example}

\begin{solution}

\end{solution}

\vfill 

\underline{In General:}

If $ad - bc \neq 0$, then
$$
	\begin{bmatrix} a & b \\ c & d \end{bmatrix}^{-1} = \tfrac{1}{ad - bc} \begin{bmatrix} d & -b \\ -c & a \end{bmatrix} .
$$

\newpage 

\section{Inverses and Linear Systems}

Recall: A system of linear equations can be written in matrix form
	\[
		A \mathbf{x} = \mathbf{b} .
	\]

\begin{theorem}
If the $n \times n$ matrix $A$ is invertible, then the system has the unique solution
	\[
		\mathbf{x} = A^{-1} \mathbf{b} .
	\]
\end{theorem}

\begin{proof}
Start from
	\[
		A \mathbf{x} = \mathbf{b} \iff A^{-1} (A \mathbf{x}) = A^{-1} \mathbf{b} \iff (A^{-1} A) \mathbf{x} = A^{-1} \mathbf{b} .
	\]
We know that $A^{-1} A = I$. Hence $I\mathbf{x} = A^{-1} \mathbf{b}$. \hfill $\square$
\end{proof}

\begin{example}
Solve the system $\left\lbrace \systeme{5x_1 - 3x_2 = -4, 7x_1 + 4x_2 = 8} \right.$.
\end{example}

\begin{solution}

\end{solution}

\newpage 

\section{An Inversion Method}

\begin{algorithm}
If $A$ is an invertible (square) matrix, there exists a sequence of elementary row operations that 
	\begin{itemize}
		\item carry $A$ to the identity matrix $I$;
		\item carry $I$ to the inverse $A^{-1}$.
	\end{itemize}
Using block matrices, the algorithm can be rewritten as followed:
	\[
		[ \, A \quad I \, ] \longrightarrow \cdots \longrightarrow [\, I \quad A^{-1} \, ] .
	\]
\end{algorithm}

\begin{example}
Find the inverse of the matrix $A = \begin{bmatrix} 1 & 0 & -1 \\ 3 & 2 & 0 \\ -1 & -1 & 0 \end{bmatrix}$.
\end{example}

\begin{solution}

\end{solution}

\newpage 

\phantom{2} 

\vfill 

\underline{Note:} If $A$ is an $n \times n$ matrix, either 
	\begin{itemize}
		\item $A$ can be reduced to $I$ and then the algorithm produces $A^{-1}$;
		\item or $A$ can't be reduced to $I$ and then $A^{-1}$ does not exist.
	\end{itemize}

\newpage 

\section{Properties of the Inverse}

\begin{theorem}
All the matrices in this statement are square matrices of the same size.
\begin{enumerate}
	\item $I$ is invertible and $I^{-1} = I$.
	\item If $A$ is invertible, then $A^{-1}$ is invertible and $(A^{-1})^{-1} = A$.
	\item If $A$ and $B$ are invertible, then $AB$ is invertible and $(AB)^{-1} = B^{-1}A^{-1}$.
	\item If $A$ is invertible and $a \neq 0$ is a number, then $aA$ is invertible and $(aA)^{-1} = \frac{1}{a} A^{-1}$.
	\item If $A$ is invertible, then $A^{\top}$ is invertible and $(A^{\top})^{-1} = (A^{-1})^\top$.
	\item If $AB = AC$, then $B = C$ (left cancellation law).
	\item If $BA = CA$, then $B = C$ (right cancellation law).
\end{enumerate}
\end{theorem}

\textcolor{red}{\underline{Warning!}}
	\begin{itemize}
		\item The statement ``If $A$ and $B$ are both invertible, then $A + B$ is invertible'' is not true.
		\item Cross cancelling is wrong. This means ``If $AB = CA$, then $B = C$`` is a false statement.
	\end{itemize}

\begin{comment}
\begin{proof}
\begin{enumerate}
	\item We have $I I = I$, so that $I$ is an inverse of $I$. But the inverse is unique, so $I^{-1} = I$.
	\item Since $A$ is invertible, we have that $A A^{-1} = A^{-1} A = I$. Let $B = A^{-1}$, we get that $BA = AB = I$, so that $B^{-1} = A$. Hence, $(A^{-1})^{-1} = A$.
	\item Assume that $A$ and $B$ are invertible. Then
		\[
			A A^{-1} = A^{-1} A = I \quad \text{ and } \quad B B^{-1} = B^{-1} B = I .
		\]
	Then, we have
		\[
			(AB) (B^{-1} A^{-1}) = A (B B^{-1}) A^{-1} = A I A^{-1} = A A^{-1} = I
		\]
	and
		\[
			(B^{-1} A^{-1}) AB = B^{-1} (A^{-1} A ) B = B^{-1} I B = B^{-1} B = I .
		\]
	Thus, $(AB)^{-1} = B^{-1} A^{-1}$.
	\item Assume that $A$ is invertible and $a \neq 0$. Then
		\[
			(aA) ( \tfrac{1}{a} A^{-1} ) = a (A \tfrac{1}{a} ) A^{-1} = (a \tfrac{1}{a} ) A A^{-1} = 1 I = I
		\]
	and also $(\frac{1}{a} A^{-1}) (aA ) = I$. Hence $(aA)^{-1} = \frac{1}{a} A^{-1}$.
	\item Assume that $A$ is invertible. Then
		\[
			A^\top (A^{-1})^\top = (A^{-1} A)^{\top} = I^\top = I
		\]
	and similarly, $(A^{-1})^\top A^\top = I$. Hence $(A^\top )^{-1} = (A^{-1})^\top$. \hfill $\square$
\end{enumerate}
\end{proof}
\end{comment}

\newpage 

\begin{example}
Find $A$ if $\Big( \begin{bmatrix} 1 & 0 \\ 2 & 1 \end{bmatrix} A\Big)^{-1} = \begin{bmatrix} 1 & 0 \\ 2 & 2 \end{bmatrix}$. 
\end{example}

\begin{solution}

\end{solution}

\vspace*{10cm}

\begin{example}
If $A$, $B$, and $C$ are $n \times n$ invertible matrices, show that $ABC$ is invertible with $(ABC)^{-1} = C^{-1} B^{-1} A^{-1}$.
\end{example}

\begin{solution}

\end{solution}

\vfill 

\underline{Note:}
	\begin{itemize}
		\item If $A_1$, $A_2$, $\ldots$, $A_k$ are invertible, then $(A_1 A_2 \cdots A_k )^{-1} = A_k^{-1} \cdots A_2^{-1} A_1^{-1}$.
		\item If $A$ is invertible and $k \geq 0$, then $(A^k)^{-1} = (A^{-1})^k$. 
	\end{itemize}



\begin{comment}

\newpage 

\begin{example}
Let $Q = \begin{bmatrix} A & 0 \\ Y & B \end{bmatrix}$ be block matrices where $A$ is $m \times m$ and $B$ is $n \times n$. Show that $Q$ is invertible if and only if $A$ and $B$ are both invertible and find an expression of $Q^{-1}$.
\end{example}

\begin{theorem}
The following conditions are equivalent for an $n \times n$ matrix $A$:
	\begin{enumerate}
		\item $A$ is invertible.
		\item The homogeneous system $A \mathbf{x} = \mathbf{0}$ has only the trivial solution $\mathbf{0}$.
		\item $A$ can be carried to the identity matrix $I_n$ by elementary row operations.
		\item The system $A \mathbf{x} = \mathbf{b}$ has a only one solution $\mathbf{x}$ for every choice of column vector $\mathbf{b}$.
		\item There exists an $n \times n$ matrix $C$ such that $AC = I_n$.
	\end{enumerate}
\end{theorem}
\end{comment}


\end{document}