\documentclass[20pt,a4paper]{extarticle}
\usepackage[utf8]{inputenc}
\usepackage[english]{babel}

\usepackage{amsmath}
\usepackage{amsfonts}
\usepackage{amssymb}
\usepackage{mathtools}
\usepackage{systeme}
\sysdelim..

\usepackage{graphicx}
\usepackage{caption}
\usepackage{subcaption}
\usepackage{lmodern}
\usepackage{tikz}
\usetikzlibrary{calc}
\usepackage{titlesec}
\usepackage{environ}
\usepackage{xcolor}
\usepackage{fancyhdr}
\usepackage[colorlinks = true, linkcolor = black]{hyperref}
\usepackage{xparse}
\usepackage{enumitem}
\usepackage{comment}
\usepackage{wrapfig}
\usepackage{soul}
\usepackage[capitalise]{cleveref}
\usepackage{circledsteps}

\usepackage[left=1cm,right=1cm,top=1cm,bottom=3cm]{geometry}
\usepackage{multicol}
\usepackage[indent=0pt]{parskip}

\newcommand{\spaceP}{\vspace*{0.5cm}}
\newcommand{\range}{\mathrm{range}\,}
\newcommand{\ra}{\rightarrow}
\newcommand{\curl}{\mathrm{curl} \,}
\newcommand{\hrint}[2]{\scalebox{2}{$\displaystyle\int_{\scalebox{0.35}{$#1$}}^{\scalebox{0.35}{$#2$}}$}\,}
\newcommand{\hint}[1]{\scalebox{2}{$\displaystyle\int_{\scalebox{0.35}{$#1$}}$}\,}
\newcommand{\hiint}[1]{\scalebox{2}{$\displaystyle\iint_{\scalebox{0.35}{$#1$}}$}\,}
\newcommand{\hiiint}[1]{\scalebox{2}{$\displaystyle\iiint_{\scalebox{0.35}{$#1$}}$}\,}
\renewcommand{\div}{\mathrm{div}\,}

\DeclareMathOperator{\Span}{span}
\DeclareMathOperator{\tr}{tr}

\makeatletter
\renewcommand*\env@matrix[1][*\c@MaxMatrixCols c]{%
  \hskip -\arraycolsep
  \let\@ifnextchar\new@ifnextchar
  \array{#1}}
\makeatother

%% Redefining sections
\newcommand{\sectionformat}[1]{%
    \begin{tikzpicture}[baseline=(title.base)]
        \node[rectangle, draw] (title) {#1};
    \end{tikzpicture}
    
    \noindent\hrulefill
}

\newif\ifhNotes 

\hNotesfalse

\ifhNotes
	\newcommand{\hideNotes}[1]{%
	\phantom{#1}
	}
	\newcommand{\hideNotesU}[1]{%
	\underline{\hspace{1mm}\phantom{#1}\hspace{1mm}}
	}
\else
	\newcommand{\hideNotes}[1]{#1}
	\newcommand{\hideNotesU}[1]{\textcolor{blue}{#1}}
\fi

% default values copied from titlesec documentation page 23
% parameters of \titleformat command are explained on page 4
\titleformat%
    {\section}% <command> is the sectioning command to be redefined, i. e., \part, \chapter, \section, \subsection, \subsubsection, \paragraph or \subparagraph.
    {\normalfont\large\scshape}% <format>
    {}% <label> the number
    {0em}% <sep> length. horizontal separation between label and title body
    {\centering\sectionformat}% code preceding the title body  (title body is taken as argument)

%% Set counters for sections to none
\setcounter{secnumdepth}{0}

%% Set the footer/headers
\pagestyle{fancy}
\fancyhf{}
\renewcommand{\headrulewidth}{0pt}
\renewcommand{\footrulewidth}{2pt}
\lfoot{P.-O. Paris{\'e}}
\cfoot{MATH 311}
\rfoot{Page \thepage}

%% Defining example environment
\newcounter{example}
\NewEnviron{example}%
	{%
	\noindent\refstepcounter{example}\fcolorbox{gray!40}{gray!40}{\textsc{\textcolor{red}{Example~\theexample.}}}%
	%\fcolorbox{black}{white}%
		{  %\parbox{0.95\textwidth}%
			{
			\BODY
			}%
		}%
	}

\newcounter{theorem}
\NewEnviron{theorem}%
	{%
	\noindent\refstepcounter{theorem}\fcolorbox{gray!40}{gray!40}{\textsc{\textcolor{black}{Theorem~\thetheorem.}}}%
	%\fcolorbox{black}{white}%
		{  %\parbox{0.95\textwidth}%
			{
			\BODY
			}%
		}%
	}

\newcounter{definition}
\NewEnviron{definition}%
	{%
	\noindent\refstepcounter{definition}\fcolorbox{gray!40}{gray!40}{\textsc{\textcolor{black}{Definition~\thedefinition.}}}%
	%\fcolorbox{black}{white}%
		{  %\parbox{0.95\textwidth}%
			{
			\BODY
			}%
		}%
	}

\newcounter{algo}
\NewEnviron{algorithm}
	{%
	\noindent\refstepcounter{algo}\fcolorbox{gray!40}{gray!40}{\textsc{\textcolor{black}{Algorithm~\thealgo.}}}%
	%\fcolorbox{black}{white}%
		{  %\parbox{0.95\textwidth}%
			{
			\BODY
			}%
		}%
	}

\NewEnviron{goal}
	{%
	\noindent\fcolorbox{gray!40}{gray!40}{\textsc{\textcolor{black}{Goal:}}}%
	%\fcolorbox{black}{white}%
		{  %\parbox{0.95\textwidth}%
			{
			\BODY
			}%
		}%
	}

\NewEnviron{solution}%
	{%
	\noindent \fcolorbox{gray!40}{gray!40}{\textsc{\textcolor{blue}{Solution.}}}%
	%\fcolorbox{black}{white}%
		{  %\parbox{0.95\textwidth}%
			{
			%\textcolor{blue}
			}%
		}%
	}

\NewEnviron{proof}%
	{%
	\noindent \fcolorbox{gray!40}{gray!40}{\textsc{\textcolor{blue}{Proof.}}}%
	%\fcolorbox{black}{white}%
		{  %\parbox{0.95\textwidth}%
			{
			\textcolor{blue}{%
			\BODY
			}
			}%
		}%
	}
%%% Ignorer les notes
%\excludecomment{notes}

%%%%
\begin{document}
\thispagestyle{empty}

\begin{center}
\vspace*{0.75cm}

{\Huge \textsc{Math 311}}

\vspace*{1.5cm}

{\LARGE \textsc{Last Chapter}} 

\vspace*{0.75cm}

\noindent\textsc{Section 10.1: Inner Product Spaces}

\vspace*{0.25cm}


\begin{footnotesize}

\tableofcontents
\end{footnotesize}

\vfill

\noindent \textsc{Created by: Pierre-Olivier Paris{\'e}} \\
\textsc{Spring 2024}
\end{center}

\newpage

\section{Definition}

For $\mathbb{R}^n$, if we define
	\[
		\left\langle \mathbf{x} , \mathbf{y} \right\rangle = \mathbf{x} \cdot \mathbf{y} = x_1 y_1 + x_2 y_2 + \cdots + x_n y_n ,
	\]
then the following properties are satisfied:
	\begin{enumerate}[label=\Circled{P\arabic*}]
		\item $\left\langle \mathbf{x} , \mathbf{y} \right\rangle$ is real number;
		\item $\left\langle \mathbf{x} , \mathbf{y} \right\rangle = \left\langle \mathbf{y} , \mathbf{x} \right\rangle$;
		\item $\left\langle \mathbf{x} + \mathbf{y} , \mathbf{z} \right\rangle = \left\langle \mathbf{x} , \mathbf{z} \right\rangle + \left\langle \mathbf{y} , \mathbf{z} \right\rangle$;
		\item $\left\langle a \mathbf{x} , \mathbf{y} \right\rangle = a \left\langle \mathbf{x} , \mathbf{y} \right\rangle$;
		\item $\mathbf{x} \neq 0$ if and only if $\left\langle \mathbf{x} , \mathbf{x} \right\rangle > 0$.
	\end{enumerate}
When P1-P5 are satisfied, we say that the dot product is an inner product and $(\mathbb{R}^n , \left\langle \cdot , \cdot \right\rangle)$ is an inner product space.

\begin{definition}
Let $V$ be a vector space. If a function $\left\langle \cdot , \cdot \right\rangle : V \times V \ra \mathbb{R}$ satisfies P1-P5, then we say that $\left\langle \cdot , \cdot \right\rangle$ is an \textbf{inner product} defined on $V$ and $(V, \left\langle \cdot , \cdot \right\rangle )$ is an \textbf{inner product space}.
\end{definition}


\underline{Remarks:}
	\begin{enumerate}[label=\Circled{\arabic*}]
		\item for $\mathbf{v} \in V$, we define $\Vert \mathbf{v} \Vert := \sqrt{\left\langle \mathbf{v} , \mathbf{v} \right\rangle}$. 
		\item $\mathbf{v} , \mathbf{w} \in V$ are orthogonal if and only if $\left\langle \mathbf{v} , \mathbf{w} \right\rangle = 0$.
		\item All notions from 5.3 and 8.1 extends to a general inner product space.
	\end{enumerate}

\section{Examples}

\subsubsection{Vectors}

\begin{example}
%Let
%	\[
%		A = \begin{bmatrix} 5 & 7 \\ 7 & 10 \end{bmatrix}.
%	\]
We can show that
	\[
		\left\langle \mathbf{x} , \mathbf{y} \right\rangle :=  5 x_1 y_1 + 7 x_1 y_2 + 7 x_2 y_1 + 10 x_2 y_2 
	\]
is an inner product on $\mathbb{R}^2$. Show that
	\begin{enumerate}[label=\alph*)]
	\item $\mathbf{x} = \begin{bmatrix} 1 & 1 \end{bmatrix}$ and $\mathbf{y} = \begin{bmatrix} -1 & 1 \end{bmatrix}$ are not orthogonal.
	\item $\mathbf{x} = \begin{bmatrix} 2 & 1 \end{bmatrix}$ and $\mathbf{y} = \begin{bmatrix} 24 & -17 \end{bmatrix}$ are orthogonal.
	\end{enumerate}
\end{example}
\begin{solution}

\end{solution}

\newpage 

\subsubsection{Matrices}

\begin{example}
For a matrix $A \in \mathbf{M_{nn}}$, we define its \textbf{trace} to be
	\[
		\tr (A) := a_{11} + a_{22} + \cdots + a_{nn} .
	\]
Then the function
	\[
		\left\langle A , B \right\rangle = \tr (A B^\top )
	\]
defines an inner product on $\mathbf{M_{nn}}$. 
\end{example}

\vspace*{16pt}

\subsubsection{Space of Continuous Functions}

\begin{example}
Let $\mathbf{C} [a, b]$ be the vector space of \textbf{real-valued continuous functions} on the interval $[a, b]$. The application
	\[
		\left\langle f , g \right\rangle = \hrint{a}{b} f (x) g (x) \, dx
	\]
is an inner product on $\mathbf{C} [a, b]$.
\end{example}


\newpage

\section{Complex Inner Product Spaces}

It is possible to have a theory of vector spaces using complex numbers. We simply replace $\mathbb{R}$ by $\mathbb{C}$, the set of complex numbers, everywhere in the definitions.

However, we have to modify the definition of an inner product.

\begin{definition}
Let $V$ be a \textbf{complex vector space}. An application $\left\langle \cdot , \cdot \right\rangle: V \times V \ra \mathbb{C}$ is a \textbf{complex inner product} if
	\begin{enumerate}[label=\Circled{P\arabic*}]
		\item $\left\langle \mathbf{x} , \mathbf{y} \right\rangle$ is a complex number;
		\item $\left\langle \mathbf{x} , \mathbf{y} \right\rangle = \overline{\left\langle \mathbf{y} , \mathbf{x} \right\rangle}$, where $\overline{w} = u - iv$ is the complex conjugate of $w = u + iv$;
		\item $\left\langle \mathbf{x} + \mathbf{y} , \mathbf{z} \right\rangle = \left\langle \mathbf{x} , \mathbf{z} \right\rangle + \left\langle \mathbf{y} , \mathbf{z} \right\rangle$;
		\item $\left\langle a \mathbf{x} , \mathbf{y} \right\rangle = a \left\langle \mathbf{x} , \mathbf{y} \right\rangle$ for any complex number $a$;
		\item $\mathbf{x} \neq 0$ if and only if $\left\langle \mathbf{x} , \mathbf{x} \right\rangle > 0$.
	\end{enumerate}
\end{definition}


\underline{Remarks:} The extension of vector space and inner product to complex numbers is used, for instance, in the foundations of Quantum Mechanics.


\end{document}