\documentclass[20pt,a4paper]{extarticle}
\usepackage[utf8]{inputenc}
\usepackage[english]{babel}

\usepackage{amsmath}
\usepackage{amsfonts}
\usepackage{amssymb}
\usepackage{mathtools}
\usepackage{systeme}
\sysdelim..

\usepackage{graphicx}
\usepackage{caption}
\usepackage{subcaption}
\usepackage{lmodern}
\usepackage{tikz}
\usetikzlibrary{calc}
\usepackage{titlesec}
\usepackage{environ}
\usepackage{xcolor}
\usepackage{fancyhdr}
\usepackage[colorlinks = true, linkcolor = black]{hyperref}
\usepackage{xparse}
\usepackage{enumitem}
\usepackage{comment}
\usepackage{wrapfig}
\usepackage{soul}
\usepackage[capitalise]{cleveref}
\usepackage{circledsteps}

\usepackage[left=1cm,right=1cm,top=1cm,bottom=3cm]{geometry}
\usepackage{multicol}
\usepackage[indent=0pt]{parskip}

\newcommand{\spaceP}{\vspace*{0.5cm}}
\newcommand{\range}{\mathrm{range}\,}
\newcommand{\ra}{\rightarrow}
\newcommand{\curl}{\mathrm{curl} \,}
\newcommand{\hint}[1]{\scalebox{2}{$\displaystyle\int_{\scalebox{0.35}{$#1$}}$}\,}
\newcommand{\hiint}[1]{\scalebox{2}{$\displaystyle\iint_{\scalebox{0.35}{$#1$}}$}\,}
\newcommand{\hiiint}[1]{\scalebox{2}{$\displaystyle\iiint_{\scalebox{0.35}{$#1$}}$}\,}
\renewcommand{\div}{\mathrm{div}\,}

\DeclareMathOperator{\Span}{span}

\makeatletter
\renewcommand*\env@matrix[1][*\c@MaxMatrixCols c]{%
  \hskip -\arraycolsep
  \let\@ifnextchar\new@ifnextchar
  \array{#1}}
\makeatother

%% Redefining sections
\newcommand{\sectionformat}[1]{%
    \begin{tikzpicture}[baseline=(title.base)]
        \node[rectangle, draw] (title) {#1};
    \end{tikzpicture}
    
    \noindent\hrulefill
}

\newif\ifhNotes 

\hNotesfalse

\ifhNotes
	\newcommand{\hideNotes}[1]{%
	\phantom{#1}
	}
	\newcommand{\hideNotesU}[1]{%
	\underline{\hspace{1mm}\phantom{#1}\hspace{1mm}}
	}
\else
	\newcommand{\hideNotes}[1]{#1}
	\newcommand{\hideNotesU}[1]{\textcolor{blue}{#1}}
\fi

% default values copied from titlesec documentation page 23
% parameters of \titleformat command are explained on page 4
\titleformat%
    {\section}% <command> is the sectioning command to be redefined, i. e., \part, \chapter, \section, \subsection, \subsubsection, \paragraph or \subparagraph.
    {\normalfont\large\scshape}% <format>
    {}% <label> the number
    {0em}% <sep> length. horizontal separation between label and title body
    {\centering\sectionformat}% code preceding the title body  (title body is taken as argument)

%% Set counters for sections to none
\setcounter{secnumdepth}{0}

%% Set the footer/headers
\pagestyle{fancy}
\fancyhf{}
\renewcommand{\headrulewidth}{0pt}
\renewcommand{\footrulewidth}{2pt}
\lfoot{P.-O. Paris{\'e}}
\cfoot{MATH 311}
\rfoot{Page \thepage}

%% Defining example environment
\newcounter{example}
\NewEnviron{example}%
	{%
	\noindent\refstepcounter{example}\fcolorbox{gray!40}{gray!40}{\textsc{\textcolor{red}{Example~\theexample.}}}%
	%\fcolorbox{black}{white}%
		{  %\parbox{0.95\textwidth}%
			{
			\BODY
			}%
		}%
	}

\newcounter{theorem}
\NewEnviron{theorem}%
	{%
	\noindent\refstepcounter{theorem}\fcolorbox{gray!40}{gray!40}{\textsc{\textcolor{black}{Theorem~\thetheorem.}}}%
	%\fcolorbox{black}{white}%
		{  %\parbox{0.95\textwidth}%
			{
			\BODY
			}%
		}%
	}

\newcounter{definition}
\NewEnviron{definition}%
	{%
	\noindent\refstepcounter{definition}\fcolorbox{gray!40}{gray!40}{\textsc{\textcolor{black}{Definition~\thedefinition.}}}%
	%\fcolorbox{black}{white}%
		{  %\parbox{0.95\textwidth}%
			{
			\BODY
			}%
		}%
	}

\newcounter{algo}
\NewEnviron{algorithm}
	{%
	\noindent\refstepcounter{algo}\fcolorbox{gray!40}{gray!40}{\textsc{\textcolor{black}{Algorithm~\thealgo.}}}%
	%\fcolorbox{black}{white}%
		{  %\parbox{0.95\textwidth}%
			{
			\BODY
			}%
		}%
	}

\NewEnviron{goal}
	{%
	\noindent\fcolorbox{gray!40}{gray!40}{\textsc{\textcolor{black}{Goal:}}}%
	%\fcolorbox{black}{white}%
		{  %\parbox{0.95\textwidth}%
			{
			\BODY
			}%
		}%
	}

\NewEnviron{solution}%
	{%
	\noindent \fcolorbox{gray!40}{gray!40}{\textsc{\textcolor{blue}{Solution.}}}%
	%\fcolorbox{black}{white}%
		{  %\parbox{0.95\textwidth}%
			{
			%\textcolor{blue}
			}%
		}%
	}

\NewEnviron{proof}%
	{%
	\noindent \fcolorbox{gray!40}{gray!40}{\textsc{\textcolor{blue}{Proof.}}}%
	%\fcolorbox{black}{white}%
		{  %\parbox{0.95\textwidth}%
			{
			\textcolor{blue}{%
			\BODY
			}
			}%
		}%
	}
%%% Ignorer les notes
%\excludecomment{notes}

%%%%
\begin{document}
\thispagestyle{empty}

\begin{center}
\vspace*{0.75cm}

{\Huge \textsc{Math 311}}

\vspace*{1.5cm}

{\LARGE \textsc{Last Chapter}} 

\vspace*{0.75cm}

\noindent\textsc{Section 5.3: Orthogonality}

\vspace*{0.25cm}


\begin{footnotesize}

\tableofcontents
\end{footnotesize}

\vfill

\noindent \textsc{Created by: Pierre-Olivier Paris{\'e}} \\
\textsc{Spring 2024}
\end{center}

\newpage

\section{Definitions}

\subsubsection{Dot Product}
If $\mathbf{x}$ is an $1 \times n$ column vector and $\mathbf{y}$ is an $n \times 1$ column vector, then recall that
	\[
		\begin{bmatrix} x_1 & x_2 & \cdots & x_n \end{bmatrix} \begin{bmatrix} y_1 \\ y_2 \\ \vdots \\ y_n \end{bmatrix} = \begin{bmatrix} x_1 y_1 + x_2 y_2 + \cdots + x_n y_n \end{bmatrix} .
	\]
The result is a $1 \times 1$ matrix that we treat as a number.

	\begin{definition}
	Let $\mathbf{x} = \begin{bmatrix} x_1 & x_2 & \ldots & x_n \end{bmatrix}$ and $\mathbf{y} = \begin{bmatrix} y_1 & y_2 & \ldots & y_n \end{bmatrix}$ be two $1 \times n$ row vectors in $\mathbb{R}^n$. Their \textbf{dot product} is defined as followed:
		\[
			\mathbf{x} \cdot \mathbf{y} := \mathbf{x} \mathbf{y}^\top = x_1 y_1 + x_2 y_2 + \cdots + x_n y_n .
		\]
	\end{definition}

\begin{example}
If $\mathbf{x} = \begin{bmatrix} 1 & -1 & -3 & 1 \end{bmatrix}$ and $\mathbf{y} = \begin{bmatrix} 2 & 1 & 1 & 0 \end{bmatrix}$. Then
	\[
		\mathbf{x} \cdot \mathbf{y} = (1)(2) + (-1)(1) + (-3)(1) + (1) (0) = - 2.
	\]
\end{example}

\underline{Notes:}
	\begin{enumerate}[label=\Circled{\arabic*}]
		\item We can use other representations of vectors in $\mathbb{R}^n$.
		\item For instance, if $\mathbf{x}$ and $\mathbf{y}$ are $n \times 1$ column vectors, then
			\[
				\mathbf{x} \cdot \mathbf{y} = x_1 y_1 + x_2 y_2 + \cdots + x_n y_n = \mathbf{x}^\top \mathbf{y} .
			\]
	\end{enumerate}

\newpage 

\subsubsection{Length}

\begin{definition}
Let $\mathbf{x} = \begin{bmatrix} x_1 & x_2 & \cdots & x_n \end{bmatrix}$. The \textbf{length} $\Vert \mathbf{x} \Vert$ is defined by
	\[
		\Vert \mathbf{x} \Vert := \sqrt{\mathbf{x} \cdot \mathbf{x}} = \sqrt{x_1^2 + x_2^2 + \cdots + x_n^2} .
	\]
\end{definition}

\begin{example}
If $\mathbf{x} = \begin{bmatrix} 1 & 3 & -2 & 0 \end{bmatrix}$, then
	\[
		\Vert \mathbf{x} \Vert = \sqrt{(1)^2 + (3)^2 + (-2)^2 + (0)^2} = \sqrt{1 + 9 + 4} = \sqrt{14} .
	\]
\end{example}

\underline{Properties:}
	\begin{enumerate}[label=\Circled{\arabic*}]
		\item $\mathbf{x} \cdot \mathbf{y} = \mathbf{y} \cdot \mathbf{x}$.
		\item $\mathbf{x} \cdot (\mathbf{y} + \mathbf{z} ) = \mathbf{x} \cdot \mathbf{y} + \mathbf{x} \cdot \mathbf{z}$.
		\item $(a \mathbf{x}) \cdot \mathbf{y} = a (\mathbf{x} \cdot \mathbf{y} ) = \mathbf{x} \cdot (a \mathbf{y} )$.
		\item $\Vert \mathbf{x} \Vert^2 = \mathbf{x} \cdot \mathbf{x}$.
		\item $\Vert \mathbf{x} \Vert \geq 0$, and $\Vert \mathbf{x} \Vert = 0$ if and only if $\mathbf{x} = \mathbf{0}$.
		\item $\Vert \mathbf{x} + \mathbf{y} \Vert^2 = \Vert \mathbf{x} \Vert^2 + 2 (\mathbf{x} \cdot \mathbf{y}) + \Vert \mathbf{y} \Vert^2$.
	\end{enumerate}

\newpage 

\section{Important Identities}

	\subsubsection{Cauchy-Schwarz Inequality}
	\begin{example}
	Let $\mathbf{x} = (a, b)$ and $\mathbf{y} = (c, d)$. Show that
		\[
			|\mathbf{x} \cdot \mathbf{y}| \leq \Vert \mathbf{x} \Vert \Vert \mathbf{y} \Vert .
		\]
	\end{example}

	\begin{solution}

	\end{solution}

	\vfill

	\begin{theorem}
	If $\mathbf{x}$ and $\mathbf{y}$ are in $\mathbb{R}^n$, then
		\[
			| \mathbf{x} \cdot \mathbf{y} | \leq \Vert \mathbf{x} \Vert \Vert \mathbf{y} \Vert .
		\]
	\end{theorem}

	\newpage 

	\subsubsection{Triangle Inequality}

	\begin{theorem}
	If $\mathbf{x}$ and $\mathbf{y}$ are in $\mathbb{R}^n$, then $\Vert \mathbf{x} + \mathbf{y} \Vert \leq \Vert \mathbf{x} \Vert + \Vert \mathbf{y} \Vert$.
	\end{theorem}

	\underline{Illustration in $\mathbb{R}^2$.}

	\vspace*{6cm}

	\subsubsection{Distance}

	\begin{definition}
	If $\mathbf{x}$ and $\mathbf{y}$ are two vectors in $\mathbb{R}^n$, the \textbf{distance} $d (\mathbf{x}, \mathbf{y})$ is defined by
		\[
			d (\mathbf{x} , \mathbf{y} ) = \Vert \mathbf{x} - \mathbf{y} \Vert .
		\]
	\end{definition}

	\underline{Illustration in $\mathbb{R}^2$.}


\newpage 

\section{Orthogonality}

	\begin{definition}
	Two vectors $\mathbf{x}$ and $\mathbf{y}$ are \textbf{orthogonal} if
		\[
			\mathbf{x} \cdot \mathbf{y} = 0 .
		\]
	If $\mathbf{x}$ and $\mathbf{y}$ are orthogonal, we write $\mathbf{x} \perp \mathbf{y}$.
	\end{definition}

	\begin{example}
	Let $\mathbf{x} = \begin{bmatrix} 1 & 1 \end{bmatrix}$ and $\mathbf{y} = \begin{bmatrix} 1 & -1 \end{bmatrix}$. 
		\begin{enumerate}[label=\alph*)]
			\item Are $\mathbf{x}$, $\mathbf{y}$ orthogonal?
			\item If they are orthogonal, then draw the vectors in a coordinates plane and give one special geometric properties.
		\end{enumerate}
	\end{example}

	\vfill

	\underline{Notes:} In $\mathbb{R}^2$, we can show that
		\[
			\mathbf{x} \cdot \mathbf{y} = \Vert x \Vert \Vert y \Vert \cos \theta
		\]
	where $\theta$ is the angle between the vectors $\mathbf{x}$ and $\mathbf{y}$.

	\newpage 

	\subsubsection{Orthogonal Sets}

	\begin{definition}
	A collection of vectors $\{ \mathbf{x_1} , \mathbf{x_2} , \ldots , \mathbf{x_k} \}$ is an \textbf{orthogonal set} if
		\begin{enumerate}[label=\Circled{\arabic*}]
			\item $\mathbf{x_i} \cdot \mathbf{x_j} = 0$ for any $i \neq j$.
			\item $\mathbf{x_i} \neq 0$ for any $i$.
		\end{enumerate}
	\end{definition}

	\vspace*{10pt}

	\begin{example}
	Let 
		\begin{enumerate}[label=\alph*)]
		\item $S_1 = \{ (0, 0, 0) , (1, 2, 3) , (-1, -1, -1) \}$.
		\item $S_2 = \{ (1, 2, 3 ) , (-1, -1, - 1) , (1, 1, 1) \}$.
		\item $S_3 = \{ (3, 4, 5) , (-4, 3, 0) , (-3, -4, 5) \}$. 
		\end{enumerate}
	Which one of these sets is an orthogonal set?
	\end{example}

	\begin{solution}

	\end{solution}

	\newpage 

	\subsubsection{Orthonormal Sets}

	\begin{definition}
	A collection of vectors $\{ \mathbf{x_1} , \mathbf{x_2} , \ldots , \mathbf{x_k} \}$ is an \textbf{orthonormal set} if
		\begin{enumerate}[label=\Circled{\arabic*}]
			\item it is an orthogonal set.
			\item $\Vert \mathbf{x_i} \Vert = 1$ for every index $i$.
		\end{enumerate}
	\end{definition}

	\vspace*{10pt}

	\begin{example}
	The standard basis $\{ \mathbf{e_1} , \mathbf{e_2} , \ldots , \mathbf{e_n} \}$ is an orthonormal set in $\mathbb{R}^n$.
	\end{example}

	We can always obtain an orthonormal set from an orthogonal set by \textbf{normalizing} the vectors in the orthogonal set.

	\begin{example}
	Obtain an orthonormal set by normalizing the following orthogonal set:
		\[
			\{ (1, -1, 2) , (0, 2, 1) , (5, 1, -2 ) \} .
		\]
	\end{example}

	\begin{solution}

	\end{solution}

	\newpage 

\section{Important Identities}

	\subsubsection{Pythagoras' Theorem}

	\begin{theorem}
	If $\{ \mathbf{x_1} , \mathbf{x_2} , \ldots , \mathbf{x_k} \}$ is an orthogonal set in $\mathbb{R}^n$, then
		\[
			\Vert \mathbf{x_1} + \mathbf{x_2} + \cdots + \mathbf{x_k} \Vert^2 = \Vert \mathbf{x_1} \Vert^2 + \Vert \mathbf{x_2} \Vert^2 + \cdots + \Vert \mathbf{x_k} \Vert^2 .
		\]
	\end{theorem}

	\underline{Illustration in $\mathbb{R}^2$.}

	\vspace*{5cm} 

	\subsubsection{Linearly Independent}

	\begin{theorem}
	If $S = \{ \mathbf{x_1}, \mathbf{x_2} , \ldots , \mathbf{x_k} \}$ is an orthogonal set in $\mathbb{R}^n$, then $S$ is linearly independent.
	\end{theorem}

	\newpage 

	\subsubsection{Fourier Expansion}

	\begin{example}
	Let $U = \Span \{ (1, -2, 3), (-1, 1, 1) \}$ and $\mathbf{x} = (13, -20, 15) \in U$. 
		\begin{enumerate}[label=\alph*)]
			\item Show $\{ (1, -2, 3) , (-1, 1, 1) \}$ is an orthogonal basis of $U$.
			\item Express $\mathbf{x}$ as a linear combination of the basis of $U$.
		\end{enumerate}
	\end{example}

	\begin{solution}

	\end{solution}

	\newpage 

	\phantom{2}

	\vfill

	\begin{theorem}
	Let $\{ \mathbf{u_1} , \mathbf{u_2} , \ldots , \mathbf{u_m} \}$ be an orthogonal basis of a subspace $U$ of $\mathbb{R}^n$. For any $\mathbf{x} \in U$, we have
		\[
			\mathbf{x} = \Big( \frac{\mathbf{x} \cdot \mathbf{u_1}}{\Vert \mathbf{u_1} \Vert^2} \Big) \mathbf{u_1} + \Big( \frac{\mathbf{x} \cdot \mathbf{u_2}}{\Vert \mathbf{u_2} \Vert^2} \Big) \mathbf{u_2} + \cdots + \Big( \frac{\mathbf{x} \cdot \mathbf{u_m}}{\Vert \mathbf{u_m} \Vert^2} \Big) \mathbf{u_m} .
		\]
	\end{theorem}

	\newpage 

	\subsubsection{Criteria to be in the Span}

	\begin{example}
	Let $U = \Span \{ \mathbf{u_1} , \mathbf{u_2} , \ldots , \mathbf{u_m} \}$ and let $\mathbf{x} \in \mathbb{R}^n$. Show that if $\mathbf{x} \neq \mathbf{0}$ and $\mathbf{x} \perp \mathbf{u_k}$ for each $1 \leq k \leq m$, then $\mathbf{x} \not\in U$.
	\end{example}

	\begin{solution}

	\end{solution}



\end{document}