\documentclass[20pt,a4paper]{extarticle}
\usepackage[utf8]{inputenc}
\usepackage[english]{babel}

\usepackage{amsmath}
\usepackage{amsfonts}
\usepackage{amssymb}
\usepackage{mathtools}
\usepackage{systeme}
\sysdelim..

\usepackage{graphicx}
\usepackage{caption}
\usepackage{subcaption}
\usepackage{lmodern}
\usepackage{tikz}
\usetikzlibrary{calc}
\usepackage{titlesec}
\usepackage{environ}
\usepackage{xcolor}
\usepackage{fancyhdr}
\usepackage[colorlinks = true, linkcolor = black]{hyperref}
\usepackage{xparse}
\usepackage{enumitem}
\usepackage{comment}
\usepackage{wrapfig}
\usepackage{soul}
\usepackage[capitalise]{cleveref}

\usepackage[left=2cm,right=2cm,top=2cm,bottom=2cm]{geometry}
\usepackage{multicol}
\usepackage[indent=0pt]{parskip}

\newcommand{\spaceP}{\vspace*{0.5cm}}
\newcommand{\Span}{\mathrm{Span}\,}
\newcommand{\range}{\mathrm{range}\,}
\newcommand{\ra}{\rightarrow}
\newcommand{\curl}{\mathrm{curl} \,}
\newcommand{\hint}[1]{\scalebox{2}{$\displaystyle\int_{\scalebox{0.35}{$#1$}}$}\,}
\newcommand{\hiint}[1]{\scalebox{2}{$\displaystyle\iint_{\scalebox{0.35}{$#1$}}$}\,}
\newcommand{\hiiint}[1]{\scalebox{2}{$\displaystyle\iiint_{\scalebox{0.35}{$#1$}}$}\,}
\renewcommand{\div}{\mathrm{div}\,}

\makeatletter
\renewcommand*\env@matrix[1][*\c@MaxMatrixCols c]{%
  \hskip -\arraycolsep
  \let\@ifnextchar\new@ifnextchar
  \array{#1}}
\makeatother

%% Redefining sections
\newcommand{\sectionformat}[1]{%
    \begin{tikzpicture}[baseline=(title.base)]
        \node[rectangle, draw] (title) {#1};
    \end{tikzpicture}
    
    \noindent\hrulefill
}

\newif\ifhNotes 

\hNotesfalse

\ifhNotes
	\newcommand{\hideNotes}[1]{%
	\phantom{#1}
	}
	\newcommand{\hideNotesU}[1]{%
	\underline{\hspace{1mm}\phantom{#1}\hspace{1mm}}
	}
\else
	\newcommand{\hideNotes}[1]{#1}
	\newcommand{\hideNotesU}[1]{\textcolor{blue}{#1}}
\fi

% default values copied from titlesec documentation page 23
% parameters of \titleformat command are explained on page 4
\titleformat%
    {\section}% <command> is the sectioning command to be redefined, i. e., \part, \chapter, \section, \subsection, \subsubsection, \paragraph or \subparagraph.
    {\normalfont\large\scshape}% <format>
    {}% <label> the number
    {0em}% <sep> length. horizontal separation between label and title body
    {\centering\sectionformat}% code preceding the title body  (title body is taken as argument)

%% Set counters for sections to none
\setcounter{secnumdepth}{0}

%% Set the footer/headers
\pagestyle{fancy}
\fancyhf{}
\renewcommand{\headrulewidth}{0pt}
\renewcommand{\footrulewidth}{2pt}
\lfoot{P.-O. Paris{\'e}}
\cfoot{MATH 311}
\rfoot{Page \thepage}

%% Defining example environment
\newcounter{example}[section]
\NewEnviron{example}%
	{%
	\noindent\refstepcounter{example}\fcolorbox{gray!40}{gray!40}{\textsc{\textcolor{red}{Example~\theexample.}}}%
	%\fcolorbox{black}{white}%
		{  %\parbox{0.95\textwidth}%
			{
			\BODY
			}%
		}%
	}

\newcounter{theorem}
\NewEnviron{theorem}%
	{%
	\noindent\refstepcounter{theorem}\fcolorbox{gray!40}{gray!40}{\textsc{\textcolor{black}{Theorem~\thetheorem.}}}%
	%\fcolorbox{black}{white}%
		{  %\parbox{0.95\textwidth}%
			{
			\BODY
			}%
		}%
	}

\newcounter{definition}[section]
\NewEnviron{definition}%
	{%
	\noindent\refstepcounter{definition}\fcolorbox{gray!40}{gray!40}{\textsc{\textcolor{black}{Definition~\thedefinition.}}}%
	%\fcolorbox{black}{white}%
		{  %\parbox{0.95\textwidth}%
			{
			\BODY
			}%
		}%
	}

\NewEnviron{algorithm}
	{%
	\noindent\refstepcounter{definition}\fcolorbox{gray!40}{gray!40}{\textsc{\textcolor{black}{Algorithm~\thedefinition.}}}%
	%\fcolorbox{black}{white}%
		{  %\parbox{0.95\textwidth}%
			{
			\BODY
			}%
		}%
	}

\NewEnviron{solution}%
	{%
	\noindent \fcolorbox{gray!40}{gray!40}{\textsc{\textcolor{blue}{Solution.}}}%
	%\fcolorbox{black}{white}%
		{  %\parbox{0.95\textwidth}%
			{
			%\textcolor{blue}
			}%
		}%
	}

\NewEnviron{proof}%
	{%
	\noindent \fcolorbox{gray!40}{gray!40}{\textsc{\textcolor{blue}{Proof.}}}%
	%\fcolorbox{black}{white}%
		{  %\parbox{0.95\textwidth}%
			{
			\textcolor{blue}{%
			\BODY
			}
			}%
		}%
	}
%%% Ignorer les notes
%\excludecomment{notes}

%%%%
\begin{document}
\thispagestyle{empty}

\begin{center}
\vspace*{2.5cm}

{\Huge \textsc{Math 311}}

\vspace*{2cm}

{\LARGE \textsc{Chapter 1}} 

\vspace*{0.75cm}

\noindent\textsc{Section 1.3: Homogeneous Equations}

\vspace*{0.75cm}

\tableofcontents

\vfill

\noindent \textsc{Created by: Pierre-Olivier Paris{\'e}} \\
\textsc{Spring 2024}
\end{center}

\newpage

\section{Terminology}

\begin{definition}
A system of linear equations in $x_1$, $\ldots$, $x_n$ is called \textbf{homogeneous} if all the constant terms are zero.
	\begin{itemize}
		\item \textbf{Trivial solution:} $x_1 = 0$, $x_2 = 0$, $\ldots$, $x_n = 0$.
		\item \textbf{Non trivial solution:} Any solution in which at least one variable has a nonzero value.
	\end{itemize}
\end{definition}



\begin{example}\label{Example:HomogeneousSystemSolution}
Show that the following homogeneous system has nontrivial solutions.
	\[
		\systeme{x_1 - x_2 + 2x_3 - x_4 = 0, 2x_1 + 2x_2 + x_4 = 0, 3x_1 + x_2 + 2x_3 - x_4 = 0}
	\]
\end{example}

\begin{solution}

\end{solution}

\vfill

\begin{theorem}
If a homogeneous system of linear equations has more variables than equations, then it has a nontrivial solution (in fact, infinitely many).
\end{theorem}

\newpage 

\section{Linear Combinations}

\vspace*{-0.5cm}

\begin{definition}
\begin{itemize}
	\item An $\mathbf{n}$-\textbf{column vector}: $\mathbf{x} = \begin{bmatrix} x_1 \\ x_2 \\ \vdots \\ x_n \end{bmatrix}$. 
	\item Set of all $n$-column vectors is denoted by $\mathbb{R}^n$.
	\item \textbf{Equality}: $\mathbf{x} = \mathbf{y}$ if $\mathbf{x}$ and $\mathbf{y}$ are of the same size and all entries are the same.
	\item \textbf{Sum} of two $n$-column vectors $\mathbf{x}$,$\mathbf{y}$ is the new $n$-column vector $\mathbf{x} + \mathbf{y}$ obtained by adding corresponding entries.
	\item \textbf{Scalar multiplication} $k\mathbf{x}$ of a $n$-vector $\mathbf{x}$ with a scalar $k$ is obtained by multiplying each entry of $\mathbf{x}$ by $k$.
	\item \textbf{Linear combination}: A sum of scalar multiples of several column vectors.
\end{itemize}
\end{definition}

\begin{example}
If $\mathbf{x} = \begin{bmatrix} 3 \\ -2 \end{bmatrix}$ and $\mathbf{y} = \begin{bmatrix} -1 \\ 1 \end{bmatrix}$, then
	\[
		\mathbf{x} + \mathbf{y} = \begin{bmatrix} 3 - 1 \\ -2 + 1 \end{bmatrix} = \begin{bmatrix} 2 \\ -1 \end{bmatrix} \text{ and } 2 \mathbf{x} = \begin{bmatrix} (2) (3) \\ (2)(-2) \end{bmatrix} = \begin{bmatrix} 6 \\ -4 \end{bmatrix}
	\]
\end{example}

\begin{example}
Let 
	\[
		\mathbf{x} = \begin{bmatrix} 1 \\ 0 \\ 1 \end{bmatrix}, \, 
		\mathbf{y} = \begin{bmatrix} 2 \\ 1 \\ 0 \end{bmatrix} , \,
		\mathbf{z} = \begin{bmatrix} 3 \\ 1 \\ 1 \end{bmatrix}, \,
		\mathbf{v} = \begin{bmatrix} 0 \\ -1 \\ 2 \end{bmatrix} , \,
		\mathbf{w} = \begin{bmatrix} 1 \\ 1 \\ 1 \end{bmatrix} .
	\]
Determine weither $\mathbf{v}$ and $\mathbf{w}$ are linear combinations of $\mathbf{x}$, $\mathbf{y}$, and $\mathbf{z}$.
\end{example}

\begin{solution}

\end{solution}

\newpage 

\phantom{2} 

\newpage 

\section{Basic Solutions}

\underline{Notation:}
	\begin{itemize}
		\item Write $n$ variables $x_1$, $x_2$, $\ldots$, $x_n$ as $\mathbf{x} = \begin{bmatrix} x_1 \\ x_2 \\ \vdots \\ x_n \end{bmatrix}$.
	\end{itemize}

The solution in Example \ref{Example:HomogeneousSystemSolution} can be written as
	\[
		\mathbf{x} = \begin{bmatrix} -t \\ t \\ t \\ 0 \end{bmatrix} = -t \begin{bmatrix} -1 \\ 1 \\ 1 \\ 0 \end{bmatrix} .
	\]

\begin{theorem}
Any linear combination of solutions to a homogeneous system is again a solution.
\end{theorem}

\begin{proof}
Let $\mathbf{x}$ and $\mathbf{y}$ be two different solutions to a homogeneous system. Let $\mathbf{z} = c\mathbf{x} + d\mathbf{y}$. Then, by definition, each component of $\mathbf{z}$ is $c x_j + d y_j$, for each $j$. Plugging that in each equation of the system:
	\begin{align*}
		a_{i1} (cx_1 + dy_1) + a_{i2} (cx_2 + dy_2) + \cdots + a_{in} (cx_n + dy_n) \\ 
		\, = c (a_{i1} x_1 + \cdots + a_{in} x_n) + d (a_{i1} y_1 + \cdots + a_{in} y_n) \\ 
		\, = c (0) + d(0) \qquad \qquad \qquad \qquad \qquad \qquad \qquad \quad \,\,\, \\ 
		\, = 0 \qquad \qquad \qquad \qquad \qquad \qquad \qquad \qquad \qquad \quad \,\,\,\,\,
	\end{align*}
Therefore, $\mathbf{z}$ is a solution to the homogeneous system.
\end{proof}

\newpage 

\begin{example}
Solve the homogeneous system with coefficient matrix
	\[
		A = \begin{bmatrix} 1 & -2 & 3 & -2 \\ -3 & 6 & 1 & 0 \\ -2 & 4 & 4 & -2 \end{bmatrix}
	\]
and express the solution as a linear combination of particular solutions.
\end{example}

\begin{solution}

\end{solution}

\newpage 

\phantom{2}

\vfill

\begin{definition}
The gaussian algorithm systematically produces solutions to any homogeneous systems of linear equations, called \textbf{basic solutions}, one for every parameter.
\end{definition}

Hence, the basic solutions in the previous example are
	\[
		\mathbf{x_1} = \begin{bmatrix} 2 \\ 1 \\ 0 \\ 0 \end{bmatrix} \quad \text{ and } \quad \mathbf{x_2} = \begin{bmatrix} \frac{1}{5} \\ 0 \\ \frac{3}{5} \\ 1 \end{bmatrix} .
	\]


\begin{theorem}
Let $A$ be the coefficient matrix of a homogeneous system of $m$ linear equations in $n$ variables. If $A$ has rank $r$, then
	\begin{enumerate}[label=\arabic*.]
	\item The system has exactly $n - r$ basic solutions, one for each parameter.
	\item Every solution is a linear combination of these basic solutions.
\end{enumerate}
\end{theorem}

\end{document}