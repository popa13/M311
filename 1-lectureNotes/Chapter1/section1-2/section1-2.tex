\documentclass[20pt,a4paper]{extarticle}
\usepackage[utf8]{inputenc}
\usepackage[english]{babel}

\usepackage{amsmath}
\usepackage{amsfonts}
\usepackage{amssymb}
\usepackage{mathtools}
\usepackage{systeme}
\sysdelim..

\usepackage{graphicx}
\usepackage{caption}
\usepackage{subcaption}
\usepackage{lmodern}
\usepackage{tikz}
\usetikzlibrary{calc}
\usepackage{titlesec}
\usepackage{environ}
\usepackage{xcolor}
\usepackage{fancyhdr}
\usepackage[colorlinks = true, linkcolor = black]{hyperref}
\usepackage{xparse}
\usepackage{enumitem}
\usepackage{comment}
\usepackage{wrapfig}
\usepackage[capitalise]{cleveref}

\usepackage[left=2cm,right=2cm,top=2cm,bottom=2cm]{geometry}
\usepackage{multicol}
\usepackage[indent=0pt]{parskip}

\newcommand{\spaceP}{\vspace*{0.5cm}}
\newcommand{\Span}{\mathrm{Span}\,}
\newcommand{\range}{\mathrm{range}\,}
\newcommand{\ra}{\rightarrow}
\newcommand{\curl}{\mathrm{curl} \,}
\newcommand{\hint}[1]{\scalebox{2}{$\displaystyle\int_{\scalebox{0.35}{$#1$}}$}\,}
\newcommand{\hiint}[1]{\scalebox{2}{$\displaystyle\iint_{\scalebox{0.35}{$#1$}}$}\,}
\newcommand{\hiiint}[1]{\scalebox{2}{$\displaystyle\iiint_{\scalebox{0.35}{$#1$}}$}\,}
\renewcommand{\div}{\mathrm{div}\,}

\makeatletter
\renewcommand*\env@matrix[1][*\c@MaxMatrixCols c]{%
  \hskip -\arraycolsep
  \let\@ifnextchar\new@ifnextchar
  \array{#1}}
\makeatother

%% Redefining sections
\newcommand{\sectionformat}[1]{%
    \begin{tikzpicture}[baseline=(title.base)]
        \node[rectangle, draw] (title) {#1};
    \end{tikzpicture}
    
    \noindent\hrulefill
}

\newif\ifhNotes 

\hNotesfalse

\ifhNotes
	\newcommand{\hideNotes}[1]{%
	\phantom{#1}
	}
	\newcommand{\hideNotesU}[1]{%
	\underline{\hspace{1mm}\phantom{#1}\hspace{1mm}}
	}
\else
	\newcommand{\hideNotes}[1]{#1}
	\newcommand{\hideNotesU}[1]{\textcolor{blue}{#1}}
\fi

% default values copied from titlesec documentation page 23
% parameters of \titleformat command are explained on page 4
\titleformat%
    {\section}% <command> is the sectioning command to be redefined, i. e., \part, \chapter, \section, \subsection, \subsubsection, \paragraph or \subparagraph.
    {\normalfont\large\scshape}% <format>
    {}% <label> the number
    {0em}% <sep> length. horizontal separation between label and title body
    {\centering\sectionformat}% code preceding the title body  (title body is taken as argument)

%% Set counters for sections to none
\setcounter{secnumdepth}{0}

%% Set the footer/headers
\pagestyle{fancy}
\fancyhf{}
\renewcommand{\headrulewidth}{0pt}
\renewcommand{\footrulewidth}{2pt}
\lfoot{P.-O. Paris{\'e}}
\cfoot{MATH 311}
\rfoot{Page \thepage}

%% Defining example environment
\newcounter{example}[section]
\NewEnviron{example}%
	{%
	\noindent\refstepcounter{example}\fcolorbox{gray!40}{gray!40}{\textsc{\textcolor{red}{Example~\theexample.}}}%
	%\fcolorbox{black}{white}%
		{  %\parbox{0.95\textwidth}%
			{
			\BODY
			}%
		}%
	}

\newcounter{theorem}
% Theorem environment
\NewEnviron{theorem}%
	{%
	\noindent\refstepcounter{theorem}\fcolorbox{gray!40}{gray!40}{\textsc{\textcolor{black}{Theorem~\thetheorem.}}}%
	%\fcolorbox{black}{white}%
		{  %\parbox{0.95\textwidth}%
			{
			\BODY
			}%
		}%
	}

\newcounter{definition}
\NewEnviron{definition}%
	{%
	\noindent\refstepcounter{definition}\fcolorbox{gray!40}{gray!40}{\textsc{\textcolor{black}{Definition~\thedefinition.}}}%
	%\fcolorbox{black}{white}%
		{  %\parbox{0.95\textwidth}%
			{
			\BODY
			}%
		}%
	}

\NewEnviron{algorithm}%
	{%
	\noindent\refstepcounter{definition}\fcolorbox{gray!40}{gray!40}{\textsc{\textcolor{black}{Algorithm~\thedefinition.}}}%
	%\fcolorbox{black}{white}%
		{  %\parbox{0.95\textwidth}%
			{
			\BODY
			}%
		}%
	}

\NewEnviron{solution}%
	{%
	\noindent \fcolorbox{gray!40}{gray!40}{\textsc{\textcolor{blue}{Solution.}}}%
	%\fcolorbox{black}{white}%
		{  %\parbox{0.95\textwidth}%
			{
			\textcolor{blue}{%
			\BODY
			}
			}%
		}%
	}

\NewEnviron{proof}%
	{%
	\noindent \fcolorbox{gray!40}{gray!40}{\textsc{\textcolor{blue}{Proof.}}}%
	%\fcolorbox{black}{white}%
		{  %\parbox{0.95\textwidth}%
			{
			\textcolor{blue}{%
			\BODY
			}
			}%
		}%
	}
%%% Ignorer les notes
%\excludecomment{notes}

%%%%
\begin{document}
\thispagestyle{empty}

\begin{center}
\vspace*{2.5cm}

{\Huge \textsc{Math 311}}

\vspace*{2cm}

{\LARGE \textsc{Chapter 1}} 

\vspace*{0.75cm}

\noindent\textsc{Section 1.2: Gaussian Elimination}

\vspace*{0.75cm}

\tableofcontents

\vfill

\noindent \textsc{Created by: Pierre-Olivier Paris{\'e}} \\
\textsc{Spring 2024}
\end{center}

\newpage

\section{Matrix and Augmented Matrix}

The system
	\[
		\systeme{x + y + z = 1, 2x + 2y  +z = 3}
	\]
can be put in \textbf{matrix form} (array of numbers):
	\[
		\begin{bmatrix}[ccc|r] 
		1 & 1 & 1 & 1 \\ 
		2 & 2 & 1 & 3 
		\end{bmatrix}
	\]

\begin{definition}
\begin{itemize}
	\item The matrix $\begin{bmatrix} 1 & 1 & 1 \\ 2 & 2 & 1 \end{bmatrix}$ is called the \textbf{coefficient matrix} of the system.
	\item The matrix $\begin{bmatrix} 1 \\ 3 \end{bmatrix}$ is called the \textbf{constant matrix} of the system.
\end{itemize}
\end{definition}

\vspace*{20pt}

Solving a system of linear equations requires to transform the coefficient matrix in the following form:
	\[
		\begin{bmatrix}
		1 & * & * \\ 0 & 1 & *
		\end{bmatrix}
		\quad \text{or} \quad 
		\begin{bmatrix}
		1 & * & * \\ 0 & 0 & 1
		\end{bmatrix} 
		\quad \text{or} \quad 
		\begin{bmatrix}
		1 & * & * \\ 
		0 & 0 & 0 
		\end{bmatrix} 
	\]
where $*$ denotes any real number.

\newpage 

\section{Row-Echelon Form of a matrix}

\begin{definition}
A matrix is said to be in \textbf{row-echelon form} (REF for short) if it satisfies the following three conditions:
	\begin{enumerate}[label=\arabic*.]
	\item All \textbf{zero rows} are at the bottom.
	\item The first nonzero entry from the left in each nonzero row is $1$, called the \textbf{leading} $\mathbf{1}$ for that row.
	\item Each leading $1$ is to the right of all leading $1$s in the rows above it
\end{enumerate}
\end{definition}

\begin{definition}
A row-echelon matrix is said to be in \textbf{reduced row-echelon form} (RREF for short) if, in addition, it satisfies the following condition:
	\begin{enumerate}
		\item[4.] Each leading $1$ is the only nonzero entry in its column.
	\end{enumerate}
\end{definition}

\vspace*{12pt}

\begin{example}
Circle the matrices in REF. Draw a square around the matrices in RREF.
\begin{align*}
\begin{bmatrix} 
1 & * & * \\ 0 & 0 & 1
\end{bmatrix}
\quad \begin{bmatrix} 1 & 2 & 3 & 4 \\ 0 & 1 & 2 & 2 \\ 0 & 0 & 0 & 1 \end{bmatrix}
\quad \begin{bmatrix} 1 & 0 & 0 & 3 \\ 0 & 0 & 1 & 4 \\ 0 & 0 & 0 & 0 \end{bmatrix} 
\quad \begin{bmatrix} 1 & 2 & 3 & 4 \\ 3 & 2 & 0 & 3 \\ 0 & 1 & 0 & 0 \end{bmatrix}
\end{align*}
\end{example}

\begin{theorem}
Every matrix can be transformed, with row operations, to a REF (or RREF).
\end{theorem}

\newpage

\begin{algorithm}\underline{\textbf{Gaussian Elinination.}}
To solve a system of linear equations proceed as follows.
	\begin{enumerate}[label=\arabic*.]
	\item Carry the augmented matrix to a reduced row-echelon form using elementary row operations.
	\item If a row $\begin{bmatrix} 0 & 0 & \ldots & 0 & 1 \end{bmatrix}$ occurs, the system is inconsistent.
	\item Otherwise, find the parametric form of the solution set to the system of equations.
	\end{enumerate}
\end{algorithm}

\begin{example}
Solve the following system:
	\[
		\systeme{3x + y - 4z = -1, x + 10z = 5, 4x + y + 6z = 1}
	\]
\end{example}

\begin{solution}

\end{solution}

\newpage 

\phantom{2} 

\newpage 

\begin{example}
Solve the following equation:
	\[
		\systeme{x_1 - 2x_2 - x_3 + 3x_4 = 1, 2x_1 - 4x_2 + x_3 = 5, x_1 - 2x_2 + 2x_3 - 3x_4 = 4}
	\]
\end{example}
\begin{solution}

\end{solution}

\newpage 

\phantom{2} 

\vfill 

\underline{Note:} The variable corresponding to the leading ones are called \textbf{leading variables}.

\newpage 

\section{Rank}

It can be proved that the number $r$ of leading $1$s must be the same in each of the row-echelon matrices.

\begin{definition}
The \textbf{rank} of a matrix $A$ is the number of leading $1$s in any REF to which $A$ can be carried by row operations. 
\end{definition}

\vspace*{20pt}

\begin{example}
Compute the rank of 
	\[
		A = \begin{bmatrix} 1 & 1 & -1 & 4 \\ 2 & 1 & 3 & 0 \\ 0 & 1 & -5 & 8 \end{bmatrix}
	\]
\end{example}

\begin{solution}

\end{solution}

\newpage 

\begin{theorem}
Suppose a system of $m$ equations in $n$ variables is consistent, and that the rank of the coefficient matrix is $r$.
	\begin{enumerate}[label=\arabic*.]
	\item The set of solutions involves $n - r$ parameters.
	\item If $r < n$, the system has infinitely many solutions.
	\item If $r = n$, the system has a unique solution.
	\end{enumerate}
\end{theorem}

\vspace*{20pt}

Three situations occur:
	\begin{itemize}
		\item No solution. 
		\item Unique solution.
		\item Infinitely many solutions.
	\end{itemize}

\begin{example}
A system of equation with $m = 4$ linear equations and $n = 5$ variables has been  carried to the following REF by row operations:
	\[
		\begin{bmatrix}[ccccc|r]
		1 & 2 & 1 & 3 & 1 & 1 \\ 
		0 & 1 & -1 & 0 & 1 & 1 \\ 
		0 & 0 & 0 & 1 & -1 & 0 \\ 
		0 & 0 & 0 & 0 & 0 & 0
		\end{bmatrix}
	\]
(a) Find the rank of the coefficient matrix. (b) Is there no solution, unique solution, or infinitely many solutions?
\end{example}

\begin{solution}

\end{solution}

\end{document}