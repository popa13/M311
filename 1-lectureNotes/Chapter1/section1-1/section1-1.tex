\documentclass[20pt,a4paper]{extarticle}
\usepackage[utf8]{inputenc}
\usepackage[english]{babel}

\usepackage{amsmath}
\usepackage{amsfonts}
\usepackage{amssymb}
\usepackage{mathtools}
\usepackage{systeme}
\sysdelim..

\usepackage{graphicx}
\usepackage{caption}
\usepackage{subcaption}
\usepackage{lmodern}
\usepackage{tikz}
\usetikzlibrary{calc}
\usepackage{titlesec}
\usepackage{environ}
\usepackage{xcolor}
\usepackage{fancyhdr}
\usepackage[colorlinks = true, linkcolor = black]{hyperref}
\usepackage{xparse}
\usepackage{enumitem}
\usepackage{comment}
\usepackage{wrapfig}
\usepackage[capitalise]{cleveref}

\usepackage[left=2cm,right=2cm,top=2cm,bottom=2cm]{geometry}
\usepackage{multicol}
\usepackage[indent=0pt]{parskip}

\newcommand{\spaceP}{\vspace*{0.5cm}}
\newcommand{\Span}{\mathrm{Span}\,}
\newcommand{\range}{\mathrm{range}\,}
\newcommand{\ra}{\rightarrow}
\newcommand{\curl}{\mathrm{curl} \,}
\newcommand{\hint}[1]{\scalebox{2}{$\displaystyle\int_{\scalebox{0.35}{$#1$}}$}\,}
\newcommand{\hiint}[1]{\scalebox{2}{$\displaystyle\iint_{\scalebox{0.35}{$#1$}}$}\,}
\newcommand{\hiiint}[1]{\scalebox{2}{$\displaystyle\iiint_{\scalebox{0.35}{$#1$}}$}\,}
\renewcommand{\div}{\mathrm{div}\,}

\makeatletter
\renewcommand*\env@matrix[1][*\c@MaxMatrixCols c]{%
  \hskip -\arraycolsep
  \let\@ifnextchar\new@ifnextchar
  \array{#1}}
\makeatother

%% Redefining sections
\newcommand{\sectionformat}[1]{%
    \begin{tikzpicture}[baseline=(title.base)]
        \node[rectangle, draw] (title) {#1};
    \end{tikzpicture}
    
    \noindent\hrulefill
}

\newif\ifhNotes 

\hNotesfalse

\ifhNotes
	\newcommand{\hideNotes}[1]{%
	\phantom{#1}
	}
	\newcommand{\hideNotesU}[1]{%
	\underline{\hspace{1mm}\phantom{#1}\hspace{1mm}}
	}
\else
	\newcommand{\hideNotes}[1]{#1}
	\newcommand{\hideNotesU}[1]{\textcolor{blue}{#1}}
\fi

% default values copied from titlesec documentation page 23
% parameters of \titleformat command are explained on page 4
\titleformat%
    {\section}% <command> is the sectioning command to be redefined, i. e., \part, \chapter, \section, \subsection, \subsubsection, \paragraph or \subparagraph.
    {\normalfont\large\scshape}% <format>
    {}% <label> the number
    {0em}% <sep> length. horizontal separation between label and title body
    {\centering\sectionformat}% code preceding the title body  (title body is taken as argument)

%% Set counters for sections to none
\setcounter{secnumdepth}{0}

%% Set the footer/headers
\pagestyle{fancy}
\fancyhf{}
\renewcommand{\headrulewidth}{0pt}
\renewcommand{\footrulewidth}{2pt}
\lfoot{P.-O. Paris{\'e}}
\cfoot{MATH 311}
\rfoot{Page \thepage}

%% Defining example environment
\newcounter{example}[section]
\NewEnviron{example}%
	{%
	\noindent\refstepcounter{example}\fcolorbox{gray!40}{gray!40}{\textsc{\textcolor{red}{Example~\theexample.}}}%
	%\fcolorbox{black}{white}%
		{  %\parbox{0.95\textwidth}%
			{
			\BODY
			}%
		}%
	}

\newcounter{theorem}
% Theorem environment
\NewEnviron{theorem}%
	{%
	\noindent\refstepcounter{theorem}\fcolorbox{gray!40}{gray!40}{\textsc{\textcolor{black}{Theorem~\thetheorem.}}}%
	%\fcolorbox{black}{white}%
		{  %\parbox{0.95\textwidth}%
			{
			\BODY
			}%
		}%
	}

\newcounter{definition}
\NewEnviron{definition}%
	{%
	\noindent\refstepcounter{definition}\fcolorbox{gray!40}{gray!40}{\textsc{\textcolor{black}{Definition~\thedefinition.}}}%
	%\fcolorbox{black}{white}%
		{  %\parbox{0.95\textwidth}%
			{
			\BODY
			}%
		}%
	}

\NewEnviron{solution}%
	{%
	\noindent \fcolorbox{gray!40}{gray!40}{\textsc{\textcolor{blue}{Solution.}}}%
	%\fcolorbox{black}{white}%
		{  %\parbox{0.95\textwidth}%
			{
			\textcolor{blue}{%
			\BODY
			}
			}%
		}%
	}

\NewEnviron{proof}%
	{%
	\noindent \fcolorbox{gray!40}{gray!40}{\textsc{\textcolor{blue}{Proof.}}}%
	%\fcolorbox{black}{white}%
		{  %\parbox{0.95\textwidth}%
			{
			\textcolor{blue}{%
			\BODY
			}
			}%
		}%
	}
%%% Ignorer les notes
%\excludecomment{notes}

%%%%
\begin{document}
\thispagestyle{empty}

\begin{center}
\vspace*{2.5cm}

{\Huge \textsc{Math 311}}

\vspace*{2cm}

{\LARGE \textsc{Chapter 1}} 

\vspace*{0.75cm}

\noindent\textsc{Section 1.1: Systems of Linear Equations}

\vspace*{0.75cm}

\tableofcontents

\vfill

\noindent \textsc{Created by: Pierre-Olivier Paris{\'e}} \\
\textsc{Spring 2024}
\end{center}

\newpage

\section{Terminology}

\begin{definition}

\begin{itemize}
	\item An equation of the form $a_1 x_1 + a_2 x_2 + \ldots + a_n x_n$ is called a \textbf{linear equation} in the $n$ variables $x_1$, $x_2$, $\ldots$, $x_n$.
	\item $a_1$, $\ldots$, $a_n$ are fixed real numbers called the \textbf{coefficients}.
	\item $b$ is a fixed real number called the \textbf{constant term}.
	\item A \textit{finite} collection of linear equations is called a \textbf{system of linear equations}.
\end{itemize}

\end{definition}

\begin{definition}

\begin{itemize}
	\item Given a linear equation $a_1 x_1 + a_2 x_2 + \cdots + a_n x_n = b$, a list $s_1$, $s_2$, $\ldots$, $s_n$ of $n$ numbers is called a \textbf{solution} to the equation if
		\[
			a_1 s_1 + a_2 s_2 + \ldots + a_n s_n = b .
		\]
	\item A list $s_1$, $s_2$, $\ldots$, $s_n$ of $n$ numbers is a \textbf{solution to a system} of linear equations if it is a solution of every linear equation of the system.
	\item Two systems of linear equations are \textbf{equivalent} if they have the same set of solutions.
\end{itemize}

\end{definition}

\newpage 

\begin{example}
Consider the following system:
	\[
		\systeme{x_1 - 2x_2 + 3x_3 + x_4 = -3, 2x_1 - x_2 + 3x_3 - x_4 = 0} .
	\]
(a) Show that $x_1 = 1$, $x_2 = 2$, $x_3=0$, and $x_4 = 0$ is a solution. 

(b) Show that, for arbitrary values of $s$ and $t$, $x_1 = t - s + 1$, $x_2 = t + s + 2$, $x_3 = s$, and $x_4 = t$ is a solution.	
\end{example}

\begin{solution}

\end{solution}

\vfill 

\begin{definition}

\begin{itemize}
	\item The quantities $s$ and $t$ are called \textbf{parameters}.
	\item The set of solutions, described with parameters, is said to be given in \textbf{parametric form} and is called the \textbf{general solution}.
\end{itemize}

\end{definition}

\newpage 


\subsection*{Geometric interpretations}

An equation in $2$ variables (namely $x_1 = x$ and $x_2 = y$) can be drawn in a cartesian plane.

Check out Desmos: \\ 
\url{https://www.desmos.com/calculator/dbnumvofgs}.

Three alternatives:
	\begin{itemize}
		\item The system has a unique solution (the lines intersect at a single point).
		\item The system has no solution.
			%\begin{itemize}
			%	\item Lines are parallel and distinct or;
			%	\item Lines do not cross in a single point.
			%\end{itemize}
		\item The system has infinitely many solutions (the lines are identical). 
	\end{itemize}

In general:
	\begin{itemize}
		\item If the system has at least one solution, the system is called \textbf{consistent}.
		\item If the system has no solution, the system is called \textbf{inconsistent}.
	\end{itemize}

\subsection*{General Presentation}
A system of $m$ linear equations in $n$ variables:
\begin{align*}
	a_{11} x_1 + a_{12} x_2 + \ldots + a_{1n} x_n &= b_1 \\ 
	a_{21} x_1 + a_{22} x_2 + \ldots + a_{2n} x_n &= b_2 \\ 
	\qquad \qquad \vdots \qquad \qquad & \\ 
	a_{m1} x_1 + a_{m2} x_2 + \ldots + a_{mn} x_n &= b_m
\end{align*}

\newpage

\section{Elementary Operations}

\underline{Goal:} Manipulate the equations in the system to reduce it to another simpler system with the same set of solutions.

\begin{example}\label{Example:FirstSystem}
Solve the system $2x + y = 7$, $x + 2y = -2$.
%\begin{enumerate}
	%\item Write the system as its \textbf{augmented matrix}:
	%\begin{small}
	%	\[
	%		\begin{matrix} 2x + y &=& 7 \\ x + 2y &=& -2 \end{matrix} \quad \longleftrightarrow \quad \begin{bmatrix}[cc|r] 2 & \,\, 1 & \, \phantom{-} 7 \\ 1 & \,\, 2 & \, -2  \end{bmatrix}
	%	\]
	%\end{small}
	%\item Manipulate the rows of the augmented matrix:
	%\begin{small}
	%\begin{align*}
	%\systeme{2x + y = 7, x + 2y = -2} & \longleftarrow 
	%\end{align*}
	%\end{small}
	%\begin{small}
%\begin{align*}
%\begin{bmatrix}[cc|r] 2 & \,\, 1 & \, 7 \\ 1 & \,\, 2 & \, -2 \end{bmatrix}
% \quad & \longrightarrow \quad \begin{bmatrix}[cc|r] 1 & \,\, 2 & \, -2 \\ 2 & \,\, 1 & \, \phantom{-} 7 \end{bmatrix} \qquad \begin{matrix}  \scriptstyle R_1 \longleftrightarrow R_2 \\ \phantom{2} \end{matrix} \\ 
%\quad & \longrightarrow \quad \begin{bmatrix}[cc|r] 1 & \,\, 2 & \, -2 \\ 0 & \,\, 3 & \, -11 \end{bmatrix} \qquad \begin{matrix} \phantom{2} \\ \scriptstyle 2R_1 - R_2 \to R_2 \end{matrix} \\ 
%\quad & \longrightarrow \quad \begin{bmatrix}[cc|r] 1 & \,\, 2 & \, -2 \\ 0 & \,\, 1 & \, -\frac{11}{3} \end{bmatrix} \qquad \begin{matrix} \phantom{2} \\ \scriptstyle (1/3) R_2 \to R_2 \end{matrix} \\ 
%\quad & \longrightarrow \quad \begin{bmatrix}[cc|r] 1 & \,\, 0 & \, \frac{16}{3} \\ 0 & \,\, 1 & \,\, - \frac{11}{3} \end{bmatrix} \qquad \begin{matrix} \scriptstyle 2R_2 - R_1 \to R_1 \\ \phantom{2} \end{matrix}
%\end{align*}
%\end{small}
%Reading off the augmented matrix:
%	\[
%		\begin{matrix} x &=& 16/3 \\ y &=& -11/3 \end{matrix}
%	\]
%\end{enumerate}
%The starting system is equivalent to the original because the operations didn't change the set of solutions.
\end{example}


\newpage 

\subsubsection*{Three Types of Elementary Operations}

\begin{enumerate}[label=\textbf{\Roman*.}]
\item Interchange two equations.
\item Multiply one equation by a nonzero number.
\item Add a multiple of one equation to a different equation.
\end{enumerate}

\begin{theorem}
Suppose that a sequence of elementary operations is performed on a system of linear equations. Then the resulting system has the same set of solutions as the original, so the two systems are equivalent.
\end{theorem}

\subsubsection*{A Little Shortcut}

\begin{definition}
The \textbf{augmented matrix} of a system is an array of numbers where each row is obtained from each equation by removing the variable.
\end{definition}

\begin{example}
Find the augmented matrix associated to the system in Example \ref{Example:FirstSystem}.
\end{example}

\begin{solution}

\end{solution}

\vfill 

\underline{Elementary operations translate to:}
\begin{enumerate}[label=\textbf{\Roman*.}]
\item Interchange two rows.
\item Multiply one row by a nonzero number.
\item Add a multiple of one row to a different row.
\end{enumerate}

\begin{comment}\begin{proof}
Assume $x_1 = s_1$, $\ldots$, $x_n = s_n$ is a solution to the system and a type III operation was performed to transform $E_1 = b_1$ into $E_1 + k E_2 = b_1 + k b_2$. Since $s_1, s_2$, $\ldots$, $s_n$ solve the equations $E_1 = b_1$ and $E_2 = b_2$, it must also solve $E_1 + k E_2 = b_1 + kb_2$.

Assume now that $x_1 = s_1$, $\ldots$, $x_n = s_n$ is a solution to the system with $E_1 = b_1$ replaced by $E_1 + k E_2 = b_1 + k b_2$. Then, replacing $E_1 + kE_2 = b_1 + kb_2$ by $(E_1 + kE_2) - k E_2 = (b_1 + kb_2) - k b_2$, we get $E_1 = b_1$. From the previous paragraph, $s_1, s_2$, $\ldots$, $s_n$ is a solution to $E_1 = b_1$. \hfill $\square$
\end{proof}
\end{comment}

\newpage

\begin{example}
Find all solutions to the following system of equations using elementary operations on the augmented matrix.
	\[
		\systeme{3x + 4y + z = \phantom{-}1, 2x + 3y = \phantom{-} 0, 4x + 3y - z = -2}
	\]
\end{example}

\begin{solution}

\end{solution}

\newpage 

\phantom{2}

\vfill 


\textbf{Note:} Any row operation can be reversed.
	\begin{enumerate}[label=\textbf{\Roman*.}]
		\item Interchanging two rows is reversed by interchanging them again.
		\item Multiplying a row by $k \neq 0$ is reversed by multiplying by $1/k$.
		\item Adding $k$ times row $p$ to a different row $q$ is reversed by adding $-k$ times row $p$ to the new row $q$. 
	\end{enumerate}



\end{document}



\begin{align*}
	\systeme{2x + y = 7, x + 2y = -2} &\longrightarrow \systeme{x + 2y = -2, 2x + y = 7}  \quad \begin{matrix} \scriptstyle R_1 \leftrightarrow R_2 \\ \phantom{2} \end{matrix}\\ 
	 &\longrightarrow \systeme{x + 2y = -2 , 3y = -11}  \quad \begin{matrix} \phantom{2} \\ \scriptstyle 2R_1 - R_2 \to R_2 \end{matrix} \\ 
	& \longrightarrow \systeme{x + 2y = 2, y = -\frac{11}{3}} \quad \begin{matrix} \phantom{2} \\ \scriptstyle (1/3)R_2 \to R_2 \end{matrix} \\ 
	& \longrightarrow \systeme{x = \frac{16}{3} , y = - \frac{11}{3}} \quad \begin{matrix} \scriptstyle 2R_2 - R_1 \to R_1 \\ \phantom{2} \end{matrix} 
\end{align*}

\begin{matrix}
2x + y & = & 7 \\ 
x + 2y & = & -2
\end{matrix} \quad & \longrightarrow \quad \begin{matrix} x + 2y &=& -2 \\ 2x + y &=& 7 \end{matrix} \qquad \begin{matrix}  \scriptstyle R_1 \longleftrightarrow R_2 \\ \phantom{2} \end{matrix} \\ 
\quad & \longrightarrow \quad \begin{matrix} x + 2y & = & -2 \\ \phantom{x +} 3y & = & -11 \end{matrix} \qquad \begin{matrix} \phantom{2} \\ \scriptstyle 2R_1 - R_2 \to R_2 \end{matrix} \\ 
\quad & \longrightarrow \quad \begin{matrix} x + 2y & = & -2 \\ \phantom{x +} y & = & - \frac{11}{3} \end{matrix} \qquad \begin{matrix} \phantom{2} \\ \scriptstyle (1/3) R_2 \to R_2 \end{matrix} \\ 
\quad & \longrightarrow \quad \begin{matrix} x & = & \frac{16}{3} \\ y & = & - \frac{11}{3} \end{matrix} \qquad \begin{matrix} \scriptstyle 2R_2 - R_1 \to R_1 \\ \phantom{2} \end{matrix} \tag*{$\triangle$}