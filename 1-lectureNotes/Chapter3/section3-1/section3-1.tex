\documentclass[20pt,a4paper]{extarticle}
\usepackage[utf8]{inputenc}
\usepackage[english]{babel}

\usepackage{amsmath}
\usepackage{amsfonts}
\usepackage{amssymb}
\usepackage{mathtools}
\usepackage{systeme}
\sysdelim..

\usepackage{graphicx}
\usepackage{caption}
\usepackage{subcaption}
\usepackage{lmodern}
\usepackage{tikz}
\usetikzlibrary{calc}
\usepackage{titlesec}
\usepackage{environ}
\usepackage{xcolor}
\usepackage{fancyhdr}
\usepackage[colorlinks = true, linkcolor = black]{hyperref}
\usepackage{xparse}
\usepackage{enumitem}
\usepackage{comment}
\usepackage{wrapfig}
\usepackage{soul}
\usepackage[capitalise]{cleveref}

\usepackage[left=1cm,right=1cm,top=1cm,bottom=3cm]{geometry}
\usepackage{multicol}
\usepackage[indent=0pt]{parskip}

\newcommand{\spaceP}{\vspace*{0.5cm}}
\newcommand{\Span}{\mathrm{Span}\,}
\newcommand{\range}{\mathrm{range}\,}
\newcommand{\ra}{\rightarrow}
\newcommand{\curl}{\mathrm{curl} \,}
\newcommand{\hint}[1]{\scalebox{2}{$\displaystyle\int_{\scalebox{0.35}{$#1$}}$}\,}
\newcommand{\hiint}[1]{\scalebox{2}{$\displaystyle\iint_{\scalebox{0.35}{$#1$}}$}\,}
\newcommand{\hiiint}[1]{\scalebox{2}{$\displaystyle\iiint_{\scalebox{0.35}{$#1$}}$}\,}
\renewcommand{\div}{\mathrm{div}\,}

\makeatletter
\renewcommand*\env@matrix[1][*\c@MaxMatrixCols c]{%
  \hskip -\arraycolsep
  \let\@ifnextchar\new@ifnextchar
  \array{#1}}
\makeatother

%% Redefining sections
\newcommand{\sectionformat}[1]{%
    \begin{tikzpicture}[baseline=(title.base)]
        \node[rectangle, draw] (title) {#1};
    \end{tikzpicture}
    
    \noindent\hrulefill
}

\newif\ifhNotes 

\hNotesfalse

\ifhNotes
	\newcommand{\hideNotes}[1]{%
	\phantom{#1}
	}
	\newcommand{\hideNotesU}[1]{%
	\underline{\hspace{1mm}\phantom{#1}\hspace{1mm}}
	}
\else
	\newcommand{\hideNotes}[1]{#1}
	\newcommand{\hideNotesU}[1]{\textcolor{blue}{#1}}
\fi

% default values copied from titlesec documentation page 23
% parameters of \titleformat command are explained on page 4
\titleformat%
    {\section}% <command> is the sectioning command to be redefined, i. e., \part, \chapter, \section, \subsection, \subsubsection, \paragraph or \subparagraph.
    {\normalfont\large\scshape}% <format>
    {}% <label> the number
    {0em}% <sep> length. horizontal separation between label and title body
    {\centering\sectionformat}% code preceding the title body  (title body is taken as argument)

%% Set counters for sections to none
\setcounter{secnumdepth}{0}

%% Set the footer/headers
\pagestyle{fancy}
\fancyhf{}
\renewcommand{\headrulewidth}{0pt}
\renewcommand{\footrulewidth}{2pt}
\lfoot{P.-O. Paris{\'e}}
\cfoot{MATH 311}
\rfoot{Page \thepage}

%% Defining example environment
\newcounter{example}
\NewEnviron{example}%
	{%
	\noindent\refstepcounter{example}\fcolorbox{gray!40}{gray!40}{\textsc{\textcolor{red}{Example~\theexample.}}}%
	%\fcolorbox{black}{white}%
		{  %\parbox{0.95\textwidth}%
			{
			\BODY
			}%
		}%
	}

\newcounter{theorem}
\NewEnviron{theorem}%
	{%
	\noindent\refstepcounter{theorem}\fcolorbox{gray!40}{gray!40}{\textsc{\textcolor{black}{Theorem~\thetheorem.}}}%
	%\fcolorbox{black}{white}%
		{  %\parbox{0.95\textwidth}%
			{
			\BODY
			}%
		}%
	}

\newcounter{definition}
\NewEnviron{definition}%
	{%
	\noindent\refstepcounter{definition}\fcolorbox{gray!40}{gray!40}{\textsc{\textcolor{black}{Definition~\thedefinition.}}}%
	%\fcolorbox{black}{white}%
		{  %\parbox{0.95\textwidth}%
			{
			\BODY
			}%
		}%
	}

\newcounter{algo}
\NewEnviron{algorithm}
	{%
	\noindent\refstepcounter{algo}\fcolorbox{gray!40}{gray!40}{\textsc{\textcolor{black}{Algorithm~\thealgo.}}}%
	%\fcolorbox{black}{white}%
		{  %\parbox{0.95\textwidth}%
			{
			\BODY
			}%
		}%
	}

\NewEnviron{goal}
	{%
	\noindent\fcolorbox{gray!40}{gray!40}{\textsc{\textcolor{black}{Goal:}}}%
	%\fcolorbox{black}{white}%
		{  %\parbox{0.95\textwidth}%
			{
			\BODY
			}%
		}%
	}

\NewEnviron{solution}%
	{%
	\noindent \fcolorbox{gray!40}{gray!40}{\textsc{\textcolor{blue}{Solution.}}}%
	%\fcolorbox{black}{white}%
		{  %\parbox{0.95\textwidth}%
			{
			%\textcolor{blue}
			}%
		}%
	}

\NewEnviron{proof}%
	{%
	\noindent \fcolorbox{gray!40}{gray!40}{\textsc{\textcolor{blue}{Proof.}}}%
	%\fcolorbox{black}{white}%
		{  %\parbox{0.95\textwidth}%
			{
			\textcolor{blue}{%
			\BODY
			}
			}%
		}%
	}
%%% Ignorer les notes
%\excludecomment{notes}

%%%%
\begin{document}
\thispagestyle{empty}

\begin{center}
\vspace*{0.5cm}

{\Huge \textsc{Math 311}}

\vspace*{1cm}

{\LARGE \textsc{Chapter 3}} 

\vspace*{0.75cm}

\noindent\textsc{Section 3.1: The Cofactor Expansion}

\vspace*{0.25cm}

\tableofcontents

\vfill

\noindent \textsc{Created by: Pierre-Olivier Paris{\'e}} \\
\textsc{Spring 2024}
\end{center}

\newpage

\section{Goal}

Recall that if $A = \begin{bmatrix} a & b \\ c & d \end{bmatrix}$, then $\det A = ad - bc$
and $A$ is invertible if and only if $\det A \neq 0$.

\begin{goal}
To Generalize the determinant to $n \times n$ matrix.
\end{goal}

\section{A Basic Example}

If $A$ is a $3 \times 3$ square matrix and if $A$ is invertible, then we know $A$ can be carried to the identity matrix $I$. 

Following the process of $A \ra I$:
\begin{align*}
A = \begin{bmatrix} a & b & c \\ d & e & f \\ g & h & i \end{bmatrix} \ra \begin{bmatrix} a & b & c \\ ad & ae & af \\ ag & ah & ai \end{bmatrix} \ra \begin{bmatrix} a & b & c \\ 0 & ae - bd & af - cd \\ 0 & ah - bg & ai - cg \end{bmatrix}
\end{align*}

Set $u = ae - bd$ and $v = ah - bg$. Then
\begin{align*}
 \begin{bmatrix} a & b & c \\ 0 & u & af - cd \\ 0 & v & ai - cg \end{bmatrix} \ra \begin{bmatrix} a & b & c \\ 0 & u & af - cd \\ 0 & vu & u (ai - cg) \end{bmatrix} \ra \begin{bmatrix} a & b & c \\ 0 & vu & v(af - cd) \\ 0 & 0 & w \end{bmatrix}
 \end{align*}
$w = u (ai - cg) - v(af - cd)$. Hence, if we want to carry on the algorithm, we need that
	\[
		w \neq 0 
	\]

\newpage 

\begin{definition}
If $A$ is a $3 \times 3$ matrix, then
	\[
		\det A := w = aei + bfg + cdh - ceg - afh - bdi .
	\]
\end{definition}

\underline{Remark:}
	\begin{itemize}
		\item Notice that $A$ is invertible if and only if $\det A \neq 0$.
		\item Notice that
			\begin{align*}
				\det A &= a (ei - fh) - b (di - fg) + c (dh - eg) \\ 
				&= a \det \begin{bmatrix} e & f \\ h & i \end{bmatrix} - b \det \begin{bmatrix} d & f \\ g & i \end{bmatrix} + c \det \begin{bmatrix} d & e \\ g & h \end{bmatrix} .
			\end{align*}
		\item The terms $+a \det \begin{bmatrix} e & f \\ h & i \end{bmatrix}$, $- b \det \begin{bmatrix} d & f \\ g & i \end{bmatrix}$ and $+c \det \begin{bmatrix} d & e \\ g & h \end{bmatrix}$ are called \textbf{cofactors} of $A$ and are denoted by $c_{11} (A)$, $c_{12} (A)$ and $c_{13} (A)$ respectively.
	\end{itemize}

\begin{example}
Compute the determinant of $A = \begin{bmatrix} 2 & 3 & 1 \\ 1 & 0 & 1 \\ 2 & 1 & 0 \end{bmatrix}$.
\end{example}

\begin{solution}

\end{solution}

\newpage 

\section{Cofactors of A Matrix} 
Notice that
	\[
		c_{12} (A) = (-1)^{1 + 2} \det \begin{bmatrix} a & b & c \\ d & e & f \\ g & h & i \end{bmatrix}  \phantom{= (-1)^{1 + 2} \det \begin{bmatrix} d & f \\ \\ g & i \end{bmatrix}}
	\]

We denote by $A_{ij}$ the $(n - 1) \times (n - 1)$ matrix obtained from $A$ by deleting row $i$ and column $j$.

\begin{definition}
Let $A$ be an $n \times n$ matrix. The $\mathbf{(i, j)}$\textbf{-cofactor} $c_{ij} (A)$ is the scalar defined by
	\[
	 	c_{ij} (A) = (-1)^{i + j} \det (A_{ij}) .
	\] 
Here, $(-1)^{i + j}$ is called the \textbf{sign} of the $(i, j)$-position.
\end{definition}

\begin{example}
Find the cofactors of positions $(3,2)$ of $$A = \begin{bmatrix} 2 & 3 & 1 \\ 1 & 0 & 1 \\ 2 & 1 & 0 \end{bmatrix}.$$
\end{example}

\newpage 

\section{Definition of the Determinant}

\begin{definition}
Let $A = [a_{ij}]$ be an $n \times n$ matrix. The \textbf{determinant} of $A$ is defined by
	\[
		\det A = a_{11} c_{11} (A) + a_{12} c_{12} (A) + \cdots + a_{1n} c_{1n} (A) .
	\]
\end{definition}

\underline{Remark:} This is called the \textbf{cofactor expansion} of $\det A$ along row $1$.

\begin{example}
compute the determinant of $A = \begin{bmatrix} 3 & 4 & 5 & 0 \\ 1 & 7 & 2 & 0\\ 9 & 8 & -6 & 3 \\ 1 & 1 & 1 & 1 \end{bmatrix}$.
\end{example}

\begin{solution}

\end{solution}

\newpage 

\begin{theorem}[Proved by Pierre-Simon de Laplace (1749-1827)]\\
The determinant of an $n \times n$ matrix $A$ can be computed by using the cofactor expansion along any row or column of $A$.
\end{theorem}

\begin{example}
Compute $\det A$ if $A = \begin{bmatrix} 1 & 1 & 1 & 1 \\ 0 & 1 & 1 & 1 \\ 0 & 0 & 1 & 1 \\ 0 & 2 & 3 & 1 \end{bmatrix}$.
\end{example}

\begin{solution}

\end{solution}

\newpage 

\section{Determinant and Row Operations}

\subsubsection{Interchanging two rows}

\begin{example}
Show that 
	\[
		\det \begin{bmatrix} 
		1 & 0 & 1 \\ 1 & 0 & 0 \\ 0 & 1 & 0
		\end{bmatrix} = - \det \begin{bmatrix} 1 & 0 & 0 \\ 1 & 0 & 1 \\ 0 & 1 & 0 \end{bmatrix} .
	\]
\end{example}

\begin{solution}

\end{solution}

\vfill 

\begin{theorem}
If $B$ is an $n \times n$ matrix obtained from interchanging two rows of an $n \times n$ matrix $A$, then
	\[
		\det (B) = - \det (A) .
	\]
\end{theorem}

\underline{Remark:} This fact is still true if we interchange two \textit{columns} (instead of rows).

\newpage 

\subsubsection{Scaling a row}

\begin{example}
Show that
	\[
		\det \begin{bmatrix} 2 & 6 & 8 \\ 1 & 1 & 1 \\ 0 & 1 & 1 \end{bmatrix} = 2 \det \begin{bmatrix} 1 & 3 & 4 \\ 1 & 1 & 1 \\ 0 & 1 & 1 \end{bmatrix} .
	\]
\end{example}

\begin{solution}

\end{solution}

\vfill 

\begin{theorem}
If $B$ is an $n \times n$ matrix for which column $j$ is obtained by multiplying $k$ times the column $j$ of an $n \times n$ matrix $A$, with $k \neq 0$, then
	\[
		\det (B) = k \det (A) .
	\]
\end{theorem}

\underline{Remark:} This fact is still true if a column $j$ of a matrix $B$ is obtained by multiplying the column $j$ of a given matrix $A$ by a nonzero scalar

\newpage 

\subsubsection{Subtracting a Multiple of a Row}

\begin{example} 
Show that 
	\[
		\det \begin{bmatrix} 1 & 1 & 1 \\ 0 & 2 & 1 \\ 1 & 2 & 3 \end{bmatrix} = \det \begin{bmatrix} 1 & 1 & 1 \\ 2 & 4 & 3 \\ 1 & 2 & 3 \end{bmatrix} .
	\]
\end{example}

\begin{solution}

\end{solution}

\vfill 

\begin{theorem}
If a the row $j$ of a matrix $B$ is obtained by subtracting a multiple of a row of a matrix $A$ to the row $j$ of $A$, then
	\[
		\det (B) = \det (A) .
	\]
\end{theorem}

\underline{Remark:} This remains true if we replace the row operation by the corresponding column operation.

\newpage 

\begin{theorem}
Let $A$ be an $n \times n$ matrix.
	\begin{enumerate}
		\item If $A$ has a row (or column) of zero, then $\det (A) = 0$.
		\item If $A$ has two identical rows (or columns), then $\det (A) = 0$.
	\end{enumerate}
\end{theorem}

\begin{proof}
\begin{enumerate}
	\item Developing $\det (A)$ along the row of zero, then $\det (A) = 0$.
	\item Assume that the two identical rows have index $p$ and $q$. Let $B$ be the matrix obtained by interchanging rows $p$ and $q$ of $A$. Then, $A = B$. But, $\det (B) = - \det (A)$, which implies that $2 \det (A) = 0$, hence $\det (A) = 0$. \hfill $\square$
\end{enumerate}
\end{proof}

\begin{example}
Find the values of $x$ for which $\det (A) = 0$, where $A = \begin{bmatrix} 1 & x & x \\ x & 1 & x \\ x & x & 1 \end{bmatrix}$.
\end{example}

\begin{solution}

\end{solution}

\newpage 

\section{Diagonal Matrices}

\begin{example}
Compute $\det (A)$ if $A = \begin{bmatrix} 1 & 0 & 0 & 0 \\ 2 & 3 & 0 & 0 \\ 3 & 4 & 5 & 0 \\ 4 & 3 & 2 & 10 \end{bmatrix}$.
\end{example}

\begin{solution}

\end{solution}

\vfill 

\begin{definition}
A matrix $A$ is
	\begin{enumerate}
		\item \textbf{lower triangle} if all the entries above the main diagonal are zero.
		\item \textbf{upper triangle} if all the entries below the main diagonal are zero.
		\item \textbf{triangular} if it is lower triangle or upper triangle.
	\end{enumerate}
\end{definition}

\begin{theorem}
If $A = [a_{ij}]$ is an $n \times n$ triangular matrix, then $\det (A) = a_{11} a_{22} \cdots a_{nn}$.
\end{theorem}

\end{document}