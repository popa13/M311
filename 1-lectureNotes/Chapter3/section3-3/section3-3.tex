\documentclass[20pt,a4paper]{extarticle}
\usepackage[utf8]{inputenc}
\usepackage[english]{babel}

\usepackage{amsmath}
\usepackage{amsfonts}
\usepackage{amssymb}
\usepackage{mathtools}
\usepackage{systeme}
\sysdelim..

\usepackage{graphicx}
\usepackage{caption}
\usepackage{subcaption}
\usepackage{lmodern}
\usepackage{tikz}
\usetikzlibrary{calc}
\usepackage{titlesec}
\usepackage{environ}
\usepackage{xcolor}
\usepackage{fancyhdr}
\usepackage[colorlinks = true, linkcolor = black]{hyperref}
\usepackage{xparse}
\usepackage{enumitem}
\usepackage{comment}
\usepackage{wrapfig}
\usepackage{soul}
\usepackage[capitalise]{cleveref}
\usepackage{circledsteps}

\usepackage[left=1cm,right=1cm,top=1cm,bottom=3cm]{geometry}
\usepackage{multicol}
\usepackage[indent=0pt]{parskip}

\newcommand{\spaceP}{\vspace*{0.5cm}}
\newcommand{\Span}{\mathrm{Span}\,}
\newcommand{\range}{\mathrm{range}\,}
\newcommand{\ra}{\rightarrow}
\newcommand{\curl}{\mathrm{curl} \,}
\newcommand{\hint}[1]{\scalebox{2}{$\displaystyle\int_{\scalebox{0.35}{$#1$}}$}\,}
\newcommand{\hiint}[1]{\scalebox{2}{$\displaystyle\iint_{\scalebox{0.35}{$#1$}}$}\,}
\newcommand{\hiiint}[1]{\scalebox{2}{$\displaystyle\iiint_{\scalebox{0.35}{$#1$}}$}\,}
\renewcommand{\div}{\mathrm{div}\,}

\makeatletter
\renewcommand*\env@matrix[1][*\c@MaxMatrixCols c]{%
  \hskip -\arraycolsep
  \let\@ifnextchar\new@ifnextchar
  \array{#1}}
\makeatother

%% Redefining sections
\newcommand{\sectionformat}[1]{%
    \begin{tikzpicture}[baseline=(title.base)]
        \node[rectangle, draw] (title) {#1};
    \end{tikzpicture}
    
    \noindent\hrulefill
}

\newif\ifhNotes 

\hNotesfalse

\ifhNotes
	\newcommand{\hideNotes}[1]{%
	\phantom{#1}
	}
	\newcommand{\hideNotesU}[1]{%
	\underline{\hspace{1mm}\phantom{#1}\hspace{1mm}}
	}
\else
	\newcommand{\hideNotes}[1]{#1}
	\newcommand{\hideNotesU}[1]{\textcolor{blue}{#1}}
\fi

% default values copied from titlesec documentation page 23
% parameters of \titleformat command are explained on page 4
\titleformat%
    {\section}% <command> is the sectioning command to be redefined, i. e., \part, \chapter, \section, \subsection, \subsubsection, \paragraph or \subparagraph.
    {\normalfont\large\scshape}% <format>
    {}% <label> the number
    {0em}% <sep> length. horizontal separation between label and title body
    {\centering\sectionformat}% code preceding the title body  (title body is taken as argument)

%% Set counters for sections to none
\setcounter{secnumdepth}{0}

%% Set the footer/headers
\pagestyle{fancy}
\fancyhf{}
\renewcommand{\headrulewidth}{0pt}
\renewcommand{\footrulewidth}{2pt}
\lfoot{P.-O. Paris{\'e}}
\cfoot{MATH 311}
\rfoot{Page \thepage}

%% Defining example environment
\newcounter{example}
\NewEnviron{example}%
	{%
	\noindent\refstepcounter{example}\fcolorbox{gray!40}{gray!40}{\textsc{\textcolor{red}{Example~\theexample.}}}%
	%\fcolorbox{black}{white}%
		{  %\parbox{0.95\textwidth}%
			{
			\BODY
			}%
		}%
	}

\newcounter{theorem}
\NewEnviron{theorem}%
	{%
	\noindent\refstepcounter{theorem}\fcolorbox{gray!40}{gray!40}{\textsc{\textcolor{black}{Theorem~\thetheorem.}}}%
	%\fcolorbox{black}{white}%
		{  %\parbox{0.95\textwidth}%
			{
			\BODY
			}%
		}%
	}

\newcounter{definition}
\NewEnviron{definition}%
	{%
	\noindent\refstepcounter{definition}\fcolorbox{gray!40}{gray!40}{\textsc{\textcolor{black}{Definition~\thedefinition.}}}%
	%\fcolorbox{black}{white}%
		{  %\parbox{0.95\textwidth}%
			{
			\BODY
			}%
		}%
	}

\newcounter{algo}
\NewEnviron{algorithm}
	{%
	\noindent\refstepcounter{algo}\fcolorbox{gray!40}{gray!40}{\textsc{\textcolor{black}{Algorithm~\thealgo.}}}%
	%\fcolorbox{black}{white}%
		{  %\parbox{0.95\textwidth}%
			{
			\BODY
			}%
		}%
	}

\NewEnviron{goal}
	{%
	\noindent\fcolorbox{gray!40}{gray!40}{\textsc{\textcolor{black}{Goal:}}}%
	%\fcolorbox{black}{white}%
		{  %\parbox{0.95\textwidth}%
			{
			\BODY
			}%
		}%
	}

\NewEnviron{solution}%
	{%
	\noindent \fcolorbox{gray!40}{gray!40}{\textsc{\textcolor{blue}{Solution.}}}%
	%\fcolorbox{black}{white}%
		{  %\parbox{0.95\textwidth}%
			{
			%\textcolor{blue}
			}%
		}%
	}

\NewEnviron{proof}%
	{%
	\noindent \fcolorbox{gray!40}{gray!40}{\textsc{\textcolor{blue}{Proof.}}}%
	%\fcolorbox{black}{white}%
		{  %\parbox{0.95\textwidth}%
			{
			\textcolor{blue}{%
			\BODY
			}
			}%
		}%
	}
%%% Ignorer les notes
%\excludecomment{notes}

%%%%
\begin{document}
\thispagestyle{empty}

\begin{center}
\vspace*{0.75cm}

{\Huge \textsc{Math 311}}

\vspace*{1.5cm}

{\LARGE \textsc{Chapter 3}} 

\vspace*{0.75cm}

\noindent\textsc{Section 3.3: Diagonalization and Eigenvalues}

\vspace*{0.25cm}

\tableofcontents

\vfill

\noindent \textsc{Created by: Pierre-Olivier Paris{\'e}} \\
\textsc{Spring 2024}
\end{center}

\newpage

\section{Why Diagonalization?}

\begin{example}\label{Ex:PowerOfAMatrix}
Let $A = \begin{bmatrix} 1 & 2 \\ 3 & 2 \end{bmatrix}$. Compute $A^{100}$.
\end{example}

\begin{solution}

\end{solution}

\vfill 

\textbf{Fact:} If $A = P D P^{-1}$, then $A^k = P D^k P^{-1}$. 

\underline{\textsc{Goal}:} Find the matrix $P$ such that $P^{-1} A P$ is a diagonal matrix.

\newpage 

\section{Eigenvalues and Eigenvectors}

\textbf{Exploration:} Consider the matrix
	\[
		A = \begin{bmatrix} 
		1 & 2 \\ 3 & 2
		\end{bmatrix} .
	\]
Set $\mathbf{x} = \begin{bmatrix} a \\ b \end{bmatrix}$ a $2 \times 1$ vector. Then
	\[
		A \mathbf{x} = \begin{bmatrix} 1 & 2 \\ 3 & 2 \end{bmatrix} \begin{bmatrix} a \\ b \end{bmatrix} = \begin{bmatrix} a + 2b \\ 3a + 2b \end{bmatrix} 
	\]
Use Desmos\footnote{\url{https://www.desmos.com/calculator/5xlrp9fd7g}} to explore and answer the following questions:
	\begin{itemize}
		\item Can you find an exceptional behavior of $A \mathbf{x}$ and $\mathbf{x}$ for certain choices of $\mathbf{x}$?
		\item Can you find a relation between $A \mathbf{x}$ and $\mathbf{x}$?
	\end{itemize}
Record your observations in the following blank space:

\newpage 

\begin{definition}
Let $A$ be an $n \times n$ matrix.
	\begin{enumerate}[label=\alph*)]
		\item A number $\lambda$ is called an \textbf{eigenvalue} of $A$ if there is a non-zero $n \times 1$ vector $\mathbf{x}$ such that $A \mathbf{x} = \lambda \mathbf{x}$.
		\item The vector $\mathbf{x}$ is called an \textbf{eigenvector} associated to $\lambda$. 
	\end{enumerate}
\end{definition}

\begin{example}
Let $A = \begin{bmatrix}  1 & 2 \\ 3 & 2  \end{bmatrix}$ and let $\mathbf{x} = \begin{bmatrix} -2 \\ 2 \end{bmatrix}$. Then
	\[
		A \mathbf{x} = \begin{bmatrix}  1 & 2 \\ 3 & 2 \end{bmatrix} \begin{bmatrix} -2 \\ 2 \end{bmatrix} = \hspace*{8cm}
	\]
\end{example}

\bigskip 

\subsubsection{Finding eigenvalues}

Notice that
\begin{align*}
	\lambda \text{ is an eigenvalue of } A &\iff A\mathbf{x} = \lambda \mathbf{x} \text{ for some } \mathbf{x} \neq 0 \\
	& \iff (\lambda I - A) \mathbf{x} = 0 \text{ for some } \mathbf{x} \neq 0 .
\end{align*}

So
	\begin{align*}
		\lambda \text{ is an eigenvalue of } A &\iff (\lambda I - A) \text{ is not invertible } \\ 
		&\iff \det (\lambda I - A) = 0 
	\end{align*}

\begin{definition}
The \textbf{characteristic polynomial} of an $n \times n$ matrix $A$ is defined by
	\[
		c_A (x) = \det (xI - A) .
	\]
\end{definition}

\textbf{Conclusion:}
\begin{align*}
		\lambda \text{ is an eigenvalue of } A \iff \lambda \text{ is a root of } c_A (x) .
	\end{align*}

\newpage 

\begin{example}
Find all eigenvalues of the matrix 
	\[
		A = \begin{bmatrix}
		1 & 2 \\ 3 & 2 
		\end{bmatrix} .
	\]
\end{example}

\begin{solution}

\end{solution}

\newpage 

\phantom{2}

\newpage 

\subsubsection{Finding Eigenvectors}

For a given eigenvalue $\lambda$, the eigenvectors associated to $\lambda$ are the solutions $\mathbf{x}$ to the system
	\[
		(\lambda I - A) \mathbf{x} = 0 .
	\]

\begin{example}
Find the eigenvectors associated to the each eigenvalue of the matrix
	\[
		A = \begin{bmatrix}
		1 & 2 \\ 3 & 2 
		\end{bmatrix} .
	\]
\end{example}

\newpage 

\begin{example}
Find all eigenvalues and associated eigenvectors of the matrix
	\[
		A = \begin{bmatrix}
			7 & 0 & -4 \\ 
			0 & 5 & 0 \\ 
			5 & 0 & -2 
			\end{bmatrix} .
	\]
\end{example}

\begin{solution}

\end{solution}

\newpage 

\phantom{2}

\newpage

\section{Diagonalization}

\begin{example}
Find a matrix $P$ such that
	\[
		P^{-1} A P
	\]
is a diagonal matrix, where $A$ is from \cref{Ex:PowerOfAMatrix}.
\end{example}

\begin{solution}

\end{solution}

\newpage 

\begin{theorem}
Let $A$ be an $n \times n$ matrix. Then if all eigenvalues of $A$ are distinct, then $A$ is diagonalizable.
\end{theorem}

Notice that if $A$ is diagonalizable and if we let $P = \begin{bmatrix} \mathbf{x}_1 & \mathbf{x}_2 & \cdots & \mathbf{x}_n \end{bmatrix}$:
	\begin{align*}
		P^{-1} A P = D & \iff AP = PD \\
		&\iff A \begin{bmatrix} \mathbf{x}_1 & \mathbf{x}_2 & \cdots & \mathbf{x}_n \end{bmatrix} = \begin{bmatrix} \mathbf{x}_1 & \mathbf{x}_2 & \cdots & \mathbf{x}_n \end{bmatrix} D \\ 
		& \iff A \mathbf{x}_1 = \lambda_1 \mathbf{x}_1 , \, A \mathbf{x}_2 = \lambda_2 \mathbf{x}_2 , \ldots , A \mathbf{x}_n = \lambda_n \mathbf{x}_n .
	\end{align*}

\begin{algorithm}
Let $A$ be an $n \times n$ matrix with distinct eigenvalues.
	\begin{enumerate}[label=\Circled{\arabic*}]
		\item Find all distinct eigenvalues of $A$.
		\item For each eigenvalue of $A$, find the corresponding set of eigenvectors.
		\item If $\mathbf{x}_1$, $\mathbf{x}_2$, $\ldots$, $\mathbf{x}_n$ are a set of $n$ distinct eigenvectors, then set
			\[
				P = \begin{bmatrix} \mathbf{x}_1 & \mathbf{x}_2 & \cdots & \mathbf{x}_n \end{bmatrix} .
			\]
	\end{enumerate}
\end{algorithm}


\textcolor{red}{\textsc{Warning!}}

Not every matrix is diagonalizable. For instance, the matrix
	\[
		A = \begin{bmatrix} 1 & 1 \\ 0 & 1 \end{bmatrix}
	\]
is not diagonalizable. 

For a more general algorithm, see \textit{Jordan Canonical Form}, Chapter 11 from the textbook. Complex numbers are required.


\end{document}