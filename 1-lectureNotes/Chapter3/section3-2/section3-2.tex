\documentclass[20pt,a4paper]{extarticle}
\usepackage[utf8]{inputenc}
\usepackage[english]{babel}

\usepackage{amsmath}
\usepackage{amsfonts}
\usepackage{amssymb}
\usepackage{mathtools}
\usepackage{systeme}
\sysdelim..

\usepackage{graphicx}
\usepackage{caption}
\usepackage{subcaption}
\usepackage{lmodern}
\usepackage{tikz}
\usetikzlibrary{calc}
\usepackage{titlesec}
\usepackage{environ}
\usepackage{xcolor}
\usepackage{fancyhdr}
\usepackage[colorlinks = true, linkcolor = black]{hyperref}
\usepackage{xparse}
\usepackage{enumitem}
\usepackage{comment}
\usepackage{wrapfig}
\usepackage{soul}
\usepackage[capitalise]{cleveref}

\usepackage[left=1cm,right=1cm,top=1cm,bottom=3cm]{geometry}
\usepackage{multicol}
\usepackage[indent=0pt]{parskip}

\newcommand{\spaceP}{\vspace*{0.5cm}}
\newcommand{\Span}{\mathrm{Span}\,}
\newcommand{\range}{\mathrm{range}\,}
\newcommand{\ra}{\rightarrow}
\newcommand{\curl}{\mathrm{curl} \,}
\newcommand{\hint}[1]{\scalebox{2}{$\displaystyle\int_{\scalebox{0.35}{$#1$}}$}\,}
\newcommand{\hiint}[1]{\scalebox{2}{$\displaystyle\iint_{\scalebox{0.35}{$#1$}}$}\,}
\newcommand{\hiiint}[1]{\scalebox{2}{$\displaystyle\iiint_{\scalebox{0.35}{$#1$}}$}\,}
\renewcommand{\div}{\mathrm{div}\,}

\makeatletter
\renewcommand*\env@matrix[1][*\c@MaxMatrixCols c]{%
  \hskip -\arraycolsep
  \let\@ifnextchar\new@ifnextchar
  \array{#1}}
\makeatother

%% Redefining sections
\newcommand{\sectionformat}[1]{%
    \begin{tikzpicture}[baseline=(title.base)]
        \node[rectangle, draw] (title) {#1};
    \end{tikzpicture}
    
    \noindent\hrulefill
}

\newif\ifhNotes 

\hNotesfalse

\ifhNotes
	\newcommand{\hideNotes}[1]{%
	\phantom{#1}
	}
	\newcommand{\hideNotesU}[1]{%
	\underline{\hspace{1mm}\phantom{#1}\hspace{1mm}}
	}
\else
	\newcommand{\hideNotes}[1]{#1}
	\newcommand{\hideNotesU}[1]{\textcolor{blue}{#1}}
\fi

% default values copied from titlesec documentation page 23
% parameters of \titleformat command are explained on page 4
\titleformat%
    {\section}% <command> is the sectioning command to be redefined, i. e., \part, \chapter, \section, \subsection, \subsubsection, \paragraph or \subparagraph.
    {\normalfont\large\scshape}% <format>
    {}% <label> the number
    {0em}% <sep> length. horizontal separation between label and title body
    {\centering\sectionformat}% code preceding the title body  (title body is taken as argument)

%% Set counters for sections to none
\setcounter{secnumdepth}{0}

%% Set the footer/headers
\pagestyle{fancy}
\fancyhf{}
\renewcommand{\headrulewidth}{0pt}
\renewcommand{\footrulewidth}{2pt}
\lfoot{P.-O. Paris{\'e}}
\cfoot{MATH 311}
\rfoot{Page \thepage}

%% Defining example environment
\newcounter{example}
\NewEnviron{example}%
	{%
	\noindent\refstepcounter{example}\fcolorbox{gray!40}{gray!40}{\textsc{\textcolor{red}{Example~\theexample.}}}%
	%\fcolorbox{black}{white}%
		{  %\parbox{0.95\textwidth}%
			{
			\BODY
			}%
		}%
	}

\newcounter{theorem}
\NewEnviron{theorem}%
	{%
	\noindent\refstepcounter{theorem}\fcolorbox{gray!40}{gray!40}{\textsc{\textcolor{black}{Theorem~\thetheorem.}}}%
	%\fcolorbox{black}{white}%
		{  %\parbox{0.95\textwidth}%
			{
			\BODY
			}%
		}%
	}

\newcounter{definition}
\NewEnviron{definition}%
	{%
	\noindent\refstepcounter{definition}\fcolorbox{gray!40}{gray!40}{\textsc{\textcolor{black}{Definition~\thedefinition.}}}%
	%\fcolorbox{black}{white}%
		{  %\parbox{0.95\textwidth}%
			{
			\BODY
			}%
		}%
	}

\newcounter{algo}
\NewEnviron{algorithm}
	{%
	\noindent\refstepcounter{algo}\fcolorbox{gray!40}{gray!40}{\textsc{\textcolor{black}{Algorithm~\thealgo.}}}%
	%\fcolorbox{black}{white}%
		{  %\parbox{0.95\textwidth}%
			{
			\BODY
			}%
		}%
	}

\NewEnviron{goal}
	{%
	\noindent\fcolorbox{gray!40}{gray!40}{\textsc{\textcolor{black}{Goal:}}}%
	%\fcolorbox{black}{white}%
		{  %\parbox{0.95\textwidth}%
			{
			\BODY
			}%
		}%
	}

\NewEnviron{solution}%
	{%
	\noindent \fcolorbox{gray!40}{gray!40}{\textsc{\textcolor{blue}{Solution.}}}%
	%\fcolorbox{black}{white}%
		{  %\parbox{0.95\textwidth}%
			{
			%\textcolor{blue}
			}%
		}%
	}

\NewEnviron{proof}%
	{%
	\noindent \fcolorbox{gray!40}{gray!40}{\textsc{\textcolor{blue}{Proof.}}}%
	%\fcolorbox{black}{white}%
		{  %\parbox{0.95\textwidth}%
			{
			\textcolor{blue}{%
			\BODY
			}
			}%
		}%
	}
%%% Ignorer les notes
%\excludecomment{notes}

%%%%
\begin{document}
\thispagestyle{empty}

\begin{center}
\vspace*{0.75cm}

{\Huge \textsc{Math 311}}

\vspace*{1.5cm}

{\LARGE \textsc{Chapter 3}} 

\vspace*{0.75cm}

\noindent\textsc{Section 3.2: Determinants and Matrix Inverses}

\vspace*{0.25cm}

\tableofcontents

\vfill

\noindent \textsc{Created by: Pierre-Olivier Paris{\'e}} \\
\textsc{Spring 2024}
\end{center}

\newpage

\section{Product Rule}

\begin{example}
Show that for any number $a$, $b$, $c$, $d$, we have the following identity
	\[
		(a^2 + b^2) (c^2 + d^2) = (ac - bd)^2 + (ad + bc)^2 .
	\]
\end{example}

\begin{solution}

\end{solution}

\newpage 


\begin{theorem}
If $A$ and $B$ are $n \times n$ matrices, then
	\[
		\det (AB) = \det (A) \det (B) .
	\]
\end{theorem}

\underline{Facts:}
	\begin{itemize}
		\item For three matrices, $\det (ABC) = \det (A) \det (B) \det (C)$.
		\item For $n$ matrices, 
			$$ 
				\det (A_1 A_2 \cdots A_n) = \det (A_1) \det (A_2) \cdots \det (A_n) .
			$$
		\item For powers of a matrix, $\det (A^k) = (\det (A))^k$ (here, $k \geq 1$).
	\end{itemize}

\begin{example}
Assume that $\det (A) = 2$, $\det (B) = 3$, and $\det (C) = -2$. Compute
	\[
		\det (A^2 BC B C^2).
	\]
\end{example}

\begin{solution}

\end{solution}

\newpage 

\section{Matrix Inverses}

Recall that
	\[
		A = \begin{bmatrix} a & b \\ c & d \end{bmatrix} \text{ is invertible } \iff \det (A) = ad - bc \neq 0 .
	\]

\begin{theorem}
Let $A$ be an $n \times n$ matrix. The matrix $A$ is invertible if and only if $\det (A) \neq 0$. In this case, we have $\det (A^{-1}) = \frac{1}{\det (A)}$.
\end{theorem}

\begin{proof}
See page 156 in the textbook for the complete proof.
\end{proof}

\vspace*{20pt}

\begin{example}
For which real value(s) of $c$ is the matrix $A = \begin{bmatrix} 0 & c & -c \\ -1 & 2 & 1 \\ c & -c & c \end{bmatrix}$ invertible?
\end{example}

\begin{solution}

\end{solution}

\newpage 

\phantom{2}

\newpage 

\section{Transpose and Determinants}

\begin{example}
Let $A = \begin{bmatrix} 5 & 1 & 3 \\ -1 & 2 & 3 \\ 1 & 4 & 8 \end{bmatrix}$. Find $\det (A)$ and $\det (A^\top)$ and compare their values.
\end{example}

\begin{solution}

\end{solution}

\vfill

\begin{theorem}
If $A$ is an $n \times n$ matrix, then $\det (A) = \det (A^\top)$. 
\end{theorem}

\vspace*{1cm}

\newpage 

\begin{example}
Assume that $\det (A) = 2$ and $\det (B) = 4$. Find the value of $\det (A A^T (B^T)^2)$.
\end{example}

\begin{solution}

\end{solution}

\vspace*{10cm}

\begin{example}
A square matrix is called \textbf{orthogonal} if $A^{-1} = A^\top$. What are the possible values of $\det (A)$ if $A$ is orthogonal?
\end{example}

\begin{solution}

\end{solution}

\end{document}


\begin{example}
Let $A = \begin{bmatrix} 5 & 1 & 3\\-1 & 2 & 3\\1 & 4 & 8 \end{bmatrix}$ and $B = \begin{bmatrix} 1 & -1 & 2\\3 & 1 & 0\\0 & -1 & 1 \end{bmatrix}$. Then, we have
	\[
		\det (A) = 13 \text{ and } \det (B) = -2.
	\]
We have
	\[
		\det (AB) = \det \begin{bmatrix} 8 & -7 & 13\\5 & 0 & 1\\13 & -5 & 10 \end{bmatrix} = -26 .
	\]
Hey, notice that $\det (AB) = (13) (-2) = \det (A) \det (B)$!!
\end{example}