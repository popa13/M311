\documentclass[20pt,a4paper]{extarticle}
\usepackage[utf8]{inputenc}
\usepackage[english]{babel}

\usepackage{amsmath}
\usepackage{amsfonts}
\usepackage{amssymb}
\usepackage{mathtools}
\usepackage{systeme}
\sysdelim..

\usepackage{graphicx}
\usepackage{caption}
\usepackage{subcaption}
\usepackage{lmodern}
\usepackage{tikz}
\usetikzlibrary{calc}
\usepackage{titlesec}
\usepackage{environ}
\usepackage{xcolor}
\usepackage{fancyhdr}
\usepackage[colorlinks = true, linkcolor = black]{hyperref}
\usepackage{xparse}
\usepackage{enumitem}
\usepackage{comment}
\usepackage{wrapfig}
\usepackage{soul}
\usepackage[capitalise]{cleveref}
\usepackage{circledsteps}

\usepackage[left=1cm,right=1cm,top=1cm,bottom=3cm]{geometry}
\usepackage{multicol}
\usepackage[indent=0pt]{parskip}

\newcommand{\spaceP}{\vspace*{0.5cm}}
\newcommand{\range}{\mathrm{range}\,}
\newcommand{\ra}{\rightarrow}
\newcommand{\curl}{\mathrm{curl} \,}
\newcommand{\hint}[1]{\scalebox{2}{$\displaystyle\int_{\scalebox{0.35}{$#1$}}$}\,}
\newcommand{\hiint}[1]{\scalebox{2}{$\displaystyle\iint_{\scalebox{0.35}{$#1$}}$}\,}
\newcommand{\hiiint}[1]{\scalebox{2}{$\displaystyle\iiint_{\scalebox{0.35}{$#1$}}$}\,}
\renewcommand{\div}{\mathrm{div}\,}

\DeclareMathOperator{\Span}{span}

\makeatletter
\renewcommand*\env@matrix[1][*\c@MaxMatrixCols c]{%
  \hskip -\arraycolsep
  \let\@ifnextchar\new@ifnextchar
  \array{#1}}
\makeatother

%% Redefining sections
\newcommand{\sectionformat}[1]{%
    \begin{tikzpicture}[baseline=(title.base)]
        \node[rectangle, draw] (title) {#1};
    \end{tikzpicture}
    
    \noindent\hrulefill
}

\newif\ifhNotes 

\hNotesfalse

\ifhNotes
	\newcommand{\hideNotes}[1]{%
	\phantom{#1}
	}
	\newcommand{\hideNotesU}[1]{%
	\underline{\hspace{1mm}\phantom{#1}\hspace{1mm}}
	}
\else
	\newcommand{\hideNotes}[1]{#1}
	\newcommand{\hideNotesU}[1]{\textcolor{blue}{#1}}
\fi

% default values copied from titlesec documentation page 23
% parameters of \titleformat command are explained on page 4
\titleformat%
    {\section}% <command> is the sectioning command to be redefined, i. e., \part, \chapter, \section, \subsection, \subsubsection, \paragraph or \subparagraph.
    {\normalfont\large\scshape}% <format>
    {}% <label> the number
    {0em}% <sep> length. horizontal separation between label and title body
    {\centering\sectionformat}% code preceding the title body  (title body is taken as argument)

%% Set counters for sections to none
\setcounter{secnumdepth}{0}

%% Set the footer/headers
\pagestyle{fancy}
\fancyhf{}
\renewcommand{\headrulewidth}{0pt}
\renewcommand{\footrulewidth}{2pt}
\lfoot{P.-O. Paris{\'e}}
\cfoot{MATH 311}
\rfoot{Page \thepage}

%% Defining example environment
\newcounter{example}
\NewEnviron{example}%
	{%
	\noindent\refstepcounter{example}\fcolorbox{gray!40}{gray!40}{\textsc{\textcolor{red}{Example~\theexample.}}}%
	%\fcolorbox{black}{white}%
		{  %\parbox{0.95\textwidth}%
			{
			\BODY
			}%
		}%
	}

\newcounter{theorem}
\NewEnviron{theorem}%
	{%
	\noindent\refstepcounter{theorem}\fcolorbox{gray!40}{gray!40}{\textsc{\textcolor{black}{Theorem~\thetheorem.}}}%
	%\fcolorbox{black}{white}%
		{  %\parbox{0.95\textwidth}%
			{
			\BODY
			}%
		}%
	}

\newcounter{definition}
\NewEnviron{definition}%
	{%
	\noindent\refstepcounter{definition}\fcolorbox{gray!40}{gray!40}{\textsc{\textcolor{black}{Definition~\thedefinition.}}}%
	%\fcolorbox{black}{white}%
		{  %\parbox{0.95\textwidth}%
			{
			\BODY
			}%
		}%
	}

\newcounter{algo}
\NewEnviron{algorithm}
	{%
	\noindent\refstepcounter{algo}\fcolorbox{gray!40}{gray!40}{\textsc{\textcolor{black}{Algorithm~\thealgo.}}}%
	%\fcolorbox{black}{white}%
		{  %\parbox{0.95\textwidth}%
			{
			\BODY
			}%
		}%
	}

\NewEnviron{goal}
	{%
	\noindent\fcolorbox{gray!40}{gray!40}{\textsc{\textcolor{black}{Goal:}}}%
	%\fcolorbox{black}{white}%
		{  %\parbox{0.95\textwidth}%
			{
			\BODY
			}%
		}%
	}

\NewEnviron{solution}%
	{%
	\noindent \fcolorbox{gray!40}{gray!40}{\textsc{\textcolor{blue}{Solution.}}}%
	%\fcolorbox{black}{white}%
		{  %\parbox{0.95\textwidth}%
			{
			%\textcolor{blue}
			}%
		}%
	}

\NewEnviron{proof}%
	{%
	\noindent \fcolorbox{gray!40}{gray!40}{\textsc{\textcolor{blue}{Proof.}}}%
	%\fcolorbox{black}{white}%
		{  %\parbox{0.95\textwidth}%
			{
			\textcolor{blue}{%
			\BODY
			}
			}%
		}%
	}
%%% Ignorer les notes
%\excludecomment{notes}

%%%%
\begin{document}
\thispagestyle{empty}

\begin{center}
\vspace*{0.75cm}

{\Huge \textsc{Math 311}}

\vspace*{1.5cm}

{\LARGE \textsc{Chapter 6}} 

\vspace*{0.75cm}

\noindent\textsc{Section 6.3: Linear Independence and Dimension}

\vspace*{0.25cm}

\tableofcontents

\vfill

\noindent \textsc{Created by: Pierre-Olivier Paris{\'e}} \\
\textsc{Spring 2024}
\end{center}

\newpage


\section{Linear Independence}

\begin{example}% See Section 1.3
Let $\mathbf{v} = \begin{bmatrix} 0 \\ -1 \\ 2 \end{bmatrix}$. Let $\mathbf{u_1} = \begin{bmatrix} 1 \\ 0 \\ 0 \end{bmatrix}$, $\mathbf{u_2} = \begin{bmatrix} 0 \\ 1 \\ 1 \end{bmatrix}$, $\mathbf{u_3} = \begin{bmatrix} 1 \\ 1 \\ 0 \end{bmatrix}$ and $\mathbf{v_1} = \begin{bmatrix} 1 \\ 0 \\ 1 \end{bmatrix}$, $\mathbf{v_2} = \begin{bmatrix} 2 \\ 1 \\ 0 \end{bmatrix}$, and $\mathbf{v_3} = \begin{bmatrix} 3 \\ 1 \\ 1 \end{bmatrix}$
\begin{enumerate}[label=\alph*)]
\item Consider the vectors. Can you write $\mathbf{v}$ as a unique linear combination of the vectors $\mathbf{u_1}$, $\mathbf{u_2}$, and $\mathbf{u_3}$? 
\item Consider the vectors. Can you write the vector $\mathbf{v}$ as a unique linear combination of the vectors $\mathbf{v_1}$, $\mathbf{v_2}$, $\mathbf{v_3}$?
\end{enumerate}
\end{example}

\begin{solution}

\end{solution}

\newpage 

\begin{definition}
A set of vectors $\{ \mathbf{v_1} , \mathbf{v_2} , \ldots , \mathbf{v_n} \}$ in a vector space $V$ is called \textbf{linearly independent} (or simply \textbf{independent}) if 
	\[
		s_1 \mathbf{v_1} + s_2 \mathbf{v_2} + \cdots + s_n \mathbf{v_n} = \mathbf{0} \quad \Rightarrow \quad s_1 = s_2 = \cdots = s_n = 0 .
	\]
A set of vectors that is not independent is said to be \textbf{linearly dependent} (or simply \textbf{dependent}).
\end{definition}

\underline{Note:} 
\begin{itemize}
	\item The \textbf{trivial linear combination} of the vectors $\mathbf{v_1}$, $\mathbf{v_2}$, $\ldots$, $\mathbf{v_n}$ is the one with every coefficient zero:
	\[
		0 \mathbf{v_1} + 0 \mathbf{v_2} + \cdots + 0 \mathbf{v_n} .
	\]
	\item So the vectors $\mathbf{v_1}$, $\mathbf{v_2}$, $\ldots$, $\mathbf{v_n}$ are linearly independent if and only if the only way to write $\mathbf{0}$ is using the trivial linear combination.
\end{itemize}

\begin{example}
Show that the set
	\[
		\left\{ \begin{bmatrix} 1 & 1 \\ 0 & 0 \end{bmatrix} , \begin{bmatrix} 1 & 0 \\ 1 & 0 \end{bmatrix} , \begin{bmatrix} 0 & 0 \\ 1 & -1 \end{bmatrix} , \begin{bmatrix} 0 & 1 \\ 0 & 1 \end{bmatrix} \right\} 
	\]
is independent.
\end{example}

\begin{solution}

\end{solution}

\newpage 

\phantom{2} 

\newpage 

\begin{example} 
Let $\{ \mathbf{x} , \mathbf{y} , \mathbf{z} , \mathbf{w} \}$ be an independent set in a vector space $V$. Which of the following set is indepedent?
	\begin{enumerate}[label=\alph*)]
		\item $\{ \mathbf{x} - \mathbf{y} , \mathbf{y} - \mathbf{z} , \mathbf{z} - \mathbf{x} \}$.
		\item $\{ \mathbf{x} - \mathbf{y} , \mathbf{y} - \mathbf{z} , \mathbf{z} - \mathbf{w} , \mathbf{w} - \mathbf{x} \}$.
	\end{enumerate}
\end{example} 

\begin{solution}

\end{solution}

\vfill 


%\subsection{Basic Properties}

%\begin{enumerate}[label=\Circled{\arabic*}]
	%\item If $\{ \mathbf{v_1} , \mathbf{v_2} , \ldots , \mathbf{v_n} \}$ is an independent set of vectors in a vector space $V$, then any vector in $\Span \{ \mathbf{v_1} , \mathbf{v_2} , \ldots , \mathbf{v_n} \}$ has a unique representation as a linear combination of the $\mathbf{x_i}$. 
	%\item \underline{Fundamental Theorem (p.346):} Suppose a vector space $V$ can be spanned by $n$ vectors. If any set of $m$ vectors in $V$ is linearly independent, then $m \leq n$.
	%\item For the vector space $V = \mathbb{R}^m$, a set of vectors $\{ \mathbf{x_1} , \mathbf{x_2} , \ldots , \mathbf{x_n} \}$ is independent if and only if 
	%	$$ 
	%	\begin{bmatrix} \mathbf{x_1} & \mathbf{x_2} & \cdots & \mathbf{x_n} \end{bmatrix} \mathbf{s} = \mathbf{0} \quad \Rightarrow \quad \mathbf{s} = \mathbf{0} .
	%	$$
	%\item 
%\end{enumerate}

\newpage 

\section{Basis}


\begin{definition}
A set $\{ \mathbf{e_1} , \mathbf{e_2} , \ldots , \mathbf{e_n} \}$ of vectors in a vector space $V$ is called a basis of $V$ if it satisfies the following two conditions:
	\begin{enumerate}[label=\Circled{\arabic*}]
		\item $\{ \mathbf{e_1} , \mathbf{e_2} , \ldots , \mathbf{e_n} \}$ is linearly independent.
		\item $V = \Span \{ \mathbf{e_1} , \mathbf{e_2} , \ldots , \mathbf{e_n} \}$. 
	\end{enumerate}
\end{definition}

\begin{example}
Let $V = \mathbb{R}^3$. Verify the following.
	\begin{enumerate}[label=\alph*)]
		\item If $\mathbf{e_1}$, $\mathbf{e_2}$, $\mathbf{e_3}$ are the columns of $I_3$, then $\{ \mathbf{e_1} , \mathbf{e_2} , \mathbf{e_3} \}$ is a basis for $\mathbb{R}^3$.
		\item $\{ \begin{bmatrix} 1 & -1 & 0 \end{bmatrix}^\top , \begin{bmatrix} 3 & 2 & -1 \end{bmatrix}^\top , \begin{bmatrix} 3 & 5 & -2 \end{bmatrix}^\top \}$ is a basis for $\mathbb{R}^3$.
	\end{enumerate}
\end{example}

\newpage 

\phantom{2}

\vfill 

\underline{Observations:}
	\begin{itemize}
		\item Invariance Theorem (p.347): If $\{ \mathbf{e}_1 , \mathbf{e}_2 , \ldots , \mathbf{e}_n \}$ is a basis for a vector space $V$ and if $\{ \mathbf{f_1} , \mathbf{f_2} , \ldots , \mathbf{f_m} \}$ is a basis for a vector space $V$, then $m = n$.
	\end{itemize}


\newpage 

\begin{definition}
If $\{ \mathbf{e_1} , \mathbf{e_2} , \ldots , \mathbf{e_n} \}$ is a basis of a nonzero vector space $V$, the number $n$ of vectors in the basis is called the \textbf{dimension}, and we write
	\[
		\dim V = n .
	\]
In the case of the zero vector space, we define $\dim \{ \mathbf{0} \} = 0$. 
\end{definition}

\underline{Note:}
	\begin{enumerate}[label=\Circled{\arabic*}]
		\item We have $\dim \mathbb{R}^m = m$ because the $n$ columns of the identity matrix $I_m$ is a basis.
		\item We have $\dim \mathbf{M_{mn}} = mn$. Let $E_{ij}$ be the matrix with a $1$ in the $(i,j)$-entry and $0$ elsewhere. A basis for $\mathbf{M_{mn}}$ is
			\[
				B = \{ M_{ij} \, : \, 1 \leq i \leq m , \, 1 \leq j \leq n \} .
			\]
		This is called the \textbf{canonical basis} or \textbf{standard basis} of $\mathbf{M_{mn}}$. For instance, if $m = n = 2$, then a basis for $\mathbf{M_{22}}$ is
			\begin{align*}
				B &= \{ M_{11} , M_{12} , M_{21} , M_{22} \} \\ 
				&= \left\{ \phantom{\begin{bmatrix} 1 & 0 \\ 0 & 0 \end{bmatrix} , \begin{bmatrix} 0 & 1 \\ 0 & 0 \end{bmatrix} , \begin{bmatrix} 0 & 0 \\ 1 & 0 \end{bmatrix} , \begin{bmatrix} 0 & 0 \\ 0 & 1 \end{bmatrix}} \hspace*{5cm} \right\}
			\end{align*}
		\item We have $\dim \mathbf{P_n} = \phantom{n + 1}$.
		\vfill 
		\item Any subspace $U$ of a vector space $V$ is a vector space. Therefore, we can find the dimension of $U$.
	\end{enumerate}


\newpage 

\subsubsection{Subspaces of $\mathbb{R}^m$}

For subspaces of $\mathbb{R}^m$, there is a really nice way to determine a basis and the dimension of a spanning set. Let
	\[
		U = \Span \{ \mathbf{v_1} , \mathbf{v_2} , \ldots , \mathbf{v_n} \} .
	\]
Let 
	\begin{itemize}
		\item $A = \begin{bmatrix} \mathbf{v_1} & \mathbf{v_2} & \cdots & \mathbf{v_n} \end{bmatrix}$. 
		\item $R$ be the RREF of $A$.
	\end{itemize}
Then
	\begin{itemize}
		\item $\dim U = $ number of pivots in $R$.
		\item A basis for $U$ is given by the vector in the same column as the pivots.
	\end{itemize}


\begin{example}
Find a basis and calculate the dimension for the following subspace of $\mathbb{R}^4$:
	\[
		U = \Span \{ (1, -1, 2, 0) , (2, 3, 0, 3) , (1, 9, -6, 6) \} .
	\]
\end{example}

\begin{solution}

\end{solution}

\newpage 

\phantom{2}

\vfill 

\underline{Note:} This trick also works for subspaces of the space of polynomials $\mathbf{P_n}$.

\newpage 

\subsubsection{Subspaces of Matrices}

\begin{example}
Define the subspace of $\mathbf{M_{22}}$ as
	\[
		U = \left\{ X \in \mathbf{M_{22}} \, : \, \begin{bmatrix} 1 & 0 \\ 1 & 0 \end{bmatrix} X = X \begin{bmatrix} 1 & 0 \\ 1 & 0 \end{bmatrix} \right\} .
	\]
Find a basis of $U$ and its dimension.
\end{example}

\begin{solution}

\end{solution}

\newpage 

\phantom{2}

\end{document}