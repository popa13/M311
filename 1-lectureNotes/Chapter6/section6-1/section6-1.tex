\documentclass[20pt,a4paper]{extarticle}
\usepackage[utf8]{inputenc}
\usepackage[english]{babel}

\usepackage{amsmath}
\usepackage{amsfonts}
\usepackage{amssymb}
\usepackage{mathtools}
\usepackage{systeme}
\sysdelim..

\usepackage{graphicx}
\usepackage{caption}
\usepackage{subcaption}
\usepackage{lmodern}
\usepackage{tikz}
\usetikzlibrary{calc}
\usepackage{titlesec}
\usepackage{environ}
\usepackage{xcolor}
\usepackage{fancyhdr}
\usepackage[colorlinks = true, linkcolor = black]{hyperref}
\usepackage{xparse}
\usepackage{enumitem}
\usepackage{comment}
\usepackage{wrapfig}
\usepackage{soul}
\usepackage[capitalise]{cleveref}
\usepackage{circledsteps}

\usepackage[left=1cm,right=1cm,top=1cm,bottom=3cm]{geometry}
\usepackage{multicol}
\usepackage[indent=0pt]{parskip}

\newcommand{\spaceP}{\vspace*{0.5cm}}
\newcommand{\range}{\mathrm{range}\,}
\newcommand{\ra}{\rightarrow}
\newcommand{\curl}{\mathrm{curl} \,}
\newcommand{\hint}[1]{\scalebox{2}{$\displaystyle\int_{\scalebox{0.35}{$#1$}}$}\,}
\newcommand{\hiint}[1]{\scalebox{2}{$\displaystyle\iint_{\scalebox{0.35}{$#1$}}$}\,}
\newcommand{\hiiint}[1]{\scalebox{2}{$\displaystyle\iiint_{\scalebox{0.35}{$#1$}}$}\,}
\renewcommand{\div}{\mathrm{div}\,}

\DeclareMathOperator{\Span}{span}

\makeatletter
\renewcommand*\env@matrix[1][*\c@MaxMatrixCols c]{%
  \hskip -\arraycolsep
  \let\@ifnextchar\new@ifnextchar
  \array{#1}}
\makeatother

%% Redefining sections
\newcommand{\sectionformat}[1]{%
    \begin{tikzpicture}[baseline=(title.base)]
        \node[rectangle, draw] (title) {#1};
    \end{tikzpicture}
    
    \noindent\hrulefill
}

\newif\ifhNotes 

\hNotesfalse

\ifhNotes
	\newcommand{\hideNotes}[1]{%
	\phantom{#1}
	}
	\newcommand{\hideNotesU}[1]{%
	\underline{\hspace{1mm}\phantom{#1}\hspace{1mm}}
	}
\else
	\newcommand{\hideNotes}[1]{#1}
	\newcommand{\hideNotesU}[1]{\textcolor{blue}{#1}}
\fi

% default values copied from titlesec documentation page 23
% parameters of \titleformat command are explained on page 4
\titleformat%
    {\section}% <command> is the sectioning command to be redefined, i. e., \part, \chapter, \section, \subsection, \subsubsection, \paragraph or \subparagraph.
    {\normalfont\large\scshape}% <format>
    {}% <label> the number
    {0em}% <sep> length. horizontal separation between label and title body
    {\centering\sectionformat}% code preceding the title body  (title body is taken as argument)

%% Set counters for sections to none
\setcounter{secnumdepth}{0}

%% Set the footer/headers
\pagestyle{fancy}
\fancyhf{}
\renewcommand{\headrulewidth}{0pt}
\renewcommand{\footrulewidth}{2pt}
\lfoot{P.-O. Paris{\'e}}
\cfoot{MATH 311}
\rfoot{Page \thepage}

%% Defining example environment
\newcounter{example}
\NewEnviron{example}%
	{%
	\noindent\refstepcounter{example}\fcolorbox{gray!40}{gray!40}{\textsc{\textcolor{red}{Example~\theexample.}}}%
	%\fcolorbox{black}{white}%
		{  %\parbox{0.95\textwidth}%
			{
			\BODY
			}%
		}%
	}

\newcounter{theorem}
\NewEnviron{theorem}%
	{%
	\noindent\refstepcounter{theorem}\fcolorbox{gray!40}{gray!40}{\textsc{\textcolor{black}{Theorem~\thetheorem.}}}%
	%\fcolorbox{black}{white}%
		{  %\parbox{0.95\textwidth}%
			{
			\BODY
			}%
		}%
	}

\newcounter{definition}
\NewEnviron{definition}%
	{%
	\noindent\refstepcounter{definition}\fcolorbox{gray!40}{gray!40}{\textsc{\textcolor{black}{Definition~\thedefinition.}}}%
	%\fcolorbox{black}{white}%
		{  %\parbox{0.95\textwidth}%
			{
			\BODY
			}%
		}%
	}

\newcounter{algo}
\NewEnviron{algorithm}
	{%
	\noindent\refstepcounter{algo}\fcolorbox{gray!40}{gray!40}{\textsc{\textcolor{black}{Algorithm~\thealgo.}}}%
	%\fcolorbox{black}{white}%
		{  %\parbox{0.95\textwidth}%
			{
			\BODY
			}%
		}%
	}

\NewEnviron{goal}
	{%
	\noindent\fcolorbox{gray!40}{gray!40}{\textsc{\textcolor{black}{Goal:}}}%
	%\fcolorbox{black}{white}%
		{  %\parbox{0.95\textwidth}%
			{
			\BODY
			}%
		}%
	}

\NewEnviron{solution}%
	{%
	\noindent \fcolorbox{gray!40}{gray!40}{\textsc{\textcolor{blue}{Solution.}}}%
	%\fcolorbox{black}{white}%
		{  %\parbox{0.95\textwidth}%
			{
			%\textcolor{blue}
			}%
		}%
	}

\NewEnviron{proof}%
	{%
	\noindent \fcolorbox{gray!40}{gray!40}{\textsc{\textcolor{blue}{Proof.}}}%
	%\fcolorbox{black}{white}%
		{  %\parbox{0.95\textwidth}%
			{
			\textcolor{blue}{%
			\BODY
			}
			}%
		}%
	}
%%% Ignorer les notes
%\excludecomment{notes}

%%%%
\begin{document}
\thispagestyle{empty}

\begin{center}
\vspace*{0.75cm}

{\Huge \textsc{Math 311}}

\vspace*{1.5cm}

{\LARGE \textsc{Chapter 6}} 

\vspace*{0.75cm}

\noindent\textsc{Section 6.1: Vector Spaces}

\vspace*{0.25cm}

\tableofcontents

\vfill

\noindent \textsc{Created by: Pierre-Olivier Paris{\'e}} \\
\textsc{Spring 2024}
\end{center}

\newpage

\section{Definition}

\subsection{Column Vectors}

Recall that 
	$$
		\mathbb{R}^n = \{ \mathbf{x} \, : \, \mathbf{x} \text{ is an } n \times 1 \text{ vector} \} .
	$$
	\begin{enumerate}[label=\Circled{\arabic*}]
		\item For addition:
			\begin{enumerate}[label=\fbox{A\arabic*.}]
				\item %$\mathbf{x} , \mathbf{y} \in \mathbb{R}^n$ $\Rightarrow$ $\mathbf{x} + \mathbf{y} \in \mathbb{R}^n$.
				\item %$\mathbf{x} + \mathbf{y} = \mathbf{y} + \mathbf{x}$.
				\item %$\mathbf{x} + (\mathbf{y} + \mathbf{z} ) = (\mathbf{x} + \mathbf{y} ) + \mathbf{z}$.
				\item %$\mathbf{x} + \mathbf{0} = \mathbf{x} = \mathbf{0} + \mathbf{x}$. 
				\item \phantom{For each $n \times 1$ row-vector $\mathbf{x}$, there is a $n \times 1$ row vector $\mathbf{y}$ such that $\mathbf{x} + \mathbf{y} = \mathbf{y} + \mathbf{x} = \mathbf{0}$, that is $\mathbf{y} = -\mathbf{x}$.}
				\vspace*{0.1cm}
			\end{enumerate}
		\item For scalar multiplication:
			\begin{enumerate}[label=\fbox{S\arabic*.}]
				\item %$\mathbf{x} \in \mathbb{R}^n$ $\Rightarrow$ $a \mathbf{x} \in \mathbb{R}^n$.
				\item %$a (\mathbf{x} + \mathbf{y}) = a \mathbf{x} + a \mathbf{y}$. 
				\item %$(a + b) \mathbf{x} = a \mathbf{x} + a \mathbf{x}$.
				\item %$a (b \mathbf{x}) = (ab) \mathbf{x}$.
				\item %$1 \mathbf{x} = \mathbf{x}$.
			\end{enumerate}
	\end{enumerate}

\underline{Conclusion:} %We say that $\mathbb{R}^n$ is a \textbf{vector space}.

\newpage 

\subsection{General Definition}

Let $V$ be a set of objects called \textbf{vectors}. Assume
	\begin{enumerate}
		\item \textbf{Vector Addition:} Two vectors $\mathbf{v}$ and $\mathbf{w}$ can be added and denote this operation by $\mathbf{v} + \mathbf{w}$.
		\item \textbf{Scalar Multiplication:} Any vector $\mathbf{v}$ can be multiplied by any number (scalar) $a$ and denote this operation by $a \mathbf{v}$.
	\end{enumerate}

The set $V$ is called a \textbf{vector space} if
	\begin{enumerate}
		\item Axioms for the vector addition:
			\begin{enumerate}[label=\fbox{A\arabic*.}]
				\item \underline{Closed:} $\mathbf{v} , \mathbf{w} \in V$ $\Rightarrow$ $\mathbf{v} + \mathbf{w} \in V$.
				\item \underline{Commutativity:} $\mathbf{v} + \mathbf{w} = \mathbf{w} + \mathbf{v}$.
				\item \underline{Associativity:} $\mathbf{v} + (\mathbf{w} + \mathbf{z} ) = (\mathbf{v} + \mathbf{w} ) + \mathbf{z}$.
				\item \underline{Existence of a zero vector:} $\mathbf{v} + \mathbf{0} = \mathbf{v} = \mathbf{0} + \mathbf{v}$. 
				\item \underline{Existence of a negative:} For each $\mathbf{v}$, there is a $\mathbf{w}$ such that $\mathbf{v} + \mathbf{w} = \mathbf{w} + \mathbf{v} = \mathbf{0}$.
			\end{enumerate}
		\item Axioms for the scalar multiplication: 
			\begin{enumerate}[label=\fbox{S\arabic*.}]
				\item $\mathbf{v} \in V$ $\Rightarrow$ $a \mathbf{v} \in V$.
				\item $a (\mathbf{v} + \mathbf{w}) = a \mathbf{v} + a \mathbf{w}$. 
				\item $(a + b) \mathbf{v} = a \mathbf{v} + b \mathbf{v}$.
				\item $a (b \mathbf{v}) = (ab) \mathbf{v}$.
				\item $1 \mathbf{v} = \mathbf{v}$.
			\end{enumerate}
	\end{enumerate}

%\underline{Note:} We can prove that the negative of $\mathbf{v}$ is $\mathbf{w} = -\mathbf{v}$. 

\newpage 

\section{Examples}

\subsection{Spaces of Matrices}

\begin{example}
Let $\mathbf{M_{mn}}$ be the set of all $m \times n$ matrices, that is
	\[
		\mathbf{M_{mn}} := \{ A \, : \, A \text{ is an } m \times n \text{ matrix.} \}
	\]
Consider the addition and scalar multiplication for matrices. Show that $\mathbf{M_{mn}}$ is a vector space.
\end{example}

\begin{solution}

\end{solution}

\newpage

\subsection{Spaces of Polynomials}

\begin{example}
Consider the space $\mathbf{P_3}$ of all polynomials of degree at most $3$, that is
	\[
		\mathbf{P} := \{ a_3 x^3 + a_2 x^2 + a_1 x + a_0 \, : \, a_i \in \mathbb{R} \} .
	\]
Define
	\begin{enumerate}
		\item Addition: for two polynomials $p(x) = a_3 x^3 + a_2 x^2 + a_1 x + a_0$ and $q(x) = b_3 x^3 + b_2 x^2 + b_1 x + b_0$, define $p + q$ as the polynomial
		\begin{small}
			\begin{align*}
				(p + q) (x) &= p (x) + q(x) \\ 
				&= { \small (a_3 + b_3) x^3 + (a_2 + b_2) x^2 + (a_1 + b_1) x + (a_0 + b_0) } .
			\end{align*}
		\end{small}
		\item Scalar multiplicatoin: for a polynomial $p (x) = a_3 x^3 + a_2 x^2 + a_1 x + a_0$, define $ap$ as the polynomial
			\[
				(a p) (x) = a p (x) = (a a_3) x^3 + (a a_2) x^2 + (a a_1)x + (a a_0) .
			\]
	\end{enumerate}
Show that $\mathbf{P_3}$, with this addition and scalar multiplication, is a vector space.
\end{example}

\begin{solution}

\end{solution}

\newpage 

\phantom{2} 

\newpage 

\phantom{2}

\newpage 

\phantom{2} 

\vfill 

\underline{Note:} 
	\begin{enumerate}[label=\Circled{\arabic*}]
		\item The space of polynomial of degree at most $n$ is denoted by $\mathbf{P_n}$ and is a vector space using the addition and scalar multiplication introduced above.
		\item The space of all polynomial of any degree is denoted by $\mathbf{P}$ and it is a vector space using the addition and scalar multiplication introduced above.
	\end{enumerate}

\newpage 

\subsection{Weird Example}

\begin{example}
Consider the set of all $2 \times 1$ vectors $\mathbb{R}^2$. Define the addition and scalar multiplication:
	\begin{enumerate}
		\item $\mathbf{x} + \mathbf{y} = \begin{bmatrix} x_1 \\ x_2 \end{bmatrix} + \begin{bmatrix} y_1 \\ y_2 \end{bmatrix} = \begin{bmatrix} x_1 + y_1 \\ x_2 + y_2 + 1 \end{bmatrix}$.
		\item $a \mathbf{x} = a \begin{bmatrix} x_1 \\ x_2 \end{bmatrix} = \begin{bmatrix} a x_1 \\ a x_2 + a - 1 \end{bmatrix}$. 
	\end{enumerate}
Show that $\mathbb{R}^2$, with these operations, is a vector space.
\end{example}

\begin{solution}

\end{solution}

\newpage

\phantom{2}

\newpage 

\phantom{2} 

\newpage 

\phantom{2} 

\newpage 

\subsection{Non-Example}

\begin{example}
Consider the set of all $2 \times 1$ vectors $\mathbb{R}^2$. Define the addition and scalar multiplication:
	\begin{enumerate}
		\item $\mathbf{x} + \mathbf{y} = \begin{bmatrix} x_1 \\ x_2 \end{bmatrix} + \begin{bmatrix} y_1 \\ y_2 \end{bmatrix} = \begin{bmatrix} x_1 + y_1 \\ x_2 + y_2 + 1 \end{bmatrix}$.
		\item $a \mathbf{x} = a \begin{bmatrix} x_1 \\ x_2 \end{bmatrix} = \begin{bmatrix} a x_1 \\ a x_2 - 1 \end{bmatrix}$. 
	\end{enumerate}
Show that $\mathbb{R}^2$, with these operations, is not a vector space.
\end{example}

\begin{solution}

\end{solution}

\newpage

\section{Properties}

Consider a general vector space $V$.

\begin{enumerate}[label=\Circled{\arabic*}]
	\item Cancellation: If $\mathbf{u}, \mathbf{v} , \mathbf{w} \in V$, then
		\[
			\mathbf{v} + \mathbf{u} = \mathbf{v} + \mathbf{w} \Longrightarrow \mathbf{u} = \mathbf{w} .
		\]
	\item Multiplying by scalar $0$: 
		\[
			0 \mathbf{v} = \mathbf{0}. 
		\]
	\item Multiplying by the zero vector:
		\[
			a \mathbf{0} = \mathbf{0} .
		\]
	\item If $a \mathbf{v} \mathbf{0}$, then $a = 0$ or $\mathbf{v} = \mathbf{0}$.
\end{enumerate}

\begin{example}
Simplify the following expression:
	\[
		3 \big( 2 (\mathbf{u} - 2 \mathbf{v} - \mathbf{w} ) + 3 (\mathbf{w} - \mathbf{v} \big) - 7 ( \mathbf{u} - 3 \mathbf{v} - \mathbf{w} ) .
	\]
\end{example}

\begin{solution}

\end{solution}

\newpage 

\phantom{2} 

\end{document}


where we have
	\begin{itemize}
		\item \underline{an addition:} $\mathbf{x} + \mathbf{y}$ is obtained by adding each respective entry.
		\item \underline{a scalar multiplication:} $a \mathbf{x}$ is obtained by multiplying each entry by $a$.
	\end{itemize}