\documentclass[20pt,a4paper]{extarticle}
\usepackage[utf8]{inputenc}
\usepackage[english]{babel}

\usepackage{amsmath}
\usepackage{amsfonts}
\usepackage{amssymb}
\usepackage{mathtools}
\usepackage{systeme}
\sysdelim..

\usepackage{graphicx}
\usepackage{caption}
\usepackage{subcaption}
\usepackage{lmodern}
\usepackage{tikz}
\usetikzlibrary{calc}
\usepackage{titlesec}
\usepackage{environ}
\usepackage{xcolor}
\usepackage{fancyhdr}
\usepackage[colorlinks = true, linkcolor = black]{hyperref}
\usepackage{xparse}
\usepackage{enumitem}
\usepackage{comment}
\usepackage{wrapfig}
\usepackage{soul}
\usepackage[capitalise]{cleveref}
\usepackage{circledsteps}

\usepackage[left=1cm,right=1cm,top=1cm,bottom=3cm]{geometry}
\usepackage{multicol}
\usepackage[indent=0pt]{parskip}

\newcommand{\spaceP}{\vspace*{0.5cm}}
\newcommand{\range}{\mathrm{range}\,}
\newcommand{\ra}{\rightarrow}
\newcommand{\curl}{\mathrm{curl} \,}
\newcommand{\hint}[1]{\scalebox{2}{$\displaystyle\int_{\scalebox{0.35}{$#1$}}$}\,}
\newcommand{\hiint}[1]{\scalebox{2}{$\displaystyle\iint_{\scalebox{0.35}{$#1$}}$}\,}
\newcommand{\hiiint}[1]{\scalebox{2}{$\displaystyle\iiint_{\scalebox{0.35}{$#1$}}$}\,}
\renewcommand{\div}{\mathrm{div}\,}

\DeclareMathOperator{\Span}{span}

\makeatletter
\renewcommand*\env@matrix[1][*\c@MaxMatrixCols c]{%
  \hskip -\arraycolsep
  \let\@ifnextchar\new@ifnextchar
  \array{#1}}
\makeatother

%% Redefining sections
\newcommand{\sectionformat}[1]{%
    \begin{tikzpicture}[baseline=(title.base)]
        \node[rectangle, draw] (title) {#1};
    \end{tikzpicture}
    
    \noindent\hrulefill
}

\newif\ifhNotes 

\hNotesfalse

\ifhNotes
	\newcommand{\hideNotes}[1]{%
	\phantom{#1}
	}
	\newcommand{\hideNotesU}[1]{%
	\underline{\hspace{1mm}\phantom{#1}\hspace{1mm}}
	}
\else
	\newcommand{\hideNotes}[1]{#1}
	\newcommand{\hideNotesU}[1]{\textcolor{blue}{#1}}
\fi

% default values copied from titlesec documentation page 23
% parameters of \titleformat command are explained on page 4
\titleformat%
    {\section}% <command> is the sectioning command to be redefined, i. e., \part, \chapter, \section, \subsection, \subsubsection, \paragraph or \subparagraph.
    {\normalfont\large\scshape}% <format>
    {}% <label> the number
    {0em}% <sep> length. horizontal separation between label and title body
    {\centering\sectionformat}% code preceding the title body  (title body is taken as argument)

%% Set counters for sections to none
\setcounter{secnumdepth}{0}

%% Set the footer/headers
\pagestyle{fancy}
\fancyhf{}
\renewcommand{\headrulewidth}{0pt}
\renewcommand{\footrulewidth}{2pt}
\lfoot{P.-O. Paris{\'e}}
\cfoot{MATH 311}
\rfoot{Page \thepage}

%% Defining example environment
\newcounter{example}
\NewEnviron{example}%
	{%
	\noindent\refstepcounter{example}\fcolorbox{gray!40}{gray!40}{\textsc{\textcolor{red}{Example~\theexample.}}}%
	%\fcolorbox{black}{white}%
		{  %\parbox{0.95\textwidth}%
			{
			\BODY
			}%
		}%
	}

\newcounter{theorem}
\NewEnviron{theorem}%
	{%
	\noindent\refstepcounter{theorem}\fcolorbox{gray!40}{gray!40}{\textsc{\textcolor{black}{Theorem~\thetheorem.}}}%
	%\fcolorbox{black}{white}%
		{  %\parbox{0.95\textwidth}%
			{
			\BODY
			}%
		}%
	}

\newcounter{definition}
\NewEnviron{definition}%
	{%
	\noindent\refstepcounter{definition}\fcolorbox{gray!40}{gray!40}{\textsc{\textcolor{black}{Definition~\thedefinition.}}}%
	%\fcolorbox{black}{white}%
		{  %\parbox{0.95\textwidth}%
			{
			\BODY
			}%
		}%
	}

\newcounter{algo}
\NewEnviron{algorithm}
	{%
	\noindent\refstepcounter{algo}\fcolorbox{gray!40}{gray!40}{\textsc{\textcolor{black}{Algorithm~\thealgo.}}}%
	%\fcolorbox{black}{white}%
		{  %\parbox{0.95\textwidth}%
			{
			\BODY
			}%
		}%
	}

\NewEnviron{goal}
	{%
	\noindent\fcolorbox{gray!40}{gray!40}{\textsc{\textcolor{black}{Goal:}}}%
	%\fcolorbox{black}{white}%
		{  %\parbox{0.95\textwidth}%
			{
			\BODY
			}%
		}%
	}

\NewEnviron{solution}%
	{%
	\noindent \fcolorbox{gray!40}{gray!40}{\textsc{\textcolor{blue}{Solution.}}}%
	%\fcolorbox{black}{white}%
		{  %\parbox{0.95\textwidth}%
			{
			%\textcolor{blue}
			}%
		}%
	}

\NewEnviron{proof}%
	{%
	\noindent \fcolorbox{gray!40}{gray!40}{\textsc{\textcolor{blue}{Proof.}}}%
	%\fcolorbox{black}{white}%
		{  %\parbox{0.95\textwidth}%
			{
			\textcolor{blue}{%
			\BODY
			}
			}%
		}%
	}
%%% Ignorer les notes
%\excludecomment{notes}

%%%%
\begin{document}
\thispagestyle{empty}

\begin{center}
\vspace*{0.75cm}

{\Huge \textsc{Math 311}}

\vspace*{1.5cm}

{\LARGE \textsc{Chapter 6}} 

\vspace*{0.75cm}

\noindent\textsc{Section 6.2: Linear Combination and Subspaces}

\vspace*{0.25cm}

\tableofcontents

\vfill

\noindent \textsc{Created by: Pierre-Olivier Paris{\'e}} \\
\textsc{Spring 2024}
\end{center}

\newpage


\section{Subspaces}

\begin{example}\label{Ex:SolutionHomoSystem}
The solution to the homogeneous system
	\[
		\begin{bmatrix}1 & 2 & 3 & 4 & 5 & 1\\2 & 3 & 4 & 1 & 2 & -2\\1 & 2 & 4 & 5 & 3 & -1\\3 & 1 & 2 & 4 & 5 & 1\end{bmatrix} \mathbf{x} = \mathbf{0} 
	\]
is
	\[
		\mathbf{x} = t \begin{bmatrix} 1 \\ 9 \\ -7 \\ 3 \end{bmatrix} + s \begin{bmatrix} 0 \\ 3 \\ -3 \\ 1 \end{bmatrix} = t \mathbf{x_1} + s \mathbf{x_2} , \quad s, t \in \mathbb{R} .
	\]
Notice that
	\begin{enumerate}[label=\fbox{S\arabic*.}]
		\item \phantom{Letting $t = s = 0$, the \underline{zero vector} is a solution.}
		\item \phantom{If $\mathbf{x} = 5 \mathbf{x_1} + 3 \mathbf{x_2}$ and $\mathbf{y} = -3 \mathbf{x_1} + 2 \mathbf{x_1}$, then}
			\[
				\phantom{\mathbf{x} + \mathbf{y} = (5 - 3) \mathbf{x_1} + (3 + 2) \mathbf{x_2} = 2 \mathbf{x_1} + 5 \mathbf{x_2}}
			\]
		\phantom{is \underline{still} a solution.}
		\item \phantom{If $\mathbf{x} = 5 \mathbf{x_1} + 3 \mathbf{x_2}$, then}
			\[
				\phantom{2 \mathbf{x} = 10 \mathbf{x_1} + 6 \mathbf{x_2}}
			\]
		\phantom{is \underline{still} a solution.}
	\end{enumerate}
If $U = \{ t \mathbf{x_1} + s \mathbf{x_2} \, : \, s, t \in \mathbb{R} \}$ is the set of all solutions, then $U$ is called a \textbf{subspace}.
\end{example}

\newpage 

\begin{definition}
A subset $U$ of a vector space $V$ is called a \textbf{subspace} of $V$ if it satisfies the following properties:
	\begin{enumerate}[label=\fbox{S\arabic*.}]
		\item The zero vector $\mathbf{0} \in U$.
		\item If $\mathbf{u_1} \in U$ and $\mathbf{u_2} \in U$, then $\mathbf{u_1} + \mathbf{u_2} \in U$. 
		\item If $\mathbf{u} \in U$ and $a$ is a scalar, then $a\mathbf{u} \in U$. 
	\end{enumerate}
\end{definition}

\underline{Remarks:}
	\begin{enumerate}[label=\Circled{\arabic*}]
		\item S2: $U$ is said to be \textbf{closed under addition}.
		\item S3: $U$ is said to be \textbf{closed under scalar multiplication}.
		\item A subspace is a vector subspace itself.
	\end{enumerate}


\begin{example}
Let $V$ be a vector space. Show that $U = \{ \mathbf{0} \}$ is a subspace of $V$. This space is called the \textbf{zero subspace}.
\end{example}

\begin{solution}

\end{solution}

\vfill 

\underline{Note:} Any subspace $U$ of $V$ such that $U \neq \{ \mathbf{0} \}$ and $U \neq V$ is called a \textbf{proper subspace}. 

\newpage 

\subsection{Important Examples}

\begin{example}
Given an $m \times n$ matrix $A$, define 
	$$ 
		\mathrm{null} A := \{ \mathbf{x} \in \mathbb{R}^n \, : \, A \mathbf{x} = \mathbf{0} \} .
	$$
Show that $\mathrm{null} A$ is a subspace of $\mathbb{R}^n$. 
\end{example}

\begin{solution}

\end{solution}

\vfill 

\underline{Note:} The subspace $\mathrm{null} A$ is called the \textbf{null space} of a matrix $A$. It is the set of all solutions to the homogeneous system $A \mathbf{x} = \mathbf{0}$. 

\newpage 

\begin{example}
Given an $m \times n$ matrix $A$, define
	\[
		\mathrm{im} A := \{ A \mathbf{x} \, : \, \mathbf{x} \in \mathbb{R}^n \} .
	\]
Show that $\mathrm{im} A$ is a subspace of $\mathbb{R}^m$. 
\end{example}

\begin{solution}

\end{solution}

\vfill 

\underline{Note:} The subspace $\mathrm{im} A$ is called the \textbf{image space} (or \textbf{range space}) of the matrix $A$. It is the set of all vectors $\mathbf{b}$ such that $A \mathbf{x} = \mathbf{b}$ has a solution.

\newpage 

\begin{example}
For an $n \times n$ matrix $A$ and a number $\lambda$, define
	\[
		E_\lambda (A) := \{ \mathbf{x} \in \mathbb{R}^n \, : \, A \mathbf{x} = \lambda \mathbf{x} \} .
	\]
Show that $E_{\lambda} (A)$ is a subspace of $\mathbb{R}^n$. 
\end{example}

\begin{solution}

\end{solution}

\vfill 

\underline{Note:} When $\lambda$ is an eigenvalue of $A$, the subspace $\mathrm{E_\lambda} (A)$ is called the \textbf{eigenspace} associated to $\lambda$. 

\newpage 

\subsection{More Examples}

\begin{example}
Let $\mathbf{M}_{nn}$ be the vector space of $n \times n$ matrices. Show that $U = \{ A \, : \, A^\top = A \}$ is a subspace of $\mathbf{M}_{nn}$. 
\end{example}

\begin{solution}

\end{solution}

\vfill 

\underline{Note:} The set $U$ is the subspace of all symmetric matrices.

\newpage 

\subsection{Non-Examples}

\begin{example}
Show that the set
	\[
		U = \{ p \, : \, p \in \mathbf{P}_3 \text{ and } p (2) = 1 \}
	\]
is not a subspace of $\mathbf{P}_3$.
\end{example}

\begin{solution}

\end{solution}

\newpage 

\section{Spanning Sets}

\begin{example}
The solutions set to the system $A \mathbf{x} = \mathbf{0}$ given in Example \ref{Ex:SolutionHomoSystem} is given by the linear combination
	\[
		t \mathbf{x_1} + s \mathbf{x_2} , \quad  t, s \in \mathbb{R} .
	\]
The set $\{ t \mathbf{x_1} + s \mathbf{x_2} \, : \, t, s \in \mathbb{R} \}$ is called the \textbf{span} of $\mathbf{x_1}$ and $\mathbf{x_2}$. 
\end{example}

\vspace*{1cm}

\begin{definition}
Let $\{ \mathbf{v_1} , \mathbf{v_2} , \ldots \, \mathbf{v_n} \}$ be a collection of vectors in a vector space $V$.
	\begin{enumerate}[label=\Circled{\arabic*}]
		\item a \textbf{linear combination} of the vectors $\mathbf{v_1}$, $\mathbf{v_2}$, $\ldots$, $\mathbf{v_n}$ is an expression of the form
			\[
				a_1 \mathbf{v_1} + a_2 \mathbf{v_2} + \cdots + a_n \mathbf{v_n} 
			\]
		where $a_1$, $a_2$, $\ldots$, $a_n$ are scalars called the \textbf{coefficients} of each vector.
		\item The set of all linear combinations of these vectors is called their \textbf{span}. %and is the set
			%\[
			%	\mathrm{span} \, \{ \mathbf{x_1} , \mathbf{x_2} , \ldots , \mathbf{x_n} \} = \{ a_1 \mathbf{x_1} + a_2 \mathbf{x_2} + \cdots + a_n \mathbf{x_n} \, : \, a_1, a_2 , \ldots , a_n \in \mathbb{R} \} .
			%\]
		\item If it happens that $V = \mathrm{span} \{ \mathbf{v_1} , \mathbf{v_2} , \ldots , \mathbf{v_n} \}$, then the vectors are called a \textbf{spanning set} for $V$. 
	\end{enumerate}
\end{definition}

\underline{Remarks:}
	\begin{itemize}
		\item $\Span \{ \mathbf{v_1} , \mathbf{v_2} , \ldots , \mathbf{v_n} \}$ is a subspace of $V$.
		%\item $\Span \{ \mathbf{v_1} , \mathbf{v_2} , \ldots , \mathbf{v_n} \}$ is the smallest subspace containing $\mathbf{v_1}$, $\mathbf{v_2}$, $\ldots$, $\mathbf{v_n}$. 
	\end{itemize}

\newpage 

\begin{example}
Consider $p_1 = 1 + x + 4x^2$ and $p_2 = 1 + 5x + x^2$, two polynomials in $\mathbf{P}_2$. 
	\begin{enumerate}[label=\alph*)]
		\item Is $p_1$ in the $\mathrm{span} \{ 1 + 2x - x^2 , 3 + 5x + 2x^2 \}$.
		\item Is $p_2$ in the $\mathrm{span} \{ 1 + 2x - x^2 , 3 + 5x + 2x^2 \}$. 
	\end{enumerate}
\end{example}

\begin{solution}

\end{solution}

\newpage 

\phantom{2} 


\end{document}