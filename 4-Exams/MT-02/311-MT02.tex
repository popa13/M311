\documentclass[addpoints, 12pt]{exam}%, answers]
\usepackage[utf8]{inputenc}
\usepackage[T1]{fontenc}

\usepackage{lmodern}
\usepackage{arydshln}
\usepackage[margin=2cm]{geometry}

\usepackage{enumitem}
\usepackage{systeme}

\usepackage{amsmath, amsthm, amsfonts, amssymb}
\usepackage{graphicx}
\usepackage{tikz}
\usetikzlibrary{arrows,calc,patterns}
\usepackage{pgfplots}
\pgfplotsset{compat=newest}
\usepackage{url}
\usepackage{multicol}
\usepackage{thmtools}
\usepackage{wrapfig}

\usepackage{caption}
\usepackage{subcaption}
\usepackage{pdfpages}

\usepackage{pifont}

% MATH commands
\newcommand{\bC}{\mathbb{C}}
\newcommand{\bR}{\mathbb{R}}
\newcommand{\bN}{\mathbb{N}}
\newcommand{\bZ}{\mathbb{Z}}
\newcommand{\bT}{\mathbb{T}}
\newcommand{\bD}{\mathbb{D}}

\newcommand{\cL}{\mathcal{L}}
\newcommand{\cM}{\mathcal{M}}
\newcommand{\cP}{\mathcal{P}}
\newcommand{\cH}{\mathcal{H}}
\newcommand{\cB}{\mathcal{B}}
\newcommand{\cK}{\mathcal{K}}
\newcommand{\cJ}{\mathcal{J}}
\newcommand{\cU}{\mathcal{U}}
\newcommand{\cO}{\mathcal{O}}
\newcommand{\cA}{\mathcal{A}}
\newcommand{\cC}{\mathcal{C}}
\newcommand{\cF}{\mathcal{F}}

\newcommand{\fK}{\mathfrak{K}}
\newcommand{\fM}{\mathfrak{M}}

\newcommand{\ga}{\left\langle}
\newcommand{\da}{\right\rangle}
\newcommand{\oa}{\left\lbrace}
\newcommand{\fa}{\right\rbrace}
\newcommand{\oc}{\left[}
\newcommand{\fc}{\right]}
\newcommand{\op}{\left(}
\newcommand{\fp}{\right)}

\newcommand{\ra}{\rightarrow}
\newcommand{\Ra}{\Rightarrow}

\renewcommand{\Re}{\mathrm{Re}\,}
\renewcommand{\Im}{\mathrm{Im}\,}
\newcommand{\Arg}{\mathrm{Arg}\,}
\newcommand{\Arctan}{\mathrm{Arctan}\,}
\newcommand{\sech}{\mathrm{sech}\,}
\newcommand{\csch}{\mathrm{csch}\,}
\newcommand{\Log}{\mathrm{Log}\,}
\newcommand{\cis}{\mathrm{cis}\,}

\newcommand{\ran}{\mathrm{ran}\,}
\newcommand{\bi}{\mathbf{i}}
\newcommand{\Sp}{\mathrm{span}\,}
\newcommand{\Inv}{\mathrm{Inv}\,}
\newcommand\smallO{
  \mathchoice
    {{\scriptstyle\mathcal{O}}}% \displaystyle
    {{\scriptstyle\mathcal{O}}}% \textstyle
    {{\scriptscriptstyle\mathcal{O}}}% \scriptstyle
    {\scalebox{.7}{$\scriptscriptstyle\mathcal{O}$}}%\scriptscriptstyle
  }
\newcommand{\HOL}{\mathrm{Hol}}
\newcommand{\cl}{\mathrm{clos}}
\newcommand{\ve}{\varepsilon}

\DeclareMathOperator{\dom}{dom}

\makeatletter
\renewcommand*\env@matrix[1][*\c@MaxMatrixCols c]{%
  \hskip -\arraycolsep
  \let\@ifnextchar\new@ifnextchar
  \array{#1}}
\makeatother

%%%%%% Définitions Theorems and al.
%\declaretheoremstyle[preheadhook = {\vskip0.2cm}, mdframed = {linewidth = 2pt, backgroundcolor = yellow}]{myThmstyle}
%\declaretheoremstyle[preheadhook = {\vskip0.2cm}, postfoothook = {\vskip0.2cm}, mdframed = {linewidth = 1.5pt, backgroundcolor=green}]{myDefstyle}
%\declaretheoremstyle[bodyfont = \normalfont , spaceabove = 0.1cm , spacebelow = 0.25cm, qed = $\blacktriangle$]{myRemstyle}

%\declaretheorem[ style = myThmstyle, name=Th\'eor\`eme]{theorem}
%\declaretheorem[style =myThmstyle, name=Proposition]{proposition}
%\declaretheorem[style = myThmstyle, name = Corollaire]{corollary}
%\declaretheorem[style = myThmstyle, name = Lemme]{lemma}
%\declaretheorem[style = myThmstyle, name = Conjecture]{conjecture}

%\declaretheorem[style = myDefstyle, name = D\'efinition]{definition}

%\declaretheorem[style = myRemstyle, name = Remarque]{remark}
%\declaretheorem[style = myRemstyle, name = Remarques]{remarks}

\newtheorem{theorem}{Théorème}
\newtheorem{corollary}{Corollaire}
\newtheorem{lemma}{Lemme}
\newtheorem{proposition}{Proposition}
\newtheorem{conjecture}{Conjecture}

\theoremstyle{definition}

\newtheorem{definition}{Définition}[section]
\newtheorem{example}{Exemple}[section]
\newtheorem{remark}{\textcolor{red}{Remarque}}[section]
\newtheorem{exer}{\textbf{Exercice}}[section]


\tikzstyle{myboxT} = [draw=black, fill=black!0,line width = 1pt,
    rectangle, rounded corners = 0pt, inner sep=8pt, inner ysep=8pt]

\begin{document}
	\noindent \hrulefill \\
	\noindent MATH-311 \hfill Created by Pierre-O. Paris{\'e}\\
	Midterm 02 (50min)\hfill March, Spring 2024\\\vspace*{-0.7cm}

\noindent\hrulefill

\vspace*{0.5cm}

\begin{center}
\begin{minipage}{0.6\textwidth}
\begin{Huge}
\textsc{University of Hawai'i}
\end{Huge}
\end{minipage}
\begin{minipage}{0.12\textwidth}
\includegraphics[scale=0.05]{../../../../manoaseal_transparent.png}
\end{minipage}
\end{center}
	
\vspace*{0.5cm}

\noindent\makebox[\textwidth]{\textbf{Last name:}\enspace \hrulefill}

\vspace*{0.5cm}

\noindent\makebox[\textwidth]{\textbf{First name:}\enspace\hrulefill}

\vspace*{1cm}

\begin{center}
\gradetable[h][questions]
\end{center}

\vspace*{1cm}

\noindent\textbf{Instructions:} 

\begin{itemize}
\item Write your complete name on your copy. 
\item Answer all 6 questions below.
\item Write your answers directly on the questionnaire.
\item Show ALL your work to have full credit.
\item Draw a square around your final answer.
\item Return your copy when you're done or at the end of the 50min period. 
\item No electronic devices allowed during the exam. 
\item Scientific calculator allowed only (no graphical calculators).
\item \textbf{Turn off your cellphone(s) during the exam}.
\item Lecture notes and the textbook are not allowed during the exam. 
\end{itemize}

\vspace{0.5cm}

\noindent\textbf{Your Signature:} \hrulefill

\vspace*{1.5cm}

\noindent\textsc{May the Force be with you!}\\
\textsc{Pierre}

\newpage % End of cover page

\phantom{2}

\newpage

\qformat{\rule{0.3\textwidth}{.4pt} \begin{large}{\textsc{Question}} \thequestion \end{large} \hspace*{0.2cm} \hrulefill \hspace*{0.1cm} \textbf{(\totalpoints\hspace*{0.1cm} pts)}}

\pointpoints{Pt}{Pts}


\begin{questions}

\question[10]
Say if the following matrix products are well-defined. If it is well-defined, then compute the matrix products.
$$
  \begin{bmatrix} 1 & 0 & -1 \\ 1 & -1 & 9 \\ 1 & 1 & -1 \end{bmatrix} \begin{bmatrix} 1 & 0 & -1 \\ 1 & -1 & -1 \\ -1 & -1 & -1 \end{bmatrix} .
$$

\newpage 

\question[10]

Find the values of the entries of the matrix $A$ if
  \[
    \Big( \begin{bmatrix} 1 & 0 \\ 2 & 1 \end{bmatrix} A \Big)^{-1} = \begin{bmatrix} 1 & 0 \\ 2 & 2 \end{bmatrix} - 2 \begin{bmatrix} 1 & 3 \\ 2 & 4 \end{bmatrix} .
  \]


\newpage 

\question[10]

Find a $2 \times 2$ elementary matrix $E$ such that
  \[
    E  \begin{bmatrix} 3 & 0 & 1 \\ 2 & -1 & 0 \end{bmatrix} = \begin{bmatrix} 1 & 1 & 1 \\ 2 & -1 & 0 \end{bmatrix} .
  \]

\newpage 

\question 
Evaluate the determinant of the matrix $A$.

\begin{multicols}{2}
\begin{parts}
\part[2] 
$A= \begin{bmatrix} 1 & -1 & 2 \\ 3 & 1 & 1 \\ 2 & -1 & 3 \end{bmatrix}$.
\part[2] 
$A = \begin{bmatrix} 1 & 1 & 1 \\ 2 & 2 & 2 \\ 3 & 3 & 3 \end{bmatrix}$.
\part[2] 
$A = \begin{bmatrix} 1 & 1 & 1 & 1 \\ 0 & 0 & 0 & 0 \\ 2 & -1 & 1 & 4 \\ 1 & -1 & 1 & -1 \end{bmatrix}$.
\part[2] 
$A = \begin{bmatrix} 1 & 45 & 3 & 4 & 3 & 4 \\ 0 & 1 & 9 & 100 & 4 & 45 \\ 0 & 0 & 1 & 45 & -3 & -2 \\ 0 & 0 & 0 & 5 & 4 & -1 \\ 0 & 0 & 0 & 0 & 1 & -1 \\ 0 & 0 & 0 & 0 & 0 & 4 \end{bmatrix}$.
\part[2] 
$A = \begin{bmatrix} 1 & 1 & 2 \\ 0 & 1 & 1 \\ 1 & 1 & 1 \end{bmatrix}$ if $\begin{vmatrix} 1 & 1 & 2 \\ 1 & 2 & 3 \\ 1 & 1 & 1 \end{vmatrix} = 1$. 
\end{parts}
\end{multicols}


\newpage 

\question[6]

Let $A$ be an $n \times n$ matrix. Assume that $A^2 = 0$ and $I - A$ is invertible, where $I$ is the $n \times n$ identity matrix. Show that
  \[
    (I - A)^{-1} = I + A .
  \]

\newpage 

\question[4]
\noaddpoints
Answer the following questions with \textbf{True} or \textbf{False}. Write down you answers on the line at the end of each question. Justify briefly your answer in the space after the statement of the problem.

  \begin{parts}
  \pointformat{(\hspace*{0.35cm}/ \thepoints)}
  \pointname{}
  \pointsinrightmargin
  
  \part[1]
  If $A$ is an $n \times n$ matrix and $A^2 = I$, then $A = \pm I$. 
  \begin{solution}[\stretch{1}]
  
  \end{solution}
  \answerline[True]
  
  \part[1]
  If $A$ and $B$ are $n \times n$ matrices, then $AB = BA$.
  \begin{solution}[\stretch{1}]
  
  \end{solution}
  \answerline[False]
  
  \part[1]
  If $A$ and $B$ are $n \times n$ invertible matrices, then $A + B$ is invertible. 
  \begin{solution}[\stretch{1}]
  
  \end{solution}
  \answerline[False]
  
  \part[1]
  If $A$ and $B$ are $n \times n$ matrices, then $(AB)^\top = A^\top B^\top$.
  \begin{solution}[\stretch{1}]
  
  \end{solution}
  \answerline[False]

  \end{parts}

\end{questions}


\end{document}