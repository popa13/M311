\documentclass[addpoints, 12pt]{exam}%, answers]
\usepackage[utf8]{inputenc}
\usepackage[T1]{fontenc}

\usepackage{lmodern}
\usepackage{arydshln}
\usepackage[margin=2cm]{geometry}

\usepackage{enumitem}
\usepackage{systeme}

\usepackage{amsmath, amsthm, amsfonts, amssymb}
\usepackage{graphicx}
\usepackage{tikz}
\usetikzlibrary{arrows,calc,patterns}
\usepackage{pgfplots}
\pgfplotsset{compat=newest}
\usepackage{url}
\usepackage{multicol}
\usepackage{thmtools}
\usepackage{wrapfig}

\usepackage{caption}
\usepackage{subcaption}
\usepackage{pdfpages}

\usepackage{pifont}

% MATH commands
\newcommand{\bC}{\mathbb{C}}
\newcommand{\bR}{\mathbb{R}}
\newcommand{\bN}{\mathbb{N}}
\newcommand{\bZ}{\mathbb{Z}}
\newcommand{\bT}{\mathbb{T}}
\newcommand{\bD}{\mathbb{D}}

\newcommand{\cL}{\mathcal{L}}
\newcommand{\cM}{\mathcal{M}}
\newcommand{\cP}{\mathcal{P}}
\newcommand{\cH}{\mathcal{H}}
\newcommand{\cB}{\mathcal{B}}
\newcommand{\cK}{\mathcal{K}}
\newcommand{\cJ}{\mathcal{J}}
\newcommand{\cU}{\mathcal{U}}
\newcommand{\cO}{\mathcal{O}}
\newcommand{\cA}{\mathcal{A}}
\newcommand{\cC}{\mathcal{C}}
\newcommand{\cF}{\mathcal{F}}

\newcommand{\fK}{\mathfrak{K}}
\newcommand{\fM}{\mathfrak{M}}

\newcommand{\ga}{\left\langle}
\newcommand{\da}{\right\rangle}
\newcommand{\oa}{\left\lbrace}
\newcommand{\fa}{\right\rbrace}
\newcommand{\oc}{\left[}
\newcommand{\fc}{\right]}
\newcommand{\op}{\left(}
\newcommand{\fp}{\right)}

\newcommand{\ra}{\rightarrow}
\newcommand{\Ra}{\Rightarrow}

\renewcommand{\Re}{\mathrm{Re}\,}
\renewcommand{\Im}{\mathrm{Im}\,}
\newcommand{\Arg}{\mathrm{Arg}\,}
\newcommand{\Arctan}{\mathrm{Arctan}\,}
\newcommand{\sech}{\mathrm{sech}\,}
\newcommand{\csch}{\mathrm{csch}\,}
\newcommand{\Log}{\mathrm{Log}\,}
\newcommand{\cis}{\mathrm{cis}\,}

\newcommand{\ran}{\mathrm{ran}\,}
\newcommand{\bi}{\mathbf{i}}
\newcommand{\Sp}{\mathrm{span}\,}
\newcommand{\Inv}{\mathrm{Inv}\,}
\newcommand\smallO{
  \mathchoice
    {{\scriptstyle\mathcal{O}}}% \displaystyle
    {{\scriptstyle\mathcal{O}}}% \textstyle
    {{\scriptscriptstyle\mathcal{O}}}% \scriptstyle
    {\scalebox{.7}{$\scriptscriptstyle\mathcal{O}$}}%\scriptscriptstyle
  }
\newcommand{\HOL}{\mathrm{Hol}}
\newcommand{\cl}{\mathrm{clos}}
\newcommand{\ve}{\varepsilon}

\DeclareMathOperator{\dom}{dom}

\makeatletter
\renewcommand*\env@matrix[1][*\c@MaxMatrixCols c]{%
  \hskip -\arraycolsep
  \let\@ifnextchar\new@ifnextchar
  \array{#1}}
\makeatother

%%%%%% Définitions Theorems and al.
%\declaretheoremstyle[preheadhook = {\vskip0.2cm}, mdframed = {linewidth = 2pt, backgroundcolor = yellow}]{myThmstyle}
%\declaretheoremstyle[preheadhook = {\vskip0.2cm}, postfoothook = {\vskip0.2cm}, mdframed = {linewidth = 1.5pt, backgroundcolor=green}]{myDefstyle}
%\declaretheoremstyle[bodyfont = \normalfont , spaceabove = 0.1cm , spacebelow = 0.25cm, qed = $\blacktriangle$]{myRemstyle}

%\declaretheorem[ style = myThmstyle, name=Th\'eor\`eme]{theorem}
%\declaretheorem[style =myThmstyle, name=Proposition]{proposition}
%\declaretheorem[style = myThmstyle, name = Corollaire]{corollary}
%\declaretheorem[style = myThmstyle, name = Lemme]{lemma}
%\declaretheorem[style = myThmstyle, name = Conjecture]{conjecture}

%\declaretheorem[style = myDefstyle, name = D\'efinition]{definition}

%\declaretheorem[style = myRemstyle, name = Remarque]{remark}
%\declaretheorem[style = myRemstyle, name = Remarques]{remarks}

\newtheorem{theorem}{Théorème}
\newtheorem{corollary}{Corollaire}
\newtheorem{lemma}{Lemme}
\newtheorem{proposition}{Proposition}
\newtheorem{conjecture}{Conjecture}

\theoremstyle{definition}

\newtheorem{definition}{Définition}[section]
\newtheorem{example}{Exemple}[section]
\newtheorem{remark}{\textcolor{red}{Remarque}}[section]
\newtheorem{exer}{\textbf{Exercice}}[section]


\tikzstyle{myboxT} = [draw=black, fill=black!0,line width = 1pt,
    rectangle, rounded corners = 0pt, inner sep=8pt, inner ysep=8pt]

\begin{document}
	\noindent \hrulefill \\
	\noindent MATH-311 \hfill Created by Pierre-O. Paris{\'e}\\
	Midterm 01 (50min)\hfill February, Spring 2024\\\vspace*{-0.7cm}

\noindent\hrulefill

\vspace*{0.5cm}

\begin{center}
\begin{minipage}{0.6\textwidth}
\begin{Huge}
\textsc{University of Hawai'i}
\end{Huge}
\end{minipage}
\begin{minipage}{0.12\textwidth}
\includegraphics[scale=0.05]{../../../../manoaseal_transparent.png}
\end{minipage}
\end{center}
	
\vspace*{0.5cm}

\noindent\makebox[\textwidth]{\textbf{Last name:}\enspace \hrulefill}

\vspace*{0.5cm}

\noindent\makebox[\textwidth]{\textbf{First name:}\enspace\hrulefill}

\vspace*{1cm}

\begin{center}
\gradetable[h][questions]
\end{center}

\vspace*{1cm}

\noindent\textbf{Instructions:} 

\begin{itemize}
\item Write your complete name on your copy. 
\item Answer all 6 questions below.
\item Write your answers directly on the questionnaire.
\item Show ALL your work to have full credit.
\item Draw a square around your final answer.
\item Return your copy when you're done or at the end of the 50min period. 
\item No electronic devices allowed during the exam. 
\item Scientific calculator allowed only (no graphical calculators).
\item \textbf{Turn off your cellphone(s) during the exam}.
\item Lecture notes and the textbook are not allowed during the exam. 
\end{itemize}

\vspace{0.5cm}

\noindent\textbf{Your Signature:} \hrulefill

\vspace*{1.5cm}

\noindent\textsc{May the Force be with you!}\\
\textsc{Pierre}

\newpage % End of cover page

\phantom{2}

\newpage

\qformat{\rule{0.3\textwidth}{.4pt} \begin{large}{\textsc{Question}} \thequestion \end{large} \hspace*{0.2cm} \hrulefill \hspace*{0.1cm} \textbf{(\totalpoints\hspace*{0.1cm} pts)}}


\begin{questions}

\question[10]

Find the solution to the following system of linear equations:
  \[
    \systeme{x_1 + x_2 + 3x_3 = 3, 2x_1 - 2x_2 + x_3 = 0, x_1 - x_2 + x_3 = 2} 
  \]
Does it have one solution, infinitely many solutions or no solution?

\newpage 

\question

Consider the following vectors:
  \[
    \mathbf{x} = \begin{bmatrix} 1 \\ 1 \\ 1 \\ 1 \end{bmatrix} , \, \mathbf{y} = \begin{bmatrix} 1 \\ -1 \\ 2 \\ -1 \end{bmatrix} , \, \mathbf{z} = \begin{bmatrix} -2 \\ 3 \\ 2 \\ 0 \end{bmatrix} , \, \mathbf{v} = \begin{bmatrix} -1 \\ 1 \\ 0 \\ 1 \end{bmatrix} , \, \mathbf{w} = \begin{bmatrix} 9 \\ -8 \\ 0 \\ 1 \end{bmatrix} .
  \]
We would like to know if $\mathbf{w}$ is a linear combination of $\mathbf{x}$, $\mathbf{y}$, $\mathbf{z}$ and $\mathbf{v}$.

\begin{parts}
\part[5] 
  Write down the system of linear equations corresponding to this problem. 
  %What is the system of linear equations to solve to know if $\mathbf{w}$ is a linear combination of $\mathbf{x}$, $\mathbf{y}$, $\mathbf{z}$, and $\mathbf{v}$? Write it down, but 
  \textbf{DO NOT SOLVE THE SYSTEM}.
\part[5]
  If the RREF of the augmented matrix of the system from part (a) is
    \[
      \begin{bmatrix}[cccc|c]
      1 & 0 & 0 & 0 & 2\\0 & 1 & 0 & 0 & 2\\0 & 0 & 1 & 0 & -3\\0 & 0 & 0 & 1 & 1
      \end{bmatrix} ,
    \]
  can you express $\mathbf{w}$ as a linear combination of $\mathbf{x}$, $\mathbf{y}$, $\mathbf{z}$, and $\mathbf{v}$? If so, write $\mathbf{w}$ as a linear combination of the other vectors.

\end{parts}


\newpage 

\question

Consider the following homogeneous system of linear equations:
  \[
    \systeme{x_1 - x_2 - x_3 + 2x_5 = 0, x_1 + 2x_2 + x_3 + 2x_4 + 2x_5 = 0, x_1 + x_2 + x_3 + 2x_4 + 2x_5 = 0, x_1 + 2x_2 + 2x_3 + 3 x_4 + 2x_5 = 0} .
  \]
  \begin{parts} 
  \part[2] Write the augmented matrix of the system.
  \part[8] The RREF of the augmented matrix of the system is
    \[
      \begin{bmatrix}[ccccc|c]
      1 & 0 & 0 & 1 & 2 & 0\\0 & 1 & 0 & 0 & 0 & 0\\0 & 0 & 1 & 1 & 0 & 0\\0 & 0 & 0 & 0 & 0 & 0
      \end{bmatrix}
    \]
  Express the solution as a linear combination of basic solution(s).
  \end{parts}

\newpage 

\question[10] 

Find the entries of the matrix $A$ if $A$ satisfies the equation:
  \[
    2A^\top - 5 \begin{bmatrix} 1 & 0 \\ -1 & 2 \end{bmatrix} = \Big( 4A - 9 \begin{bmatrix} 1 & 1 \\ -1 & 0 \end{bmatrix} \Big)^{\top}.
  \]


\newpage 

\question[6] 

Show that if $A$ is a square matrix and $A = k A^\top$ for some scalar $k \neq \pm 1$, then $A = 0$. 

\newpage 

\question[4]
\noaddpoints
Answer the following questions with \textbf{True} or \textbf{False}. Write down you answers on the line at the end of each question. Justify briefly your answer in the space after the statement of the problem.

  \begin{parts}
  \pointformat{(\hspace*{0.35cm}/ \thepoints)}
  \pointname{}
  \pointsinrightmargin
  
  \part[1]
  A matrix $B$ with RREF $\begin{bmatrix} 1 & 1 & 0 \\ 0 & 0 & 1 \end{bmatrix}$ has $\mathrm{rank}\, (B) = 2$. 
  \begin{solution}[\stretch{1}]
  
  \end{solution}
  \answerline[True]

  \part[1]
  A system of $3$ linear equations in $2$ variables with a augmented matrix of rank $2$ has a unique solution.
  \begin{solution}[\stretch{1}]
  
  \end{solution}
  \answerline[False]

    \part[1]
  If $\mathbf{x_1}$ and $\mathbf{x_2}$ are solutions to a system of homogeneous linear equations denoted by $(S)$, then $2\mathbf{x_1} - \mathbf{x_2}$ is also a solution of the system $(S)$. 
  \begin{solution}[\stretch{1}]
  
  \end{solution}
  \answerline[False]
  
  
  \part[1]
  A system of linear equations can have no solution.
  \begin{solution}[\stretch{1}]
  
  \end{solution}
  \answerline[False]

  \end{parts}
  


\end{questions}


\end{document}