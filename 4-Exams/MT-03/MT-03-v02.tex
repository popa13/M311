\documentclass[addpoints, 12pt]{exam}%, answers]
\usepackage[utf8]{inputenc}
\usepackage[T1]{fontenc}

\usepackage{lmodern}
\usepackage{arydshln}
\usepackage[margin=2cm]{geometry}

\usepackage{enumitem}
\usepackage{systeme}

\usepackage{amsmath, amsthm, amsfonts, amssymb}
\usepackage{graphicx}
\usepackage{tikz}
\usetikzlibrary{arrows,calc,patterns}
\usepackage{pgfplots}
\pgfplotsset{compat=newest}
\usepackage{url}
\usepackage{multicol}
\usepackage{thmtools}
\usepackage{wrapfig}

\usepackage{caption}
\usepackage{subcaption}
\usepackage{pdfpages}

\usepackage{pifont}

% MATH commands
\newcommand{\bC}{\mathbb{C}}
\newcommand{\bR}{\mathbb{R}}
\newcommand{\bN}{\mathbb{N}}
\newcommand{\bZ}{\mathbb{Z}}
\newcommand{\bT}{\mathbb{T}}
\newcommand{\bD}{\mathbb{D}}

\newcommand{\cL}{\mathcal{L}}
\newcommand{\cM}{\mathcal{M}}
\newcommand{\cP}{\mathcal{P}}
\newcommand{\cH}{\mathcal{H}}
\newcommand{\cB}{\mathcal{B}}
\newcommand{\cK}{\mathcal{K}}
\newcommand{\cJ}{\mathcal{J}}
\newcommand{\cU}{\mathcal{U}}
\newcommand{\cO}{\mathcal{O}}
\newcommand{\cA}{\mathcal{A}}
\newcommand{\cC}{\mathcal{C}}
\newcommand{\cF}{\mathcal{F}}

\newcommand{\fK}{\mathfrak{K}}
\newcommand{\fM}{\mathfrak{M}}

\newcommand{\ga}{\left\langle}
\newcommand{\da}{\right\rangle}
\newcommand{\oa}{\left\lbrace}
\newcommand{\fa}{\right\rbrace}
\newcommand{\oc}{\left[}
\newcommand{\fc}{\right]}
\newcommand{\op}{\left(}
\newcommand{\fp}{\right)}

\newcommand{\ra}{\rightarrow}
\newcommand{\Ra}{\Rightarrow}

\renewcommand{\Re}{\mathrm{Re}\,}
\renewcommand{\Im}{\mathrm{Im}\,}
\newcommand{\Arg}{\mathrm{Arg}\,}
\newcommand{\Arctan}{\mathrm{Arctan}\,}
\newcommand{\sech}{\mathrm{sech}\,}
\newcommand{\csch}{\mathrm{csch}\,}
\newcommand{\Log}{\mathrm{Log}\,}
\newcommand{\cis}{\mathrm{cis}\,}

\newcommand{\ran}{\mathrm{ran}\,}
\newcommand{\bi}{\mathbf{i}}
\newcommand{\Sp}{\mathrm{span}\,}
\newcommand{\Inv}{\mathrm{Inv}\,}
\newcommand\smallO{
  \mathchoice
    {{\scriptstyle\mathcal{O}}}% \displaystyle
    {{\scriptstyle\mathcal{O}}}% \textstyle
    {{\scriptscriptstyle\mathcal{O}}}% \scriptstyle
    {\scalebox{.7}{$\scriptscriptstyle\mathcal{O}$}}%\scriptscriptstyle
  }
\newcommand{\HOL}{\mathrm{Hol}}
\newcommand{\cl}{\mathrm{clos}}
\newcommand{\ve}{\varepsilon}

\DeclareMathOperator{\dom}{dom}

\makeatletter
\renewcommand*\env@matrix[1][*\c@MaxMatrixCols c]{%
  \hskip -\arraycolsep
  \let\@ifnextchar\new@ifnextchar
  \array{#1}}
\makeatother

%%%%%% Définitions Theorems and al.
%\declaretheoremstyle[preheadhook = {\vskip0.2cm}, mdframed = {linewidth = 2pt, backgroundcolor = yellow}]{myThmstyle}
%\declaretheoremstyle[preheadhook = {\vskip0.2cm}, postfoothook = {\vskip0.2cm}, mdframed = {linewidth = 1.5pt, backgroundcolor=green}]{myDefstyle}
%\declaretheoremstyle[bodyfont = \normalfont , spaceabove = 0.1cm , spacebelow = 0.25cm, qed = $\blacktriangle$]{myRemstyle}

%\declaretheorem[ style = myThmstyle, name=Th\'eor\`eme]{theorem}
%\declaretheorem[style =myThmstyle, name=Proposition]{proposition}
%\declaretheorem[style = myThmstyle, name = Corollaire]{corollary}
%\declaretheorem[style = myThmstyle, name = Lemme]{lemma}
%\declaretheorem[style = myThmstyle, name = Conjecture]{conjecture}

%\declaretheorem[style = myDefstyle, name = D\'efinition]{definition}

%\declaretheorem[style = myRemstyle, name = Remarque]{remark}
%\declaretheorem[style = myRemstyle, name = Remarques]{remarks}

\newtheorem{theorem}{Théorème}
\newtheorem{corollary}{Corollaire}
\newtheorem{lemma}{Lemme}
\newtheorem{proposition}{Proposition}
\newtheorem{conjecture}{Conjecture}

\theoremstyle{definition}

\newtheorem{definition}{Définition}[section]
\newtheorem{example}{Exemple}[section]
\newtheorem{remark}{\textcolor{red}{Remarque}}[section]
\newtheorem{exer}{\textbf{Exercice}}[section]


\tikzstyle{myboxT} = [draw=black, fill=black!0,line width = 1pt,
    rectangle, rounded corners = 0pt, inner sep=8pt, inner ysep=8pt]

\begin{document}
	\noindent \hrulefill \\
	\noindent MATH-311 \hfill Created by Pierre-O. Paris{\'e}\\
	Midterm 03 (50min)\hfill April, Spring 2024\\\vspace*{-0.7cm}

\noindent\hrulefill

\vspace*{0.5cm}

\begin{center}
\begin{minipage}{0.6\textwidth}
\begin{Huge}
\textsc{University of Hawai'i}
\end{Huge}
\end{minipage}
\begin{minipage}{0.12\textwidth}
\includegraphics[scale=0.05]{../../../../manoaseal_transparent.png}
\end{minipage}
\end{center}
	
\vspace*{0.5cm}

\noindent\makebox[\textwidth]{\textbf{Last name:}\enspace \hrulefill}

\vspace*{0.5cm}

\noindent\makebox[\textwidth]{\textbf{First name:}\enspace\hrulefill}

\vspace*{1cm}

\begin{center}
\gradetable[h][questions]
\end{center}

\vspace*{1cm}

\noindent\textbf{Instructions:} 

\begin{itemize}
\item Write your complete name on your copy. 
\item Answer all 5 questions below.
\item Write your answers directly on the questionnaire.
\item Show ALL your work to have full credit.
\item Draw a square around your final answer.
\item Return your copy when you're done or at the end of the 50min period. 
\item No electronic devices allowed during the exam. 
\item Scientific calculator allowed only (no graphical calculators).
\item \textbf{Turn off your cellphone(s) during the exam}.
\item Lecture notes and the textbook are not allowed during the exam. 
\end{itemize}

\vspace{0.5cm}

\noindent\textbf{Your Signature:} \hrulefill

\vspace*{1.5cm}

\noindent\textsc{May the Force be with you!}\\
\textsc{Pierre}

\newpage % End of cover page

\qformat{\rule{0.3\textwidth}{.4pt} \begin{large}{\textsc{Question}} \thequestion \end{large} \hspace*{0.2cm} \hrulefill \hspace*{0.1cm} \textbf{(\totalpoints\hspace*{0.1cm} pts)}}

\pointpoints{Pt}{Pts}


\begin{questions}

\question
Let $A = \begin{bmatrix} 6 & -5 \\ 2 & -1 \end{bmatrix}$.
  \begin{parts}
  \part[5]
  Find the eigenvalues of the matrix $A$.
  \part[10]
  Find the eigenvectors associated to each eigenvalue.
  \part[5]
  Is $A$ diagonalizable? If so, find the matrix $P$ such that $P^{-1} A P$ is a diagonal matrix.
  \end{parts}

\newpage 

\phantom{2}

\newpage

\question[10]

Let $\mathbf{x}, \mathbf{y}, \mathbf{z}, \mathbf{w}$ be vectors in a vector space $V$. Simplify the following expression: 
  $$
  2(\mathbf{x} - \mathbf{y}) + 4(\mathbf{z} - \mathbf{y}) + 4(\mathbf{w} - \mathbf{z}) + (\mathbf{x} - 4\mathbf{w}) .
  $$


\newpage 

\question

Which of the following are subspaces of $\mathbf{M_{22}}$, the vector space of all $2 \times 2$ matrices with usual addition and scalar multiplication of matrices.

  \begin{parts}
  \part[5]
  $U = \{ A \, : \, A \in \mathbf{M_{22}} \text{ and } A = A^\top \}$.
  \part[5]
  $U = \{ A \, : \, A \in \mathbf{M_{22}} \text{ and } A^4 = I \}$. 
  \end{parts}

\newpage 

\question

Answer the following questions:
\begin{parts}
\part[3]
Assume that $A$ is an $3 \times 3$ matrix and that $c_A (x)$ is the characteristic polynomial of $A$. Show that 
  $$ 
    c_{A^2} (x^2) = (-1) c_A (x) c_A (-x).
  $$
\textit{[Hint: Use the following property of determinants: $\det (XY) = \det (X) \det (Y)$.]}
\part[2]
What does the word ``eigen'' in ``eigen-vectors'' and ``eigen-values'' mean in English?
\end{parts}

\newpage 

\question
%\noaddpoints
Answer the following questions with \textbf{True} or \textbf{False}. Write down you answers on the line at the end of each question. Justify briefly your answer in the space after the statement of the problem. 

  \begin{parts}
  \pointformat{(\hspace*{0.35cm}/ \thepoints)}
  \pointname{}
  \pointsinrightmargin
  
    \part[1]
  If $A$ is a $2 \times 2$ matrix with two distinct eigenvectors, then $A$ is diagonalizable.
  \begin{solution}[\stretch{1}]
  
  \end{solution}
  \answerline[False]

   \part[1]
  If $A$ is a $2 \times 2$ matrix with eigenvalues $\lambda_1 = 3$ and $\lambda_2 = -1$, then $P^{-1} A P = \begin{bmatrix} 3 & 0 \\ 0 & -1 \end{bmatrix}$.
  \begin{solution}[\stretch{1}]
  
  \end{solution}
  \answerline[False]
  
  \part[1]
   If the solution to $A \mathbf{x} = \lambda \mathbf{x}$ is only $\mathbf{x} = \mathbf{0}$, then $\lambda$ is an eigenvalue.
  \begin{solution}[\stretch{1}]
 
  \end{solution}
  \answerline[False]
  
  \part[1]
  If a matrix $A$ has $\lambda = 0$ as an eigenvalue, then $A$ is not invertible.
  \begin{solution}[\stretch{1}]
  
  \end{solution}
  \answerline[False]
  


 
  \part[1]
  The set $U = \{ p \, : \, p \in \mathbf{P_3} \text{ and } p (0) = 1 \}$ is a subspace of $\mathbf{P_3}$.
  \begin{solution}[\stretch{1}]
  
  \end{solution}
  \answerline[False]

  \end{parts}

\end{questions}


\end{document}