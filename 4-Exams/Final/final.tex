\documentclass[addpoints, 12pt]{exam}%, answers]
\usepackage[utf8]{inputenc}
\usepackage[T1]{fontenc}

\usepackage{lmodern}
\usepackage{arydshln}
\usepackage[margin=2cm]{geometry}

\usepackage{enumitem}
\usepackage{systeme}

\usepackage{amsmath, amsthm, amsfonts, amssymb}
\usepackage{graphicx}
\usepackage{tikz}
\usetikzlibrary{arrows,calc,patterns}
\usepackage{pgfplots}
\pgfplotsset{compat=newest}
\usepackage{url}
\usepackage{multicol}
\usepackage{thmtools}
\usepackage{wrapfig}

\usepackage{caption}
\usepackage{subcaption}
\usepackage{pdfpages}

\usepackage{pifont}

% MATH commands
\newcommand{\bC}{\mathbb{C}}
\newcommand{\bR}{\mathbb{R}}
\newcommand{\bN}{\mathbb{N}}
\newcommand{\bZ}{\mathbb{Z}}
\newcommand{\bT}{\mathbb{T}}
\newcommand{\bD}{\mathbb{D}}

\newcommand{\cL}{\mathcal{L}}
\newcommand{\cM}{\mathcal{M}}
\newcommand{\cP}{\mathcal{P}}
\newcommand{\cH}{\mathcal{H}}
\newcommand{\cB}{\mathcal{B}}
\newcommand{\cK}{\mathcal{K}}
\newcommand{\cJ}{\mathcal{J}}
\newcommand{\cU}{\mathcal{U}}
\newcommand{\cO}{\mathcal{O}}
\newcommand{\cA}{\mathcal{A}}
\newcommand{\cC}{\mathcal{C}}
\newcommand{\cF}{\mathcal{F}}

\newcommand{\fK}{\mathfrak{K}}
\newcommand{\fM}{\mathfrak{M}}

\newcommand{\ga}{\left\langle}
\newcommand{\da}{\right\rangle}
\newcommand{\oa}{\left\lbrace}
\newcommand{\fa}{\right\rbrace}
\newcommand{\oc}{\left[}
\newcommand{\fc}{\right]}
\newcommand{\op}{\left(}
\newcommand{\fp}{\right)}

\newcommand{\ra}{\rightarrow}
\newcommand{\Ra}{\Rightarrow}

\renewcommand{\Re}{\mathrm{Re}\,}
\renewcommand{\Im}{\mathrm{Im}\,}
\newcommand{\Arg}{\mathrm{Arg}\,}
\newcommand{\Arctan}{\mathrm{Arctan}\,}
\newcommand{\sech}{\mathrm{sech}\,}
\newcommand{\csch}{\mathrm{csch}\,}
\newcommand{\Log}{\mathrm{Log}\,}
\newcommand{\cis}{\mathrm{cis}\,}

\newcommand{\ran}{\mathrm{ran}\,}
\newcommand{\bi}{\mathbf{i}}
\newcommand{\Sp}{\mathrm{span}\,}
\newcommand{\Inv}{\mathrm{Inv}\,}
\newcommand\smallO{
  \mathchoice
    {{\scriptstyle\mathcal{O}}}% \displaystyle
    {{\scriptstyle\mathcal{O}}}% \textstyle
    {{\scriptscriptstyle\mathcal{O}}}% \scriptstyle
    {\scalebox{.7}{$\scriptscriptstyle\mathcal{O}$}}%\scriptscriptstyle
  }
\newcommand{\HOL}{\mathrm{Hol}}
\newcommand{\cl}{\mathrm{clos}}
\newcommand{\ve}{\varepsilon}

\DeclareMathOperator{\dom}{dom}

\makeatletter
\renewcommand*\env@matrix[1][*\c@MaxMatrixCols c]{%
  \hskip -\arraycolsep
  \let\@ifnextchar\new@ifnextchar
  \array{#1}}
\makeatother

%%%%%% Définitions Theorems and al.
%\declaretheoremstyle[preheadhook = {\vskip0.2cm}, mdframed = {linewidth = 2pt, backgroundcolor = yellow}]{myThmstyle}
%\declaretheoremstyle[preheadhook = {\vskip0.2cm}, postfoothook = {\vskip0.2cm}, mdframed = {linewidth = 1.5pt, backgroundcolor=green}]{myDefstyle}
%\declaretheoremstyle[bodyfont = \normalfont , spaceabove = 0.1cm , spacebelow = 0.25cm, qed = $\blacktriangle$]{myRemstyle}

%\declaretheorem[ style = myThmstyle, name=Th\'eor\`eme]{theorem}
%\declaretheorem[style =myThmstyle, name=Proposition]{proposition}
%\declaretheorem[style = myThmstyle, name = Corollaire]{corollary}
%\declaretheorem[style = myThmstyle, name = Lemme]{lemma}
%\declaretheorem[style = myThmstyle, name = Conjecture]{conjecture}

%\declaretheorem[style = myDefstyle, name = D\'efinition]{definition}

%\declaretheorem[style = myRemstyle, name = Remarque]{remark}
%\declaretheorem[style = myRemstyle, name = Remarques]{remarks}

\newtheorem{theorem}{Théorème}
\newtheorem{corollary}{Corollaire}
\newtheorem{lemma}{Lemme}
\newtheorem{proposition}{Proposition}
\newtheorem{conjecture}{Conjecture}

\theoremstyle{definition}

\newtheorem{definition}{Définition}[section]
\newtheorem{example}{Exemple}[section]
\newtheorem{remark}{\textcolor{red}{Remarque}}[section]
\newtheorem{exer}{\textbf{Exercice}}[section]


\tikzstyle{myboxT} = [draw=black, fill=black!0,line width = 1pt,
    rectangle, rounded corners = 0pt, inner sep=8pt, inner ysep=8pt]

\begin{document}
  \noindent \hrulefill \\
  \noindent MATH-311 \hfill Created by Pierre-O. Paris{\'e}\\
  Final (2h)\hfill May, Spring 2024\\\vspace*{-0.7cm}

\noindent\hrulefill

\vspace*{0.5cm}

\begin{center}
\begin{minipage}{0.6\textwidth}
\begin{Huge}
\textsc{University of Hawai'i}
\end{Huge}
\end{minipage}
\begin{minipage}{0.12\textwidth}
\includegraphics[scale=0.05]{../../../../manoaseal_transparent.png}
\end{minipage}
\end{center}
  
\vspace*{0.5cm}

\noindent\makebox[\textwidth]{\textbf{Last name:}\enspace \hrulefill}

\vspace*{0.5cm}

\noindent\makebox[\textwidth]{\textbf{First name:}\enspace\hrulefill}

\vspace*{1cm}

\begin{center}
\gradetable[h][questions]
\end{center}

\vspace*{1cm}

\noindent\textbf{Instructions:} 

\begin{itemize}
\item Write your complete name on your copy. 
\item Answer all 4 questions below.
\item Write your answers directly on the questionnaire.
\item Show ALL your work to have full credit.
\item Draw a square around your final answer.
\item Return your copy when you're done or at the end of the 50min period. 
\item No electronic devices allowed during the exam. 
\item Scientific calculator allowed only (no graphical calculators).
\item \textbf{Turn off your cellphone(s) during the exam}.
\item Lecture notes and the textbook are not allowed during the exam. 
\end{itemize}

\vspace{0.5cm}

\noindent\textbf{Your Signature:} \hrulefill

\vspace*{1.5cm}

\noindent\textsc{May the Force be with you!}\\
\textsc{Pierre}

\newpage % End of cover page

\phantom{2}

\newpage

\qformat{\rule{0.3\textwidth}{.4pt} \begin{large}{\textsc{Question}} \thequestion \end{large} \hspace*{0.2cm} \hrulefill \hspace*{0.1cm} \textbf{(\totalpoints\hspace*{0.1cm} pts)}}

\pointpoints{Pt}{Pts}


\begin{questions}

\question 

Let $T : \mathbb{R}^3 \ra \mathbb{R}^3$ be defined by $T (x, y, z) = (2y, z, z)$.
  \begin{parts}
  \part[5]
  Show that $T$ is a linear transformation.
  \part[5]
  Find the kernel of $T$ and its dimension.
  \part[5]
  Using the Dimension Theorem, deduce the rank of $T$.
  \end{parts}

\newpage 

\question 

Let $V = \mathbb{R}^3$. Let $B$ be the standard basis $\{ (1, 0, 0), (0, 1, 0) , (0, 0, 1) \}$ and let the set $D = \{ (1, 2, 0), (0, -1, 2), (0, 2, 0) \}$ be another basis of $\mathbb{R}^3$.

  \begin{parts}
  \part[10]
  Knowing that  
     \begin{itemize}[label=\textbullet]
     \item $(1, 0, 0) = a (1, 2, 0) + b (0, -1, 2) + c (0, 2, 0)$, where $a = 1$, $b = 0$, $c = -1$;
     \item $(0, 1, 0) = a(1, 2, 0) + b (0, -1, 2) + c (0, 2, 0)$, where $a = 0$, $b = 0$, $c = 1/2$;
     \item $(0, 0, 1) = a(1, 2, 0) + b (0, -1, 2) + c (0, 2, 0)$, where $a = 0$, $b = 1/2$, $c = 1/4$;
     \end{itemize}
    Find the change matrix $P_{D \leftarrow B}$. Justify carefully your answer.
  \part[5]
  Let $T (x, y, z) = (2y, z, z)$. Find the matrix representation of $T$ on the basis $B$, that is $M_B (T)$.
  \end{parts}

\newpage 

\question 

Let $U = \mathrm{span} \, \{ (-1, 0, 3) , (0, -3, 2) \}$. 

  \begin{parts}
    \part[3] 
    Are $(-1, 0, 3)$ and $(0, -3, 2)$ orthogonal?
    \part[5]
    What is the dimension of $U$. Justify carefully your answer.
    \part[5]
    Using the Gram-Schmidt process, transform the set of vectors $\{ (-1, 0, 3) , (0, -3, 2) \}$ in a set of orthogonal vectors $F$.
    \part[2]
    Illustrate visually the Gram-Schmidt orthogonalization process for two vectors in $\mathbb{R}^2$.
  \end{parts}

\newpage 

\question

Assume that $T : V \ra \mathbb{R}^2$ is a linear transformation where $V$ is a vector space and $\mathbb{R}$ is the vector space of real numbers. Let $\{ \mathbf{v_1} , \mathbf{v_2} , \mathbf{v_3} \}$ be a basis of $V$. 
  \begin{parts}
  \part[1]
  If $T (\mathbf{v_1}) = (2, 2)$, $T (\mathbf{v_2} ) = (1, 2)$ and $T (\mathbf{v_3}) = (2, 1)$, show that $-3\mathbf{v_1} + 2\mathbf{v_2} + 2 \mathbf{v_3} \in \ker T$. Explain carefully your answer.
  \part[4]
  If $T (\mathbf{v_1})$, $T (\mathbf{v_2})$ and $T( \mathbf{v_3})$ are defined as in part (a), find the nullity of $T$. Explain carefully your answer.
  \end{parts}




\end{questions}

\end{document}