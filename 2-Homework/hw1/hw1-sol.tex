\documentclass[12pt]{article}
\usepackage[utf8]{inputenc}

\usepackage{enumitem}
\usepackage[margin=2cm]{geometry}

\usepackage{amsmath, amsfonts, amssymb}
\usepackage{graphicx}
\usepackage{tikz}
\usepackage{pgfplots}
\usepackage{multicol}

\usepackage{comment}
\usepackage{url}
\usepackage{calc}
\usepackage{subcaption}
\usepackage{circledsteps}
\usepackage{wrapfig}
\usepackage{array}

\setlength\parindent{0pt}

\usepackage{fancyhdr}
\pagestyle{fancy}
\fancyhf{}
\renewcommand{\headrulewidth}{2pt}
\renewcommand{\footrulewidth}{0pt}
\rfoot{\thepage}
\lhead{\textsc{Math} 311}
\chead{\textsc{Homework 1}}
\rhead{Fall 2023}

\pgfplotsset{compat=1.16}

% MATH commands
\newcommand{\ga}{\left\langle}
\newcommand{\da}{\right\rangle}
\newcommand{\oa}{\left\lbrace}
\newcommand{\fa}{\right\rbrace}
\newcommand{\oc}{\left[}
\newcommand{\fc}{\right]}
\newcommand{\op}{\left(}
\newcommand{\fp}{\right)}

\newcommand{\bi}{\mathbf{i}}
\newcommand{\bj}{\mathbf{j}}
\newcommand{\bk}{\mathbf{k}}
\newcommand{\bF}{\mathbf{F}}

\newcommand{\ra}{\rightarrow}
\newcommand{\Ra}{\Rightarrow}

\newcommand{\sech}{\mathrm{sech}\,}
\newcommand{\csch}{\mathrm{csch}\,}
\newcommand{\curl}{\mathrm{curl}\,}
\newcommand{\dive}{\mathrm{div}\,}

\newcommand{\ve}{\varepsilon}
\newcommand{\spc}{\vspace*{0.5cm}}

\DeclareMathOperator{\Ran}{Ran}
\DeclareMathOperator{\Dom}{Dom}

\newcommand{\exo}[3]{\noindent\textcolor{red}{\fbox{\textbf{Appendix {#1}, Problem {#2}}}\hrulefill   \textbf{({#3} Pts})}\vspace*{10pt}}

\begin{document}
\thispagestyle{empty}
	\noindent \hrulefill \newline
	MATH-311 \hfill Pierre-Olivier Paris{\'e}\newline
	Homework 1 Solutions \hfill Spring 2024\newline \vspace*{-0.7cm}
	
	\noindent\hrulefill
	
	\spc
	
	\exo{L}{1}{4}
	\begin{enumerate}[label=\alph*)]
	\item Yes it is a statement. The statement is false since $|-12| = 12$ (absolute value turns negative numbers into positive numbers).
	\item No, this is not a statement. The value of $x$ is not specify, so there is no truth value that can be associated to this sentence.
	\item No, this is not a statement. A question is not a statement.
	\item Yes, this is a statement. It is true, because assuming that $a = 2$ and $b = 4$, we have $a + b = 2 + 4 = 6$.
	\end{enumerate}

	\vspace*{0.5cm}

	\exo{L}{2}{6}
	\begin{enumerate}[label=\alph*)]
 	\item \textbf{Converse:} If Angela sleeps in, then it is a Saturday. \\
 	\textbf{Contrapositive:} If Angela does not sleep in, then it is not Saturday.
 	\item \textbf{Converse:} If I use my umbrella, then it rains outside. \\
 	\textbf{Contrapositive:} If I don't use my umbrella, then it does not rain outside.
 	\item \textbf{Converse:} If the surf was bigger than $4$ feet high, then I went surfing.\\ 
 	\textbf{Contrapositive:} If the surf was smaller than 4 feet high, then I did not go surfing.
 	\end{enumerate}

 	\vspace*{0.5cm}

 	\exo{L}{3}{8}
 	\begin{enumerate}[label=\alph*)]
 	\item The negation is ``It is not the case that it is raining and Charlie is cold.''. The negation of a statement $P \wedge Q$, is $(\neg P) \vee (\neg Q)$. So, letting $P$: ``It is raining'' and $Q$: ``Charlie is cold'', a useful reformulation of the negation is ``it is not raining or Charlie is not cold''.
 	\item The negation is ``It is not the case that if is raining, then Charlie is cold''. The negation of a statement $P \Rightarrow Q$ is $P \wedge (\neg Q)$. So, a useful reformulation of the negation is ``It is raining and Charlie is not cold''.
 	\item Let's simplify the statement using mathematical symbols. We can equivalently and compactly rewrite the statement as `` $\forall x$ real, $\exists y$ real such that $x + y = 0$''. The negation is then ``It is not the case that $\forall x$ real, $\exists y$ real such that $x + y = 0$''. The negation of a universal statement ``$\forall x$, $P (x)$'' is ``$\exists x$, $\neg P(x)$''. Let $P (x)$: ``$\exists y$ real such that $x + y = 0$''. Then we can rewrite the negation of the statement as ``$\exists x$ real such that $\neg P (x)$'' or
 		\begin{center}
 		$\exists x$ real such that it is not the case that there exists $y$ real such that $x + y = 0$.
 		\end{center}
 	The negation of an existential ``$\exists y$, $Q(y)$'' is ``$\forall y$, $\neg Q(y)$. For a fixed $x$, let $Q (y)$: ``$x + y = 0$''. Then we can rewrite the negation of ``$\exists y$ real such that $x + y = 0$'' as ``$\forall x$ real, $\neg Q (y)$'', or ``$\forall x$ real, $x + y \neq 0$''. Therefore, the negation of the whole statement is
 		\begin{center}
 		$\exists x$ real such that $\forall y$ real, $x + y \neq 0$ .
 		\end{center}
 	\item Let $P:$ ``$|a| > 0$'' and $Q:$ ``$a \neq 0$''. The statement $P \iff Q$ can be written as
 		\[
 			(P \Rightarrow Q ) \wedge (Q \Rightarrow P) .
 		\]
 	Therefore, the negation of the $P \iff Q$ is
 		\[
 			(P \wedge \neg Q) \vee (Q \wedge \neg P) .
 		\]
 	Replacing what is $P$ and $Q$, the statement $P \wedge \neg Q$ becomes `` $|a| > 0$ and $a = 0$ '' and the statement $Q \wedge \neg P$ becomes ``$a \neq 0$ and $|a| \leq 0$''. Hence, the negation of the full statement is
 		\[
 			( |a| > 0 \text{ and } a = 0 ) \text{ or } (a \neq 0 \text{ and } |a| \leq 0 ) .
 		\]
	\end{enumerate}

	\hspace*{0.5cm}

	\exo{B}{1}{12}
	\begin{enumerate}
		\item[a)] \textbf{Proof of the implication.} Assume that $n$ is an even integer. Then $n = 2k$, for some integer $k$. Hence, $n^2 = (2k)^2 = 4k^2$ and $n^2$ is a multiple of $4$.\\ 
		\textbf{Proof of the converse.} Assume that $n^2$ is a multiple of $4$. Then $n^2 = 4k$, for some integer $k$. Rearranging the equation for $n$ and $k$, we get $(n/2)^2 = k$. If $n$ was an odd integer, then $(n/2)^2$ would be a rational number (a fraction). However, from the equation $(n/2)^2 = k$, $(n/2)^2$ should be a whole number. Therefore, $n$ should be even.
		\item[d)] \textbf{Proof of the implication.} Assume that $x^2 - 5x + 6 = 0$. Factoring the polynomial, we find that $(x - 2) (x - 3) = 0$. Therefore, $x - 2 = 0$ or $x - 3 = 0$. Hence, $x = 2$ or $x = 3$.\\ 
		\textbf{Proof of the converse.} There are two cases to verify. Assume $x = 2$. Then $2^2 - 5(2) + 6 = 4 - 10 + 6 = 0$. So $x = 2$ satisfies the conclusion. Now assume that $x = 3$. Then $3^2 - 5(3) + 6 = 9 - 15 + 6 = 0$. So $x = 3$ satisfies the conclusion.
	\end{enumerate}

	\vspace*{0.5cm} 

	\exo{B}{2b}{5} 
	\\ 
	Assume that $n$ is an odd integer. Then $n = 2m + 1$, for some integer $m$. Squaring $n$, we get 
		$$ 
			n^2 = (2m + 1)^2 = 4m^2 + 4m + 1 = 4 (m^2 + m) + 1 .
		$$
	Since $m$ is an integer, it can be ether odd or even. We can split the proof into two cases:
		\begin{enumerate}
			\item \textbf{Assume that $m$ is even.} Then $m = 2j$, for some integer $j$. Replacing it in the expression for $n^2$, we get
				\[
					n^2 = 4 (4j^2 + 2j) + 1 = 4 ( 2 (2j^2 + j)) + 1 = 8 (2j^2 + j) + 1 .
				\]
			Let $k = 2j^2 + 2$, then $n^2 = 8k + 1$.
			\item \textbf{Assume that $m$ is odd.} Then $m = 2j + 1$, for some integer $j$. Replacing it in the expression for $n^2$, we get
				\[
					n^2 = 4 (4j^2 + 4j + 1 + 2j + 1) + 1 = 4 (4j^2 + 6j + 2) + 1 = 8 (2j^2 + 3j + 1) + 1 .
				\]
			Let $k = 2j^2 + 3j + 1$, then $n^2 = 8k + 1$.
		\end{enumerate}
	Thus, in both cases, we can write $n^2 = 8k + 1$, for some integer $k$.

	\hspace*{0.5cm}

	\exo{B}{3a}{10}
	\\ 
	Assume that $n > 2$ and $n$ is a prime. To give a proof by contradiction, assume further that $n$ is not an odd integer. In other words, assume that $n$ is an even integer. Since $n$ is even, then $n$ is divisible by $2$. But a prime number is only divisible by itself and $1$. Since $n > 2$, $n$ can not be divisible by $2$. This is a contradiction. Hence, $n$ should be an odd integer.

	The converse is false though. For example, take $n = 9$. Then $n$ is an odd integer, but it is divisible by $3$, so it is not a prime.

	\hspace*{0.5cm} 

	\exo{B}{4a}{5}
	\\ 
	Assume that $x$ and $y$ are positive numbers. Assume further that the conclusion is not true, so that $\sqrt{x + y}= \sqrt{x} + \sqrt{y}$. Then squaring the last equation:
		\[
			x + y = (\sqrt{x} + \sqrt{y})^2 = x + 2\sqrt{x}\sqrt{y} = y
		\]
	which, after simplifications, turns out to be 
		\[
			0 = xy .
		\]
	However, two numbers are zero only when one of them is $0$. But, we assumed that both $x$ and $y$ are not zero. A contradiction. Hence,
		\[
			\sqrt{x + y} \neq \sqrt{x} + \sqrt{y} .
		\]

	
\end{document}