\documentclass[12pt]{article}
\usepackage[utf8]{inputenc}

\usepackage{enumitem}
\usepackage[margin=2cm]{geometry}

\usepackage{amsmath, amsfonts, amssymb}
\usepackage{graphicx}
\usepackage{tikz}
\usepackage{pgfplots}
\usepackage{multicol}

\usepackage{comment}
\usepackage{url}
\usepackage{calc}
\usepackage{subcaption}
\usepackage{circledsteps}
\usepackage{wrapfig}
\usepackage{array}
\usepackage{systeme}
\sysdelim..

\setlength\parindent{0pt}

\usepackage{fancyhdr}
\pagestyle{fancy}
\fancyhf{}
\renewcommand{\headrulewidth}{2pt}
\renewcommand{\footrulewidth}{0pt}
\rfoot{\thepage}
\lhead{\textsc{Math} 311}
\chead{\textsc{Homework 9}}
\rhead{Spring 2024}

\pgfplotsset{compat=1.16}

% MATH commands
\newcommand{\ga}{\left\langle}
\newcommand{\da}{\right\rangle}
\newcommand{\oa}{\left\lbrace}
\newcommand{\fa}{\right\rbrace}
\newcommand{\oc}{\left[}
\newcommand{\fc}{\right]}
\newcommand{\op}{\left(}
\newcommand{\fp}{\right)}

\newcommand{\bi}{\mathbf{i}}
\newcommand{\bj}{\mathbf{j}}
\newcommand{\bk}{\mathbf{k}}
\newcommand{\bF}{\mathbf{F}}

\newcommand{\ra}{\rightarrow}
\newcommand{\Ra}{\Rightarrow}

\newcommand{\sech}{\mathrm{sech}\,}
\newcommand{\csch}{\mathrm{csch}\,}
\newcommand{\curl}{\mathrm{curl}\,}
\newcommand{\dive}{\mathrm{div}\,}

\newcommand{\ve}{\varepsilon}
\newcommand{\spc}{\vspace*{0.5cm}}

\DeclareMathOperator{\Ran}{Ran}
\DeclareMathOperator{\Dom}{Dom}

\newcommand{\exo}[3]{\noindent\textcolor{red}{\fbox{\textbf{Section {#1} | Problem {#2}}}\hrulefill   \textbf{({#3} Pts})}\vspace*{10pt}}

\makeatletter
\renewcommand*\env@matrix[1][*\c@MaxMatrixCols c]{%
  \hskip -\arraycolsep
  \let\@ifnextchar\new@ifnextchar
  \array{#1}}
\makeatother

\begin{document}
\thispagestyle{empty}
	\noindent \hrulefill \newline
	MATH-311 \hfill Pierre-Olivier Paris{\'e}\newline
	Homework 9 solutions \hfill Spring 2024\newline \vspace*{-0.7cm}
	
	\noindent\hrulefill
	
	\spc

	\exo{5.1}{1}{10} 
	\begin{enumerate}
		\item[a.] This is not a subspace because the vector $\mathbf{0} = (0 , 0, 0)$ is not in $U$.
		\item[b.] This is a subspace because it satisfies S1-S3.
			\begin{enumerate}[label=S\arabic*.]
				\item Set $s = t = 0$, then $(0, 0, 0) \in U$.
				\item Let $\mathbf{u_1} = (0, s_1, t_1)$ and $\mathbf{u_2} = (0, s_2, t_2)$. Then, we get
					\[
						\mathbf{u_1} + \mathbf{u_2} = (0 + 0, s_1 + s_2 , t_1 + t_2) = (0, s , t)
					\]
				where $s := s_1 + s_2$ and $t := t_1 + t_2$. Hence, $\mathbf{u_1} + \mathbf{u_2} \in U$.
				\item Let $\mathbf{u_1} = (0, s_1, t_1)$ and let $a \in \mathbb{R}$. Then
					\[
						a \mathbf{u_1} = (a (0) , a s_1 , a t_1 ) = (0, s, t) ,
					\]
				where $s = a s_1$ and $t = at_1$. Hence $a\mathbf{u_1} \in U$. 
			\end{enumerate}
	\end{enumerate}

	\spc 

	\exo{5.1}{17a}{10}
	\\ 
	Notice that
		\[
			A \mathbf{x} = B \mathbf{x} \iff A \mathbf{x} - B \mathbf{x} = \mathbf{0} \iff (A - B) \mathbf{x} = \mathbf{0} .
		\]
	Hence,
		\[
			U = \{ \mathbf{x} \in \mathbb{R}^n \, : \, A\mathbf{x} = B \mathbf{x} \} = \{ \mathbf{x} \in \mathbb{R}^n \, : \, (A - B) \mathbf{x} = \mathbf{0} \} = \mathrm{null}\, (A - B) .
		\]
	Since $C = A - B$ is a $m \times n$ matrix, from the lecture notes, we know that $\mathrm{null} (A - B)$ is a subspace of $\mathbb{R}^n$. 

	\spc 

	\exo{6.2}{1}{15}
	\begin{enumerate}
		\item[b.] This is a subset of $\mathbf{P_3}$ because $x$ times a polynomial of degree at most 2 results in a polynomial of degree at most 3.
			\begin{enumerate}
				\item[S1.] Set $g (x) = 0$ for any $x$, then $x g (x) = 0$. Therefore, the zero polynomial is in $U$.
				\item[S2.] Let $p_1 (x) = x g_1 (x)$ and $p_2 (x) = x g_2 (x)$, where $g_1$ and $g_2$ are in $\mathbf{P_2}$. Then
					\[
						p_1 (x) + p_2 (x) = x g_1 (x) + x g_2 (x) = x (g_1 (x) + g_2 (x)) = x g (x)
					\]
				where $g(x) = g_1 (x) + g_2 (x) \in \mathbf{P_2}$. Hence, $p_1 + p_2 \in U$.
				\item[S3.] Let $p (x) = x g_1 (x)$, where $g_1 \in \mathbf{P_2}$. Then
					\[
						a p(x) = a x g_1 (x) = x (a g_1 (x)) = x g (x) 
					\]
				with $g (x) = a g_1 (x) \in \mathbf{P_2}$. Hence, $ap \in U$.
			\end{enumerate}
		Since S1, S2, S3 are satisfied, we conclude that $U$ is a subspace of $\mathbf{P_3}$.
		\item[f.] Since the zero polynomial is of degree $0$, it is not in $U$. Therefore, $U$ is not a subspace.
	\end{enumerate}

	\spc 

	\exo{6.2}{2}{15}
	\begin{enumerate}
		\item[d.] Since the product of $2 \times 2$ matrices stay a $2 \times 2$ matrix, the set $U$ is a subset of $\mathbf{M_{22}}$.
			\begin{enumerate}
				\item[S1.] If $A = \mathbf{0}$, then $\mathbf{0} B = \mathbf{0}$. Hence, $\mathbf{0} \in U$.
				\item[S2.] Assume that $A$ and $C$ are in $U$. Then $AB = \mathbf{0}$ and $CB = \mathbf{0}$. Therefore
					\[
						(A + C) B = AB + CB = \mathbf{0} + \mathbf{0} = \mathbf{0} .
					\]
				Hence, $A + C \in U$.
				\item[S3.] Assume that $A \in U$ and $a \in \mathbb{R}$. Therefore, $AB = 0$ and
					\[
						(aA) B = a (AB) = a \mathbf{0} = \mathbf{0} .
					\]
				Hence, $aA \in U$.
			\end{enumerate}
		Since S1, S2, and S3 are satisfied, we get that $U$ is a subspace.
		\item[e.] $U$ is a subset of $\mathbf{M_{22}}$.
			\begin{enumerate}
				\item[S1.] Notice that $\mathbf{0}^2 = \mathbf{0}$ and therefore $\mathbf{0} \in U$.
				\item[S2.] Let $A, B \in U$, so that $A^2 = A$ and $B^2 = B$. Now, we have
					\[
						(A + B)^2 = A^2 + AB + BA + B^2 = A + AB + BA + B .
					\]
				Let $A = \begin{bmatrix} 1 & 1 \\ 0 & 0 \end{bmatrix}$ and $B = \begin{bmatrix} 0 & 1 \\ 0 & 1 \end{bmatrix}$. We get check that $A^2 = A$ and $B^2 = B$. However,
					\[
						A + B = \begin{bmatrix} 1 & 1 \\ 0 & 1 \end{bmatrix} \quad \Rightarrow \quad (A + B)^2 = \begin{bmatrix} 1 & 2 \\ 0 & 1 \end{bmatrix} \neq A + B .
					\]
				Hence, $A + B \not\in U$.
			\end{enumerate}
		Since S2 is not satisfied, $U$ is not a subspace.
	\end{enumerate}



\end{document}


%\item[a.] The zero polynomial is not in $U$ because $0(2) = 0 \neq 1$. So, $U$ is not a subspace of $\mathbf{P}_3$.
%\item[d.] $U$ is a subset of $\mathbf{P_3}$ because every element of the form $x g(x) + (1 - x) h (x)$ is a polynomial of degree at most 3 when $g$ and $h$ are polynomials of degree at most 2.
			\begin{enumerate}
				\item[S1.] Set $g (x) = h (x) = 0$ for any $x$. Then $x g (x) + (1 - x) h (x) = 0$ for any $x$ and therefore the zero polynomial is in $U$.
				\item[S2.] Let $p_1 (x) = x g_1 (x) + (1 - x) h_1 (x)$ and $p_2 (x) = x g_2 (x) + (1 - x) h_2 (x)$. Then,
					\begin{align*}
						p_1 (x) + p_2 (x) &= x g_1 (x) + (1 - x) h_1 (x) + x g_2 (x) + (1 - x) h_2 (x) \\ 
						&= x (g_1 (x) + g_2 (x)) + (1 - x) (h_1 (x) + h_2 (x)) \\ 
						&= x g (x) + (1 - x) h (x) 
					\end{align*}
				where $g (x) = g_1 (x) + g_2 (x) \in \mathbf{P_2}$ and $h (x) = h_1 (x) + h_2 (x) \in \mathbf{P_2}$. Therefore $p_1 (x) + p_2 (x) \in U$.
				\item[S3.] Let $p (x) = x g_1 (x) + (1 - x) h_1 (x)$. Then
					\[
						a p (x) = a(xg_1 (x) + (1 - x) h_1 (x)) = ax g_1 (x) + a (1 - x) h_1 (x) = x g(x) + (1 - x) h (x) 
					\]
				where $g (x) = a g_1 (x) \in \mathbf{P_2}$ and $h (x) = a h_1 (x) \in \mathbf{P_2}$. Therefore $a p (x) \in U$.
			\end{enumerate}
			Since S1, S2, and S3 are satisfied, $U$ is a subspace of $\mathbf{P_3}$.