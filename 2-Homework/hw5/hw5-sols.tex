\documentclass[12pt]{article}
\usepackage[utf8]{inputenc}

\usepackage{enumitem}
\usepackage[margin=2cm]{geometry}

\usepackage{amsmath, amsfonts, amssymb}
\usepackage{graphicx}
\usepackage{tikz}
\usepackage{pgfplots}
\usepackage{multicol}

\usepackage{comment}
\usepackage{url}
\usepackage{calc}
\usepackage{subcaption}
\usepackage{circledsteps}
\usepackage{wrapfig}
\usepackage{array}
\usepackage{systeme}
\sysdelim..

\setlength\parindent{0pt}

\usepackage{fancyhdr}
\pagestyle{fancy}
\fancyhf{}
\renewcommand{\headrulewidth}{2pt}
\renewcommand{\footrulewidth}{0pt}
\rfoot{\thepage}
\lhead{\textsc{Math} 311}
\chead{\textsc{Homework 1}}
\rhead{Spring 2024}

\pgfplotsset{compat=1.16}

% MATH commands
\newcommand{\ga}{\left\langle}
\newcommand{\da}{\right\rangle}
\newcommand{\oa}{\left\lbrace}
\newcommand{\fa}{\right\rbrace}
\newcommand{\oc}{\left[}
\newcommand{\fc}{\right]}
\newcommand{\op}{\left(}
\newcommand{\fp}{\right)}

\newcommand{\bi}{\mathbf{i}}
\newcommand{\bj}{\mathbf{j}}
\newcommand{\bk}{\mathbf{k}}
\newcommand{\bF}{\mathbf{F}}

\newcommand{\ra}{\rightarrow}
\newcommand{\Ra}{\Rightarrow}

\newcommand{\sech}{\mathrm{sech}\,}
\newcommand{\csch}{\mathrm{csch}\,}
\newcommand{\curl}{\mathrm{curl}\,}
\newcommand{\dive}{\mathrm{div}\,}

\newcommand{\ve}{\varepsilon}
\newcommand{\spc}{\vspace*{0.5cm}}

\DeclareMathOperator{\Ran}{Ran}
\DeclareMathOperator{\Dom}{Dom}

\newcommand{\exo}[3]{\noindent\textcolor{red}{\fbox{\textbf{Section {#1} | Problem {#2}}}\hrulefill   \textbf{({#3} Pts})}\vspace*{10pt}}

\makeatletter
\renewcommand*\env@matrix[1][*\c@MaxMatrixCols c]{%
  \hskip -\arraycolsep
  \let\@ifnextchar\new@ifnextchar
  \array{#1}}
\makeatother

\begin{document}
\thispagestyle{empty}
	\noindent \hrulefill \newline
	MATH-311 \hfill Pierre-Olivier Paris{\'e}\newline
	Homework 5 solutions \hfill Spring 2024\newline \vspace*{-0.7cm}
	
	\noindent\hrulefill
	
	\spc
	
	\exo{2.4}{1}{4}

	\begin{enumerate}
		\item[b.] We have
			\[
				\begin{bmatrix}
				3 & 0 \\ 
				1 & -4
				\end{bmatrix} 
				(\tfrac{1}{2}) 
				\begin{bmatrix} 
				4 & 0 \\ 
				1 & -3 
				\end{bmatrix}
				= (\tfrac{1}{2}) \begin{bmatrix}12 & 0\\0 & 12\end{bmatrix} = \begin{bmatrix} 6 & 0 \\ 0 & 6 \end{bmatrix} .
			\]
		Here, the matrix are not inverse of each other because we don't have $AB = I$. We can stop here and we don't have to calculate $BA$.
		\item[d.] We have
			\[
				\begin{bmatrix}
				3 & 0 \\ 
				0 & 5
				\end{bmatrix} 
				\begin{bmatrix} 
				\tfrac{1}{3} & 0 \\ 
				0 & \tfrac{1}{5} 
				\end{bmatrix} = \begin{bmatrix} 1 & 0 \\ 0 & 1 \end{bmatrix}
			\]
			and
			\[
				\begin{bmatrix}
				\tfrac{1}{3} & 0 \\ 
				0 & \tfrac{1}{5} 
				\end{bmatrix} 
				\begin{bmatrix} 
				3 & 0 \\ 
				0 & 5 
				\end{bmatrix} = \begin{bmatrix} 1 & 0 \\ 0 & 1 \end{bmatrix} .
			\]
			Therefore the two matrices are inverses of each other.
	\end{enumerate}

	\spc 

	\exo{2.4}{3b}{6}
	\\ 
	The system can be put in matrix form:
		\[
			\begin{bmatrix} 
			2 & -3 \\ 
			1 & -4 
			\end{bmatrix} 
			\begin{bmatrix} 
			x\\ y 
			\end{bmatrix}
			= 
			\begin{bmatrix}
			0 \\ 1
			\end{bmatrix}
		\]
	Using the formula for the inverse of a $2 \times 2$ matrix with $a = 2$, $b = -3$, $c = 1$, and $d = -4$, we have
		\[
			\begin{bmatrix} 
			2 & -3 \\ 
			1 & -4 
			\end{bmatrix}^{-1} = -\tfrac{1}{5} \begin{bmatrix} -4 & 3 \\ -1 & 2 \end{bmatrix}
		\]
	Hence the solution is given by
		\[
			\begin{bmatrix}
			x \\ y
			\end{bmatrix}
			= -\tfrac{1}{5} \begin{bmatrix} -4 & 3 \\ -1 & 2 \end{bmatrix} \begin{bmatrix}0 \\ 1 \end{bmatrix} = \begin{bmatrix} -3/5 \\ -2/5 \end{bmatrix} .
		\]

	\spc 

	\exo{2.4}{5}{10}

	\begin{enumerate}
		\item[d.] The inverse does not distribute on the addition nor the substraction. We first take the inverse on each side to get
			\[
				( (I - 2A^\top)^{-1})^{-1} = \begin{bmatrix} 2 & 1 \\ 1 & 1 \end{bmatrix}^{-1} \iff ((I - 2A^\top)^{-1})^{-1} = \begin{bmatrix} 1 & -1 \\ -1 & 2 \end{bmatrix}.
			\]
		The left-hand side becomes simply $I - 2A^\top$ and therefore
			\[
				I - 2A^\top = \begin{bmatrix} 1 & -1 \\ -1 & 2 \end{bmatrix} \iff \begin{bmatrix} 0 & 1 \\ 1 & -1 \end{bmatrix} = 2A^\top \iff \begin{bmatrix} 0 & 1/2 \\ 1/2 & -1/2 \end{bmatrix} = A^\top .
			\]
		We take the transpose on both side and since $(A^\top)^\top = A$, we get
			\[
				\begin{bmatrix} 0 & 1/2 \\ 1/2 & -1/2 \end{bmatrix}^\top = A \iff A = \begin{bmatrix} 0 & 1/2 \\ 1/2 & -1/2 \end{bmatrix} .
			\]

		\item[g.] We apply the same stategy. We start by taking the inverse on each side:
			\[
				( (A^\top - 2I)^{-1})^{-1} = \Big( 2 \begin{bmatrix} 1 & 1 \\ 2 & 3 \end{bmatrix} \Big)^{-1} \iff A^\top - 2I = \tfrac{1}{2} \begin{bmatrix} 3 & -1 \\ -2 & 1 \end{bmatrix} .
			\]
		After multiplying by $1/2$ on the right and move the $2I$ on the other side, we get
			\[
				A^\top = \begin{bmatrix} 3/2 & -1/2 \\ -1 & 1/2 \end{bmatrix} + 2I \iff A^\top = \begin{bmatrix} 7/2 & -1/2 \\ -1 & 5/2 \end{bmatrix} \iff A = \begin{bmatrix} 7/2 & -1 \\ -1/2 & 5/2 \end{bmatrix} .
			\]
	\end{enumerate}

\spc 

\exo{2.4}{9}{4}

\begin{enumerate}
	\item[b.] This is false. For example, $I - I = O$ is not inversibe, but $I$ is invertible.
	\item[c.] This is true. If $A$ and $B$ are invertible, then $A^{-1}$ and $B$ are invertible. Therefore, from the properties of inverses, $A^{-1}B$ is invertible. Again, from the properties of the inverse, we know that the conjugate of an invertible matrix will be invertible, hence $(A^{-1} B)^{\top}$ is invertible.
\end{enumerate}

\spc 

\exo{2.4}{39a}{5}
\\ 
Assume that $P$ is idempotent and invertible, but $P \neq I$. We have $P^2 = P$, which can be rewritten as $P^2- P = 0$. Factoring one $P$ on the left, we get
	\[
		P (P - I) = O \iff P^{-1} P (P - I) = P^{-1} O \iff P - I = O \iff P = I .
	\]
We get $P \neq I$ and $P = I$. This is a contradiction and the only invertible idempotent is $I$.

\spc 

\exo{2.5}{1}{6}

\begin{enumerate}
	\item[b.] Let $R_1$, $R_2$, and $R_3$ be the rows of an arbitrary $3 \times 3$ matrix $A$. The elementary matrix $E$ interchanges $R_1$ with $R_3$. The inverse $E^{-1}$ must therefore undo what $E$ does, so it must interchange $R_1$ and $R_3$ again:
		\[
			E^{-1} = \begin{bmatrix}
			0 & 0 & 1 \\ 
			0 & 1 & 0 \\ 
			1 & 0 & 0
			\end{bmatrix} .
		\]
	\item[d.]  Let $R_1$, $R_2$, and $R_3$ be the rows of an arbitrary $3 \time 3$ matrix $A$. The elementary matrix $E$ replace the second row of $A$ by $-2R_1 + R_2$. The inverse $E^{-1}$ must therefore undo what $E$ does, so it must replace the second row by $2R_1 + R_2$. Therefore the inverse of $E$ is
		\[
			E^{-1} = \begin{bmatrix}
			1 & 0 & 0 \\ 
			2 & 1 & 0 \\ 
			0 & 0 & 1 
			\end{bmatrix} .
		\]
\end{enumerate}

\spc 

\exo{2.5}{6b}{15}
\\ 
The first operation is $R_2 - 5 R_1$, so the elementary matrix corresponding to that operation is
	\[
		E_1 = \begin{bmatrix} 1 & 0 \\ -5 & 1 \end{bmatrix} .
	\]
	Multiplying by $E_1$ to the left of $A$, we get
	\[
		E_1 A
		 = \begin{bmatrix}1 & 2 & 1\\0 & 2 & -6\end{bmatrix}
	\]

The second operation is $R_1 - R_2$, so the elementary matrix corresponding to that second operation is
	\[
		E_2 = \begin{bmatrix} 1 & -1 \\ 0 & 1 \end{bmatrix} . 
	\]
Multiplying by $E_2$ to the left of $E_1 A$, we get
	\[
		E_2 E_1 A = \begin{bmatrix}1 & 0 & 7\\0 & 2 & -6\end{bmatrix}
	\]

The third operation is $\tfrac{1}{2} R_2$, so the elementary matrix corresponding to that third operation is
	\[
		E_3 = \begin{bmatrix} 1 & 0 \\ 0 & 1/2 \end{bmatrix} .
	\]
Multiplying by $E_3$ to the left of $E_2 E_1 A$, we get
	\[
		E_3 E_2 E_1 A = \begin{bmatrix}1 & 0 & 7\\0 & 1 & -3\end{bmatrix}.
	\]

Hence, we get that
	\[
		U = E_3 E_2 E_1 \quad \text{ and } \quad R = \begin{bmatrix}1 & 0 & 7\\0 & 1 & -3\end{bmatrix} .
	\]


\end{document}