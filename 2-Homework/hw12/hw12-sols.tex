\documentclass[12pt]{article}
\usepackage[utf8]{inputenc}

\usepackage{enumitem}
\usepackage[margin=2cm]{geometry}

\usepackage{amsmath, amsfonts, amssymb}
\usepackage{graphicx}
\usepackage{tikz}
\usepackage{pgfplots}
\usepackage{multicol}

\usepackage{comment}
\usepackage{url}
\usepackage{calc}
\usepackage{subcaption}
\usepackage{circledsteps}
\usepackage{wrapfig}
\usepackage{array}
\usepackage{systeme}
\sysdelim..

\setlength\parindent{0pt}

\usepackage{fancyhdr}
\pagestyle{fancy}
\fancyhf{}
\renewcommand{\headrulewidth}{2pt}
\renewcommand{\footrulewidth}{0pt}
\rfoot{\thepage}
\lhead{\textsc{Math} 311}
\chead{\textsc{Homework 12}}
\rhead{Spring 2024}

\pgfplotsset{compat=1.16}

% MATH commands
\newcommand{\ga}{\left\langle}
\newcommand{\da}{\right\rangle}
\newcommand{\oa}{\left\lbrace}
\newcommand{\fa}{\right\rbrace}
\newcommand{\oc}{\left[}
\newcommand{\fc}{\right]}
\newcommand{\op}{\left(}
\newcommand{\fp}{\right)}

\newcommand{\bi}{\mathbf{i}}
\newcommand{\bj}{\mathbf{j}}
\newcommand{\bk}{\mathbf{k}}
\newcommand{\bF}{\mathbf{F}}

\newcommand{\ra}{\rightarrow}
\newcommand{\Ra}{\Rightarrow}

\newcommand{\sech}{\mathrm{sech}\,}
\newcommand{\csch}{\mathrm{csch}\,}
\newcommand{\curl}{\mathrm{curl}\,}
\newcommand{\dive}{\mathrm{div}\,}

\newcommand{\ve}{\varepsilon}
\newcommand{\spc}{\vspace*{0.5cm}}

\DeclareMathOperator{\Img}{Im}
\DeclareMathOperator{\Dom}{Dom}
\DeclareMathOperator{\Spn}{span}

\newcommand{\exo}[3]{\noindent\textcolor{red}{\fbox{\textbf{Section {#1} | Problem {#2}}}\hrulefill   \textbf{({#3} Pts})}\vspace*{10pt}}

\makeatletter
\renewcommand*\env@matrix[1][*\c@MaxMatrixCols c]{%
  \hskip -\arraycolsep
  \let\@ifnextchar\new@ifnextchar
  \array{#1}}
\makeatother

\begin{document}
\thispagestyle{empty}
	\noindent \hrulefill \newline
	MATH-311 \hfill Pierre-Olivier Paris{\'e}\newline
	Homework 12 solutions \hfill Spring 2024\newline \vspace*{-0.7cm}

\noindent\hrulefill
	
	\spc

\exo{9.1}{1}{10}

	\begin{enumerate}
		\item[a.] We want to find $a$, $b$, and $c$ such that
			\[
				a (x + 1) + b (x^2) + c (3) = \mathbf{v} = 2x^2 + x - 1 .
			\]
		Therefore	
			\[
				bx^2 + ax + (a + 3c) = 2x^2 + x - 1 .
			\]
		and $b = 2$, $a = 1$ and $1 + 3c = -1$. Therefore
			\[
				b = 2 , \, a = 1 \text{ and } c = -2/3 .
			\]
		Hence
			\[
				C_B (\mathbf{v}) = \begin{bmatrix} 1 \\ 2 \\ -2/3 \end{bmatrix} .
			\]
		\item[c.] We want to find $a$, $b$, and $c$ such that
			\[
				\mathbf{v} = (1, -1, 2) = a (1, -1, 0) + b (1, 1, 1) + c (0, 1, 1) .
			\]
		Therefore
			\[
				(1, -1, 2) = (a + b, -a + b + c, b + c) 
			\]
		and $a + b = 1$, $-a + b + c = -1$, and $b + c = 2$. Adding the first equation to the second equation, we get $2b + c = 0$ and subtracting the third equation to this last equation:
			\[
				2b + c - b - c = -2 \quad \Ra \quad b = -2 .
			\]
		We then find $a = 3$ and $c = 4$. Hence
			\[
				C_B (\mathbf{v}) = \begin{bmatrix} 3 \\ -2 \\ 4 \end{bmatrix} .
			\]
	\end{enumerate}

\exo{9.1}{4}{10}

	\begin{enumerate}
		\item[a.] Since $B$ and $D$ are the standard basis, it is more easy to find the matrix representation. 

		We have
			\[
				T (1, 0, 0) = (1, 0, 0, 1) \quad \Ra \quad C_D (T (1, 0, 0)) = \begin{bmatrix} 1 \\ 0 \\ 0 \\ 1 \end{bmatrix} ,
			\]
		then
			\[
				T (0, 1, 0) = (0, 0, 1, 2) \quad \Ra \quad C_D (T (0, 1, 0)) = \begin{bmatrix} 0 \\ 0 \\ 1 \\ 2 \end{bmatrix} ,
			\]
		and then
			\[
				T (0, 0, 1) = (1, 2, -1, 0) \quad \Ra \quad C_D (T (0, 0, 1)) = \begin{bmatrix} 1 \\ 2 \\ -1 \\ 0 \end{bmatrix} .
			\]
		Therefore
			\[
				A = \begin{bmatrix} 1 & 0 & 1 \\ 0 & 0 & 2 \\ 0 & 1 & -1 \\ 1 & 2 & 0 \end{bmatrix}
			\]
		is the matrix representing $T$ on the standard basis $B$ and $D$. 

		We have
			\[
				C_B (\mathbf{v}) = \begin{bmatrix} 1 \\ -1 \\ 3 \end{bmatrix} \quad \Ra \quad C_D (T (\mathbf{v})) = \begin{bmatrix} 1 & 0 & 1 \\ 0 & 0 & 2 \\ 0 & 1 & -1 \\ 1 & 2 & 0 \end{bmatrix} \begin{bmatrix} 1 \\ -1 \\ 3 \end{bmatrix} = \begin{bmatrix} 4 \\ 6 \\ -4 \\ -1 \end{bmatrix} .
			\]
		\item[c.] Let $\mathbf{b_1} = 1$, $\mathbf{b_2} = x$, and $\mathbf{b_3} = x^2$. Let $\mathbf{d_1} = (1, 0)$ and $\mathbf{d_2} = (1, -1)$.

		We have
			\[
				T (\mathbf{b_1}) = T (1 + 0x + 0x^2) = (1, 0) \quad \Ra \quad C_D (T (\mathbf{b_1})) = (1) \mathbf{d_1} + (0) \mathbf{d_2} = \begin{bmatrix} 1 \\ 0 \end{bmatrix} ,
			\]
		then
			\[
				T (\mathbf{b_2}) = T (0 + 1x + 0x^2) = (0, 2) \quad \Ra \quad C_D (T (\mathbf{b_2})) = (2) \mathbf{d_1} + (-2) \mathbf{d_2} = \begin{bmatrix} 2 \\ -2 \end{bmatrix} ,
			\]
		and
			\[
				T (\mathbf{b_3}) = T (0 + 0 x + 1x^2) = (1, 0) \quad \Ra \quad C_D (T (\mathbf{b_3})) = (1) \mathbf{d_1} + (0) \mathbf{d_2} = \begin{bmatrix} 1 \\ 0 \end{bmatrix} .
			\]
		Therefore
			\[
				A = \begin{bmatrix} 1 & 2 & 1 \\ 0 & -2 & 0 \end{bmatrix}
			\]
		is the matrix representing $T$ on the basis $B$ and $D$. 

		We have $C_B (\mathbf{v}) = \begin{bmatrix} a & b & c \end{bmatrix}^\top$ and therefore
			\[
				C_D (T(\mathbf{v})) =  \begin{bmatrix} 1 & 2 & 1 \\ 0 & -2 & 0 \end{bmatrix} \begin{bmatrix} a \\ b \\ c \end{bmatrix} = \begin{bmatrix} a + 2b + c \\ -2b \end{bmatrix} . 
			\]
	\end{enumerate}

\exo{9.1}{21a}{5}

Asssume that $S$ and $T$ are linear transformations. Let $\mathbf{v} \in \ker S \cap \ker T$. This means $\mathbf{v} \in \ker S$ and $\mathbf{v} \in \ker T$. Therefore, $S (\mathbf{v}) = \mathbf{0}$ and $T (\mathbf{v}) = \mathbf{0}$. Hence
			\[
				(S + T) (\mathbf{v}) = S (\mathbf{v}) + T (\mathbf{v}) = \mathbf{0} + \mathbf{0} = \mathbf{0} 
			\]
		and $\mathbf{v} \in \ker (S + T)$.

	\spc 

\exo{9.2}{1a}{10}

We have
	\[
		 (0, -1) = (-1) (0, 1) + (0) (1, 1) \quad \Ra \quad C_D ( (0, -1)) = \begin{bmatrix} -1 \\ 0 \end{bmatrix} 
	\]
and
	\[
		(2, 1) = (-1) (0, 1) + (2) (1, 1) \quad \Ra \quad C_D ( (2, 1)) = \begin{bmatrix} -1 \\ 2 \end{bmatrix} .
	\]
Hence, we obtain
	\[
		P_{D \leftarrow B} = \begin{bmatrix} -1 & -1 \\ 0 & 2 \end{bmatrix} .
	\]

For $\mathbf{v} = (3, -5)$, we have
	\[
		C_B (\mathbf{v}) = \begin{bmatrix} 13/2 \\ 3/2 \end{bmatrix} \text{ and } C_D (\mathbf{v}) = \begin{bmatrix} -8 \\ 3 \end{bmatrix} .
	\]
We therefore have
	\[
		P_{D \leftarrow B} C_B (\mathbf{v}) = \begin{bmatrix} -1 & -1 \\ 0 & 2 \end{bmatrix} \begin{bmatrix} 13/2 \\ 3/2 \end{bmatrix} = \begin{bmatrix} -8 \\ 3 \end{bmatrix} = C_D (\mathbf{v}) .
	\]

\spc 

\exo{9.2}{7b}{10}

In the basis $B_0$, we have
	\[
		C_{B_0} (T (1)) = C_{B_0} (1 + x^2) = \begin{bmatrix} 1 \\ 0 \\ 1 \end{bmatrix} , \quad 	C_{B_0} (T (x)) = C_{B_0} (1 + x ) = \begin{bmatrix} 1 \\ 1 \\ 0 \end{bmatrix} , \quad C_{B_0} (T (x^2)) C_{B_0} (x + x^2 ) = \begin{bmatrix} 0 \\ 1 \\ 1 \end{bmatrix} .
	\]
Therefore
	\[
		M_{B_0} (T) = \begin{bmatrix} 1 & 1 & 0 \\ 0 & 1 & 1 \\ 1 & 0 & 1 \end{bmatrix} .
	\]

In the basis $B$, we have
	\[
		C_{B} (T (1 - x^2)) = C_{B} (1 - x ) = \begin{bmatrix} -2 \\ 3 \\ -2 \end{bmatrix} , \quad C_B (T (1 + x)) = C_B (2 + x + x^2) = \begin{bmatrix} -3 \\ 5 \\ -2 \end{bmatrix}
	\]
and
	\[
		C_B (T (2x + x^2)) = C_B (2 + 3x + x^2 ) = \begin{bmatrix} -1 \\ 3 \\ 0 \end{bmatrix} .
	\]
Therefore
	\[
		M_B (T) = \begin{bmatrix} -2 & -3 & -1 \\ 3 & 5 & 3 \\ -2 & -2 & 0 \end{bmatrix} .
	\]

It is straightforward to obtain
	\[
		P = P_{B_0 \leftarrow B} = \begin{bmatrix} 1 & 1 & 0\\ 0 & 1 & 2 \\ -1 & 0 & 1 \end{bmatrix} .
	\]
Hence, we get
	\[
		P^{-1} = \begin{bmatrix} -1 & 1 & -2\\2 & -1 & 2\\-1 & 1 & -1 \end{bmatrix} .
	\]
Hence, we get
	\[
		P^{-1} M_{B_0} (T) P = \begin{bmatrix} -1 & 1 & -2\\2 & -1 & 2\\-1 & 1 & -1 \end{bmatrix}  \begin{bmatrix} 1 & 1 & 0 \\ 0 & 1 & 1 \\ 1 & 0 & 1 \end{bmatrix} \begin{bmatrix} 1 & 1 & 0\\ 0 & 1 & 2 \\ -1 & 0 & 1 \end{bmatrix} = \begin{bmatrix} -2 & -3 & -1\\3 & 5 & 3\\-2 & -2 & 0 \end{bmatrix} = M_B (T) .
	\]

\spc 

\exo{9.2}{8b}{5}

Using python, we get
	\[
		P^{-1} = \begin{bmatrix} 5 & -2\\-7 & 3 \end{bmatrix} 
	\]
and so
	\[
		P^{-1} A P = \begin{bmatrix} 5 & -2\\-7 & 3 \end{bmatrix} \begin{bmatrix} 29 & -12 \\ 70 & -29 \end{bmatrix} \begin{bmatrix} 3 & 2 \\ 7 & 5 \end{bmatrix} = \begin{bmatrix} 1 & 0 \\ 0 & -1 \end{bmatrix} = D .
	\]

Using the columns of $P$, we define
	\[
		B = \left\lbrace \begin{bmatrix} 3 \\ 7 \end{bmatrix} , \begin{bmatrix} 2 \\ 5 \end{bmatrix} \right\rbrace .
	\]



\end{document}