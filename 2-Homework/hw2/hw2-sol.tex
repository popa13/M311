\documentclass[12pt]{article}
\usepackage[utf8]{inputenc}

\usepackage{enumitem}
\usepackage[margin=2cm]{geometry}

\usepackage{amsmath, amsfonts, amssymb}
\usepackage{graphicx}
\usepackage{tikz}
\usepackage{pgfplots}
\usepackage{multicol}

\usepackage{comment}
\usepackage{url}
\usepackage{calc}
\usepackage{subcaption}
\usepackage{circledsteps}
\usepackage{wrapfig}
\usepackage{array}
\usepackage{systeme}
\sysdelim..

\setlength\parindent{0pt}

\usepackage{fancyhdr}
\pagestyle{fancy}
\fancyhf{}
\renewcommand{\headrulewidth}{2pt}
\renewcommand{\footrulewidth}{0pt}
\rfoot{\thepage}
\lhead{\textsc{Math} 311}
\chead{\textsc{Homework 1}}
\rhead{Fall 2023}

\pgfplotsset{compat=1.16}

% MATH commands
\newcommand{\ga}{\left\langle}
\newcommand{\da}{\right\rangle}
\newcommand{\oa}{\left\lbrace}
\newcommand{\fa}{\right\rbrace}
\newcommand{\oc}{\left[}
\newcommand{\fc}{\right]}
\newcommand{\op}{\left(}
\newcommand{\fp}{\right)}

\newcommand{\bi}{\mathbf{i}}
\newcommand{\bj}{\mathbf{j}}
\newcommand{\bk}{\mathbf{k}}
\newcommand{\bF}{\mathbf{F}}

\newcommand{\ra}{\rightarrow}
\newcommand{\Ra}{\Rightarrow}

\newcommand{\sech}{\mathrm{sech}\,}
\newcommand{\csch}{\mathrm{csch}\,}
\newcommand{\curl}{\mathrm{curl}\,}
\newcommand{\dive}{\mathrm{div}\,}

\newcommand{\ve}{\varepsilon}
\newcommand{\spc}{\vspace*{0.5cm}}

\DeclareMathOperator{\Ran}{Ran}
\DeclareMathOperator{\Dom}{Dom}

\newcommand{\exo}[3]{\noindent\textcolor{red}{\fbox{\textbf{Section {#1} | Problem {#2}}}\hrulefill   \textbf{({#3} Pts})}\vspace*{10pt}}

\makeatletter
\renewcommand*\env@matrix[1][*\c@MaxMatrixCols c]{%
  \hskip -\arraycolsep
  \let\@ifnextchar\new@ifnextchar
  \array{#1}}
\makeatother

\begin{document}
\thispagestyle{empty}
	\noindent \hrulefill \newline
	MATH-311 \hfill Pierre-Olivier Paris{\'e}\newline
	Homework 2 Solutions \hfill Spring 2024\newline \vspace*{-0.7cm}
	
	\noindent\hrulefill
	
	\spc
	
	\exo{1.1}{1a}{5}
	\\ 
	Replacing in the first equation, we get
		\[
			2 (19t - 35) + 3 (25  - 13t) + t = 38t - 70 + 75 - 39t + t = 5
		\]
	and replacing in the second equation
		\[
			5 (19t - 35) + 7 (25-13t) - 4t = 95t - 175 + 175 - 91t - 4t = 0.
		\]
	Hence, it is a solution to the system of linear equations.

	\vspace*{0.5cm}

	\exo{1.1}{7}{6} 
	\begin{enumerate} 
		\item[a.] $\begin{bmatrix}[cc|c] 1 & -3 & 5 \\ 2 & 1 & 1 \end{bmatrix}$.
		\item[b.] $\begin{bmatrix}[cc|c] 1 & 2 & 0 \\ 0 & 1 & 1 \end{bmatrix}$. 
	\end{enumerate}

	\vspace*{0.5cm}

	\exo{1.1}{8a}{5} 
	\\ 
	Here is the system associated to the augmented matrix
		\[
			\begin{bmatrix}[ccc|c] 
			1 & -1 & 6 & 0 \\ 
			0 & 1 & 0 & 3 \\ 
			2 & -1 & 0 & 1
			\end{bmatrix} .
		\]

	\vspace*{0.5cm} 

	\exo{1.1}{14}{12} 
	\begin{enumerate}[label=\alph*.]
		\item False in general. For example
			\[
				\systeme{2x + 3y + 2z = 0, x + 3y + 2z = 1} \quad \longrightarrow \quad \begin{bmatrix}[ccc|c] 2 & 3 & 2 & 0 \\ 1 & 3 & 2 & 1 \end{bmatrix}
			\]
		The augmented matrix has $2$ rows only, but there are $n = 3$ variables.
		\item False in general. For an example, see Example 2 in Section 1.1.
		\item True, because any system, say $S_1$ obtained from row operations applied to another system, say $S_2$, have the same set of solutions. Therefore, $S_2$ is consistent, then $S_1$ is consistent.
		\item True. If the system $S_1$ is inconsistent, then there is no solutions to the system $S_1$, so the set of solutions is the empty set $\varnothing$. Since the new system $S_2$ obtained from the series of row operations has the same set of solutions, then $S_2$ must have no solution, that is the set of solution is $\varnothing$.
	\end{enumerate}

	\newpage 

	\exo{1.2}{1}{2} 
	\\ 
	\textbf{REF:} c, d and e.\\
	\textbf{RREF:} None.

	\vspace*{0.5cm}

	\exo{1.2}{5}{12} 
	\begin{enumerate}
		\item[a.] We have 
			\[
				\begin{bmatrix}[ccc|c]
				1 & 1 & 2 & 8 \\ 
				3 & -1 & 1 & 0 \\ 
				-1 & 3 & 4 & -4
				\end{bmatrix}
				\longrightarrow 
				\begin{bmatrix}[ccc|c] 
				1 & 0 & 0 & 17 \\ 
				0 & 1 & 0 & 31 \\ 
				0 & 0 & 1 & -20 
				\end{bmatrix}
			\]
		Hence, $x = 17$, $y = 31$, $z = -20$.
		\item[c.] We have 
			\[
				\begin{bmatrix}[ccc|c]
				1 & 1 & -1 & 10 \\ 
				-1 & 4 & 5 & -5 \\ 
				1 & 6 & 3 & 15 
				\end{bmatrix} 
				\longrightarrow 
				\begin{bmatrix}[ccc|c]
				1 & 0 & -9/5 & 9 \\ 
				0 & 1 & 4/5 & 1 \\ 
				0 & 0 & 0 & 0
				\end{bmatrix}
			\]
		The system is consistent and $x = 9 + 9t/5$, $y = 1 -4t/5$, and $z = t$, for $t \in \mathbb{R}$.
		\item[d.] We have 
			\[
				\begin{bmatrix}[ccc|c]
				1 & 2 & -1 & 2 \\ 
				2 & 5 & -3 & 1 \\ 
				1 & 4 & -3 & 3
				\end{bmatrix} 
				\longrightarrow 
				\begin{bmatrix}[ccc|c]
				1 & 0 & 1 & 0\\0 & 1 & -1 & 0\\0 & 0 & 0 & 1
				\end{bmatrix}
			\]
		The system is inconsistent.
	\end{enumerate}

	\vspace*{0.5cm}

	\exo{1.2}{11}{8}
	\begin{enumerate}
		\item[a.] We obtain
			\[
				\begin{bmatrix}
				1 & 1 & 2\\3 & -1 & 1\\-1 & 3 & 4
				\end{bmatrix}
				\longrightarrow 
				\begin{bmatrix}
				1 & 0 & 0\\
				0 & 1 & 0\\
				0 & 0 & 1
				\end{bmatrix}
			\]
		Hence, $\mathrm{rank} (A) = 3$.
		\item[b.] We obtain 
			\[
				\begin{bmatrix}
				-2 & 3 & 3\\3 & -4 & 1\\-5 & 7 & 2
				\end{bmatrix}
				\longrightarrow 
				\begin{bmatrix}
				1 & 0 & 15\\0 & 1 & 11\\0 & 0 & 0\end{bmatrix}
			\]
		Hence, $\mathrm{rank} (A) = 2$.
	\end{enumerate}

\end{document}