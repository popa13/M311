\documentclass[12pt]{article}
\usepackage[utf8]{inputenc}

\usepackage{enumitem}
\usepackage[margin=2cm]{geometry}

\usepackage{amsmath, amsfonts, amssymb}
\usepackage{graphicx}
\usepackage{tikz}
\usepackage{pgfplots}
\usepackage{multicol}

\usepackage{comment}
\usepackage{url}
\usepackage{calc}
\usepackage{subcaption}
\usepackage{circledsteps}
\usepackage{wrapfig}
\usepackage{array}

\setlength\parindent{0pt}

\usepackage{fancyhdr}
\pagestyle{fancy}
\fancyhf{}
\renewcommand{\headrulewidth}{2pt}
\renewcommand{\footrulewidth}{0pt}
\rfoot{\thepage}
\lhead{\textsc{Math} 311}
\chead{\textsc{Homework 1}}
\rhead{Fall 2023}

\pgfplotsset{compat=1.16}

% MATH commands
\newcommand{\ga}{\left\langle}
\newcommand{\da}{\right\rangle}
\newcommand{\oa}{\left\lbrace}
\newcommand{\fa}{\right\rbrace}
\newcommand{\oc}{\left[}
\newcommand{\fc}{\right]}
\newcommand{\op}{\left(}
\newcommand{\fp}{\right)}

\newcommand{\bi}{\mathbf{i}}
\newcommand{\bj}{\mathbf{j}}
\newcommand{\bk}{\mathbf{k}}
\newcommand{\bF}{\mathbf{F}}

\newcommand{\ra}{\rightarrow}
\newcommand{\Ra}{\Rightarrow}

\newcommand{\sech}{\mathrm{sech}\,}
\newcommand{\csch}{\mathrm{csch}\,}
\newcommand{\curl}{\mathrm{curl}\,}
\newcommand{\dive}{\mathrm{div}\,}

\newcommand{\ve}{\varepsilon}
\newcommand{\spc}{\vspace*{0.5cm}}

\DeclareMathOperator{\Ran}{Ran}
\DeclareMathOperator{\Dom}{Dom}

\newcommand{\exo}[3]{\noindent\textcolor{red}{\fbox{\textbf{Appendix {#1}, Problem {#2}}}\hrulefill   \textbf{({#3} Pts})}\vspace*{10pt}}

\begin{document}
\thispagestyle{empty}
	\noindent \hrulefill \newline
	MATH-311 \hfill Pierre-Olivier Paris{\'e}\newline
	Homework 2 Solutions \hfill Spring 2024\newline \vspace*{-0.7cm}
	
	\noindent\hrulefill
	
	\spc
	
	\exo{1.1}{1a}{}
	\\ 
	Replacing in the first equation, we get
		\[
			2 (19t - 35) + 3 (25  - 13t) + t = 38t - 70 + 75 - 39t + t = 5
		\]
	and replacing in the second equation
		\[
			5 (19t - 35) + 7 (25-13t) - 4t = 95t - 175 + 175 - 91t - 4t = 0.
		\]
	Hence, it is a solution to the system of linear equations.

	\hspace*{0.5cm}

	\exo{1.1}{7}{} 
	\begin{enumerate}
		\item[a)] False in general. For example
			\[
				\systeme{2x + 3y + 2z = 0, x + 3y + 2z = 1} \quad \longrightarrow \quad \begin{bmatrix} 2 & 3 & 2 & 0 \\ 1 & 3 & 2 & 1 \end{bmatrix}
			\]
	\end{enumerate}



\end{document}