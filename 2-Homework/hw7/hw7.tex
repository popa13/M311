\documentclass[12pt]{article}
\usepackage[utf8]{inputenc}

\usepackage{enumitem}
\usepackage[margin=2cm]{geometry}

\usepackage{amsmath, amsfonts, amssymb}
\usepackage{graphicx}
\usepackage{tikz}
\usepackage{pgfplots}
\usepackage{multicol}

\usepackage{comment}
\usepackage{url}
\usepackage{calc}
\usepackage{subcaption}
\usepackage{circledsteps}
\usepackage{wrapfig}
\usepackage{array}
\usepackage{systeme}
\sysdelim..

\setlength\parindent{0pt}

\usepackage{fancyhdr}
\pagestyle{fancy}
\fancyhf{}
\renewcommand{\headrulewidth}{2pt}
\renewcommand{\footrulewidth}{0pt}
\rfoot{\thepage}
\lhead{\textsc{Math} 311}
\chead{\textsc{Homework 7}}
\rhead{Fall 2023}

\pgfplotsset{compat=1.16}

% MATH commands
\newcommand{\ga}{\left\langle}
\newcommand{\da}{\right\rangle}
\newcommand{\oa}{\left\lbrace}
\newcommand{\fa}{\right\rbrace}
\newcommand{\oc}{\left[}
\newcommand{\fc}{\right]}
\newcommand{\op}{\left(}
\newcommand{\fp}{\right)}

\newcommand{\bi}{\mathbf{i}}
\newcommand{\bj}{\mathbf{j}}
\newcommand{\bk}{\mathbf{k}}
\newcommand{\bF}{\mathbf{F}}

\newcommand{\ra}{\rightarrow}
\newcommand{\Ra}{\Rightarrow}

\newcommand{\sech}{\mathrm{sech}\,}
\newcommand{\csch}{\mathrm{csch}\,}
\newcommand{\curl}{\mathrm{curl}\,}
\newcommand{\dive}{\mathrm{div}\,}

\newcommand{\ve}{\varepsilon}
\newcommand{\spc}{\vspace*{0.5cm}}

\DeclareMathOperator{\Ran}{Ran}
\DeclareMathOperator{\Dom}{Dom}

\newcommand{\exo}[3]{\noindent\textcolor{red}{\fbox{\textbf{Section {#1} | Problem {#2}}}\hrulefill   \textbf{({#3} Pts})}\vspace*{10pt}}

\makeatletter
\renewcommand*\env@matrix[1][*\c@MaxMatrixCols c]{%
  \hskip -\arraycolsep
  \let\@ifnextchar\new@ifnextchar
  \array{#1}}
\makeatother

\begin{document}
\thispagestyle{empty}
	\noindent \hrulefill \newline
	MATH-311 \hfill Pierre-Olivier Paris{\'e}\newline
	Homework 7 solutions \hfill Spring 2024\newline \vspace*{-0.7cm}
	
	\noindent\hrulefill
	
	\spc

	\exo{3.2}{2}{10}
	\begin{enumerate}
		\item[a.] Denote by $A$ the matrix. We compute
			\[
				\det A = 37 - 5c .
			\]
		So $A$ is invertible $\iff$ $\det A \neq 0$ $\iff$ $37 - 5c = 0$ $\iff$ $c \neq 37/5$.
		\item[d.] Denote by $D$ the matrix. Using column and row operations, we see that
			\[
				\det D = \begin{vmatrix} 1 & c & 3 \\ 0 & 2 & c \\ 1 & c & 4 \end{vmatrix} = \begin{vmatrix} 1 & c & 3 \\ 0 & 2 & c \\ 0 & 0 & 1 \end{vmatrix} = 2 .
			\]
		The first equality comes from replacing $C_1$ of $A$ by $C_1 - C_3$ and the second equality comes form replace $R_3$ by $R_3 - R_1$. 

		Hence, $\det D \neq 0$ and the matrix $D$ is invertible for any values of $c$.
	\end{enumerate}

	\spc 

	\exo{3.2}{3}{10}
	\begin{enumerate}
		\item[a.] From the properties of determinant, we compute
			\[
				\det (A^3 B C^\top B^{-1}) = \det (A^3) \det (B) \det (C^\top) \det (B^{-1}) = (\det A)^3 \det B \det C \big( \tfrac{1}{\det B} \big)
			\]
		Notice that $\det B (\tfrac{1}{\det B}) = 1$ and therefore, replacing the values for each determinant, we get
			\[
				\det (A^3 B C^\top B^{-1}) = (\det A)^3 \det C = (-1)^3 (3) = -3
			\]
		\item[b.] From the properties of determinant again, we compute
			\begin{align*}
				\det (B^2 C^{-1} A B^{-1} C^\top ) &= \det (B^2) \det (C^{-1}) \det (A) \det (B^{-1}) \det (C^\top ) \\ 
				&= (\det B )^2 (\tfrac{1}{\det C}) \det (A) (\tfrac{1}{\det B}) \det C \\ 
				&= \det B \det A 
			\end{align*}
		Plugging in the values for the determinants, we find that
			\[
				\det (B^2 C^{-1} A B^{-1} C^\top ) = (2) (-1) = -2 .
			\]
	\end{enumerate}

	\spc 

	\exo{3.2}{7a}{10}
	\\ 
	Denote the matrix by $A$. We can extract a $2$ from the last column and the first row so that we get
		\[
			\det A = 4 \begin{vmatrix} 1 & -1 & 0 \\ c + 1 & -1 & a \\ d - 2 & 2 & b \end{vmatrix}
		\]
	Now, we will replace $R_2$ by $R_2 - R_1$ and $R_3$ by $R_3 + 2R_1$. These two elementary operations don't change the value of the determinant and therefore	
		\[
			\begin{vmatrix} 1 & -1 & 0 \\ c + 1 & -1 & a \\ d - 2 & 2 & b \end{vmatrix} = \begin{vmatrix} 1 & -1 & 0 \\ c & 0 & a \\ d & 0 & b \end{vmatrix} = (-1)(-1) \begin{vmatrix} c & a \\ d & b \end{vmatrix} .
		\]
		Thus,
			\[
				\det A = 4 \begin{vmatrix} c & a \\ d & b \end{vmatrix} .
			\]
		We are almost there! Notice that, by interchanging the first column and the second column, we get
			\[
				\begin{vmatrix} c & a \\ d & b \end{vmatrix} = (-1) \begin{vmatrix} a & c \\ b & d \end{vmatrix} .
			\]
		Now notice again that
			\[
				\begin{bmatrix} a & c \\ b & d \end{bmatrix} = \begin{bmatrix} a & b \\ c & d \end{bmatrix}^\top \quad \Rightarrow \quad \begin{vmatrix} a & c \\ b & d \end{vmatrix} = \begin{vmatrix} a & b \\ c & d \end{vmatrix} .
			\]
		Therefore, we get
			\[
				\det A = (4) (-1) \begin{vmatrix} a & b \\ c & d \end{vmatrix} = (-4) \begin{vmatrix} a & b \\ c & d \end{vmatrix} = (-4) (-2) = 8 .
			\]
	\spc 

	\exo{3.2}{8a}{10}
	\\ 
	The matrix of coefficients and the constant vector are
		\[
			A = \begin{bmatrix} 2 & 1 \\ 3 & 7 \end{bmatrix} \quad \text{ and } \quad \vec{b} = \begin{bmatrix} 1 \\ -2 \end{bmatrix} .
		\]
	We have $\det A = 14 - 3 = 11$.

	To get the value of $x$, we replace the first column of $A$ and set
		\[
			x = \frac{\begin{vmatrix} 1 & 1 \\ -2 & 7 \end{vmatrix}}{\det A} = \frac{7 + 2}{11} = \frac{9}{11} .  
		\]
	To get the value of $y$, we replace the first column of $A$ and set
		\[
			y = \frac{\begin{vmatrix} 2 & 1 \\ 3 & -2 \end{vmatrix}}{\det A} = \frac{-4 - 3}{11} = -\frac{7}{11} .
		\]

	\spc 

	\exo{3.2}{13}{10}
	\\ 
	Assume that $A$ and $B$ are two $n \times n$ matrices. From the product rule, we get
		\[
			\det (AB) = \det (A) \det (B) .
		\]
	Using the product rule on $BA$, we get
		\[
			\det (BA) = \det (B) \det (A) .
		\]
	Using the commutativity of the multiplication of real numbers, we get
		\[
			\det (AB) = \det (A) \det (B) = \det (B) \det (A) = \det (BA) .
		\]
	This completes the proof. \hfill $\square$

\end{document}