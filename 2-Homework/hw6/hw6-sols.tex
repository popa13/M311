\documentclass[12pt]{article}
\usepackage[utf8]{inputenc}

\usepackage{enumitem}
\usepackage[margin=2cm]{geometry}

\usepackage{amsmath, amsfonts, amssymb}
\usepackage{graphicx}
\usepackage{tikz}
\usepackage{pgfplots}
\usepackage{multicol}

\usepackage{comment}
\usepackage{url}
\usepackage{calc}
\usepackage{subcaption}
\usepackage{circledsteps}
\usepackage{wrapfig}
\usepackage{array}
\usepackage{systeme}
\sysdelim..

\setlength\parindent{0pt}

\usepackage{fancyhdr}
\pagestyle{fancy}
\fancyhf{}
\renewcommand{\headrulewidth}{2pt}
\renewcommand{\footrulewidth}{0pt}
\rfoot{\thepage}
\lhead{\textsc{Math} 311}
\chead{\textsc{Homework 1}}
\rhead{Fall 2023}

\pgfplotsset{compat=1.16}

% MATH commands
\newcommand{\ga}{\left\langle}
\newcommand{\da}{\right\rangle}
\newcommand{\oa}{\left\lbrace}
\newcommand{\fa}{\right\rbrace}
\newcommand{\oc}{\left[}
\newcommand{\fc}{\right]}
\newcommand{\op}{\left(}
\newcommand{\fp}{\right)}

\newcommand{\bi}{\mathbf{i}}
\newcommand{\bj}{\mathbf{j}}
\newcommand{\bk}{\mathbf{k}}
\newcommand{\bF}{\mathbf{F}}

\newcommand{\ra}{\rightarrow}
\newcommand{\Ra}{\Rightarrow}

\newcommand{\sech}{\mathrm{sech}\,}
\newcommand{\csch}{\mathrm{csch}\,}
\newcommand{\curl}{\mathrm{curl}\,}
\newcommand{\dive}{\mathrm{div}\,}

\newcommand{\ve}{\varepsilon}
\newcommand{\spc}{\vspace*{0.5cm}}

\DeclareMathOperator{\Ran}{Ran}
\DeclareMathOperator{\Dom}{Dom}

\newcommand{\exo}[3]{\noindent\textcolor{red}{\fbox{\textbf{Section {#1} | Problem {#2}}}\hrulefill   \textbf{({#3} Pts})}\vspace*{10pt}}

\makeatletter
\renewcommand*\env@matrix[1][*\c@MaxMatrixCols c]{%
  \hskip -\arraycolsep
  \let\@ifnextchar\new@ifnextchar
  \array{#1}}
\makeatother

\begin{document}
\thispagestyle{empty}
	\noindent \hrulefill \newline
	MATH-311 \hfill Pierre-Olivier Paris{\'e}\newline
	Homework 6 solutions \hfill Spring 2024\newline \vspace*{-0.7cm}
	
	\noindent\hrulefill
	
	\spc
	
	\exo{3.1}{1}{20}
	\begin{enumerate}
		\item[e.] We have
			\[
			 	\det \begin{bmatrix} \cos \theta & -\sin \theta \\ \sin \theta & \cos \theta \end{bmatrix} = (\cos \theta ) (\cos \theta ) - (-\sin \theta) (\sin \theta ) = \cos^2 \theta + \sin^2 \theta = 1 .
			 \] 
		\item[f.] We develop along the first row. We get
			\begin{align*}
				\det \begin{bmatrix} 2 & 0 & -3 \\ 1 & 2 & 5 \\ 0 & 3 & 0 \end{bmatrix} &= 2 \det \begin{bmatrix} 2 & 5 \\ 3 & 0 \end{bmatrix} - (0) \det \begin{bmatrix} 1 & 5 \\ 0 & 0 \end{bmatrix} + (-3) \det \begin{bmatrix} 1 & 2 \\ 0 & 3 \end{bmatrix} \\ 
				&= 2 (-15) + 0 - 3 (3) = -39 .
			\end{align*}
		\item[k.] We develop the determinant along the first row again. We get
			\begin{align*}
			\det \begin{bmatrix} 0 & 1 & -1 & 0 \\ 3 & 0 & 0 & 2 \\ 0 & 1 & 2 & 1 \\ 5 & 0 & 0 & 7 \end{bmatrix} &= -(1) \det \begin{bmatrix} 3 & 0 & 2 \\ 0 & 2 & 1 \\ 5 & 0 & 7 \end{bmatrix} + (-1) \det \begin{bmatrix} 3 & 0 & 2 \\ 0 & 1 & 1 \\ 5 & 0 & 7 \end{bmatrix} \\ 
			&= (-1) (2) \det \begin{bmatrix} 3 & 2 \\ 5 & 7 \end{bmatrix} + (-1) (1) \det \begin{bmatrix} 3 & 2 \\ 5 & 7 \end{bmatrix} \\ 
			&= (-2)(11) - (21 - 10) \\ 
			&= -33 .
			\end{align*}
		\item[n.] Your answer should be $-56$.
	\end{enumerate}

\spc 

	\exo{3.1}{7a}{20}
	\\ 
	Let $A$ be the matrix such that $\det A = -1$. Let $B$ be the matrix in part a. Here are the steps to go from $A \ra B$.
		\begin{enumerate}
			\item Replaced row 1 of $A$ by $R_1 + 3R_2$. Call this matrix $A_1$. We have $\det A_1 = \det A$. 
			\item Replaced $R_3$ of $A_1$ by $-R_3$. Call this new matrix $A_2$. Then $\det A_2 = -\det A_1 = - \det A$. 
			\item Replace $R_2$ of $A_2$ by $2R_2$. Call this new matrix $A_3$. Then $\det A_3 = 2 \det A_2 = -2 \det A$. 
			\item Swap $R_1$ with $R_2$ of $A_3$. Call this new matrix $A_4$. Then $\det A_4 = -\det A_3 = 2 \det A$.
			\item Swap $R_2$ with $R_3$ of $A_4$. Call this new matrix $A_5$ which is now $B$. Then $\det A_5 = -\det A_4 = -2 \det A$.
		\end{enumerate}
	Hence $\det B = -2 \det A = 2$. 

	Below is the matrix obtained after each row operations:
		\begin{align*}
			\begin{bmatrix} 
			a & b & c \\ p & q & r \\ x & y & z 
			\end{bmatrix}
			&\longrightarrow 
			\begin{bmatrix}
			a + 3p & b + 3q & c + 3r \\ 
			p & q & r \\ 
			x & y & z
			\end{bmatrix}
			\longrightarrow 
			\begin{bmatrix}
			a + 3p & b + 3q & c + 3r \\ 
			p & q & r \\ 
			-x & -y & -z
			\end{bmatrix} \\ 
			& \longrightarrow 
			\begin{bmatrix}
			a + 3p & b + 3q & c + 3r \\ 
			2p & 2q & 2r \\ 
			-x & -y & -z
			\end{bmatrix}
			\longrightarrow 
			\begin{bmatrix}
			-x & -y & -z \\ 
			2p & 2q & 2r \\ 
			a + 3p & b + 3q & c + 3r
			\end{bmatrix} \\ 
			& \longrightarrow 
			\begin{bmatrix}
			-x & -y & -z \\ 
			a + 3p & b + 3q & c + 3r \\ 
			2p & 2q & 2r
			\end{bmatrix}
		\end{align*}

\spc 

	\exo{3.1}{15a}{5}
	\\ 
	The simple trick is to develop along the last column of the matrix and find that the coefficient $b$ in front of $y$ is
		\[
			-\det \begin{bmatrix} 5 & -1 \\ -5 & 4 \end{bmatrix} = -(20 - 5) = -15 .
		\]

\spc 

	\exo{3.1}{17}{5}
	\\
	Using row operations, we see that
		\[
			\det \begin{bmatrix} 0 & 1 & 1 & 1 \\ 1 & 0 & x & x \\ 1 & x & 0 & x \\ 1 & x & x & 0 \end{bmatrix} = \det \begin{bmatrix} 0 & 1 & 1 & 1 \\ 0 & -x & 0 & x \\ 0 & 0 & -x & x \\ 1 & x & x & 0 \end{bmatrix} = - \det \begin{bmatrix} 1 & x & x & 0 \\ 0 & -x & 0 & x \\ 0 & 0 & -x & x \\ 0 & 1 & 1 & 1 \end{bmatrix}
		\]
	and
		\[
			- \det \begin{bmatrix} 1 & x & x & 0 \\ 0 & -x & 0 & x \\ 0 & 0 & -x & x \\ 0 & 1 & 1 & 1 \end{bmatrix} = -x^2 \det \begin{bmatrix} 1 & x & x & 0 \\ 0 & -1 & 0 & 1 \\ 0 & 0 & -1 & 1 \\ 0 & 1 & 1 & 1 \end{bmatrix} = -x^2 \det \begin{bmatrix} 1 & x & x & 0 \\ 0 & -1 & 0 & 1 \\ 0 & 0 & -1 & 1 \\ 0 & 0 & 0 & 3 \end{bmatrix} = - 3x^2 .
		\]
	If $A$ denote the matrix from the beginning, then $\det A = -3x^2$. 


\end{document}