\documentclass[12pt]{article}
\usepackage[utf8]{inputenc}

\usepackage{enumitem}
\usepackage[margin=2cm]{geometry}

\usepackage{amsmath, amsfonts, amssymb}
\usepackage{graphicx}
\usepackage{tikz}
\usepackage{pgfplots}
\usepackage{multicol}

\usepackage{comment}
\usepackage{url}
\usepackage{calc}
\usepackage{subcaption}
\usepackage{circledsteps}
\usepackage{wrapfig}
\usepackage{array}
\usepackage{systeme}
\sysdelim..

\setlength\parindent{0pt}

\usepackage{fancyhdr}
\pagestyle{fancy}
\fancyhf{}
\renewcommand{\headrulewidth}{2pt}
\renewcommand{\footrulewidth}{0pt}
\rfoot{\thepage}
\lhead{\textsc{Math} 311}
\chead{\textsc{Homework 9}}
\rhead{Spring 2024}

\pgfplotsset{compat=1.16}

% MATH commands
\newcommand{\ga}{\left\langle}
\newcommand{\da}{\right\rangle}
\newcommand{\oa}{\left\lbrace}
\newcommand{\fa}{\right\rbrace}
\newcommand{\oc}{\left[}
\newcommand{\fc}{\right]}
\newcommand{\op}{\left(}
\newcommand{\fp}{\right)}

\newcommand{\bi}{\mathbf{i}}
\newcommand{\bj}{\mathbf{j}}
\newcommand{\bk}{\mathbf{k}}
\newcommand{\bF}{\mathbf{F}}

\newcommand{\ra}{\rightarrow}
\newcommand{\Ra}{\Rightarrow}

\newcommand{\sech}{\mathrm{sech}\,}
\newcommand{\csch}{\mathrm{csch}\,}
\newcommand{\curl}{\mathrm{curl}\,}
\newcommand{\dive}{\mathrm{div}\,}

\newcommand{\ve}{\varepsilon}
\newcommand{\spc}{\vspace*{0.5cm}}

\DeclareMathOperator{\Ran}{Ran}
\DeclareMathOperator{\Dom}{Dom}

\newcommand{\exo}[3]{\noindent\textcolor{red}{\fbox{\textbf{Section {#1} | Problem {#2}}}\hrulefill   \textbf{({#3} Pts})}\vspace*{10pt}}

\makeatletter
\renewcommand*\env@matrix[1][*\c@MaxMatrixCols c]{%
  \hskip -\arraycolsep
  \let\@ifnextchar\new@ifnextchar
  \array{#1}}
\makeatother

\begin{document}
\thispagestyle{empty}
	\noindent \hrulefill \newline
	MATH-311 \hfill Pierre-Olivier Paris{\'e}\newline
	Homework 10 solutions \hfill Spring 2024\newline \vspace*{-0.7cm}
	
	\noindent\hrulefill
	
	\spc

	\exo{6.2}{6}{10} 
		\begin{enumerate}
			\item[a.] We want to find $a$, $b$, and $c$ such that
				\[
					a (x + 1) + b(x^2 + x) + c (x^2 + 2) = x^2 + 3x + 2
				\]
			which can be rewritten as
				\[
					(b + c)x^2 + (a + b) x + (a + 2c) = x^2 + 3x + 2 .
				\]
			Therefore, $a$, $b$, $c$ are solutions to the system
				\[
					\left\{ \systeme{b + c = 1, a + b = 3, a + 2c = 2} \right.
				\]
			The solution to this system is $a = 2$, $b = 1$ and $c = 0$. Therefore,
				\[
					2(x + 1) + (x^2 + x) = x^2 + 3x + 2 .
				\]
			\item[c.] We want to find $a$, $b$, and $c$ such that
				\[
					a (x + 1) + b (x^2 + x) + c (x^2 + 2) = x^2 + 1
				\]
			which can be rewritten as
				\[
					(b + c)x^2 + (a + b) x + (a + 2c) = x^2 + 1 .
				\]
			Therefore, $a$, $b$, $c$ are solutions to the system
				\[
					\left\{ \systeme{b + c = 1, a + b = 0, a + 2c = 1} \right.
				\]
			The solution to this system is $a = -1/3$, $b = 1/3$, and $c = 2/3$. Therefore
				\[
					-\tfrac{1}{3} (x + 1) + \tfrac{1}{3} (x^2 + x) + \tfrac{2}{3} (x^2 + 2) = x^2 + 1 .
				\]
		\end{enumerate}

	\newpage

	\exo{6.2}{9}{10}
		\begin{enumerate}
			\item[a.] Collect the vectors in a matrix
				\[
					A = \begin{bmatrix} 1 & 1 & 0\\0 & 1 & 1\\1 & 0 & 1 \end{bmatrix} .
				\]
			The goal is to write any vector $(a, b, c)$ from $\mathbb{R}^3$ as a linear combination of the vectors in the collection, that is solving the system
				\[
					A \mathbf{x} = \begin{bmatrix} a \\ b \\ c \end{bmatrix} .
				\]
			We have $\det (A) = 2 \neq 0$. Therefore the matrix $A$ is invertible and there is always a solution to the above system. Hence, the vectors span $\mathbb{R}^3$.
			\item[c.] Let $M = \begin{bmatrix} a & b \\ c & d \end{bmatrix}$ be a $2 \times 2$ matrix. We want to know if
				\[
					M = x_1 \begin{bmatrix} 1 & 0 \\ 0 & 0 \end{bmatrix} + x_2 \begin{bmatrix} 1 & 0 \\ 0 & 1 \end{bmatrix} + x_3 \begin{bmatrix} 0 & 1 \\ 1 & 0 \end{bmatrix} + x_4 \begin{bmatrix} 1 & 1 \\ 0 & 1 \end{bmatrix} .
				\]
			Using the operations on matrices, we then get
				\[
					\begin{bmatrix} a & b \\ c & d \end{bmatrix} = \begin{bmatrix} x_1 + x_2 + x_4 & x_3 + x_4 \\ x_3 & x_2 + x_4 \end{bmatrix} .
				\]
			Therefore, the vector $\mathbf{x}$ is solution to the system
				\[
					\begin{bmatrix} 1 & 1 & 0 & 1 \\ 0 & 0 &1 & 1 \\ 0 & 0 & 1 & 0 \\ 0 & 1 & 0 & 1 \end{bmatrix} \begin{bmatrix} x_1 \\ x_2 \\ x_3 \\ x_4 \end{bmatrix} = \begin{bmatrix} a \\ b \\ c \\ d \end{bmatrix} .
				\]
			If $A$ is the matrix of coefficients, then $\det (A) = -1 \neq 0$. Therefore, the matrix $A$ is invertible and the above system always has a solution. Hence, the set of matrices span $\mathbf{M_{22}}$.
		\end{enumerate}

	\spc

	\exo{6.2}{14}{5}

	If it was possible, then a linear combination with the two given vectors will give $(c + d, 2c + d, d)$, for $c, d \in \mathbb{R}$. If $d \neq 0$, then it won't take the form of the vectors in $U$. So it is impossible.
	
	\spc

	\exo{5.2}{3}{10}
		\begin{enumerate}
			\item[a.] We put the vectors in a matrix
				\[
					A = \begin{bmatrix}1 & 2 & 1\\-1 & 3 & 9\\2 & 0 & -6\\0 & 3 & 6\end{bmatrix} .
				\]
				We find the RREF of $A$:
					\[
						\begin{bmatrix}1 & 0 & -3\\0 & 1 & 2\\0 & 0 & 0\\0 & 0 & 0\end{bmatrix}
					\]
				The dimension of $U$ is $2$ because the number of pivots in the RREF is $2$ and a basis for $U$ is the first two columns of $A$.
			\item[c.] We put the vectors in a matrix
				\[
					A =\begin{bmatrix}-1 & 2 & 4 & 3\\2 & 0 & 4 & -2\\1 & 3 & 11 & 2\\0 & -1 & -3 & -1\end{bmatrix} .
				\]
			We find the RREF of $A$:
				\[
					\begin{bmatrix}1 & 0 & 2 & -1\\0 & 1 & 3 & 1\\0 & 0 & 0 & 0\\0 & 0 & 0 & 0\end{bmatrix} .
				\]
			The dimension of $U$ is $2$ because the number of pivots in the RREF is $2$ and a basis for $U$ is the first two columns of $A$.
		\end{enumerate}

	\spc

	\exo{6.3}{1d}{5}
	
	Set
		\[
			a \begin{bmatrix} 1 & 1 \\ 1 & 0 \end{bmatrix} + b \begin{bmatrix} 0 & 1 \\ 1 & 1 \end{bmatrix} + c \begin{bmatrix} 1 & 0 \\ 1 & 1 \end{bmatrix} + d \begin{bmatrix} 1 & 1 \\ 0 & 1 \end{bmatrix} = \begin{bmatrix} 0 & 0 \\ 0 & 0 \end{bmatrix} .
		\]
	We obtain, after using the matrix operations,
		\[
			\begin{bmatrix} a + c + d & a + b + d \\ a + b + c & b + c + d \end{bmatrix} = \begin{bmatrix} 0 & 0 \\ 0 & 0 \end{bmatrix}
		\]
	We therefore get the following system
		\[
			\begin{bmatrix} 1 & 0 & 1 & 1\\1 & 1 & 0 & 1\\1 & 1 & 1 & 0\\0 & 1 & 1 & 1 \end{bmatrix} \begin{bmatrix} a \\ b \\ c \\ d \end{bmatrix} = \begin{bmatrix} 0 \\ 0 \\ 0 \\ 0 \end{bmatrix} .
		\]
	If $A$ is the matrix of coefficients, then $\det (A) = 3$ and hence the matrix $A$ is invertible. Therefore, the unique solution to the system is $A^{-1} \mathbf{0} = \mathbf{0}$. Hence $a = b = c = d = 0$ and the matrices are linearly independent.

	\spc

	

	\exo{6.3}{6}{10}
		\begin{enumerate}
			\item[c.] Let $p (x) = ax^2 + bx + c$ be an element of $U$. This means $p (1) = 0$, and therefore $a + b + c = 0$. Hence
				\begin{align*}
					U = \{ ax^2 + bx + c \, : \, a + b + c = 0 \} &= \{ ax^2 + bx + (- b - a) \, : \, a, b \in \mathbb{R} \} \\ 
					&= \{ a (x^2 - 1) + b (x - 1) \, : \, a, b, c \in \mathbb{R} \}.
				\end{align*}
			Hence we have $U = \mathrm{span} \{ x^2 - 1 , x - 1 \}$ and $\{ x^2 - 1 , x - 1 \}$ is linear independent. Therefore $\{ x^2 - 1 , x - 1 \}$ is a basis for $U$ and $\dim U = 2$.
			\item[d.] Let $p (x) = ax^2 + bx + c$ be an element of $U$. This means $p (x) = p (-x)$ and therefore
				\[
					ax^2 + b x + c = ax^2 - bx + c \iff 2b x = 0 \iff b = 0 .
				\]
			Hence
				\begin{align*}
				U = \{ ax^2 + c \, : \, a , c \in \mathbb{R} \} = \mathrm{span} \, \{ x^2 , 1 \} .
				\end{align*}
			Also $\{ x^2 , 1 \}$ are linear independent and therefore form a basis for $U$. We then get $\dim U = 2$. 
		\end{enumerate}






\end{document}