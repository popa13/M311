\documentclass[12pt]{article}
\usepackage[utf8]{inputenc}

\usepackage{enumitem}
\usepackage[margin=2cm]{geometry}

\usepackage{amsmath, amsfonts, amssymb}
\usepackage{graphicx}
\usepackage{tikz}
\usepackage{pgfplots}
\usepackage{multicol}

\usepackage{comment}
\usepackage{url}
\usepackage{calc}
\usepackage{subcaption}
\usepackage{circledsteps}
\usepackage{wrapfig}
\usepackage{array}
\usepackage{systeme}
\sysdelim..

\setlength\parindent{0pt}

\usepackage{fancyhdr}
\pagestyle{fancy}
\fancyhf{}
\renewcommand{\headrulewidth}{2pt}
\renewcommand{\footrulewidth}{0pt}
\rfoot{\thepage}
\lhead{\textsc{Math} 311}
\chead{\textsc{Homework 1}}
\rhead{Fall 2023}

\pgfplotsset{compat=1.16}

% MATH commands
\newcommand{\ga}{\left\langle}
\newcommand{\da}{\right\rangle}
\newcommand{\oa}{\left\lbrace}
\newcommand{\fa}{\right\rbrace}
\newcommand{\oc}{\left[}
\newcommand{\fc}{\right]}
\newcommand{\op}{\left(}
\newcommand{\fp}{\right)}

\newcommand{\bi}{\mathbf{i}}
\newcommand{\bj}{\mathbf{j}}
\newcommand{\bk}{\mathbf{k}}
\newcommand{\bF}{\mathbf{F}}

\newcommand{\ra}{\rightarrow}
\newcommand{\Ra}{\Rightarrow}

\newcommand{\sech}{\mathrm{sech}\,}
\newcommand{\csch}{\mathrm{csch}\,}
\newcommand{\curl}{\mathrm{curl}\,}
\newcommand{\dive}{\mathrm{div}\,}

\newcommand{\ve}{\varepsilon}
\newcommand{\spc}{\vspace*{0.5cm}}

\DeclareMathOperator{\Ran}{Ran}
\DeclareMathOperator{\Dom}{Dom}

\newcommand{\exo}[3]{\noindent\textcolor{red}{\fbox{\textbf{Section {#1} | Problem {#2}}}\hrulefill   \textbf{({#3} Pts})}\vspace*{10pt}}

\makeatletter
\renewcommand*\env@matrix[1][*\c@MaxMatrixCols c]{%
  \hskip -\arraycolsep
  \let\@ifnextchar\new@ifnextchar
  \array{#1}}
\makeatother

\begin{document}
\thispagestyle{empty}
	\noindent \hrulefill \newline
	MATH-311 \hfill Pierre-Olivier Paris{\'e}\newline
	Homework 3 solutions \hfill Spring 2024\newline \vspace*{-0.7cm}
	
	\noindent\hrulefill
	
	\spc
	
	\exo{2.2}{2a}{5}
	\\
	We let $A = \begin{bmatrix} 1 & -1 & 3 \\ -3 & 1 & 1 \\ 5 & -8 & 0 \end{bmatrix}$ and $\mathbf{b} = \begin{bmatrix} 5 \\ -6 \\ 9 \end{bmatrix}$. Then the system can be rewritten as
		\[
			A \mathbf{x} = \mathbf{b} \iff \begin{bmatrix} 1 & -1 & 3 \\ -3 & 1 & 1 \\ 5 & -8 & 0 \end{bmatrix} \begin{bmatrix} x_1 \\ x_2 \\ x_3 \end{bmatrix} = \begin{bmatrix} 5 \\ -6 \\ 9 \end{bmatrix} .
		\]

	\spc

	\exo{2.3}{1}{20}
	\begin{enumerate}
		\item[b.] The answer should be
			\[
				\begin{bmatrix}1 & -1 & 2\\2 & 0 & 4\end{bmatrix} \begin{bmatrix}2 & 3 & 1\\1 & 9 & 7\\-1 & 0 & 2\end{bmatrix}= \begin{bmatrix}-1 & -6 & -2\\0 & 6 & 10\end{bmatrix} .
			\]
		\item[d.] The answer should be
			\[
					\begin{bmatrix}1 & 3 & -3\end{bmatrix} \begin{bmatrix}3 & 0\\-2 & 1\\0 & 6\end{bmatrix} = \begin{bmatrix}-3 & -15\end{bmatrix} .
			\]
		\item[f.] The answer should be
			\[
				\begin{bmatrix}1 & -1 & 3\end{bmatrix}\begin{bmatrix}2\\1\\-8\end{bmatrix} = \begin{bmatrix}-23\end{bmatrix}
			\]
		\item[g.] The answer should be 
			\[
				\begin{bmatrix}2\\1\\-7\end{bmatrix} \begin{bmatrix}1 & -1 & 3\end{bmatrix} = \begin{bmatrix}2 & -2 & 6\\1 & -1 & 3\\-7 & 7 & -21\end{bmatrix} .
			\]
	\end{enumerate}

	\spc 

	\exo{2.3}{2a}{10}
	\\ 
	We first consider the squares $A^2$, $B^2$, $C^2$.
	\begin{itemize}
		\item The matrix $A$ is a $2 \times 3$ and hence $A^2 = A A$ is not defined. 
		\item Similarly, the matrix $C$ is a $3 \times 2$ matrix and hence $C^2 = C C$ is not defined.
		\item The matrix $B$ is a $2 \times 2$ and the product $B^2$ is defined. The result is
			\[
				\begin{bmatrix}1 & -2\\1/2 & 3\end{bmatrix}\begin{bmatrix}1 & -2\\1/2 & 3\end{bmatrix} = \begin{bmatrix}	0 & -8\\2 & 8\end{bmatrix}
			\]
	\end{itemize}

	Now we consider the possible products between $A$ and $B$.
	\begin{itemize}
		\item The matrix $A$ is $2 \times 3$ and $B$ is $2 \times 2$. So $AB$ is undefined.
		\item The matrix $B$ is $2 \times 2$ and $A$ is $2 \times 3$. Hence, $BA$ is defined. Then
			\[
				BA = \begin{bmatrix}1 & -2\\1/2 & 3\end{bmatrix}\begin{bmatrix}1 & 2 & 3\\-1 & 0 & 0\end{bmatrix} = \begin{bmatrix}3 & 2 & 3\\-5/2 & 1 & 3/2\end{bmatrix} .
			\]
	\end{itemize}

	Now we consider the possible products between $A$ and $C$.
	\begin{itemize}
		\item The matrix $A$ is $2 \times 3$ and $C$ is $3 \times 2$. So $AC$ is defined. The result is
			\[
				AC = \begin{bmatrix}1 & 2 & 3\\-1 & 0 & 0\end{bmatrix}\begin{bmatrix}-1 & 0\\2 & 5\\0 & 3\end{bmatrix}= \begin{bmatrix}3 & 19\\1 & 0\end{bmatrix} .
			\]
		\item The matrix $C$ is $3 \times 2$ and $A$ is $2 \times 3$, so $CA$ is also defined. The result is
			\[
				CA = \begin{bmatrix}-1 & 0\\2 & 5\\0 & 3\end{bmatrix}\begin{bmatrix}1 & 2 & 3\\-1 & 0 & 0\end{bmatrix} = \begin{bmatrix}-1 & -2 & -3\\-3 & 4 & 6\\-3 & 0 & 0\end{bmatrix} .
			\]
	\end{itemize}

	Now we consider the possible products between $B$ and $C$.
	\begin{itemize}
		\item The matrix $B$ is $2 \times 2$ and $C$ is $3 \times 2$. Hence $BC$ is undefined.
		\item The matrix $C$ is $3 \times 2$ and $B$ is $2 \times 2$. Hence $CB$ is defined. The result is
			\[
				CB = \begin{bmatrix}-1 & 0\\2 & 5\\0 & 3\end{bmatrix}\begin{bmatrix}1 & -2\\0.5 & 3\end{bmatrix} = \begin{bmatrix}-1 & 2\\4.5 & 11\\1.5 & 9\end{bmatrix}
			\]
	\end{itemize}

	\spc

	\exo{2.3}{16d}{5}
	\\ 
	We have
		\[
			(A - B) (C - A) = AC - A^2 - BC + (-B) (-A) = AC - A^2 - BC + BA 
		\]
	and
		\[
			(C - B) (A - C) = CA + C(-C) - BA + BC = CA - C^2 - BA + BC
		\]
	and
		\[
			(C - A)^2 = (C - A) (C - A) = C^2 - CA - AC + A^2 .
		\]
	Plugging that into the original equation, denote it by $E$, we get
		\begin{align*}
			E &= AC - A^2 - BC + BA + CA - C^2 - BA + BC + C^2 - CA - AC + A^2 \\ 
			&= (AC- AC) + (A^2 - A^2) + (BC - BC) + (BA - BA) + (CA - CA) - C^2 + C^2 \\ 
			&= 0 .
		\end{align*}

	\spc 

	\exo{2.3}{34a}{10}
	\\
	Assume that $AB = BA$. Then
		\[
			(A + B)^2 = (A + B) (A + B) = A^2 + AB + BA + B^2 .
		\]
	Since $AB = BA$, then $AB + BA = AB + AB = 2AB$ and hence
		\[
			(A + B)^2 = A^2 + 2AB + B^2 .
		\]

	Now, assume that $(A + B)^2 = A^2 + 2AB + B^2$. Then we get
		\[
			A^2 + AB + BA + B^2 = A^2 + 2AB + B^2 .
		\]
	Substracting $A^2$, $B^2$ and $AB$ on each side, we get
		\[
			BA = 2AB - AB = AB .
		\]
	Hence $BA = AB$. 

	\end{document}


	

	\exo{2.3}{12a}{}
	\\ 
	Let $S$, $T$, $U$, and $V$ be the respective blocs of $A$. Then
		\[
			A^2 = \begin{bmatrix} S & T \\ U & V \end{bmatrix} \begin{bmatrix} S & T \\ U & V \end{bmatrix} = \begin{bmatrix} S^2 + TU & ST + TV \\ US + VU & UT + V^2 \end{bmatrix} 
		\]
	We will compute each entry of the bloc matrix $A^2$.
	\begin{itemize}
		\item Notice that $S^2 = [1]$ always and 
		\[
			TU = \begin{bmatrix} 0 & 0 \end{bmatrix} \begin{bmatrix} 1 \\ 1 \end{bmatrix} = \begin{bmatrix} 0 \end{bmatrix} .
		\]
	Hence, $S^2 + TU = [1] + [0] = [1]$. 
	\item	Now, we have
		\[
			ST = \begin{bmatrix} 1 \end{bmatrix} \begin{bmatrix} 0 & 0 \end{bmatrix} = \begin{bmatrix} 0 & 0 \end{bmatrix}
		\]
	and
		\[
			T V = \begin{bmatrix} 0 & 0 \end{bmatrix} \begin{bmatrix} -1 & 1 \\ 0 & 1 \end{bmatrix} = \begin{bmatrix} 0 & 0 \end{bmatrix} .
		\]
	hence, $ST + TV = \begin{bmatrix} 0 & 0 \end{bmatrix}$. 
	\item Notice that 	
		\[
			US = \begin{bmatrix} 1 \\ 1 \end{bmatrix} \begin{bmatrix} 1 \end{bmatrix} = \begin{bmatrix} 1 \\ 1 \end{bmatrix}
		\]
		and
		\[
			VU = \begin{bmatrix} -1 & 1 \\ 0 & 1 \end{bmatrix} \begin{bmatrix} 1 \\ 1 \end{bmatrix} = \begin{bmatrix} 0 \\ 1 \end{bmatrix} .
		\]
		Hence $US + VU = \begin{bmatrix} 1 \\ 2 \end{bmatrix}$.
	\item Finally, we have
		\[
			UT = \begin{bmatrix} 1 \\ 1 \end{bmatrix} \begin{bmatrix} 0 & 0 \end{bmatrix} = \begin{bmatrix} 0 & 0 \\ 0 & 0 \end{bmatrix}
		\]
	and
		\[
			V^2 = \begin{bmatrix} -1 & 1 \\ 0 & 1 \end{bmatrix} \begin{bmatrix} -1 & 1 \\ 0 & 1 \end{bmatrix} = \begin{bmatrix} 1 & 0 \\ 0 & 1 \end{bmatrix} .
		\]
	Hence, $UT + V^2 = \begin{bmatrix} 1 & 0 \\ 0 & 1 \end{bmatrix}$.
	\end{itemize}

	Combining everything, we find that
		\[
			A^2 = \begin{bmatrix} 1 & 0 \\ 0 & 1 \end{bmatrix} .
		\]

	\spc 