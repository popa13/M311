\documentclass[12pt]{article}
\usepackage[utf8]{inputenc}

\usepackage{enumitem}
\usepackage[margin=2cm]{geometry}

\usepackage{amsmath, amsfonts, amssymb}
\usepackage{graphicx}
\usepackage{tikz}
\usepackage{pgfplots}
\usepackage{multicol}

\usepackage{comment}
\usepackage{url}
\usepackage{calc}
\usepackage{subcaption}
\usepackage{circledsteps}
\usepackage{wrapfig}
\usepackage{array}
\usepackage{systeme}
\sysdelim..

\setlength\parindent{0pt}

\usepackage{fancyhdr}
\pagestyle{fancy}
\fancyhf{}
\renewcommand{\headrulewidth}{2pt}
\renewcommand{\footrulewidth}{0pt}
\rfoot{\thepage}
\lhead{\textsc{Math} 311}
\chead{\textsc{Homework 8}}
\rhead{Spring 2024}

\pgfplotsset{compat=1.16}

% MATH commands
\newcommand{\ga}{\left\langle}
\newcommand{\da}{\right\rangle}
\newcommand{\oa}{\left\lbrace}
\newcommand{\fa}{\right\rbrace}
\newcommand{\oc}{\left[}
\newcommand{\fc}{\right]}
\newcommand{\op}{\left(}
\newcommand{\fp}{\right)}

\newcommand{\bi}{\mathbf{i}}
\newcommand{\bj}{\mathbf{j}}
\newcommand{\bk}{\mathbf{k}}
\newcommand{\bF}{\mathbf{F}}

\newcommand{\ra}{\rightarrow}
\newcommand{\Ra}{\Rightarrow}

\newcommand{\sech}{\mathrm{sech}\,}
\newcommand{\csch}{\mathrm{csch}\,}
\newcommand{\curl}{\mathrm{curl}\,}
\newcommand{\dive}{\mathrm{div}\,}

\newcommand{\ve}{\varepsilon}
\newcommand{\spc}{\vspace*{0.5cm}}

\DeclareMathOperator{\Ran}{Ran}
\DeclareMathOperator{\Dom}{Dom}

\newcommand{\exo}[3]{\noindent\textcolor{red}{\fbox{\textbf{Section {#1} | Problem {#2}}}\hrulefill   \textbf{({#3} Pts})}\vspace*{10pt}}

\makeatletter
\renewcommand*\env@matrix[1][*\c@MaxMatrixCols c]{%
  \hskip -\arraycolsep
  \let\@ifnextchar\new@ifnextchar
  \array{#1}}
\makeatother

\begin{document}
\thispagestyle{empty}
	\noindent \hrulefill \newline
	MATH-311 \hfill Pierre-Olivier Paris{\'e}\newline
	Homework 8 solutions \hfill Spring 2024\newline \vspace*{-0.7cm}
	
	\noindent\hrulefill
	
	\spc

	\exo{3.3}{1b}{15}
	\\
		\underline{Characteristic polynomial:} We have
				\[
					c_A (x) = \det (x I - A) = \begin{vmatrix} x - 2 & 4 \\ 1 & x + 1 \end{vmatrix} = (x-2)(x+1) - 4 = x^2 - x - 6 .
				\]
		\underline{Eigenvalues:} We have
			\[
				c_A (x) = 0 \iff x^2 - x - 6 = 0 \iff (x + 2) (x - 3) = 0 .
			\]
		Hence $\lambda_1 = -2$ and $\lambda_2 = 3$. 

		\underline{Eigenvectors:} We have two eigenvalues.
			\begin{itemize}
				\item $\lambda = -2$. We have to solve the system
					\[
						\left( -2 \begin{bmatrix} 1 & 0 \\ 0 & 1 \end{bmatrix} - \begin{bmatrix} 2 & -4 \\ -1 & -1 \end{bmatrix} \right) \begin{bmatrix} x \\ y \end{bmatrix} = \begin{bmatrix} 0 \\ 0 \end{bmatrix} \iff \begin{bmatrix} -4 & 4 \\ 1 & -1 \end{bmatrix} \begin{bmatrix} x \\ y \end{bmatrix} = \begin{bmatrix} 0 \\ 0 \end{bmatrix} .
					\]
				The system is $-4x + 4y = 0$ and $x - y = 0$. This reduces to the single equation
					\[
						x - y = 0 \iff x = y .
					\]
				Hence, we have
					\[
						\mathbf{x_1} = \begin{bmatrix} x \\ y \end{bmatrix} = \begin{bmatrix} x \\ x \end{bmatrix} = x \begin{bmatrix} 1 \\ 1 \end{bmatrix} .
					\]
				\item $\lambda = 3$. We have to solve the system
					\[
						\begin{bmatrix} 1 & 4 \\ 1 & 4 \end{bmatrix} \begin{bmatrix} x \\ y \end{bmatrix} = \begin{bmatrix} 0 \\ 0 \end{bmatrix} .
					\]
				The system of equations is $x + 4y = 0$ and $x + 4y =0$. This can be rewritten as the single equation
					\[
						x + 4y = 0 \iff x = -4y .
					\]
				Hence
					\[
						\mathbf{x_2} = \begin{bmatrix} x \\ y \end{bmatrix} = \begin{bmatrix} -4y \\ y \end{bmatrix} = y \begin{bmatrix} -4 \\ 1 \end{bmatrix} .
					\]
			\end{itemize}

		\underline{Diagonalization:} Since the eigenvalues are all different, the matrix $P$ exists. In the expression of $\mathbf{x_1}$, we let $x = 1$, so that
			\[
				\mathbf{x_1} = \begin{bmatrix} 1 \\ 1 \end{bmatrix} .
			\]
		In the expression of $\mathbf{x_2}$, we let $y = 1$, so that
			\[
				\mathbf{x_2} = \begin{bmatrix} -4 \\ 1 \end{bmatrix} .
			\]
		Hence, $P = \begin{bmatrix} \mathbf{x_1} & \mathbf{x_2} \end{bmatrix}$, so that
			\[
				P = \begin{bmatrix} 1 & -4 \\ 1 & 1 \end{bmatrix} .
			\]

		\newpage 

		\exo{3.3}{3}{5}
		\\ 
		Assume that $A$ has $\lambda = 0$ as an eigvalue. The caracteristic polynomial is $c_A (x) = \det (x I - A)$. Since $\lambda =0$ is an eigenvalue, then $c_A (0) = 0$. This means that $\det (0I - A) = \det (A) = 0$. Hence, $A$ is not invertible.

		\vspace*{0.25cm}

		Now assume that $A$ is not invertible. Then $\det (A) = 0$. Using the characteristic polynomial, we see that $c_A (0) = \det (A) = 0$, hence $\lambda = 0$ is an eigenvalue of $A$.

		\spc 

		\exo{6.1}{1}{10} 
		\begin{enumerate}
			\item[a.] Since the addition is defined as the one we define on $\mathbb{R}^3$, the axioms A1-A5 are satisfied. We have to check if S1-S5 are satisfied.
				\begin{enumerate}[label=S\arabic*.]
					\item From the definition $a (x_1, x_2, x_3) = (ax_1, x_2, ax_3)$ is a vector in $\mathbb{R}^3$. We're good!
					\item We have to check if $a (\mathbf{x} + \mathbf{y} ) = a \mathbf{x} + a\mathbf{y}$. We have
						\[
							a (\mathbf{x} + \mathbf{y}) = a (x_1 + y_1 , x_2 + y_2 , x_3 + y_3) = ( a (x_1 + y_1), x_2 + y_2 , a (x_3 + y_3) ) 
						\]
					and
						\[
							a \mathbf{x} + b \mathbf{y} = (a x_1 , x_2 , a x_3) + (ay_1 , y_2 , ay_3) = (a x_1 + ay_1 , x_2 + y_2 , a x_3 + a y_3 ) 
						\]
					Factoring $a$ from the first and third entries, we see that
						\[
							a \mathbf{x} + a \mathbf{y} = a (\mathbf{x} + \mathbf{y} ) .
						\]
					\item We have now to check that $(a + b) \mathbf{x} = a \mathbf{x} + b \mathbf{x}$. We have
						\[
							(a + b) \mathbf{x} = ( (a + b) x_1 , x_2 , (a + b) x_3 ) = (a x_1 + b x_1 , x_2 , a x_3 + b x_3 ) 
						\]
					and
						\[
							a \mathbf{x} + b \mathbf{x} = (a x_1 , x_2 , a x_3) + (b x_1 , x_2 , b x_3) = (ax_1 + bx_1 , x_2 + x_2 , a x_3 + b x_3 )
						\]
					We see that the second entry is now $2x_2$ which is different from $x_2$! In general, we won't have $(a + b) \mathbf{x} = a \mathbf{x} + b \mathbf{x}$.

					To support that claim, let's give an example. Let $\mathbf{x} = (1, 1, 1)$. Then
						\[
							(1 + 2 ) \mathbf{x} = (3) \mathbf{x} = (3, 1, 3) 
						\]
					but
						\[
							(1) \mathbf{x} + (2) \mathbf{x} = (1, 1, 1) + (2, 1, 2) = (3, 2, 3) .
						\]
					We see that $(3, 1, 3) \neq (3, 2, 3)$. 
				\end{enumerate}
				\underline{Conclusion:} Axiom S3 is not satisfied with this scalar multiplication and $\mathbb{R}^3$ is not a vector space if we consider this scalar multiplication.
			\item[b.] Since the addition is defined as the one defined on $\mathbb{R}^3$, the axioms A1-A5 are satisfied. We will check if axioms S1-S5 are satisfied. In fact, we can go straight to the last one! Indeed, to satisfy S5, we must show that 
				\[
					1 \mathbf{x} = \mathbf{x} .
				\]
			However, based on the definition of the scalar multiplication, we have
				\[
					1 (x_1 , x_2 , x_3) = ( (1) x_1 , 0 , (1) x_3) = (x_1 , 0 , x_3) .
				\]
			However, $(x_1 , 0 , x_3) \neq (x_1 , x_2 , x_3)$ in general. To support that claim, let's consider the following example. Let $\mathbf{x} = (1, 1, 1)$, then
				\[
					1 \mathbf{x} = (1, 0, 1) \neq (1, 1, 1) .
				\]
			\underline{Conclusion:} Axiom S5 is not satisfied with this scalar multiplication and $\mathbb{R}^3$ is not a vector space if we consider this scalar multiplication.
		\end{enumerate}

		\spc 

		\exo{6.1}{2}{10}
		\begin{enumerate}
			\item[e.] We'll check if all the axioms are satisfied. The addition and scalar multiplication is the same as the ones used for $2 \times 2$ matrices.

			Let $A = \begin{bmatrix} a_{11} & a_{12} \\ 0 & a_{22} \end{bmatrix}$, $B = \begin{bmatrix} b_{11} & b_{12} \\ 0 & b_{22} \end{bmatrix}$, and $C = \begin{bmatrix} c_{11} & c_{12} \\ 0 & c_{22} \end{bmatrix}$ be $2 \times 2$ matrices from $V$.

			We first start with the axioms A1-A5.
				\begin{enumerate}
				\item[A1.] We have
					\[
						A + B = \begin{bmatrix} a_{11} + b_{11} & a_{12} + b_{12} \\ 0 + 0 & a_{22} + b_{22} \end{bmatrix} = \begin{bmatrix} a_{11} + b_{11} & a_{12} + b_{12} \\ 0 & a_{22} + b_{22} \end{bmatrix} .
					\]
					We see that $A + B$ is of the same type as the matrices in $V$. We're good!
					\item[A2.] We have
						\[
							A + B = \begin{bmatrix} a_{11} + b_{11} & a_{12} + b_{12} \\ 0 & a_{22} + b_{22} \end{bmatrix} = \begin{bmatrix} b_{11} + a_{11} & b_{12} + a_{12} \\ 0 + 0 & b_{22} + a_{22} \end{bmatrix} = \begin{bmatrix} b_{11} & b_{12} \\ 0 & b_{22} \end{bmatrix} + \begin{bmatrix} a_{11} & a_{12} \\ 0 & a_{22} \end{bmatrix} = B + A .
						\]
						So we're good!
					\item[A3.] Now we have
						\[
							(A + B) + C = \begin{bmatrix} a_{11} + b_{11} & a_{12} + b_{12} \\ 0 & a_{22} + b_{22} \end{bmatrix} + \begin{bmatrix} c_{11} & c_{12} \\ 0 & c_{22} \end{bmatrix} = \begin{bmatrix} a_{11} + b_{11} + c_{11} & a_{12} + b_{12} + c_{12} \\ 0 + 0 & a_{22} + b_{22} + c_{22} \end{bmatrix}
						\]
					and
						\[
							A + (B + C) = \begin{bmatrix} a_{11} & a_{12} \\ 0 & a_{22} \end{bmatrix} + \begin{bmatrix} b_{11} + c_{11} & b_{12} + c_{12} \\ 0 & b_{22} + c_{22} \end{bmatrix} = \begin{bmatrix} a_{11} + b_{11} + c_{11} & a_{12} + b_{12} + c_{12} \\ 0 + 0 & a_{22} + b_{22} + c_{22} \end{bmatrix} .
						\]
					Hence, $(A + B) + C = A + (B + C)$ and we're good!
					\item[A4.] Let $\mathbf{0} = \begin{bmatrix} 0 & 0 \\ 0 & 0 \end{bmatrix}$. Then, we see that
						\[
							A + \mathbf{0} = \begin{bmatrix} a_{11} + 0 & a_{12} + 0 \\ 0 + 0 & a_{22} + 0 \end{bmatrix} = \begin{bmatrix} a_{11} & a_{12} \\ 0 & a_{22} \end{bmatrix} = A .
						\]
					We're still good!
					\item[A5.] Let $-A = \begin{bmatrix} -a_{11} & -a_{12} \\ 0 & -a_{22} \end{bmatrix}$. Then
						\[
							A + (-A) = \begin{bmatrix} a_{11} - a_{11} & a_{12} - a_{12} \\ 0 - 0 & a_{22} - a_{22} \end{bmatrix} = \begin{bmatrix} 0 & 0 \\ 0 & 0 \end{bmatrix} = \mathbf{0} .
						\]
					We are doing pretty great so far!
			\end{enumerate}

		We now verify if S1-S5 are satisfied.
			\begin{enumerate}[label=S\arabic*.]
				\item We have 
					\[
						a A = \begin{bmatrix} a a_{11} & a a_{12} \\ (a)(0) & a a_{22} \end{bmatrix} = \begin{bmatrix} a a_{11} & a a_{12} \\ 0 & a a_{22} \end{bmatrix} .
					\]
				Then we see that $aA$ is a matrix from $V$. We're good!
				\item We have
					\begin{align*}
						a (A + B) = a \begin{bmatrix} a_{11} + b_{11} & a_{12} + b_{12} \\ 0 & a_{22} + b_{22} \end{bmatrix} &= \begin{bmatrix} a (a_{11} + b_{11}) & a (a_{12} + b_{12}) \\ (a) (0) & a (a_{22} + b_{22} ) \end{bmatrix} \\ 
						&= \begin{bmatrix} a a_{11} + a b_{11} & a a_{12} + a b_{12} \\ 0 & a a_{22} + a b_{22} \end{bmatrix} 
					\end{align*}
				and
					\begin{align*}
						aA + aB = \begin{bmatrix} a a_{11} & a a_{12} \\ (a) (0) & a a_{22} \end{bmatrix} + \begin{bmatrix} a b_{11} & a b_{12} \\ (a) (0) & a b_{22} \end{bmatrix} &= \begin{bmatrix} aa_{11} + a b_{11} & a a_{12} + a b_{12} \\ 0 + 0 & a a_{22} + a b_{22} \end{bmatrix} \\ 
						&= \begin{bmatrix} aa_{11} + a b_{11} & a a_{12} + a b_{12} \\ 0  & a a_{22} + a b_{22} \end{bmatrix}
					\end{align*}
				Hence $a (A + B) = aA + aB$. We're good!
				\item We have
					\[
						(a + b) A = \begin{bmatrix} (a + b) a_{11} & (a + b) a_{12} \\ 0 & (a + b) a_{22} \end{bmatrix} = \begin{bmatrix} a a_{11} + b a_{11} & a a_{12} + b a_{12} \\ 0 & a a_{22} + b a_{22} \end{bmatrix} 
					\]
				and
					\[
						aA + bA = \begin{bmatrix} a a_{11} & a a_{12} \\ 0 & a a_{22} \end{bmatrix} + \begin{bmatrix} b a_{11} & b a_{12} \\ 0 & a b_{22} \end{bmatrix} = \begin{bmatrix} aa_{11} + b a_{11} & a a_{12} + b a_{12} \\ 0 & a a_{22} + b a_{22} \end{bmatrix} .
					\]
				Hence $(a + b)A = aA + bA$. We're still good!
				\item We have
					\[
						a (bA) = a \begin{bmatrix} ba_{11} & b a_{12} \\ (b)(0) & b a_{22} \end{bmatrix} = \begin{bmatrix} ab a_{11} & ab a_{12} \\ (a)(0) & ab a_{22} \end{bmatrix} = \begin{bmatrix} ab a_{11} & ab a_{12} \\ 0 & ab a_{22} \end{bmatrix} 
					\]
				and
					\[
						(ab) A = \begin{bmatrix} ab a_{11} & ab a_{12} \\ (ab) (0) & ab a_{22} \end{bmatrix} = \begin{bmatrix} ab a_{11} & ab a_{12} \\ 0 & ab a_{22} \end{bmatrix} .
					\]
				We see that $a (bA) = (ab) A$. Hey, you know what? We're still good!
				\item Finally, we have
					\[
						1 A = \begin{bmatrix} (1) a_{11} & (1) a_{12} \\ (1) (0) & (1) a_{22} \end{bmatrix} = \begin{bmatrix} a_{11} & a_{12} \\ 0 & a_{22} \end{bmatrix} = A .
					\]
				Hence, we are good!
			\end{enumerate}

		\underline{Conclusion:} The set $V$ equipped with the addition and scalar multiplication of $2 \times 2$ matrices is a vector space because A1-A5 and S1-S5 are satisfied.

		\item[g.] Here, it's gonna be pretty quick. Let
			\[
				A = \begin{bmatrix} 0 & 1 \\ 0 & 0 \end{bmatrix} \quad \text{ and } \quad B = \begin{bmatrix} 0 & 0 \\ 1 & 0 \end{bmatrix} .
			\]
		Then, after calculations, we get $\det (A) = \det (B) = 0$. 

		But $A + B = \begin{bmatrix} 0 & 1 \\ 1 & 0 \end{bmatrix}$ and $\det (A + B) = -1$. So $A + B$ is not part of $V$. Axiom A1 is not satisfied!

		\underline{Conclusion:} The set $V$ is not a vector space if we use the usual matrix addition and scalar multiplication as operations.
		\end{enumerate}

		\spc 

		\exo{6.1}{5a}{5}
		\\ 
		From the first equation, we find
			\[
				2\mathbf{x} + \mathbf{y} + (-\mathbf{y}) = \mathbf{u} + (-\mathbf{y}) \iff 2 \mathbf{x} = \mathbf{u} - \mathbf{y} \iff \mathbf{x} = \tfrac{1}{2} \mathbf{u} - \tfrac{1}{2} \mathbf{y} .
			\]
		Plugging $\mathbf{x}$ in the second equation, we find that
			\[
				5 (\tfrac{1}{2} \mathbf{u} - \tfrac{1}{2} \mathbf{y}) + 3 \mathbf{y} = \mathbf{v} \iff \tfrac{5}{2} \mathbf{u} - \tfrac{5}{2} \mathbf{y} + 3 \mathbf{y} = \mathbf{v} \iff \tfrac{5}{2} \mathbf{u} + \tfrac{1}{2} \mathbf{y} = \mathbf{v} 
			\]
		moving $\tfrac{5}{2} \mathbf{u}$ on the other side, we get
			\[
				\tfrac{1}{2} \mathbf{y} = \mathbf{v} - \tfrac{5}{2} \mathbf{u} \iff \mathbf{y} = 2 \mathbf{v} - 5 \mathbf{u} .
			\]
		Plugging the expression of $\mathbf{y}$ back in the expression of $\mathbf{x}$, then
			\[
				\mathbf{x} = \tfrac{1}{2} \mathbf{u} - \tfrac{1}{2} (2 \mathbf{v} - 5 \mathbf{u} ) = \tfrac{1}{2} \mathbf{u} - \mathbf{v} + \tfrac{5}{2} \mathbf{u} = 3 \mathbf{u} - \mathbf{v} .
			\]

		\spc 

		\exo{6.1}{7b}{5}
		\\ 
		We have
			\[
				4 (3 \mathbf{u} - \mathbf{v} + \mathbf{w}) - 2 [ (3 \mathbf{u} - 2 \mathbf{v}) - 3 (\mathbf{v} - \mathbf{w})] 
			\]
		which is equal to
			\begin{align*}
				12 \mathbf{u} - 4 \mathbf{v} + 4 \mathbf{w} - 2 (3 \mathbf{u} - 2 \mathbf{v}) + 6 (\mathbf{v} - \mathbf{w}) &= 12 \mathbf{u} - 4 \mathbf{v} + 4 \mathbf{w} - 6 \mathbf{u} + 4 \mathbf{v} + 6 \mathbf{v} - 6 \mathbf{w} \\ 
				&= 6 \mathbf{u} + 6 \mathbf{v} -2 \mathbf{w} .
			\end{align*}
		and
			\[
				6 (\mathbf{w} - \mathbf{u} - \mathbf{v}) = 6 \mathbf{w} - 6 \mathbf{u} - 6 \mathbf{v} .
			\]
		Hence, the answer is $4 \mathbf{w}$.
			



\end{document}